% !TeX TXS-program:compile = txs:///arara
% arara: lualatex: {shell: no, synctex: no, interaction: batchmode}

\documentclass{article}
\usepackage[margin=0.75in]{geometry}
\usepackage{cascadiamono-otf}
\usepackage{listings}
\usepackage{xcolor}
\setlength{\parindent}{0pt}

\newcommand\demotext{For \textsterling 45, almost anything can be found floating in fields.\\
	!`THE DAZED BROWN FOX QUICKLY GAVE 12345--67890 JUMPS!\\
	--- ?`But aren't Kafka's Schlo\ss{} and \AE sop's \OE uvres often na\"\i ve vis-\`a-vis the d\ae monic ph\oe nix's official r\^ole in fluffy souffl\'es?
}

\newcommand\samplettxt{oO08 iIlL1 g9qCGQ <=>}
\newcommand\samplett[1][]{#1\samplettxt}
\newcommand\samplettit[1][]{\textit{#1\samplettxt}}
\newcommand\samplettbf[1][]{\textbf{#1\samplettxt}}
\newcommand\samplettbfit[1][]{\textbf{\textit{#1\samplettxt}}}
\newcommand\sampletttbl[1][]{& \samplett[#1] &  \samplettit[#1] & \samplettbf[#1] & \samplettbfit[#1]}

\begin{document}

\part*{cascadiamono-otf (v 0.3)}

\section{Usages}

With \lstinline[language={[latex]TeX},basicstyle=\ttfamily]|\usepackage{fontspec}| (so with \lstinline[language={[latex]TeX},basicstyle=\ttfamily]|XeTeX| or \lstinline[language={[latex]TeX},basicstyle=\ttfamily]|LuaLaTeX| compilation), you can use \texttt{CascadiaCode} fonts, and \textit{remove} ligature's features, in order to use (in fact) \texttt{CascadiaMono} fonts.

\smallskip

Following OpenType fonts are supported (it's based on \texttt{CascadiaCode} without the ligatures) :

\begin{lstlisting}[language={[latex]TeX},basicstyle=\footnotesize\ttfamily,commentstyle=\itshape\color{gray},keywordstyle=\color{magenta},tabsize=4,frame=single]
CascadiaCode-Bold.otf                      CascadiaCode-BoldItalic.otf
CascadiaCode-ExtraLight.otf                CascadiaCode-ExtraLightItalic.otf
CascadiaCode-Italic.otf                    CascadiaCode-Light.otf
CascadiaCode-LightItalic.otf               CascadiaCode-Regular.otf
CascadiaCode-SemiBold.otf                  CascadiaCode-SemiBoldItalic.otf
CascadiaCode-SemiLight.otf                 CascadiaCode-SemiLightItalic.otf
\end{lstlisting}

\section{The default settings}

The \texttt{fontspec} config for the \textit{normal} version :

\begin{lstlisting}[language={[latex]TeX},basicstyle=\footnotesize\ttfamily,commentstyle=\itshape\color{gray},keywordstyle=\color{magenta},tabsize=4,frame=single]
\defaultfontfeatures[CascadiaMono]
    {Extension=.otf,
    UprightFont=CascadiaCode-Regular,
    ItalicFont=CascadiaCode-Italic,
    BoldFont=CascadiaCode-Bold,
    BoldItalicFont=CascadiaCode-BoldItalic,
    Contextuals=AlternateOff}
\end{lstlisting}

The \texttt{fontspec} config for the other versions are :

\begin{lstlisting}[language={[latex]TeX},basicstyle=\footnotesize\ttfamily,commentstyle=\itshape\color{gray},keywordstyle=\color{magenta},tabsize=4,frame=single]
\defaultfontfeatures[CascadiaMono-Light]
    {Extension=.otf,
    UprightFont=CascadiaCode-Light,
    ItalicFont=CascadiaCode-LightItalic,
    BoldFont=CascadiaCode-Regular,
    BoldItalicFont=CascadiaCode-Italic,
    Contextuals=AlternateOff}
\end{lstlisting}

\begin{lstlisting}[language={[latex]TeX},basicstyle=\footnotesize\ttfamily,commentstyle=\itshape\color{gray},keywordstyle=\color{magenta},tabsize=4,frame=single]
\defaultfontfeatures[CascadiaMono-SemiLight]
    {Extension=.otf,
    UprightFont=CascadiaCode-SemiLight,
    ItalicFont=CascadiaCode-SemiLightItalic,
    BoldFont=CascadiaCode-SemiBold,
    BoldItalicFont=CascadiaCode-SemiBoldItalic,
    Contextuals=AlternateOff}
\end{lstlisting}

\begin{lstlisting}[language={[latex]TeX},basicstyle=\footnotesize\ttfamily,commentstyle=\itshape\color{gray},keywordstyle=\color{magenta},tabsize=4,frame=single]
\defaultfontfeatures[CascadiaMono-ExtraLight]
    {Extension=.otf,
    UprightFont=CascadiaCode-ExtraLight,
    ItalicFont=CascadiaCode-ExtraLightItalic,
    BoldFont=CascadiaCode-SemiLight,
    BoldItalicFont=CascadiaCode-SemiLightItalic,
    Contextuals=AlternateOff}
\end{lstlisting}

It's possible to use ligatures with \texttt{Code} alias, instead of \texttt{Mono}.

\pagebreak

\subsection{With config files}

The idea is to propose \texttt{fontspec} config files to load correctly \texttt{CascadiaMono} features.

\begin{lstlisting}[language={[latex]TeX},basicstyle=\footnotesize\ttfamily,commentstyle=\itshape\color{gray},keywordstyle=\color{magenta},tabsize=4,frame=single]
\usepackage{fontspec}
%w/o ligatures
\setmonofont{CascadiaMono}[options]                %version regular
\setmonofont{CascadiaMono-SemiLight}[options]      %version semilight
\setmonofont{CascadiaMono-Light}[options]          %version light
\setmonofont{CascadiaMono-ExtraLight}[options]     %version extralight
%w ligatures
\setmonofont{cascadiacode}[options]                %version regular
\setmonofont{cascadiacode-semilight}[options]      %version semilight
\setmonofont{cascadiacode-light}[options]          %version light
\setmonofont{cascadiacode-extralight}[options]     %version extralight
\end{lstlisting}

\subsection{With the package loading}

With \lstinline[language={[latex]TeX},basicstyle=\ttfamily]|\usepackage[scale=...]{cascadiamono-otf}|, \lstinline[language={[latex]TeX},basicstyle=\ttfamily]|fontspec| is loaded, and \lstinline[language={[latex]TeX},basicstyle=\ttfamily]|fontfamily| are defined :

\begin{lstlisting}[language={[latex]TeX},basicstyle=\footnotesize\ttfamily,commentstyle=\itshape\color{gray},keywordstyle=\color{magenta},tabsize=4,frame=single]
%w/o ligatures
\newfontfamily\cascadiamono{CascadiaMono}
\newfontfamily\cascadiamonosemilight{CascadiaMono-SemiLight}
\newfontfamily\cascadiamonolight{CascadiaMono-Light}
\newfontfamily\cascadiamonoextralight{CascadiaMono-ExtraLight}
%w ligatures
\newfontfamily\cascadiacode{cascadiacode}
\newfontfamily\cascadiacodesemilight{cascadiacode-semilight}
\newfontfamily\cascadiacodelight{cascadiacode-light}
\newfontfamily\cascadiacodeextralight{cascadiacode-extralight}
\end{lstlisting}

\section{Font Samples}

\subsection{Normal version (Regular - Italic - Bold - BoldItalic)}

\setmonofont{CascadiaMono}[Scale=MatchLowercase]

\texttt{\demotext}\par\bigskip

\texttt{\textit{\demotext}}\par\bigskip

\texttt{\textbf{\demotext}}\par\bigskip

\texttt{\textbf{\textit{\demotext}}}\par

\subsection{SemiLight version (SemiLight - SemiLightItalic - SemiBold - SemiBoldItalic)}

\setmonofont{CascadiaMono-SemiLight}[Scale=MatchLowercase]

\texttt{\demotext}\par\bigskip

\texttt{\itshape\demotext}\par\bigskip

\texttt{\bfseries\demotext}\par\bigskip

\texttt{\bfseries\itshape\demotext}\par

\subsection{Light version (Light - LightItalic - Regular - Italic)}

\setmonofont{CascadiaMono-Light}[Scale=MatchLowercase]

\texttt{\demotext}\par\bigskip

\texttt{\itshape\demotext}\par\bigskip

\texttt{\bfseries\demotext}\par\bigskip

\texttt{\bfseries\itshape\demotext}\par

\subsection{ExtraLight version (ExtraLight - ExtraLightItalic - SemiLightItalic - SemiLightItalic)}

\setmonofont{CascadiaMono-ExtraLight}[Scale=MatchLowercase]

\texttt{\demotext}\par\bigskip

\texttt{\itshape\demotext}\par\bigskip

\texttt{\bfseries\demotext}\par\bigskip

\texttt{\bfseries\itshape\demotext}\par

\pagebreak

\section{Simple code samples}

\setmonofont{CMU Typewriter Text}[Scale=MatchLowercase]

\noindent{\small \begin{tabular}{lllll}
	\hline
	Type & Normal & Italic & Bold & BoldItalic \\ \hline
	ttdefault \sampletttbl[\ttfamily] \\ \hline
	Cmonoregular \sampletttbl[\cascadiamono]\\ \hline
	Cmonosemilight \sampletttbl[\cascadiamonosemilight] \\ \hline
	Cmonolight \sampletttbl[\cascadiamonolight] \\ \hline
	Cmonoextralight \sampletttbl[\cascadiamonoextralight]\\ \hline
	Ccoderegular \sampletttbl[\cascadiacode]\\ \hline
	Ccodesemilight \sampletttbl[\cascadiacodesemilight] \\ \hline
	Ccodelight \sampletttbl[\cascadiacodelight] \\ \hline
	Ccodeextralight \sampletttbl[\cascadiacodeextralight]\\ \hline
\end{tabular}}

\section{Algorithm samples, without ligatures}

\subsection{Normal version}

\setmonofont{CascadiaMono}[Scale=MatchLowercase]

\begin{lstlisting}[language=python,basicstyle=\footnotesize\ttfamily,commentstyle=\itshape\color{gray},keywordstyle=\bfseries\color{magenta},tabsize=4,frame=single]
def Fibonacci(n) :
  # Check if input is 0 then it will print incorrect input
  if n < 0 :
    print("Incorrect input")
  elif n == 0 :
    return 0
  elif n == 1 or n == 2 :
    return 1
  else :
    return Fibonacci(n-1) + Fibonacci(n-2)
\end{lstlisting}

\subsection{SemiLight version}

\setmonofont{CascadiaMono-SemiLight}[Scale=MatchLowercase]

\begin{lstlisting}[language=python,basicstyle=\footnotesize\ttfamily,commentstyle=\itshape\color{gray},keywordstyle=\bfseries\color{magenta},tabsize=4,frame=single]
def Fibonacci(n) :
  # Check if input is 0 then it will print incorrect input
  if n < 0 :
    print("Incorrect input")
  elif n == 0 :
    return 0
  elif n == 1 or n == 2 :
    return 1
  else :
    return Fibonacci(n-1) + Fibonacci(n-2)
\end{lstlisting}

\subsection{Light version}

\setmonofont{CascadiaMono-Light}[Scale=MatchLowercase]

\begin{lstlisting}[language=python,basicstyle=\footnotesize\ttfamily,commentstyle=\itshape\color{gray},keywordstyle=\bfseries\color{magenta},tabsize=4,frame=single]
def Fibonacci(n) :
  # Check if input is 0 then it will print incorrect input
  if n < 0 :
    print("Incorrect input")
  elif n == 0 :
    return 0
  elif n == 1 or n == 2 :
    return 1
  else :
    return Fibonacci(n-1) + Fibonacci(n-2)
\end{lstlisting}

\subsection{ExtraLight version}

\setmonofont{CascadiaMono-ExtraLight}[Scale=MatchLowercase]

\begin{lstlisting}[language=python,basicstyle=\footnotesize\ttfamily,commentstyle=\itshape\color{gray},keywordstyle=\bfseries\color{magenta},tabsize=4,frame=single]
def Fibonacci(n) :
  # Check if input is 0 then it will print incorrect input
  if n < 0 :
    print("Incorrect input")
  elif n == 0 :
    return 0
  elif n == 1 or n == 2 :
    return 1
  else :
    return Fibonacci(n-1) + Fibonacci(n-2)
\end{lstlisting}

\pagebreak

\section{Algorithm code, with ligatures}

\setmonofont{CMU Typewriter Text}[Scale=MatchLowercase]

Regular version of the fonts, with ligatures enable, can be uses with \texttt{code} alias (\lstinline[language={[latex]TeX},basicstyle=\ttfamily]|\cascadiacode|).

\makeatletter
\renewcommand*\verbatim@nolig@list{}
\makeatother

\begin{lstlisting}[language=python,basicstyle=\footnotesize\cascadiacode,commentstyle=\itshape\color{gray},keywordstyle=\bfseries\color{magenta},tabsize=4,frame=single,columns=flexible,showstringspaces=false]
\lstset{
  language=python,
  basicstyle=\footnotesize\cascadiacode,
  commentstyle=\itshape\color{gray},
  keywordstyle=\bfseries\color{magenta},
  tabsize=4,
  frame=single,
  columns=flexible,
  showstringspaces=false
}
\end{lstlisting}
\begin{lstlisting}[language=python,basicstyle=\footnotesize\cascadiacode,commentstyle=\itshape\color{gray},keywordstyle=\bfseries\color{magenta},tabsize=4,frame=single,columns=flexible,showstringspaces=false]
const similar = "oO08 iIlL1 g9qCGQ"
const diacritics_etc = "â é ù ï ø ç à Ē Æ œ"

window.toggleFavorite = (alias) => {
  try {
    let favorites = JSON.parse(localStorage.getItem('favorites')) || []
    if (favorites.indexOf(alias) > -1) {
      favorites = favorites.filter((v) => {
        return v !== alias
      })
    } else {
      favorites.push(alias)
    }
    localStorage.setItem('favorites', JSON.stringify(Array.from(new Set(favorites))))
  } catch (err) {
    // eslint-disable-next-line no-console
    console.error('could not save favorite', err)
  }
  renderSelectList()
  return false
}
\end{lstlisting}

\begin{lstlisting}[language=python,basicstyle=\footnotesize\cascadiacode,commentstyle=\itshape\color{gray},keywordstyle=\bfseries\color{magenta},tabsize=4,frame=single,columns=flexible,showstringspaces=false]
def Fibonacci(n) :
  # Check if input is 0 then it will print incorrect input
  if n < 0 :
    print("Incorrect input")
  elif n == 0 :
    return 0
  elif 1 <= n <= 2 :
    return 1
  else :
    return Fibonacci(n-1) + Fibonacci(n-2)
\end{lstlisting}

\pagebreak


\section{History}

\setmonofont{CMU Typewriter Text}[Scale=MatchLowercase]

\begin{verbatim}
v0.3 Package loading with fontfamily + 'code' version
v0.2 New weight options
v0.1 Initial version
\end{verbatim}

\end{document}
