% !TeX TXS-program:compile = txs:///arara
% arara: pdflatex: {shell: no, synctex: no, interaction: batchmode}
% arara: pdflatex: {shell: no, synctex: no, interaction: batchmode}

\documentclass{article}
\usepackage[margin=1in]{geometry}
\usepackage{openmoji}
\usepackage{longtable}
\usepackage{booktabs}
\usepackage{hyperref}
\usepackage{fancyvrb}
\usepackage{shortvrb}
\MakeShortVerb{\|}
\setlength{\parindent}{0pt}

\newenvironment{openmojicase}
{%
	\begin{longtable}{cll}%icon + name  + page
		\cmidrule[\heavyrulewidth]{1-3}% \toprule
		\bfseries Icon  & \bfseries PDF page & \bfseries Name\\
		\cmidrule{1-3}
		\endhead
}%
{%
		\cmidrule[\heavyrulewidth]{1-3}
	\end{longtable}%
}

\NewDocumentCommand{\showcaseopenmojicolor}{mm}{%name + macro + page
	\Large\openmoji{#1} & \sffamily #2 & \itshape #1 \\
}

\NewDocumentCommand{\showcaseopenmojiblack}{mm}{%name + macro + page
	\Large\openmoji[black]{#1} & \sffamily #2 & \itshape #1 \\
}

\begin{document}

\title{The \textsf{openmoji} package (0.1.0) -- \today}
\author{%
	Cédric Pierquet\\%
	\url{https://openmoji.org} (Source version 15.1.0)\\%
	\url{https://github.com/cpierquet/latex-packages/tree/main/openmoji} (\LaTeX{} package)
}
\date{cpierquet -- at -- outlook . fr}\maketitle

This package provides \LaTeX{} support for the openmoji project (\url{https://openmoji.org}).

\medskip

\hspace*{5mm}\fbox{\begin{minipage}{0.75\linewidth}{\emph{A set of thousands CC-BY-SA-4.0 licensed high-quality SVG/PDF icons for you to use in your web projects.}}\end{minipage}}

\medskip

To use \textsf{openmoji} in your document, include the package with |\usepackage{openmoji}|.

An icon can be accessed using the icon name. A list of all included icons with their respective \textit{page} can be found at the end of this document.

\section{Example}

\begin{Verbatim}[frame=single]
...
\usepackage{openmoji}
...
\begin{document}
...
Inline\openmoji{adhesive bandage}version
...
\end{document}
\end{Verbatim}

Result: Inline\openmoji{adhesive bandage}version

\section{Usage}

\subsection{Generic macro}

\begin{Verbatim}[frame=single]
\openmoji[black=TF,height=...,(d)strut=...]<includegraphics options>{name}
\end{Verbatim}

The boolean \texttt{[black]} activate \textsf{black} version of icon (if available). And key \texttt{[height=...]} can be:

\begin{itemize}
	\item \texttt{dauto} (by default): inline insertion with height/depth;
	\item \texttt{auto}: inline insertion with height/nodepth;
	\item \texttt{a length}: normal insertion with fixed height;
	\item \texttt{manuel height/depth}: normal insertion with fixed height/depth.
\end{itemize}

The keys \texttt{[(d)strut=...]} can be used to set the \textit{reference} box for dimension calculations.

\medskip

By default, icons are given with a small border, so the given height is the height of the \textit{full} box, not the hight of the icons.

\medskip

Example: Inline{\setlength\fboxsep{0pt}\fbox{\openmoji{adhesive bandage}}}version or inline {\setlength\fboxsep{0pt}\fbox{\openmoji{adhesive bandage}}} version

\pagebreak

\subsection{Examples}

\begin{Verbatim}[frame=single]
%normal version: inline with automatic height/depth (from qX characters)
\openmoji[height=dauto]{adhesive bandage} %or \openmoji{adhesive bandage}
\end{Verbatim}

{\Huge q\openmoji{adhesive bandage}X}

\begin{Verbatim}[frame=single]
%normal version: inline with automatic height/depth (from specified characters)
(q\openmoji[height=dauto,dstrut=(qB)]{adhesive bandage}B)
\end{Verbatim}

{\Huge (q\openmoji[height=dauto,dstrut=(qB)]{adhesive bandage}B)}

\begin{Verbatim}[frame=single]
%normal version: inline with automatic height (from X character) default version
\openmoji[height=auto]{adhesive bandage}
\end{Verbatim}

{\Huge X\openmoji[height=auto]{adhesive bandage}X}

\begin{Verbatim}[frame=single]
%normal version: inline with manual height/depth
\openmoji[height={0.8em/-0.4em}]{adhesive bandage}
\end{Verbatim}

{\Huge qb\openmoji[height={0.8em/-0.4em}]{adhesive bandage}Xz}

\begin{Verbatim}[frame=single]
%fixed height
\openmoji[height=1.5in]{adhesive bandage}
\openmoji[black,height=1.5in]{adhesive bandage}
\end{Verbatim}

\openmoji[height=1.5in]{adhesive bandage}\openmoji[black,height=1.5in]{adhesive bandage}

\subsection{Used packages}

\texttt{ifthen}, \texttt{calc}, \texttt{graphicx}, \texttt{xstring} and \texttt{simplekv} are loaded and used by the package.

\subsection{Bugs}

For bug reports and feature requests, report on the GitHub repository \url{https://github.com/cpierquet/latex-packages/issues}.

\newpage

\section{List of icons}

\subsection{Color version (4171 icons)}

The \textsf{PDF} file is \texttt{openmoji-color-all.pdf}

Direct command is: {\ttfamily\textbackslash openmoji\{<Name>\}}.

\begin{openmojicase}
\showcaseopenmojicolor{1st place medal}{1}
\showcaseopenmojicolor{2nd place medal}{2}
\showcaseopenmojicolor{3rd place medal}{3}
\showcaseopenmojicolor{a button (blood type)}{4}
\showcaseopenmojicolor{ab button (blood type)}{5}
\showcaseopenmojicolor{abacus}{6}
\showcaseopenmojicolor{accordion}{7}
\showcaseopenmojicolor{add button}{8}
\showcaseopenmojicolor{add contact}{9}
\showcaseopenmojicolor{adhesive bandage}{10}
\showcaseopenmojicolor{admission tickets}{11}
\showcaseopenmojicolor{aerial tramway}{12}
\showcaseopenmojicolor{airplane arrival}{13}
\showcaseopenmojicolor{airplane departure}{14}
\showcaseopenmojicolor{airplane}{15}
\showcaseopenmojicolor{alabama flag}{16}
\showcaseopenmojicolor{alarm clock}{17}
\showcaseopenmojicolor{alembic}{18}
\showcaseopenmojicolor{alien monster}{19}
\showcaseopenmojicolor{alien}{20}
\showcaseopenmojicolor{ambulance}{21}
\showcaseopenmojicolor{american football}{22}
\showcaseopenmojicolor{amphora}{23}
\showcaseopenmojicolor{anatomical heart}{24}
\showcaseopenmojicolor{anchor}{25}
\showcaseopenmojicolor{andalusia flag}{26}
\showcaseopenmojicolor{android}{27}
\showcaseopenmojicolor{anger symbol}{28}
\showcaseopenmojicolor{angry face with horns}{29}
\showcaseopenmojicolor{angry face}{30}
\showcaseopenmojicolor{anguished face}{31}
\showcaseopenmojicolor{annoyed face with tongue}{32}
\showcaseopenmojicolor{ant}{33}
\showcaseopenmojicolor{antenna bars}{34}
\showcaseopenmojicolor{anticlockwise triangle-headed top u-shaped arrow}{35}
\showcaseopenmojicolor{anxious face with sweat}{36}
\showcaseopenmojicolor{apple}{37}
\showcaseopenmojicolor{aquarius}{38}
\showcaseopenmojicolor{aragon flag}{39}
\showcaseopenmojicolor{archive}{40}
\showcaseopenmojicolor{arduino}{41}
\showcaseopenmojicolor{aries}{42}
\showcaseopenmojicolor{armchair}{43}
\showcaseopenmojicolor{arrow turn right}{44}
\showcaseopenmojicolor{articulated lorry}{45}
\showcaseopenmojicolor{artist palette}{46}
\showcaseopenmojicolor{artist- dark skin tone}{47}
\showcaseopenmojicolor{artist- light skin tone}{48}
\showcaseopenmojicolor{artist- medium skin tone}{49}
\showcaseopenmojicolor{artist- medium-dark skin tone}{50}
\showcaseopenmojicolor{artist- medium-light skin tone}{51}
\showcaseopenmojicolor{artist}{52}
\showcaseopenmojicolor{artstation}{53}
\showcaseopenmojicolor{assembly group}{54}
\showcaseopenmojicolor{assembly point}{55}
\showcaseopenmojicolor{astonished face}{56}
\showcaseopenmojicolor{astronaut- dark skin tone}{57}
\showcaseopenmojicolor{astronaut- light skin tone}{58}
\showcaseopenmojicolor{astronaut- medium skin tone}{59}
\showcaseopenmojicolor{astronaut- medium-dark skin tone}{60}
\showcaseopenmojicolor{astronaut- medium-light skin tone}{61}
\showcaseopenmojicolor{astronaut}{62}
\showcaseopenmojicolor{asturian flag}{63}
\showcaseopenmojicolor{atm sign}{64}
\showcaseopenmojicolor{atom bomb}{65}
\showcaseopenmojicolor{atom symbol}{66}
\showcaseopenmojicolor{augmented reality}{67}
\showcaseopenmojicolor{authority building}{68}
\showcaseopenmojicolor{authority instruction}{69}
\showcaseopenmojicolor{authority}{70}
\showcaseopenmojicolor{auto rickshaw}{71}
\showcaseopenmojicolor{automobile}{72}
\showcaseopenmojicolor{autonomous car}{73}
\showcaseopenmojicolor{avalanche}{74}
\showcaseopenmojicolor{avocado}{75}
\showcaseopenmojicolor{axe}{76}
\showcaseopenmojicolor{b button (blood type)}{77}
\showcaseopenmojicolor{baby angel- dark skin tone}{78}
\showcaseopenmojicolor{baby angel- light skin tone}{79}
\showcaseopenmojicolor{baby angel- medium skin tone}{80}
\showcaseopenmojicolor{baby angel- medium-dark skin tone}{81}
\showcaseopenmojicolor{baby angel- medium-light skin tone}{82}
\showcaseopenmojicolor{baby angel}{83}
\showcaseopenmojicolor{baby bottle}{84}
\showcaseopenmojicolor{baby chick}{85}
\showcaseopenmojicolor{baby symbol}{86}
\showcaseopenmojicolor{baby- dark skin tone}{87}
\showcaseopenmojicolor{baby- light skin tone}{88}
\showcaseopenmojicolor{baby- medium skin tone}{89}
\showcaseopenmojicolor{baby- medium-dark skin tone}{90}
\showcaseopenmojicolor{baby- medium-light skin tone}{91}
\showcaseopenmojicolor{baby}{92}
\showcaseopenmojicolor{back arrow}{93}
\showcaseopenmojicolor{backache}{94}
\showcaseopenmojicolor{backhand index pointing down- dark skin tone}{95}
\showcaseopenmojicolor{backhand index pointing down- light skin tone}{96}
\showcaseopenmojicolor{backhand index pointing down- medium skin tone}{97}
\showcaseopenmojicolor{backhand index pointing down- medium-dark skin tone}{98}
\showcaseopenmojicolor{backhand index pointing down- medium-light skin tone}{99}
\showcaseopenmojicolor{backhand index pointing down}{100}
\showcaseopenmojicolor{backhand index pointing left- dark skin tone}{101}
\showcaseopenmojicolor{backhand index pointing left- light skin tone}{102}
\showcaseopenmojicolor{backhand index pointing left- medium skin tone}{103}
\showcaseopenmojicolor{backhand index pointing left- medium-dark skin tone}{104}
\showcaseopenmojicolor{backhand index pointing left- medium-light skin tone}{105}
\showcaseopenmojicolor{backhand index pointing left}{106}
\showcaseopenmojicolor{backhand index pointing right- dark skin tone}{107}
\showcaseopenmojicolor{backhand index pointing right- light skin tone}{108}
\showcaseopenmojicolor{backhand index pointing right- medium skin tone}{109}
\showcaseopenmojicolor{backhand index pointing right- medium-dark skin tone}{110}
\showcaseopenmojicolor{backhand index pointing right- medium-light skin tone}{111}
\showcaseopenmojicolor{backhand index pointing right}{112}
\showcaseopenmojicolor{backhand index pointing up- dark skin tone}{113}
\showcaseopenmojicolor{backhand index pointing up- light skin tone}{114}
\showcaseopenmojicolor{backhand index pointing up- medium skin tone}{115}
\showcaseopenmojicolor{backhand index pointing up- medium-dark skin tone}{116}
\showcaseopenmojicolor{backhand index pointing up- medium-light skin tone}{117}
\showcaseopenmojicolor{backhand index pointing up}{118}
\showcaseopenmojicolor{backpack}{119}
\showcaseopenmojicolor{bacon}{120}
\showcaseopenmojicolor{badger}{121}
\showcaseopenmojicolor{badminton}{122}
\showcaseopenmojicolor{bagel}{123}
\showcaseopenmojicolor{baggage claim}{124}
\showcaseopenmojicolor{baguette bread}{125}
\showcaseopenmojicolor{balance scale}{126}
\showcaseopenmojicolor{bald}{127}
\showcaseopenmojicolor{balearic islands flag}{128}
\showcaseopenmojicolor{ballet shoes}{129}
\showcaseopenmojicolor{balloon}{130}
\showcaseopenmojicolor{ballot box with ballot}{131}
\showcaseopenmojicolor{banana}{132}
\showcaseopenmojicolor{bandage change}{133}
\showcaseopenmojicolor{bandage scissors}{134}
\showcaseopenmojicolor{banjo}{135}
\showcaseopenmojicolor{bank}{136}
\showcaseopenmojicolor{bar chart}{137}
\showcaseopenmojicolor{barber pole}{138}
\showcaseopenmojicolor{barcode}{139}
\showcaseopenmojicolor{barista}{140}
\showcaseopenmojicolor{baseball}{141}
\showcaseopenmojicolor{basket}{142}
\showcaseopenmojicolor{basketball}{143}
\showcaseopenmojicolor{basque flag}{144}
\showcaseopenmojicolor{bat}{145}
\showcaseopenmojicolor{bathtub}{146}
\showcaseopenmojicolor{battery}{147}
\showcaseopenmojicolor{bavaria flag}{148}
\showcaseopenmojicolor{beach with umbrella}{149}
\showcaseopenmojicolor{beaming face with smiling eyes}{150}
\showcaseopenmojicolor{beans}{151}
\showcaseopenmojicolor{bear}{152}
\showcaseopenmojicolor{beating heart}{153}
\showcaseopenmojicolor{beaver}{154}
\showcaseopenmojicolor{bed linen}{155}
\showcaseopenmojicolor{bed}{156}
\showcaseopenmojicolor{beer mug}{157}
\showcaseopenmojicolor{beetle}{158}
\showcaseopenmojicolor{bell pepper}{159}
\showcaseopenmojicolor{bell with slash}{160}
\showcaseopenmojicolor{bell}{161}
\showcaseopenmojicolor{bellhop bell}{162}
\showcaseopenmojicolor{beluga}{163}
\showcaseopenmojicolor{bento box}{164}
\showcaseopenmojicolor{berber flag}{165}
\showcaseopenmojicolor{berlin flag}{166}
\showcaseopenmojicolor{beverage box}{167}
\showcaseopenmojicolor{bicycle}{168}
\showcaseopenmojicolor{bikini}{169}
\showcaseopenmojicolor{billed cap}{170}
\showcaseopenmojicolor{biohazard}{171}
\showcaseopenmojicolor{bird}{172}
\showcaseopenmojicolor{birthday cake}{173}
\showcaseopenmojicolor{bison}{174}
\showcaseopenmojicolor{biting lip}{175}
\showcaseopenmojicolor{black bird}{176}
\showcaseopenmojicolor{black cat}{177}
\showcaseopenmojicolor{black circle}{178}
\showcaseopenmojicolor{black flag}{179}
\showcaseopenmojicolor{black heart}{180}
\showcaseopenmojicolor{black hexagon}{181}
\showcaseopenmojicolor{black hole}{182}
\showcaseopenmojicolor{black large circle}{183}
\showcaseopenmojicolor{black large square}{184}
\showcaseopenmojicolor{black medium square}{185}
\showcaseopenmojicolor{black medium-small square}{186}
\showcaseopenmojicolor{black nib}{187}
\showcaseopenmojicolor{black octagon}{188}
\showcaseopenmojicolor{black pentagon}{189}
\showcaseopenmojicolor{black rectangle}{190}
\showcaseopenmojicolor{black small square}{191}
\showcaseopenmojicolor{black square button}{192}
\showcaseopenmojicolor{black star}{193}
\showcaseopenmojicolor{black vertical ellipse}{194}
\showcaseopenmojicolor{black vertical rectangle}{195}
\showcaseopenmojicolor{blood transfusion}{196}
\showcaseopenmojicolor{blossom}{197}
\showcaseopenmojicolor{blowfish}{198}
\showcaseopenmojicolor{blue book}{199}
\showcaseopenmojicolor{blue circle}{200}
\showcaseopenmojicolor{blue flag}{201}
\showcaseopenmojicolor{blue heart}{202}
\showcaseopenmojicolor{blue hexagon}{203}
\showcaseopenmojicolor{blue square}{204}
\showcaseopenmojicolor{blueberries}{205}
\showcaseopenmojicolor{boar}{206}
\showcaseopenmojicolor{bomb}{207}
\showcaseopenmojicolor{bone}{208}
\showcaseopenmojicolor{bookmark tabs}{209}
\showcaseopenmojicolor{bookmark}{210}
\showcaseopenmojicolor{books}{211}
\showcaseopenmojicolor{boomerang}{212}
\showcaseopenmojicolor{bottle with popping cork}{213}
\showcaseopenmojicolor{boule bread}{214}
\showcaseopenmojicolor{bouquet}{215}
\showcaseopenmojicolor{bow and arrow}{216}
\showcaseopenmojicolor{bowl with spoon}{217}
\showcaseopenmojicolor{bowling}{218}
\showcaseopenmojicolor{boxing glove}{219}
\showcaseopenmojicolor{boy- dark skin tone}{220}
\showcaseopenmojicolor{boy- light skin tone}{221}
\showcaseopenmojicolor{boy- medium skin tone}{222}
\showcaseopenmojicolor{boy- medium-dark skin tone}{223}
\showcaseopenmojicolor{boy- medium-light skin tone}{224}
\showcaseopenmojicolor{boy}{225}
\showcaseopenmojicolor{brain}{226}
\showcaseopenmojicolor{bread}{227}
\showcaseopenmojicolor{breast-feeding- dark skin tone}{228}
\showcaseopenmojicolor{breast-feeding- light skin tone}{229}
\showcaseopenmojicolor{breast-feeding- medium skin tone}{230}
\showcaseopenmojicolor{breast-feeding- medium-dark skin tone}{231}
\showcaseopenmojicolor{breast-feeding- medium-light skin tone}{232}
\showcaseopenmojicolor{breast-feeding}{233}
\showcaseopenmojicolor{bretagne flag}{234}
\showcaseopenmojicolor{brick}{235}
\showcaseopenmojicolor{bridge at night}{236}
\showcaseopenmojicolor{briefcase}{237}
\showcaseopenmojicolor{briefs}{238}
\showcaseopenmojicolor{bright button}{239}
\showcaseopenmojicolor{broccoli}{240}
\showcaseopenmojicolor{broken chain}{241}
\showcaseopenmojicolor{broken heart}{242}
\showcaseopenmojicolor{broom}{243}
\showcaseopenmojicolor{brown circle}{244}
\showcaseopenmojicolor{brown flag}{245}
\showcaseopenmojicolor{brown heart}{246}
\showcaseopenmojicolor{brown hexagon}{247}
\showcaseopenmojicolor{brown mushroom}{248}
\showcaseopenmojicolor{brown square}{249}
\showcaseopenmojicolor{browncoat flag}{250}
\showcaseopenmojicolor{bubble tea}{251}
\showcaseopenmojicolor{bubbles}{252}
\showcaseopenmojicolor{bucket}{253}
\showcaseopenmojicolor{bug}{254}
\showcaseopenmojicolor{building construction}{255}
\showcaseopenmojicolor{bullet train}{256}
\showcaseopenmojicolor{bullseye}{257}
\showcaseopenmojicolor{burrito}{258}
\showcaseopenmojicolor{bus stop}{259}
\showcaseopenmojicolor{bus}{260}
\showcaseopenmojicolor{bust in silhouette}{261}
\showcaseopenmojicolor{busts in silhouette}{262}
\showcaseopenmojicolor{butter}{263}
\showcaseopenmojicolor{butterfly}{264}
\showcaseopenmojicolor{c}{265}
\showcaseopenmojicolor{cable}{266}
\showcaseopenmojicolor{cactus}{267}
\showcaseopenmojicolor{cafeteria}{268}
\showcaseopenmojicolor{cake}{269}
\showcaseopenmojicolor{calendar}{270}
\showcaseopenmojicolor{california flag}{271}
\showcaseopenmojicolor{call me hand- dark skin tone}{272}
\showcaseopenmojicolor{call me hand- light skin tone}{273}
\showcaseopenmojicolor{call me hand- medium skin tone}{274}
\showcaseopenmojicolor{call me hand- medium-dark skin tone}{275}
\showcaseopenmojicolor{call me hand- medium-light skin tone}{276}
\showcaseopenmojicolor{call me hand}{277}
\showcaseopenmojicolor{camel}{278}
\showcaseopenmojicolor{camera with flash}{279}
\showcaseopenmojicolor{camera}{280}
\showcaseopenmojicolor{camping}{281}
\showcaseopenmojicolor{canary islands flag}{282}
\showcaseopenmojicolor{cancer}{283}
\showcaseopenmojicolor{candle}{284}
\showcaseopenmojicolor{candy}{285}
\showcaseopenmojicolor{canned food}{286}
\showcaseopenmojicolor{canoe}{287}
\showcaseopenmojicolor{cantabria flag}{288}
\showcaseopenmojicolor{capricorn}{289}
\showcaseopenmojicolor{card file box}{290}
\showcaseopenmojicolor{card index dividers}{291}
\showcaseopenmojicolor{card index}{292}
\showcaseopenmojicolor{carousel horse}{293}
\showcaseopenmojicolor{carp streamer}{294}
\showcaseopenmojicolor{carpentry saw}{295}
\showcaseopenmojicolor{carrot}{296}
\showcaseopenmojicolor{castile and leon flag}{297}
\showcaseopenmojicolor{castile-la mancha flag}{298}
\showcaseopenmojicolor{castle}{299}
\showcaseopenmojicolor{cat face}{300}
\showcaseopenmojicolor{cat with tears of joy}{301}
\showcaseopenmojicolor{cat with wry smile}{302}
\showcaseopenmojicolor{cat}{303}
\showcaseopenmojicolor{catalonia flag}{304}
\showcaseopenmojicolor{ceuta flag}{305}
\showcaseopenmojicolor{chains}{306}
\showcaseopenmojicolor{chair}{307}
\showcaseopenmojicolor{champignon brown}{308}
\showcaseopenmojicolor{champignon white}{309}
\showcaseopenmojicolor{charge plug}{310}
\showcaseopenmojicolor{chart decreasing}{311}
\showcaseopenmojicolor{chart increasing with yen}{312}
\showcaseopenmojicolor{chart increasing}{313}
\showcaseopenmojicolor{chats}{314}
\showcaseopenmojicolor{check box with check}{315}
\showcaseopenmojicolor{check mark button}{316}
\showcaseopenmojicolor{check mark}{317}
\showcaseopenmojicolor{cheese wedge}{318}
\showcaseopenmojicolor{chequered flag}{319}
\showcaseopenmojicolor{cherries}{320}
\showcaseopenmojicolor{cherry blossom}{321}
\showcaseopenmojicolor{chess pawn}{322}
\showcaseopenmojicolor{chestnut}{323}
\showcaseopenmojicolor{chicken}{324}
\showcaseopenmojicolor{child- dark skin tone}{325}
\showcaseopenmojicolor{child- light skin tone}{326}
\showcaseopenmojicolor{child- medium skin tone}{327}
\showcaseopenmojicolor{child- medium-dark skin tone}{328}
\showcaseopenmojicolor{child- medium-light skin tone}{329}
\showcaseopenmojicolor{child}{330}
\showcaseopenmojicolor{children crossing}{331}
\showcaseopenmojicolor{chipmunk}{332}
\showcaseopenmojicolor{chocolate bar}{333}
\showcaseopenmojicolor{chopsticks}{334}
\showcaseopenmojicolor{christmas tree}{335}
\showcaseopenmojicolor{chrome canary}{336}
\showcaseopenmojicolor{chrome}{337}
\showcaseopenmojicolor{chromium}{338}
\showcaseopenmojicolor{church}{339}
\showcaseopenmojicolor{cigarette}{340}
\showcaseopenmojicolor{cinema}{341}
\showcaseopenmojicolor{circle with left half black}{342}
\showcaseopenmojicolor{circle with right half black}{343}
\showcaseopenmojicolor{circled anticlockwise arrow}{344}
\showcaseopenmojicolor{circled c with overlaid backslash}{345}
\showcaseopenmojicolor{circled cc}{346}
\showcaseopenmojicolor{circled dollar sign with overlaid backslash}{347}
\showcaseopenmojicolor{circled equals}{348}
\showcaseopenmojicolor{circled human figure}{349}
\showcaseopenmojicolor{circled m}{350}
\showcaseopenmojicolor{circled zero with slash}{351}
\showcaseopenmojicolor{circuit}{352}
\showcaseopenmojicolor{circus tent}{353}
\showcaseopenmojicolor{cityscape at dusk}{354}
\showcaseopenmojicolor{cityscape}{355}
\showcaseopenmojicolor{cl button}{356}
\showcaseopenmojicolor{clamp}{357}
\showcaseopenmojicolor{clapper board}{358}
\showcaseopenmojicolor{clapping hands- dark skin tone}{359}
\showcaseopenmojicolor{clapping hands- light skin tone}{360}
\showcaseopenmojicolor{clapping hands- medium skin tone}{361}
\showcaseopenmojicolor{clapping hands- medium-dark skin tone}{362}
\showcaseopenmojicolor{clapping hands- medium-light skin tone}{363}
\showcaseopenmojicolor{clapping hands}{364}
\showcaseopenmojicolor{classical building}{365}
\showcaseopenmojicolor{clinical thermometer}{366}
\showcaseopenmojicolor{clinking beer mugs}{367}
\showcaseopenmojicolor{clinking glasses}{368}
\showcaseopenmojicolor{clipboard}{369}
\showcaseopenmojicolor{clockwise vertical arrows}{370}
\showcaseopenmojicolor{close}{371}
\showcaseopenmojicolor{closed book}{372}
\showcaseopenmojicolor{closed mailbox with lowered flag}{373}
\showcaseopenmojicolor{closed mailbox with raised flag}{374}
\showcaseopenmojicolor{closed umbrella}{375}
\showcaseopenmojicolor{cloud with lightning and rain}{376}
\showcaseopenmojicolor{cloud with lightning}{377}
\showcaseopenmojicolor{cloud with rain}{378}
\showcaseopenmojicolor{cloud with snow}{379}
\showcaseopenmojicolor{cloud}{380}
\showcaseopenmojicolor{clown face}{381}
\showcaseopenmojicolor{club suit}{382}
\showcaseopenmojicolor{clutch bag}{383}
\showcaseopenmojicolor{coat}{384}
\showcaseopenmojicolor{cockroach}{385}
\showcaseopenmojicolor{cocktail glass}{386}
\showcaseopenmojicolor{coconut}{387}
\showcaseopenmojicolor{code editor}{388}
\showcaseopenmojicolor{coffee grinder}{389}
\showcaseopenmojicolor{coffin}{390}
\showcaseopenmojicolor{coin}{391}
\showcaseopenmojicolor{cold face}{392}
\showcaseopenmojicolor{collaboration}{393}
\showcaseopenmojicolor{collision}{394}
\showcaseopenmojicolor{colorado flag}{395}
\showcaseopenmojicolor{colossus of rhodes}{396}
\showcaseopenmojicolor{comet}{397}
\showcaseopenmojicolor{comment}{398}
\showcaseopenmojicolor{compass}{399}
\showcaseopenmojicolor{compose}{400}
\showcaseopenmojicolor{computer disk}{401}
\showcaseopenmojicolor{computer mouse}{402}
\showcaseopenmojicolor{confetti ball}{403}
\showcaseopenmojicolor{confounded face}{404}
\showcaseopenmojicolor{confused face}{405}
\showcaseopenmojicolor{construction worker- dark skin tone}{406}
\showcaseopenmojicolor{construction worker- light skin tone}{407}
\showcaseopenmojicolor{construction worker- medium skin tone}{408}
\showcaseopenmojicolor{construction worker- medium-dark skin tone}{409}
\showcaseopenmojicolor{construction worker- medium-light skin tone}{410}
\showcaseopenmojicolor{construction worker}{411}
\showcaseopenmojicolor{construction}{412}
\showcaseopenmojicolor{contacts}{413}
\showcaseopenmojicolor{control knobs}{414}
\showcaseopenmojicolor{convenience store}{415}
\showcaseopenmojicolor{cook- dark skin tone}{416}
\showcaseopenmojicolor{cook- light skin tone}{417}
\showcaseopenmojicolor{cook- medium skin tone}{418}
\showcaseopenmojicolor{cook- medium-dark skin tone}{419}
\showcaseopenmojicolor{cook- medium-light skin tone}{420}
\showcaseopenmojicolor{cook}{421}
\showcaseopenmojicolor{cooked rice}{422}
\showcaseopenmojicolor{cookie}{423}
\showcaseopenmojicolor{cooking}{424}
\showcaseopenmojicolor{cool button}{425}
\showcaseopenmojicolor{copy}{426}
\showcaseopenmojicolor{copyleft symbol}{427}
\showcaseopenmojicolor{copyright}{428}
\showcaseopenmojicolor{coral}{429}
\showcaseopenmojicolor{couch and lamp}{430}
\showcaseopenmojicolor{counterclockwise arrows button}{431}
\showcaseopenmojicolor{couple with heart- dark skin tone}{432}
\showcaseopenmojicolor{couple with heart- light skin tone}{433}
\showcaseopenmojicolor{couple with heart- man, man, dark skin tone, light skin tone}{434}
\showcaseopenmojicolor{couple with heart- man, man, dark skin tone, medium skin tone}{435}
\showcaseopenmojicolor{couple with heart- man, man, dark skin tone, medium-dark skin tone}{436}
\showcaseopenmojicolor{couple with heart- man, man, dark skin tone, medium-light skin tone}{437}
\showcaseopenmojicolor{couple with heart- man, man, dark skin tone}{438}
\showcaseopenmojicolor{couple with heart- man, man, light skin tone, dark skin tone}{439}
\showcaseopenmojicolor{couple with heart- man, man, light skin tone, medium skin tone}{440}
\showcaseopenmojicolor{couple with heart- man, man, light skin tone, medium-dark skin tone}{441}
\showcaseopenmojicolor{couple with heart- man, man, light skin tone, medium-light skin tone}{442}
\showcaseopenmojicolor{couple with heart- man, man, light skin tone}{443}
\showcaseopenmojicolor{couple with heart- man, man, medium skin tone, dark skin tone}{444}
\showcaseopenmojicolor{couple with heart- man, man, medium skin tone, light skin tone}{445}
\showcaseopenmojicolor{couple with heart- man, man, medium skin tone, medium-dark skin tone}{446}
\showcaseopenmojicolor{couple with heart- man, man, medium skin tone, medium-light skin tone}{447}
\showcaseopenmojicolor{couple with heart- man, man, medium skin tone}{448}
\showcaseopenmojicolor{couple with heart- man, man, medium-dark skin tone, dark skin tone}{449}
\showcaseopenmojicolor{couple with heart- man, man, medium-dark skin tone, light skin tone}{450}
\showcaseopenmojicolor{couple with heart- man, man, medium-dark skin tone, medium skin tone}{451}
\showcaseopenmojicolor{couple with heart- man, man, medium-dark skin tone, medium-light skin tone}{452}
\showcaseopenmojicolor{couple with heart- man, man, medium-dark skin tone}{453}
\showcaseopenmojicolor{couple with heart- man, man, medium-light skin tone, dark skin tone}{454}
\showcaseopenmojicolor{couple with heart- man, man, medium-light skin tone, light skin tone}{455}
\showcaseopenmojicolor{couple with heart- man, man, medium-light skin tone, medium skin tone}{456}
\showcaseopenmojicolor{couple with heart- man, man, medium-light skin tone, medium-dark skin tone}{457}
\showcaseopenmojicolor{couple with heart- man, man, medium-light skin tone}{458}
\showcaseopenmojicolor{couple with heart- man, man}{459}
\showcaseopenmojicolor{couple with heart- medium skin tone}{460}
\showcaseopenmojicolor{couple with heart- medium-dark skin tone}{461}
\showcaseopenmojicolor{couple with heart- medium-light skin tone}{462}
\showcaseopenmojicolor{couple with heart- person, person, dark skin tone, light skin tone}{463}
\showcaseopenmojicolor{couple with heart- person, person, dark skin tone, medium skin tone}{464}
\showcaseopenmojicolor{couple with heart- person, person, dark skin tone, medium-dark skin tone}{465}
\showcaseopenmojicolor{couple with heart- person, person, dark skin tone, medium-light skin tone}{466}
\showcaseopenmojicolor{couple with heart- person, person, light skin tone, dark skin tone}{467}
\showcaseopenmojicolor{couple with heart- person, person, light skin tone, medium skin tone}{468}
\showcaseopenmojicolor{couple with heart- person, person, light skin tone, medium-dark skin tone}{469}
\showcaseopenmojicolor{couple with heart- person, person, light skin tone, medium-light skin tone}{470}
\showcaseopenmojicolor{couple with heart- person, person, medium skin tone, dark skin tone}{471}
\showcaseopenmojicolor{couple with heart- person, person, medium skin tone, light skin tone}{472}
\showcaseopenmojicolor{couple with heart- person, person, medium skin tone, medium-dark skin tone}{473}
\showcaseopenmojicolor{couple with heart- person, person, medium skin tone, medium-light skin tone}{474}
\showcaseopenmojicolor{couple with heart- person, person, medium-dark skin tone, dark skin tone}{475}
\showcaseopenmojicolor{couple with heart- person, person, medium-dark skin tone, light skin tone}{476}
\showcaseopenmojicolor{couple with heart- person, person, medium-dark skin tone, medium skin tone}{477}
\showcaseopenmojicolor{couple with heart- person, person, medium-dark skin tone, medium-light skin tone}{478}
\showcaseopenmojicolor{couple with heart- person, person, medium-light skin tone, dark skin tone}{479}
\showcaseopenmojicolor{couple with heart- person, person, medium-light skin tone, light skin tone}{480}
\showcaseopenmojicolor{couple with heart- person, person, medium-light skin tone, medium skin tone}{481}
\showcaseopenmojicolor{couple with heart- person, person, medium-light skin tone, medium-dark skin tone}{482}
\showcaseopenmojicolor{couple with heart- woman, man, dark skin tone, light skin tone}{483}
\showcaseopenmojicolor{couple with heart- woman, man, dark skin tone, medium skin tone}{484}
\showcaseopenmojicolor{couple with heart- woman, man, dark skin tone, medium-dark skin tone}{485}
\showcaseopenmojicolor{couple with heart- woman, man, dark skin tone, medium-light skin tone}{486}
\showcaseopenmojicolor{couple with heart- woman, man, dark skin tone}{487}
\showcaseopenmojicolor{couple with heart- woman, man, light skin tone, dark skin tone}{488}
\showcaseopenmojicolor{couple with heart- woman, man, light skin tone, medium skin tone}{489}
\showcaseopenmojicolor{couple with heart- woman, man, light skin tone, medium-dark skin tone}{490}
\showcaseopenmojicolor{couple with heart- woman, man, light skin tone, medium-light skin tone}{491}
\showcaseopenmojicolor{couple with heart- woman, man, light skin tone}{492}
\showcaseopenmojicolor{couple with heart- woman, man, medium skin tone, dark skin tone}{493}
\showcaseopenmojicolor{couple with heart- woman, man, medium skin tone, light skin tone}{494}
\showcaseopenmojicolor{couple with heart- woman, man, medium skin tone, medium-dark skin tone}{495}
\showcaseopenmojicolor{couple with heart- woman, man, medium skin tone, medium-light skin tone}{496}
\showcaseopenmojicolor{couple with heart- woman, man, medium skin tone}{497}
\showcaseopenmojicolor{couple with heart- woman, man, medium-dark skin tone, dark skin tone}{498}
\showcaseopenmojicolor{couple with heart- woman, man, medium-dark skin tone, light skin tone}{499}
\showcaseopenmojicolor{couple with heart- woman, man, medium-dark skin tone, medium skin tone}{500}
\showcaseopenmojicolor{couple with heart- woman, man, medium-dark skin tone, medium-light skin tone}{501}
\showcaseopenmojicolor{couple with heart- woman, man, medium-dark skin tone}{502}
\showcaseopenmojicolor{couple with heart- woman, man, medium-light skin tone, dark skin tone}{503}
\showcaseopenmojicolor{couple with heart- woman, man, medium-light skin tone, light skin tone}{504}
\showcaseopenmojicolor{couple with heart- woman, man, medium-light skin tone, medium skin tone}{505}
\showcaseopenmojicolor{couple with heart- woman, man, medium-light skin tone, medium-dark skin tone}{506}
\showcaseopenmojicolor{couple with heart- woman, man, medium-light skin tone}{507}
\showcaseopenmojicolor{couple with heart- woman, man}{508}
\showcaseopenmojicolor{couple with heart- woman, woman, dark skin tone, light skin tone}{509}
\showcaseopenmojicolor{couple with heart- woman, woman, dark skin tone, medium skin tone}{510}
\showcaseopenmojicolor{couple with heart- woman, woman, dark skin tone, medium-dark skin tone}{511}
\showcaseopenmojicolor{couple with heart- woman, woman, dark skin tone, medium-light skin tone}{512}
\showcaseopenmojicolor{couple with heart- woman, woman, dark skin tone}{513}
\showcaseopenmojicolor{couple with heart- woman, woman, light skin tone, dark skin tone}{514}
\showcaseopenmojicolor{couple with heart- woman, woman, light skin tone, medium skin tone}{515}
\showcaseopenmojicolor{couple with heart- woman, woman, light skin tone, medium-dark skin tone}{516}
\showcaseopenmojicolor{couple with heart- woman, woman, light skin tone, medium-light skin tone}{517}
\showcaseopenmojicolor{couple with heart- woman, woman, light skin tone}{518}
\showcaseopenmojicolor{couple with heart- woman, woman, medium skin tone, dark skin tone}{519}
\showcaseopenmojicolor{couple with heart- woman, woman, medium skin tone, light skin tone}{520}
\showcaseopenmojicolor{couple with heart- woman, woman, medium skin tone, medium-dark skin tone}{521}
\showcaseopenmojicolor{couple with heart- woman, woman, medium skin tone, medium-light skin tone}{522}
\showcaseopenmojicolor{couple with heart- woman, woman, medium skin tone}{523}
\showcaseopenmojicolor{couple with heart- woman, woman, medium-dark skin tone, dark skin tone}{524}
\showcaseopenmojicolor{couple with heart- woman, woman, medium-dark skin tone, light skin tone}{525}
\showcaseopenmojicolor{couple with heart- woman, woman, medium-dark skin tone, medium skin tone}{526}
\showcaseopenmojicolor{couple with heart- woman, woman, medium-dark skin tone, medium-light skin tone}{527}
\showcaseopenmojicolor{couple with heart- woman, woman, medium-dark skin tone}{528}
\showcaseopenmojicolor{couple with heart- woman, woman, medium-light skin tone, dark skin tone}{529}
\showcaseopenmojicolor{couple with heart- woman, woman, medium-light skin tone, light skin tone}{530}
\showcaseopenmojicolor{couple with heart- woman, woman, medium-light skin tone, medium skin tone}{531}
\showcaseopenmojicolor{couple with heart- woman, woman, medium-light skin tone, medium-dark skin tone}{532}
\showcaseopenmojicolor{couple with heart- woman, woman, medium-light skin tone}{533}
\showcaseopenmojicolor{couple with heart- woman, woman}{534}
\showcaseopenmojicolor{couple with heart}{535}
\showcaseopenmojicolor{cow face}{536}
\showcaseopenmojicolor{cow}{537}
\showcaseopenmojicolor{cowboy hat face}{538}
\showcaseopenmojicolor{cplusplus}{539}
\showcaseopenmojicolor{crab}{540}
\showcaseopenmojicolor{crayon}{541}
\showcaseopenmojicolor{credit card}{542}
\showcaseopenmojicolor{crescent moon}{543}
\showcaseopenmojicolor{cricket game}{544}
\showcaseopenmojicolor{cricket}{545}
\showcaseopenmojicolor{crocodile}{546}
\showcaseopenmojicolor{croissant}{547}
\showcaseopenmojicolor{cross mark button}{548}
\showcaseopenmojicolor{cross mark}{549}
\showcaseopenmojicolor{crossed fingers- dark skin tone}{550}
\showcaseopenmojicolor{crossed fingers- light skin tone}{551}
\showcaseopenmojicolor{crossed fingers- medium skin tone}{552}
\showcaseopenmojicolor{crossed fingers- medium-dark skin tone}{553}
\showcaseopenmojicolor{crossed fingers- medium-light skin tone}{554}
\showcaseopenmojicolor{crossed fingers}{555}
\showcaseopenmojicolor{crossed flags}{556}
\showcaseopenmojicolor{crossed swords}{557}
\showcaseopenmojicolor{crown}{558}
\showcaseopenmojicolor{crutch}{559}
\showcaseopenmojicolor{crutches}{560}
\showcaseopenmojicolor{crying cat}{561}
\showcaseopenmojicolor{crying face}{562}
\showcaseopenmojicolor{crystal ball}{563}
\showcaseopenmojicolor{csharp}{564}
\showcaseopenmojicolor{ct scan}{565}
\showcaseopenmojicolor{cucumber}{566}
\showcaseopenmojicolor{cup with straw}{567}
\showcaseopenmojicolor{cupcake}{568}
\showcaseopenmojicolor{curling stone}{569}
\showcaseopenmojicolor{curly hair}{570}
\showcaseopenmojicolor{curly loop}{571}
\showcaseopenmojicolor{currency exchange}{572}
\showcaseopenmojicolor{curry rice}{573}
\showcaseopenmojicolor{cursor}{574}
\showcaseopenmojicolor{custard}{575}
\showcaseopenmojicolor{customs}{576}
\showcaseopenmojicolor{cut of meat}{577}
\showcaseopenmojicolor{cyclone}{578}
\showcaseopenmojicolor{dagger}{579}
\showcaseopenmojicolor{dango}{580}
\showcaseopenmojicolor{dark skin tone}{581}
\showcaseopenmojicolor{dashing away}{582}
\showcaseopenmojicolor{deaf man- dark skin tone}{583}
\showcaseopenmojicolor{deaf man- light skin tone}{584}
\showcaseopenmojicolor{deaf man- medium skin tone}{585}
\showcaseopenmojicolor{deaf man- medium-dark skin tone}{586}
\showcaseopenmojicolor{deaf man- medium-light skin tone}{587}
\showcaseopenmojicolor{deaf man}{588}
\showcaseopenmojicolor{deaf person- dark skin tone}{589}
\showcaseopenmojicolor{deaf person- light skin tone}{590}
\showcaseopenmojicolor{deaf person- medium skin tone}{591}
\showcaseopenmojicolor{deaf person- medium-dark skin tone}{592}
\showcaseopenmojicolor{deaf person- medium-light skin tone}{593}
\showcaseopenmojicolor{deaf person}{594}
\showcaseopenmojicolor{deaf woman- dark skin tone}{595}
\showcaseopenmojicolor{deaf woman- light skin tone}{596}
\showcaseopenmojicolor{deaf woman- medium skin tone}{597}
\showcaseopenmojicolor{deaf woman- medium-dark skin tone}{598}
\showcaseopenmojicolor{deaf woman- medium-light skin tone}{599}
\showcaseopenmojicolor{deaf woman}{600}
\showcaseopenmojicolor{deciduous tree}{601}
\showcaseopenmojicolor{deep blue flag}{602}
\showcaseopenmojicolor{deep brown flag}{603}
\showcaseopenmojicolor{deep green flag}{604}
\showcaseopenmojicolor{deep orange flag}{605}
\showcaseopenmojicolor{deep purple flag}{606}
\showcaseopenmojicolor{deep red flag}{607}
\showcaseopenmojicolor{deep yellow flag}{608}
\showcaseopenmojicolor{deer}{609}
\showcaseopenmojicolor{dejected face}{610}
\showcaseopenmojicolor{delete}{611}
\showcaseopenmojicolor{delivery truck}{612}
\showcaseopenmojicolor{department store}{613}
\showcaseopenmojicolor{derelict house}{614}
\showcaseopenmojicolor{desert island}{615}
\showcaseopenmojicolor{desert}{616}
\showcaseopenmojicolor{desktop computer}{617}
\showcaseopenmojicolor{details}{618}
\showcaseopenmojicolor{detective- dark skin tone}{619}
\showcaseopenmojicolor{detective- light skin tone}{620}
\showcaseopenmojicolor{detective- medium skin tone}{621}
\showcaseopenmojicolor{detective- medium-dark skin tone}{622}
\showcaseopenmojicolor{detective- medium-light skin tone}{623}
\showcaseopenmojicolor{detective}{624}
\showcaseopenmojicolor{diamond suit}{625}
\showcaseopenmojicolor{diamond with a dot}{626}
\showcaseopenmojicolor{dim button}{627}
\showcaseopenmojicolor{disappointed face}{628}
\showcaseopenmojicolor{discord}{629}
\showcaseopenmojicolor{disguised face}{630}
\showcaseopenmojicolor{disinfect surface}{631}
\showcaseopenmojicolor{divide}{632}
\showcaseopenmojicolor{diving mask}{633}
\showcaseopenmojicolor{diya lamp}{634}
\showcaseopenmojicolor{dizzy}{635}
\showcaseopenmojicolor{dj man}{636}
\showcaseopenmojicolor{dj woman}{637}
\showcaseopenmojicolor{dj}{638}
\showcaseopenmojicolor{dna}{639}
\showcaseopenmojicolor{dodo}{640}
\showcaseopenmojicolor{doe}{641}
\showcaseopenmojicolor{dog face}{642}
\showcaseopenmojicolor{dog}{643}
\showcaseopenmojicolor{dollar banknote}{644}
\showcaseopenmojicolor{dolphin}{645}
\showcaseopenmojicolor{donkey}{646}
\showcaseopenmojicolor{door}{647}
\showcaseopenmojicolor{dotnet}{648}
\showcaseopenmojicolor{dotted line face}{649}
\showcaseopenmojicolor{dotted six-pointed star}{650}
\showcaseopenmojicolor{double curly loop}{651}
\showcaseopenmojicolor{double exclamation mark}{652}
\showcaseopenmojicolor{double tap}{653}
\showcaseopenmojicolor{doughnut}{654}
\showcaseopenmojicolor{dove}{655}
\showcaseopenmojicolor{down arrow}{656}
\showcaseopenmojicolor{down-left arrow}{657}
\showcaseopenmojicolor{down-right arrow}{658}
\showcaseopenmojicolor{downcast face with sweat}{659}
\showcaseopenmojicolor{download}{660}
\showcaseopenmojicolor{downwards button}{661}
\showcaseopenmojicolor{dragon face}{662}
\showcaseopenmojicolor{dragon}{663}
\showcaseopenmojicolor{dress}{664}
\showcaseopenmojicolor{drip coffee maker}{665}
\showcaseopenmojicolor{drone}{666}
\showcaseopenmojicolor{drooling face}{667}
\showcaseopenmojicolor{drop cover hold}{668}
\showcaseopenmojicolor{drop of blood}{669}
\showcaseopenmojicolor{droplet}{670}
\showcaseopenmojicolor{drum}{671}
\showcaseopenmojicolor{drunk person}{672}
\showcaseopenmojicolor{duck}{673}
\showcaseopenmojicolor{dumpling}{674}
\showcaseopenmojicolor{duplicate}{675}
\showcaseopenmojicolor{dvd}{676}
\showcaseopenmojicolor{e-mail}{677}
\showcaseopenmojicolor{eagle}{678}
\showcaseopenmojicolor{ear of corn}{679}
\showcaseopenmojicolor{ear with hearing aid- dark skin tone}{680}
\showcaseopenmojicolor{ear with hearing aid- light skin tone}{681}
\showcaseopenmojicolor{ear with hearing aid- medium skin tone}{682}
\showcaseopenmojicolor{ear with hearing aid- medium-dark skin tone}{683}
\showcaseopenmojicolor{ear with hearing aid- medium-light skin tone}{684}
\showcaseopenmojicolor{ear with hearing aid}{685}
\showcaseopenmojicolor{ear- dark skin tone}{686}
\showcaseopenmojicolor{ear- light skin tone}{687}
\showcaseopenmojicolor{ear- medium skin tone}{688}
\showcaseopenmojicolor{ear- medium-dark skin tone}{689}
\showcaseopenmojicolor{ear- medium-light skin tone}{690}
\showcaseopenmojicolor{ear}{691}
\showcaseopenmojicolor{earache}{692}
\showcaseopenmojicolor{earthquake}{693}
\showcaseopenmojicolor{ecg waves}{694}
\showcaseopenmojicolor{edge}{695}
\showcaseopenmojicolor{edit}{696}
\showcaseopenmojicolor{egg}{697}
\showcaseopenmojicolor{eggplant}{698}
\showcaseopenmojicolor{eiffel tower}{699}
\showcaseopenmojicolor{eight o clock}{700}
\showcaseopenmojicolor{eight-pointed star}{701}
\showcaseopenmojicolor{eight-spoked asterisk}{702}
\showcaseopenmojicolor{eight-thirty}{703}
\showcaseopenmojicolor{eject button}{704}
\showcaseopenmojicolor{electric coffee percolator}{705}
\showcaseopenmojicolor{electric plug red}{706}
\showcaseopenmojicolor{electric plug}{707}
\showcaseopenmojicolor{element}{708}
\showcaseopenmojicolor{elephant}{709}
\showcaseopenmojicolor{elevator}{710}
\showcaseopenmojicolor{eleven o clock}{711}
\showcaseopenmojicolor{eleven-thirty}{712}
\showcaseopenmojicolor{elf- dark skin tone}{713}
\showcaseopenmojicolor{elf- light skin tone}{714}
\showcaseopenmojicolor{elf- medium skin tone}{715}
\showcaseopenmojicolor{elf- medium-dark skin tone}{716}
\showcaseopenmojicolor{elf- medium-light skin tone}{717}
\showcaseopenmojicolor{elf}{718}
\showcaseopenmojicolor{emergency exit door}{719}
\showcaseopenmojicolor{emergency exit}{720}
\showcaseopenmojicolor{empty nest}{721}
\showcaseopenmojicolor{end arrow}{722}
\showcaseopenmojicolor{enraged face}{723}
\showcaseopenmojicolor{envelope with arrow}{724}
\showcaseopenmojicolor{envelope}{725}
\showcaseopenmojicolor{esperanto flag}{726}
\showcaseopenmojicolor{espresso machine}{727}
\showcaseopenmojicolor{euro banknote}{728}
\showcaseopenmojicolor{european name badge}{729}
\showcaseopenmojicolor{evacuate downstairs}{730}
\showcaseopenmojicolor{evacuate fire}{731}
\showcaseopenmojicolor{evacuate to shelter}{732}
\showcaseopenmojicolor{evacuate vertical}{733}
\showcaseopenmojicolor{evacuate}{734}
\showcaseopenmojicolor{evergreen tree}{735}
\showcaseopenmojicolor{ewe}{736}
\showcaseopenmojicolor{exclamation question mark}{737}
\showcaseopenmojicolor{exhaust gases car}{738}
\showcaseopenmojicolor{exhaust gases factory}{739}
\showcaseopenmojicolor{exhausted face}{740}
\showcaseopenmojicolor{exit}{741}
\showcaseopenmojicolor{exploding head}{742}
\showcaseopenmojicolor{expressionless face}{743}
\showcaseopenmojicolor{extremadura flag}{744}
\showcaseopenmojicolor{eye in speech bubble}{745}
\showcaseopenmojicolor{eye pain}{746}
\showcaseopenmojicolor{eye}{747}
\showcaseopenmojicolor{eyes}{748}
\showcaseopenmojicolor{face blowing a kiss}{749}
\showcaseopenmojicolor{face exhaling}{750}
\showcaseopenmojicolor{face holding back tears}{751}
\showcaseopenmojicolor{face in clouds}{752}
\showcaseopenmojicolor{face savoring food}{753}
\showcaseopenmojicolor{face screaming in fear}{754}
\showcaseopenmojicolor{face vomiting}{755}
\showcaseopenmojicolor{face with crossed-out eyes}{756}
\showcaseopenmojicolor{face with diagonal mouth}{757}
\showcaseopenmojicolor{face with hand over mouth}{758}
\showcaseopenmojicolor{face with head-bandage}{759}
\showcaseopenmojicolor{face with medical mask}{760}
\showcaseopenmojicolor{face with monocle}{761}
\showcaseopenmojicolor{face with open eyes and hand over mouth}{762}
\showcaseopenmojicolor{face with open mouth}{763}
\showcaseopenmojicolor{face with peeking eye}{764}
\showcaseopenmojicolor{face with raised eyebrow}{765}
\showcaseopenmojicolor{face with rolling eyes}{766}
\showcaseopenmojicolor{face with spiral eyes}{767}
\showcaseopenmojicolor{face with steam from nose}{768}
\showcaseopenmojicolor{face with symbols on mouth}{769}
\showcaseopenmojicolor{face with tears of joy}{770}
\showcaseopenmojicolor{face with thermometer}{771}
\showcaseopenmojicolor{face with tongue}{772}
\showcaseopenmojicolor{face without mouth}{773}
\showcaseopenmojicolor{facebook}{774}
\showcaseopenmojicolor{factory worker- dark skin tone}{775}
\showcaseopenmojicolor{factory worker- light skin tone}{776}
\showcaseopenmojicolor{factory worker- medium skin tone}{777}
\showcaseopenmojicolor{factory worker- medium-dark skin tone}{778}
\showcaseopenmojicolor{factory worker- medium-light skin tone}{779}
\showcaseopenmojicolor{factory worker}{780}
\showcaseopenmojicolor{factory}{781}
\showcaseopenmojicolor{fairy- dark skin tone}{782}
\showcaseopenmojicolor{fairy- light skin tone}{783}
\showcaseopenmojicolor{fairy- medium skin tone}{784}
\showcaseopenmojicolor{fairy- medium-dark skin tone}{785}
\showcaseopenmojicolor{fairy- medium-light skin tone}{786}
\showcaseopenmojicolor{fairy}{787}
\showcaseopenmojicolor{falafel}{788}
\showcaseopenmojicolor{fallen leaf}{789}
\showcaseopenmojicolor{family- adult, adult, child, child}{790}
\showcaseopenmojicolor{family- adult, adult, child}{791}
\showcaseopenmojicolor{family- adult, child, child}{792}
\showcaseopenmojicolor{family- adult, child}{793}
\showcaseopenmojicolor{family- man, boy, boy}{794}
\showcaseopenmojicolor{family- man, boy}{795}
\showcaseopenmojicolor{family- man, girl, boy}{796}
\showcaseopenmojicolor{family- man, girl, girl}{797}
\showcaseopenmojicolor{family- man, girl}{798}
\showcaseopenmojicolor{family- man, man, boy, boy}{799}
\showcaseopenmojicolor{family- man, man, boy}{800}
\showcaseopenmojicolor{family- man, man, girl, boy}{801}
\showcaseopenmojicolor{family- man, man, girl, girl}{802}
\showcaseopenmojicolor{family- man, man, girl}{803}
\showcaseopenmojicolor{family- man, woman, boy, boy}{804}
\showcaseopenmojicolor{family- man, woman, boy}{805}
\showcaseopenmojicolor{family- man, woman, girl, boy}{806}
\showcaseopenmojicolor{family- man, woman, girl, girl}{807}
\showcaseopenmojicolor{family- man, woman, girl}{808}
\showcaseopenmojicolor{family- woman, boy, boy}{809}
\showcaseopenmojicolor{family- woman, boy}{810}
\showcaseopenmojicolor{family- woman, girl, boy}{811}
\showcaseopenmojicolor{family- woman, girl, girl}{812}
\showcaseopenmojicolor{family- woman, girl}{813}
\showcaseopenmojicolor{family- woman, woman, boy, boy}{814}
\showcaseopenmojicolor{family- woman, woman, boy}{815}
\showcaseopenmojicolor{family- woman, woman, girl, boy}{816}
\showcaseopenmojicolor{family- woman, woman, girl, girl}{817}
\showcaseopenmojicolor{family- woman, woman, girl}{818}
\showcaseopenmojicolor{family}{819}
\showcaseopenmojicolor{farmer- dark skin tone}{820}
\showcaseopenmojicolor{farmer- light skin tone}{821}
\showcaseopenmojicolor{farmer- medium skin tone}{822}
\showcaseopenmojicolor{farmer- medium-dark skin tone}{823}
\showcaseopenmojicolor{farmer- medium-light skin tone}{824}
\showcaseopenmojicolor{farmer}{825}
\showcaseopenmojicolor{fast down button}{826}
\showcaseopenmojicolor{fast reverse button}{827}
\showcaseopenmojicolor{fast up button}{828}
\showcaseopenmojicolor{fast-forward button}{829}
\showcaseopenmojicolor{fax machine}{830}
\showcaseopenmojicolor{fearful face}{831}
\showcaseopenmojicolor{feather}{832}
\showcaseopenmojicolor{female doctor}{833}
\showcaseopenmojicolor{female nurse}{834}
\showcaseopenmojicolor{female sign}{835}
\showcaseopenmojicolor{ferris wheel}{836}
\showcaseopenmojicolor{ferry}{837}
\showcaseopenmojicolor{field hockey}{838}
\showcaseopenmojicolor{file cabinet}{839}
\showcaseopenmojicolor{file folder}{840}
\showcaseopenmojicolor{film frames}{841}
\showcaseopenmojicolor{film projector}{842}
\showcaseopenmojicolor{filter}{843}
\showcaseopenmojicolor{finger pushing button}{844}
\showcaseopenmojicolor{fire engine}{845}
\showcaseopenmojicolor{fire extinguisher}{846}
\showcaseopenmojicolor{fire}{847}
\showcaseopenmojicolor{firecracker}{848}
\showcaseopenmojicolor{firefighter- dark skin tone}{849}
\showcaseopenmojicolor{firefighter- light skin tone}{850}
\showcaseopenmojicolor{firefighter- medium skin tone}{851}
\showcaseopenmojicolor{firefighter- medium-dark skin tone}{852}
\showcaseopenmojicolor{firefighter- medium-light skin tone}{853}
\showcaseopenmojicolor{firefighter}{854}
\showcaseopenmojicolor{firefox developer}{855}
\showcaseopenmojicolor{firefox nightly}{856}
\showcaseopenmojicolor{firefox}{857}
\showcaseopenmojicolor{fireworks}{858}
\showcaseopenmojicolor{first aid bag}{859}
\showcaseopenmojicolor{first aid kit}{860}
\showcaseopenmojicolor{first aid}{861}
\showcaseopenmojicolor{first quarter moon face}{862}
\showcaseopenmojicolor{first quarter moon}{863}
\showcaseopenmojicolor{fish cake with swirl}{864}
\showcaseopenmojicolor{fish}{865}
\showcaseopenmojicolor{fisheye}{866}
\showcaseopenmojicolor{fishing pole}{867}
\showcaseopenmojicolor{five o clock}{868}
\showcaseopenmojicolor{five-thirty}{869}
\showcaseopenmojicolor{flag in hole}{870}
\showcaseopenmojicolor{flag- afghanistan}{871}
\showcaseopenmojicolor{flag- albania}{872}
\showcaseopenmojicolor{flag- algeria}{873}
\showcaseopenmojicolor{flag- american samoa}{874}
\showcaseopenmojicolor{flag- andorra}{875}
\showcaseopenmojicolor{flag- angola}{876}
\showcaseopenmojicolor{flag- anguilla}{877}
\showcaseopenmojicolor{flag- antarctica}{878}
\showcaseopenmojicolor{flag- antigua and barbuda}{879}
\showcaseopenmojicolor{flag- argentina}{880}
\showcaseopenmojicolor{flag- armenia}{881}
\showcaseopenmojicolor{flag- aruba}{882}
\showcaseopenmojicolor{flag- ascension island}{883}
\showcaseopenmojicolor{flag- australia}{884}
\showcaseopenmojicolor{flag- austria}{885}
\showcaseopenmojicolor{flag- azerbaijan}{886}
\showcaseopenmojicolor{flag- bahamas}{887}
\showcaseopenmojicolor{flag- bahrain}{888}
\showcaseopenmojicolor{flag- bangladesh}{889}
\showcaseopenmojicolor{flag- barbados}{890}
\showcaseopenmojicolor{flag- belarus}{891}
\showcaseopenmojicolor{flag- belgium}{892}
\showcaseopenmojicolor{flag- belize}{893}
\showcaseopenmojicolor{flag- benin}{894}
\showcaseopenmojicolor{flag- bermuda}{895}
\showcaseopenmojicolor{flag- bhutan}{896}
\showcaseopenmojicolor{flag- bolivia}{897}
\showcaseopenmojicolor{flag- bosnia and herzegovina}{898}
\showcaseopenmojicolor{flag- botswana}{899}
\showcaseopenmojicolor{flag- bouvet island}{900}
\showcaseopenmojicolor{flag- brazil}{901}
\showcaseopenmojicolor{flag- british indian ocean territory}{902}
\showcaseopenmojicolor{flag- british virgin islands}{903}
\showcaseopenmojicolor{flag- brunei}{904}
\showcaseopenmojicolor{flag- bulgaria}{905}
\showcaseopenmojicolor{flag- burkina faso}{906}
\showcaseopenmojicolor{flag- burundi}{907}
\showcaseopenmojicolor{flag- cambodia}{908}
\showcaseopenmojicolor{flag- cameroon}{909}
\showcaseopenmojicolor{flag- canada}{910}
\showcaseopenmojicolor{flag- canary islands}{911}
\showcaseopenmojicolor{flag- cape verde}{912}
\showcaseopenmojicolor{flag- caribbean netherlands}{913}
\showcaseopenmojicolor{flag- cayman islands}{914}
\showcaseopenmojicolor{flag- central african republic}{915}
\showcaseopenmojicolor{flag- ceuta and melilla}{916}
\showcaseopenmojicolor{flag- chad}{917}
\showcaseopenmojicolor{flag- chile}{918}
\showcaseopenmojicolor{flag- china}{919}
\showcaseopenmojicolor{flag- christmas island}{920}
\showcaseopenmojicolor{flag- clipperton island}{921}
\showcaseopenmojicolor{flag- cocos (keeling) islands}{922}
\showcaseopenmojicolor{flag- colombia}{923}
\showcaseopenmojicolor{flag- comoros}{924}
\showcaseopenmojicolor{flag- congo - brazzaville}{925}
\showcaseopenmojicolor{flag- congo - kinshasa}{926}
\showcaseopenmojicolor{flag- cook islands}{927}
\showcaseopenmojicolor{flag- costa rica}{928}
\showcaseopenmojicolor{flag- croatia}{929}
\showcaseopenmojicolor{flag- cuba}{930}
\showcaseopenmojicolor{flag- curacao}{931}
\showcaseopenmojicolor{flag- cyprus}{932}
\showcaseopenmojicolor{flag- czechia}{933}
\showcaseopenmojicolor{flag- cote d ivoire}{934}
\showcaseopenmojicolor{flag- denmark}{935}
\showcaseopenmojicolor{flag- diego garcia}{936}
\showcaseopenmojicolor{flag- djibouti}{937}
\showcaseopenmojicolor{flag- dominica}{938}
\showcaseopenmojicolor{flag- dominican republic}{939}
\showcaseopenmojicolor{flag- ecuador}{940}
\showcaseopenmojicolor{flag- egypt}{941}
\showcaseopenmojicolor{flag- el salvador}{942}
\showcaseopenmojicolor{flag- england}{943}
\showcaseopenmojicolor{flag- equatorial guinea}{944}
\showcaseopenmojicolor{flag- eritrea}{945}
\showcaseopenmojicolor{flag- estonia}{946}
\showcaseopenmojicolor{flag- eswatini}{947}
\showcaseopenmojicolor{flag- ethiopia}{948}
\showcaseopenmojicolor{flag- european union}{949}
\showcaseopenmojicolor{flag- falkland islands}{950}
\showcaseopenmojicolor{flag- faroe islands}{951}
\showcaseopenmojicolor{flag- fiji}{952}
\showcaseopenmojicolor{flag- finland}{953}
\showcaseopenmojicolor{flag- france}{954}
\showcaseopenmojicolor{flag- french guiana}{955}
\showcaseopenmojicolor{flag- french polynesia}{956}
\showcaseopenmojicolor{flag- french southern territories}{957}
\showcaseopenmojicolor{flag- gabon}{958}
\showcaseopenmojicolor{flag- gambia}{959}
\showcaseopenmojicolor{flag- georgia}{960}
\showcaseopenmojicolor{flag- germany}{961}
\showcaseopenmojicolor{flag- ghana}{962}
\showcaseopenmojicolor{flag- gibraltar}{963}
\showcaseopenmojicolor{flag- greece}{964}
\showcaseopenmojicolor{flag- greenland}{965}
\showcaseopenmojicolor{flag- grenada}{966}
\showcaseopenmojicolor{flag- guadeloupe}{967}
\showcaseopenmojicolor{flag- guam}{968}
\showcaseopenmojicolor{flag- guatemala}{969}
\showcaseopenmojicolor{flag- guernsey}{970}
\showcaseopenmojicolor{flag- guinea-bissau}{971}
\showcaseopenmojicolor{flag- guinea}{972}
\showcaseopenmojicolor{flag- guyana}{973}
\showcaseopenmojicolor{flag- haiti}{974}
\showcaseopenmojicolor{flag- heard and mcdonald islands}{975}
\showcaseopenmojicolor{flag- honduras}{976}
\showcaseopenmojicolor{flag- hong kong sar china}{977}
\showcaseopenmojicolor{flag- hungary}{978}
\showcaseopenmojicolor{flag- iceland}{979}
\showcaseopenmojicolor{flag- india}{980}
\showcaseopenmojicolor{flag- indonesia}{981}
\showcaseopenmojicolor{flag- iran}{982}
\showcaseopenmojicolor{flag- iraq}{983}
\showcaseopenmojicolor{flag- ireland}{984}
\showcaseopenmojicolor{flag- isle of man}{985}
\showcaseopenmojicolor{flag- israel}{986}
\showcaseopenmojicolor{flag- italy}{987}
\showcaseopenmojicolor{flag- jamaica}{988}
\showcaseopenmojicolor{flag- japan}{989}
\showcaseopenmojicolor{flag- jersey}{990}
\showcaseopenmojicolor{flag- jordan}{991}
\showcaseopenmojicolor{flag- kazakhstan}{992}
\showcaseopenmojicolor{flag- kenya}{993}
\showcaseopenmojicolor{flag- kiribati}{994}
\showcaseopenmojicolor{flag- kosovo}{995}
\showcaseopenmojicolor{flag- kuwait}{996}
\showcaseopenmojicolor{flag- kyrgyzstan}{997}
\showcaseopenmojicolor{flag- laos}{998}
\showcaseopenmojicolor{flag- latvia}{999}
\showcaseopenmojicolor{flag- lebanon}{1000}
\showcaseopenmojicolor{flag- lesotho}{1001}
\showcaseopenmojicolor{flag- liberia}{1002}
\showcaseopenmojicolor{flag- libya}{1003}
\showcaseopenmojicolor{flag- liechtenstein}{1004}
\showcaseopenmojicolor{flag- lithuania}{1005}
\showcaseopenmojicolor{flag- luxembourg}{1006}
\showcaseopenmojicolor{flag- macao sar china}{1007}
\showcaseopenmojicolor{flag- madagascar}{1008}
\showcaseopenmojicolor{flag- malawi}{1009}
\showcaseopenmojicolor{flag- malaysia}{1010}
\showcaseopenmojicolor{flag- maldives}{1011}
\showcaseopenmojicolor{flag- mali}{1012}
\showcaseopenmojicolor{flag- malta}{1013}
\showcaseopenmojicolor{flag- marshall islands}{1014}
\showcaseopenmojicolor{flag- martinique}{1015}
\showcaseopenmojicolor{flag- mauritania}{1016}
\showcaseopenmojicolor{flag- mauritius}{1017}
\showcaseopenmojicolor{flag- mayotte}{1018}
\showcaseopenmojicolor{flag- mexico}{1019}
\showcaseopenmojicolor{flag- micronesia}{1020}
\showcaseopenmojicolor{flag- moldova}{1021}
\showcaseopenmojicolor{flag- monaco}{1022}
\showcaseopenmojicolor{flag- mongolia}{1023}
\showcaseopenmojicolor{flag- montenegro}{1024}
\showcaseopenmojicolor{flag- montserrat}{1025}
\showcaseopenmojicolor{flag- morocco}{1026}
\showcaseopenmojicolor{flag- mozambique}{1027}
\showcaseopenmojicolor{flag- myanmar (burma)}{1028}
\showcaseopenmojicolor{flag- namibia}{1029}
\showcaseopenmojicolor{flag- nauru}{1030}
\showcaseopenmojicolor{flag- nepal}{1031}
\showcaseopenmojicolor{flag- netherlands}{1032}
\showcaseopenmojicolor{flag- new caledonia}{1033}
\showcaseopenmojicolor{flag- new zealand}{1034}
\showcaseopenmojicolor{flag- nicaragua}{1035}
\showcaseopenmojicolor{flag- niger}{1036}
\showcaseopenmojicolor{flag- nigeria}{1037}
\showcaseopenmojicolor{flag- niue}{1038}
\showcaseopenmojicolor{flag- norfolk island}{1039}
\showcaseopenmojicolor{flag- north korea}{1040}
\showcaseopenmojicolor{flag- north macedonia}{1041}
\showcaseopenmojicolor{flag- northern mariana islands}{1042}
\showcaseopenmojicolor{flag- norway}{1043}
\showcaseopenmojicolor{flag- oman}{1044}
\showcaseopenmojicolor{flag- pakistan}{1045}
\showcaseopenmojicolor{flag- palau}{1046}
\showcaseopenmojicolor{flag- palestinian territories}{1047}
\showcaseopenmojicolor{flag- panama}{1048}
\showcaseopenmojicolor{flag- papua new guinea}{1049}
\showcaseopenmojicolor{flag- paraguay}{1050}
\showcaseopenmojicolor{flag- peru}{1051}
\showcaseopenmojicolor{flag- philippines}{1052}
\showcaseopenmojicolor{flag- pitcairn islands}{1053}
\showcaseopenmojicolor{flag- poland}{1054}
\showcaseopenmojicolor{flag- portugal}{1055}
\showcaseopenmojicolor{flag- puerto rico}{1056}
\showcaseopenmojicolor{flag- qatar}{1057}
\showcaseopenmojicolor{flag- romania}{1058}
\showcaseopenmojicolor{flag- russia}{1059}
\showcaseopenmojicolor{flag- rwanda}{1060}
\showcaseopenmojicolor{flag- reunion}{1061}
\showcaseopenmojicolor{flag- samoa}{1062}
\showcaseopenmojicolor{flag- san marino}{1063}
\showcaseopenmojicolor{flag- saudi arabia}{1064}
\showcaseopenmojicolor{flag- scotland}{1065}
\showcaseopenmojicolor{flag- senegal}{1066}
\showcaseopenmojicolor{flag- serbia}{1067}
\showcaseopenmojicolor{flag- seychelles}{1068}
\showcaseopenmojicolor{flag- sierra leone}{1069}
\showcaseopenmojicolor{flag- singapore}{1070}
\showcaseopenmojicolor{flag- sint maarten}{1071}
\showcaseopenmojicolor{flag- slovakia}{1072}
\showcaseopenmojicolor{flag- slovenia}{1073}
\showcaseopenmojicolor{flag- solomon islands}{1074}
\showcaseopenmojicolor{flag- somalia}{1075}
\showcaseopenmojicolor{flag- south africa}{1076}
\showcaseopenmojicolor{flag- south georgia and south sandwich islands}{1077}
\showcaseopenmojicolor{flag- south korea}{1078}
\showcaseopenmojicolor{flag- south sudan}{1079}
\showcaseopenmojicolor{flag- spain}{1080}
\showcaseopenmojicolor{flag- sri lanka}{1081}
\showcaseopenmojicolor{flag- st}{1082}
\showcaseopenmojicolor{flag- sudan}{1083}
\showcaseopenmojicolor{flag- suriname}{1084}
\showcaseopenmojicolor{flag- svalbard and jan mayen}{1085}
\showcaseopenmojicolor{flag- sweden}{1086}
\showcaseopenmojicolor{flag- switzerland}{1087}
\showcaseopenmojicolor{flag- syria}{1088}
\showcaseopenmojicolor{flag- sao toma and pincipe}{1089}
\showcaseopenmojicolor{flag- taiwan}{1090}
\showcaseopenmojicolor{flag- tajikistan}{1091}
\showcaseopenmojicolor{flag- tanzania}{1092}
\showcaseopenmojicolor{flag- thailand}{1093}
\showcaseopenmojicolor{flag- timor-leste}{1094}
\showcaseopenmojicolor{flag- togo}{1095}
\showcaseopenmojicolor{flag- tokelau}{1096}
\showcaseopenmojicolor{flag- tonga}{1097}
\showcaseopenmojicolor{flag- trinidad and tobago}{1098}
\showcaseopenmojicolor{flag- tristan da cunha}{1099}
\showcaseopenmojicolor{flag- tunisia}{1100}
\showcaseopenmojicolor{flag- turkmenistan}{1101}
\showcaseopenmojicolor{flag- turks and caicos islands}{1102}
\showcaseopenmojicolor{flag- tuvalu}{1103}
\showcaseopenmojicolor{flag- turkiye}{1104}
\showcaseopenmojicolor{flag- u}{1105}
\showcaseopenmojicolor{flag- uganda}{1106}
\showcaseopenmojicolor{flag- ukraine}{1107}
\showcaseopenmojicolor{flag- united arab emirates}{1108}
\showcaseopenmojicolor{flag- united kingdom}{1109}
\showcaseopenmojicolor{flag- united nations}{1110}
\showcaseopenmojicolor{flag- united states}{1111}
\showcaseopenmojicolor{flag- uruguay}{1112}
\showcaseopenmojicolor{flag- uzbekistan}{1113}
\showcaseopenmojicolor{flag- vanuatu}{1114}
\showcaseopenmojicolor{flag- vatican city}{1115}
\showcaseopenmojicolor{flag- venezuela}{1116}
\showcaseopenmojicolor{flag- vietnam}{1117}
\showcaseopenmojicolor{flag- wales}{1118}
\showcaseopenmojicolor{flag- wallis and futuna}{1119}
\showcaseopenmojicolor{flag- western sahara}{1120}
\showcaseopenmojicolor{flag- yemen}{1121}
\showcaseopenmojicolor{flag- zambia}{1122}
\showcaseopenmojicolor{flag- zimbabwe}{1123}
\showcaseopenmojicolor{flag- aland islands}{1124}
\showcaseopenmojicolor{flagged building}{1125}
\showcaseopenmojicolor{flagged point}{1126}
\showcaseopenmojicolor{flamingo}{1127}
\showcaseopenmojicolor{flashlight}{1128}
\showcaseopenmojicolor{flat shoe}{1129}
\showcaseopenmojicolor{flatbread}{1130}
\showcaseopenmojicolor{fleur-de-lis}{1131}
\showcaseopenmojicolor{flexed biceps- dark skin tone}{1132}
\showcaseopenmojicolor{flexed biceps- light skin tone}{1133}
\showcaseopenmojicolor{flexed biceps- medium skin tone}{1134}
\showcaseopenmojicolor{flexed biceps- medium-dark skin tone}{1135}
\showcaseopenmojicolor{flexed biceps- medium-light skin tone}{1136}
\showcaseopenmojicolor{flexed biceps}{1137}
\showcaseopenmojicolor{floating ice broken}{1138}
\showcaseopenmojicolor{floating ice}{1139}
\showcaseopenmojicolor{flood}{1140}
\showcaseopenmojicolor{floppy disk}{1141}
\showcaseopenmojicolor{flower playing cards}{1142}
\showcaseopenmojicolor{flushed face}{1143}
\showcaseopenmojicolor{flute}{1144}
\showcaseopenmojicolor{fly}{1145}
\showcaseopenmojicolor{flying disc}{1146}
\showcaseopenmojicolor{flying saucer}{1147}
\showcaseopenmojicolor{fog}{1148}
\showcaseopenmojicolor{foggy mountain}{1149}
\showcaseopenmojicolor{foggy}{1150}
\showcaseopenmojicolor{folded hands- dark skin tone}{1151}
\showcaseopenmojicolor{folded hands- light skin tone}{1152}
\showcaseopenmojicolor{folded hands- medium skin tone}{1153}
\showcaseopenmojicolor{folded hands- medium-dark skin tone}{1154}
\showcaseopenmojicolor{folded hands- medium-light skin tone}{1155}
\showcaseopenmojicolor{folded hands}{1156}
\showcaseopenmojicolor{folding hand fan}{1157}
\showcaseopenmojicolor{fondue}{1158}
\showcaseopenmojicolor{foot- dark skin tone}{1159}
\showcaseopenmojicolor{foot- light skin tone}{1160}
\showcaseopenmojicolor{foot- medium skin tone}{1161}
\showcaseopenmojicolor{foot- medium-dark skin tone}{1162}
\showcaseopenmojicolor{foot- medium-light skin tone}{1163}
\showcaseopenmojicolor{foot}{1164}
\showcaseopenmojicolor{footprints}{1165}
\showcaseopenmojicolor{forceps}{1166}
\showcaseopenmojicolor{fork and knife with plate}{1167}
\showcaseopenmojicolor{fork and knife}{1168}
\showcaseopenmojicolor{fortune cookie}{1169}
\showcaseopenmojicolor{forward}{1170}
\showcaseopenmojicolor{fountain pen}{1171}
\showcaseopenmojicolor{fountain}{1172}
\showcaseopenmojicolor{four leaf clover}{1173}
\showcaseopenmojicolor{four o clock}{1174}
\showcaseopenmojicolor{four-thirty}{1175}
\showcaseopenmojicolor{fox}{1176}
\showcaseopenmojicolor{fracture leg}{1177}
\showcaseopenmojicolor{framed picture}{1178}
\showcaseopenmojicolor{free button}{1179}
\showcaseopenmojicolor{french fries}{1180}
\showcaseopenmojicolor{french press}{1181}
\showcaseopenmojicolor{fried shrimp}{1182}
\showcaseopenmojicolor{frog}{1183}
\showcaseopenmojicolor{front-facing baby chick}{1184}
\showcaseopenmojicolor{frowning face with open mouth}{1185}
\showcaseopenmojicolor{frowning face}{1186}
\showcaseopenmojicolor{fuel pump}{1187}
\showcaseopenmojicolor{full moon face}{1188}
\showcaseopenmojicolor{full moon}{1189}
\showcaseopenmojicolor{funeral urn}{1190}
\showcaseopenmojicolor{galicia flag}{1191}
\showcaseopenmojicolor{game die}{1192}
\showcaseopenmojicolor{gardener man}{1193}
\showcaseopenmojicolor{gardener woman}{1194}
\showcaseopenmojicolor{gardening gloves}{1195}
\showcaseopenmojicolor{garlic}{1196}
\showcaseopenmojicolor{gear}{1197}
\showcaseopenmojicolor{geiger counter}{1198}
\showcaseopenmojicolor{gem stone}{1199}
\showcaseopenmojicolor{gemini}{1200}
\showcaseopenmojicolor{genie}{1201}
\showcaseopenmojicolor{ghost}{1202}
\showcaseopenmojicolor{ginger root}{1203}
\showcaseopenmojicolor{giraffe}{1204}
\showcaseopenmojicolor{girl- dark skin tone}{1205}
\showcaseopenmojicolor{girl- light skin tone}{1206}
\showcaseopenmojicolor{girl- medium skin tone}{1207}
\showcaseopenmojicolor{girl- medium-dark skin tone}{1208}
\showcaseopenmojicolor{girl- medium-light skin tone}{1209}
\showcaseopenmojicolor{girl}{1210}
\showcaseopenmojicolor{github}{1211}
\showcaseopenmojicolor{gitlab}{1212}
\showcaseopenmojicolor{glass bottle}{1213}
\showcaseopenmojicolor{glass of milk}{1214}
\showcaseopenmojicolor{glasses}{1215}
\showcaseopenmojicolor{globe showing americas}{1216}
\showcaseopenmojicolor{globe showing asia-australia}{1217}
\showcaseopenmojicolor{globe showing europe-africa}{1218}
\showcaseopenmojicolor{globe with meridians}{1219}
\showcaseopenmojicolor{gloves}{1220}
\showcaseopenmojicolor{glowing star}{1221}
\showcaseopenmojicolor{goal net}{1222}
\showcaseopenmojicolor{goat}{1223}
\showcaseopenmojicolor{goblin}{1224}
\showcaseopenmojicolor{goggles}{1225}
\showcaseopenmojicolor{golang}{1226}
\showcaseopenmojicolor{goldfish}{1227}
\showcaseopenmojicolor{goose}{1228}
\showcaseopenmojicolor{gorilla}{1229}
\showcaseopenmojicolor{gps}{1230}
\showcaseopenmojicolor{graduation cap}{1231}
\showcaseopenmojicolor{grapes}{1232}
\showcaseopenmojicolor{great pyramid of giza}{1233}
\showcaseopenmojicolor{green apple}{1234}
\showcaseopenmojicolor{green book}{1235}
\showcaseopenmojicolor{green circle}{1236}
\showcaseopenmojicolor{green flag}{1237}
\showcaseopenmojicolor{green heart}{1238}
\showcaseopenmojicolor{green hexagon}{1239}
\showcaseopenmojicolor{green salad}{1240}
\showcaseopenmojicolor{green square}{1241}
\showcaseopenmojicolor{greta thunberg}{1242}
\showcaseopenmojicolor{grey heart}{1243}
\showcaseopenmojicolor{grimacing face}{1244}
\showcaseopenmojicolor{grinning cat with smiling eyes}{1245}
\showcaseopenmojicolor{grinning cat}{1246}
\showcaseopenmojicolor{grinning face with big eyes}{1247}
\showcaseopenmojicolor{grinning face with smiling eyes}{1248}
\showcaseopenmojicolor{grinning face with sweat}{1249}
\showcaseopenmojicolor{grinning face}{1250}
\showcaseopenmojicolor{grinning squinting face}{1251}
\showcaseopenmojicolor{growing heart}{1252}
\showcaseopenmojicolor{guard- dark skin tone}{1253}
\showcaseopenmojicolor{guard- light skin tone}{1254}
\showcaseopenmojicolor{guard- medium skin tone}{1255}
\showcaseopenmojicolor{guard- medium-dark skin tone}{1256}
\showcaseopenmojicolor{guard- medium-light skin tone}{1257}
\showcaseopenmojicolor{guard}{1258}
\showcaseopenmojicolor{guide dog}{1259}
\showcaseopenmojicolor{guitar}{1260}
\showcaseopenmojicolor{guy fawkes mask}{1261}
\showcaseopenmojicolor{hacker cat}{1262}
\showcaseopenmojicolor{hair pick}{1263}
\showcaseopenmojicolor{hal 9000}{1264}
\showcaseopenmojicolor{half orange fruit}{1265}
\showcaseopenmojicolor{hamburger menu}{1266}
\showcaseopenmojicolor{hamburger}{1267}
\showcaseopenmojicolor{hammer and pick}{1268}
\showcaseopenmojicolor{hammer and wrench}{1269}
\showcaseopenmojicolor{hammer}{1270}
\showcaseopenmojicolor{hamsa}{1271}
\showcaseopenmojicolor{hamster}{1272}
\showcaseopenmojicolor{hand with fingers splayed- dark skin tone}{1273}
\showcaseopenmojicolor{hand with fingers splayed- light skin tone}{1274}
\showcaseopenmojicolor{hand with fingers splayed- medium skin tone}{1275}
\showcaseopenmojicolor{hand with fingers splayed- medium-dark skin tone}{1276}
\showcaseopenmojicolor{hand with fingers splayed- medium-light skin tone}{1277}
\showcaseopenmojicolor{hand with fingers splayed}{1278}
\showcaseopenmojicolor{hand with index finger and thumb crossed- dark skin tone}{1279}
\showcaseopenmojicolor{hand with index finger and thumb crossed- light skin tone}{1280}
\showcaseopenmojicolor{hand with index finger and thumb crossed- medium skin tone}{1281}
\showcaseopenmojicolor{hand with index finger and thumb crossed- medium-dark skin tone}{1282}
\showcaseopenmojicolor{hand with index finger and thumb crossed- medium-light skin tone}{1283}
\showcaseopenmojicolor{hand with index finger and thumb crossed}{1284}
\showcaseopenmojicolor{handbag}{1285}
\showcaseopenmojicolor{handshake- dark skin tone, light skin tone}{1286}
\showcaseopenmojicolor{handshake- dark skin tone, medium skin tone}{1287}
\showcaseopenmojicolor{handshake- dark skin tone, medium-dark skin tone}{1288}
\showcaseopenmojicolor{handshake- dark skin tone, medium-light skin tone}{1289}
\showcaseopenmojicolor{handshake- dark skin tone}{1290}
\showcaseopenmojicolor{handshake- light skin tone, dark skin tone}{1291}
\showcaseopenmojicolor{handshake- light skin tone, medium skin tone}{1292}
\showcaseopenmojicolor{handshake- light skin tone, medium-dark skin tone}{1293}
\showcaseopenmojicolor{handshake- light skin tone, medium-light skin tone}{1294}
\showcaseopenmojicolor{handshake- light skin tone}{1295}
\showcaseopenmojicolor{handshake- medium skin tone, dark skin tone}{1296}
\showcaseopenmojicolor{handshake- medium skin tone, light skin tone}{1297}
\showcaseopenmojicolor{handshake- medium skin tone, medium-dark skin tone}{1298}
\showcaseopenmojicolor{handshake- medium skin tone, medium-light skin tone}{1299}
\showcaseopenmojicolor{handshake- medium skin tone}{1300}
\showcaseopenmojicolor{handshake- medium-dark skin tone, dark skin tone}{1301}
\showcaseopenmojicolor{handshake- medium-dark skin tone, light skin tone}{1302}
\showcaseopenmojicolor{handshake- medium-dark skin tone, medium skin tone}{1303}
\showcaseopenmojicolor{handshake- medium-dark skin tone, medium-light skin tone}{1304}
\showcaseopenmojicolor{handshake- medium-dark skin tone}{1305}
\showcaseopenmojicolor{handshake- medium-light skin tone, dark skin tone}{1306}
\showcaseopenmojicolor{handshake- medium-light skin tone, light skin tone}{1307}
\showcaseopenmojicolor{handshake- medium-light skin tone, medium skin tone}{1308}
\showcaseopenmojicolor{handshake- medium-light skin tone, medium-dark skin tone}{1309}
\showcaseopenmojicolor{handshake- medium-light skin tone}{1310}
\showcaseopenmojicolor{handshake}{1311}
\showcaseopenmojicolor{hanging gardens of babylon}{1312}
\showcaseopenmojicolor{hatching chick}{1313}
\showcaseopenmojicolor{head shaking horizontally}{1314}
\showcaseopenmojicolor{head shaking vertically}{1315}
\showcaseopenmojicolor{headache}{1316}
\showcaseopenmojicolor{headphone}{1317}
\showcaseopenmojicolor{headstone}{1318}
\showcaseopenmojicolor{health worker- dark skin tone}{1319}
\showcaseopenmojicolor{health worker- light skin tone}{1320}
\showcaseopenmojicolor{health worker- medium skin tone}{1321}
\showcaseopenmojicolor{health worker- medium-dark skin tone}{1322}
\showcaseopenmojicolor{health worker- medium-light skin tone}{1323}
\showcaseopenmojicolor{health worker}{1324}
\showcaseopenmojicolor{hear-no-evil monkey}{1325}
\showcaseopenmojicolor{heart decoration}{1326}
\showcaseopenmojicolor{heart exclamation}{1327}
\showcaseopenmojicolor{heart hands- dark skin tone}{1328}
\showcaseopenmojicolor{heart hands- light skin tone}{1329}
\showcaseopenmojicolor{heart hands- medium skin tone}{1330}
\showcaseopenmojicolor{heart hands- medium-dark skin tone}{1331}
\showcaseopenmojicolor{heart hands- medium-light skin tone}{1332}
\showcaseopenmojicolor{heart hands}{1333}
\showcaseopenmojicolor{heart on fire}{1334}
\showcaseopenmojicolor{heart suit}{1335}
\showcaseopenmojicolor{heart with arrow}{1336}
\showcaseopenmojicolor{heart with ribbon}{1337}
\showcaseopenmojicolor{heavy circle}{1338}
\showcaseopenmojicolor{heavy dollar sign}{1339}
\showcaseopenmojicolor{heavy equals sign}{1340}
\showcaseopenmojicolor{hedgehog}{1341}
\showcaseopenmojicolor{helicopter}{1342}
\showcaseopenmojicolor{help others}{1343}
\showcaseopenmojicolor{herb}{1344}
\showcaseopenmojicolor{hibiscus}{1345}
\showcaseopenmojicolor{high voltage}{1346}
\showcaseopenmojicolor{high-heeled shoe}{1347}
\showcaseopenmojicolor{high-speed train}{1348}
\showcaseopenmojicolor{hiking boot}{1349}
\showcaseopenmojicolor{hindu temple}{1350}
\showcaseopenmojicolor{hippopotamus}{1351}
\showcaseopenmojicolor{hold}{1352}
\showcaseopenmojicolor{hole}{1353}
\showcaseopenmojicolor{hollow red circle}{1354}
\showcaseopenmojicolor{home button}{1355}
\showcaseopenmojicolor{honey pot}{1356}
\showcaseopenmojicolor{honeybee}{1357}
\showcaseopenmojicolor{hook}{1358}
\showcaseopenmojicolor{horizontal black hexagon}{1359}
\showcaseopenmojicolor{horizontal black octagon}{1360}
\showcaseopenmojicolor{horizontal traffic light}{1361}
\showcaseopenmojicolor{horse face}{1362}
\showcaseopenmojicolor{horse jumping hurdle}{1363}
\showcaseopenmojicolor{horse racing- dark skin tone}{1364}
\showcaseopenmojicolor{horse racing- light skin tone}{1365}
\showcaseopenmojicolor{horse racing- medium skin tone}{1366}
\showcaseopenmojicolor{horse racing- medium-dark skin tone}{1367}
\showcaseopenmojicolor{horse racing- medium-light skin tone}{1368}
\showcaseopenmojicolor{horse racing}{1369}
\showcaseopenmojicolor{horse riding}{1370}
\showcaseopenmojicolor{horse}{1371}
\showcaseopenmojicolor{hospital}{1372}
\showcaseopenmojicolor{hot beverage}{1373}
\showcaseopenmojicolor{hot dog}{1374}
\showcaseopenmojicolor{hot face}{1375}
\showcaseopenmojicolor{hot pepper}{1376}
\showcaseopenmojicolor{hot springs}{1377}
\showcaseopenmojicolor{hot-water bottle}{1378}
\showcaseopenmojicolor{hotel}{1379}
\showcaseopenmojicolor{hourglass done}{1380}
\showcaseopenmojicolor{hourglass not done}{1381}
\showcaseopenmojicolor{house with garden}{1382}
\showcaseopenmojicolor{house}{1383}
\showcaseopenmojicolor{houses}{1384}
\showcaseopenmojicolor{hundred points}{1385}
\showcaseopenmojicolor{hushed face}{1386}
\showcaseopenmojicolor{hut}{1387}
\showcaseopenmojicolor{hyacinth}{1388}
\showcaseopenmojicolor{hyphen-minus}{1389}
\showcaseopenmojicolor{ibeacon}{1390}
\showcaseopenmojicolor{ice core sample}{1391}
\showcaseopenmojicolor{ice cream}{1392}
\showcaseopenmojicolor{ice hockey}{1393}
\showcaseopenmojicolor{ice shelf melting}{1394}
\showcaseopenmojicolor{ice shelf}{1395}
\showcaseopenmojicolor{ice skate}{1396}
\showcaseopenmojicolor{ice}{1397}
\showcaseopenmojicolor{iceberg}{1398}
\showcaseopenmojicolor{id button}{1399}
\showcaseopenmojicolor{identification card}{1400}
\showcaseopenmojicolor{inaturalist}{1401}
\showcaseopenmojicolor{inbox tray}{1402}
\showcaseopenmojicolor{inbox}{1403}
\showcaseopenmojicolor{incoming envelope}{1404}
\showcaseopenmojicolor{incredulous face}{1405}
\showcaseopenmojicolor{index pointing at the viewer- dark skin tone}{1406}
\showcaseopenmojicolor{index pointing at the viewer- light skin tone}{1407}
\showcaseopenmojicolor{index pointing at the viewer- medium skin tone}{1408}
\showcaseopenmojicolor{index pointing at the viewer- medium-dark skin tone}{1409}
\showcaseopenmojicolor{index pointing at the viewer- medium-light skin tone}{1410}
\showcaseopenmojicolor{index pointing at the viewer}{1411}
\showcaseopenmojicolor{index pointing up- dark skin tone}{1412}
\showcaseopenmojicolor{index pointing up- light skin tone}{1413}
\showcaseopenmojicolor{index pointing up- medium skin tone}{1414}
\showcaseopenmojicolor{index pointing up- medium-dark skin tone}{1415}
\showcaseopenmojicolor{index pointing up- medium-light skin tone}{1416}
\showcaseopenmojicolor{index pointing up}{1417}
\showcaseopenmojicolor{infinity}{1418}
\showcaseopenmojicolor{information}{1419}
\showcaseopenmojicolor{input latin letters}{1420}
\showcaseopenmojicolor{input latin lowercase}{1421}
\showcaseopenmojicolor{input latin uppercase}{1422}
\showcaseopenmojicolor{input numbers}{1423}
\showcaseopenmojicolor{input symbols}{1424}
\showcaseopenmojicolor{instagram}{1425}
\showcaseopenmojicolor{internet explorer}{1426}
\showcaseopenmojicolor{interview}{1427}
\showcaseopenmojicolor{intestine}{1428}
\showcaseopenmojicolor{intricate}{1429}
\showcaseopenmojicolor{jack-o-lantern}{1430}
\showcaseopenmojicolor{japanese castle}{1431}
\showcaseopenmojicolor{japanese dolls}{1432}
\showcaseopenmojicolor{japanese post office}{1433}
\showcaseopenmojicolor{japanese symbol for beginner}{1434}
\showcaseopenmojicolor{japanese acceptable button}{1435}
\showcaseopenmojicolor{japanese application button}{1436}
\showcaseopenmojicolor{japanese bargain button}{1437}
\showcaseopenmojicolor{japanese congratulations button}{1438}
\showcaseopenmojicolor{japanese discount button}{1439}
\showcaseopenmojicolor{japanese free of charge button}{1440}
\showcaseopenmojicolor{japanese here button}{1441}
\showcaseopenmojicolor{japanese monthly amount button}{1442}
\showcaseopenmojicolor{japanese no vacancy button}{1443}
\showcaseopenmojicolor{japanese not free of charge button}{1444}
\showcaseopenmojicolor{japanese open for business button}{1445}
\showcaseopenmojicolor{japanese passing grade button}{1446}
\showcaseopenmojicolor{japanese prohibited button}{1447}
\showcaseopenmojicolor{japanese reserved button}{1448}
\showcaseopenmojicolor{japanese secret button}{1449}
\showcaseopenmojicolor{japanese service charge button}{1450}
\showcaseopenmojicolor{japanese vacancy button}{1451}
\showcaseopenmojicolor{jar with blue content}{1452}
\showcaseopenmojicolor{jar with brown content}{1453}
\showcaseopenmojicolor{jar with green content}{1454}
\showcaseopenmojicolor{jar with orange content}{1455}
\showcaseopenmojicolor{jar with purple content}{1456}
\showcaseopenmojicolor{jar with red content}{1457}
\showcaseopenmojicolor{jar with yellow content}{1458}
\showcaseopenmojicolor{jar}{1459}
\showcaseopenmojicolor{javascript}{1460}
\showcaseopenmojicolor{jeans}{1461}
\showcaseopenmojicolor{jellyfin}{1462}
\showcaseopenmojicolor{jellyfish}{1463}
\showcaseopenmojicolor{joint pain}{1464}
\showcaseopenmojicolor{joker}{1465}
\showcaseopenmojicolor{joystick}{1466}
\showcaseopenmojicolor{judge- dark skin tone}{1467}
\showcaseopenmojicolor{judge- light skin tone}{1468}
\showcaseopenmojicolor{judge- medium skin tone}{1469}
\showcaseopenmojicolor{judge- medium-dark skin tone}{1470}
\showcaseopenmojicolor{judge- medium-light skin tone}{1471}
\showcaseopenmojicolor{judge}{1472}
\showcaseopenmojicolor{kaaba}{1473}
\showcaseopenmojicolor{kangaroo}{1474}
\showcaseopenmojicolor{kehrwoche}{1475}
\showcaseopenmojicolor{key}{1476}
\showcaseopenmojicolor{keyboard}{1477}
\showcaseopenmojicolor{keycap- hashtag}{1478}
\showcaseopenmojicolor{keycap- -}{1479}
\showcaseopenmojicolor{keycap- 0}{1480}
\showcaseopenmojicolor{keycap- 1}{1481}
\showcaseopenmojicolor{keycap- 2}{1482}
\showcaseopenmojicolor{keycap- 3}{1483}
\showcaseopenmojicolor{keycap- 4}{1484}
\showcaseopenmojicolor{keycap- 5}{1485}
\showcaseopenmojicolor{keycap- 6}{1486}
\showcaseopenmojicolor{keycap- 7}{1487}
\showcaseopenmojicolor{keycap- 8}{1488}
\showcaseopenmojicolor{keycap- 9}{1489}
\showcaseopenmojicolor{keycap- 10}{1490}
\showcaseopenmojicolor{khanda}{1491}
\showcaseopenmojicolor{kick scooter}{1492}
\showcaseopenmojicolor{kidney}{1493}
\showcaseopenmojicolor{kimono}{1494}
\showcaseopenmojicolor{kiss mark}{1495}
\showcaseopenmojicolor{kiss- dark skin tone}{1496}
\showcaseopenmojicolor{kiss- light skin tone}{1497}
\showcaseopenmojicolor{kiss- man, man, dark skin tone, light skin tone}{1498}
\showcaseopenmojicolor{kiss- man, man, dark skin tone, medium skin tone}{1499}
\showcaseopenmojicolor{kiss- man, man, dark skin tone, medium-dark skin tone}{1500}
\showcaseopenmojicolor{kiss- man, man, dark skin tone, medium-light skin tone}{1501}
\showcaseopenmojicolor{kiss- man, man, dark skin tone}{1502}
\showcaseopenmojicolor{kiss- man, man, light skin tone, dark skin tone}{1503}
\showcaseopenmojicolor{kiss- man, man, light skin tone, medium skin tone}{1504}
\showcaseopenmojicolor{kiss- man, man, light skin tone, medium-dark skin tone}{1505}
\showcaseopenmojicolor{kiss- man, man, light skin tone, medium-light skin tone}{1506}
\showcaseopenmojicolor{kiss- man, man, light skin tone}{1507}
\showcaseopenmojicolor{kiss- man, man, medium skin tone, dark skin tone}{1508}
\showcaseopenmojicolor{kiss- man, man, medium skin tone, light skin tone}{1509}
\showcaseopenmojicolor{kiss- man, man, medium skin tone, medium-dark skin tone}{1510}
\showcaseopenmojicolor{kiss- man, man, medium skin tone, medium-light skin tone}{1511}
\showcaseopenmojicolor{kiss- man, man, medium skin tone}{1512}
\showcaseopenmojicolor{kiss- man, man, medium-dark skin tone, dark skin tone}{1513}
\showcaseopenmojicolor{kiss- man, man, medium-dark skin tone, light skin tone}{1514}
\showcaseopenmojicolor{kiss- man, man, medium-dark skin tone, medium skin tone}{1515}
\showcaseopenmojicolor{kiss- man, man, medium-dark skin tone, medium-light skin tone}{1516}
\showcaseopenmojicolor{kiss- man, man, medium-dark skin tone}{1517}
\showcaseopenmojicolor{kiss- man, man, medium-light skin tone, dark skin tone}{1518}
\showcaseopenmojicolor{kiss- man, man, medium-light skin tone, light skin tone}{1519}
\showcaseopenmojicolor{kiss- man, man, medium-light skin tone, medium skin tone}{1520}
\showcaseopenmojicolor{kiss- man, man, medium-light skin tone, medium-dark skin tone}{1521}
\showcaseopenmojicolor{kiss- man, man, medium-light skin tone}{1522}
\showcaseopenmojicolor{kiss- man, man}{1523}
\showcaseopenmojicolor{kiss- medium skin tone}{1524}
\showcaseopenmojicolor{kiss- medium-dark skin tone}{1525}
\showcaseopenmojicolor{kiss- medium-light skin tone}{1526}
\showcaseopenmojicolor{kiss- person, person, dark skin tone, light skin tone}{1527}
\showcaseopenmojicolor{kiss- person, person, dark skin tone, medium skin tone}{1528}
\showcaseopenmojicolor{kiss- person, person, dark skin tone, medium-dark skin tone}{1529}
\showcaseopenmojicolor{kiss- person, person, dark skin tone, medium-light skin tone}{1530}
\showcaseopenmojicolor{kiss- person, person, light skin tone, dark skin tone}{1531}
\showcaseopenmojicolor{kiss- person, person, light skin tone, medium skin tone}{1532}
\showcaseopenmojicolor{kiss- person, person, light skin tone, medium-dark skin tone}{1533}
\showcaseopenmojicolor{kiss- person, person, light skin tone, medium-light skin tone}{1534}
\showcaseopenmojicolor{kiss- person, person, medium skin tone, dark skin tone}{1535}
\showcaseopenmojicolor{kiss- person, person, medium skin tone, light skin tone}{1536}
\showcaseopenmojicolor{kiss- person, person, medium skin tone, medium-dark skin tone}{1537}
\showcaseopenmojicolor{kiss- person, person, medium skin tone, medium-light skin tone}{1538}
\showcaseopenmojicolor{kiss- person, person, medium-dark skin tone, dark skin tone}{1539}
\showcaseopenmojicolor{kiss- person, person, medium-dark skin tone, light skin tone}{1540}
\showcaseopenmojicolor{kiss- person, person, medium-dark skin tone, medium skin tone}{1541}
\showcaseopenmojicolor{kiss- person, person, medium-dark skin tone, medium-light skin tone}{1542}
\showcaseopenmojicolor{kiss- person, person, medium-light skin tone, dark skin tone}{1543}
\showcaseopenmojicolor{kiss- person, person, medium-light skin tone, light skin tone}{1544}
\showcaseopenmojicolor{kiss- person, person, medium-light skin tone, medium skin tone}{1545}
\showcaseopenmojicolor{kiss- person, person, medium-light skin tone, medium-dark skin tone}{1546}
\showcaseopenmojicolor{kiss- woman, man, dark skin tone, light skin tone}{1547}
\showcaseopenmojicolor{kiss- woman, man, dark skin tone, medium skin tone}{1548}
\showcaseopenmojicolor{kiss- woman, man, dark skin tone, medium-dark skin tone}{1549}
\showcaseopenmojicolor{kiss- woman, man, dark skin tone, medium-light skin tone}{1550}
\showcaseopenmojicolor{kiss- woman, man, dark skin tone}{1551}
\showcaseopenmojicolor{kiss- woman, man, light skin tone, dark skin tone}{1552}
\showcaseopenmojicolor{kiss- woman, man, light skin tone, medium skin tone}{1553}
\showcaseopenmojicolor{kiss- woman, man, light skin tone, medium-dark skin tone}{1554}
\showcaseopenmojicolor{kiss- woman, man, light skin tone, medium-light skin tone}{1555}
\showcaseopenmojicolor{kiss- woman, man, light skin tone}{1556}
\showcaseopenmojicolor{kiss- woman, man, medium skin tone, dark skin tone}{1557}
\showcaseopenmojicolor{kiss- woman, man, medium skin tone, light skin tone}{1558}
\showcaseopenmojicolor{kiss- woman, man, medium skin tone, medium-dark skin tone}{1559}
\showcaseopenmojicolor{kiss- woman, man, medium skin tone, medium-light skin tone}{1560}
\showcaseopenmojicolor{kiss- woman, man, medium skin tone}{1561}
\showcaseopenmojicolor{kiss- woman, man, medium-dark skin tone, dark skin tone}{1562}
\showcaseopenmojicolor{kiss- woman, man, medium-dark skin tone, light skin tone}{1563}
\showcaseopenmojicolor{kiss- woman, man, medium-dark skin tone, medium skin tone}{1564}
\showcaseopenmojicolor{kiss- woman, man, medium-dark skin tone, medium-light skin tone}{1565}
\showcaseopenmojicolor{kiss- woman, man, medium-dark skin tone}{1566}
\showcaseopenmojicolor{kiss- woman, man, medium-light skin tone, dark skin tone}{1567}
\showcaseopenmojicolor{kiss- woman, man, medium-light skin tone, light skin tone}{1568}
\showcaseopenmojicolor{kiss- woman, man, medium-light skin tone, medium skin tone}{1569}
\showcaseopenmojicolor{kiss- woman, man, medium-light skin tone, medium-dark skin tone}{1570}
\showcaseopenmojicolor{kiss- woman, man, medium-light skin tone}{1571}
\showcaseopenmojicolor{kiss- woman, man}{1572}
\showcaseopenmojicolor{kiss- woman, woman, dark skin tone, light skin tone}{1573}
\showcaseopenmojicolor{kiss- woman, woman, dark skin tone, medium skin tone}{1574}
\showcaseopenmojicolor{kiss- woman, woman, dark skin tone, medium-dark skin tone}{1575}
\showcaseopenmojicolor{kiss- woman, woman, dark skin tone, medium-light skin tone}{1576}
\showcaseopenmojicolor{kiss- woman, woman, dark skin tone}{1577}
\showcaseopenmojicolor{kiss- woman, woman, light skin tone, dark skin tone}{1578}
\showcaseopenmojicolor{kiss- woman, woman, light skin tone, medium skin tone}{1579}
\showcaseopenmojicolor{kiss- woman, woman, light skin tone, medium-dark skin tone}{1580}
\showcaseopenmojicolor{kiss- woman, woman, light skin tone, medium-light skin tone}{1581}
\showcaseopenmojicolor{kiss- woman, woman, light skin tone}{1582}
\showcaseopenmojicolor{kiss- woman, woman, medium skin tone, dark skin tone}{1583}
\showcaseopenmojicolor{kiss- woman, woman, medium skin tone, light skin tone}{1584}
\showcaseopenmojicolor{kiss- woman, woman, medium skin tone, medium-dark skin tone}{1585}
\showcaseopenmojicolor{kiss- woman, woman, medium skin tone, medium-light skin tone}{1586}
\showcaseopenmojicolor{kiss- woman, woman, medium skin tone}{1587}
\showcaseopenmojicolor{kiss- woman, woman, medium-dark skin tone, dark skin tone}{1588}
\showcaseopenmojicolor{kiss- woman, woman, medium-dark skin tone, light skin tone}{1589}
\showcaseopenmojicolor{kiss- woman, woman, medium-dark skin tone, medium skin tone}{1590}
\showcaseopenmojicolor{kiss- woman, woman, medium-dark skin tone, medium-light skin tone}{1591}
\showcaseopenmojicolor{kiss- woman, woman, medium-dark skin tone}{1592}
\showcaseopenmojicolor{kiss- woman, woman, medium-light skin tone, dark skin tone}{1593}
\showcaseopenmojicolor{kiss- woman, woman, medium-light skin tone, light skin tone}{1594}
\showcaseopenmojicolor{kiss- woman, woman, medium-light skin tone, medium skin tone}{1595}
\showcaseopenmojicolor{kiss- woman, woman, medium-light skin tone, medium-dark skin tone}{1596}
\showcaseopenmojicolor{kiss- woman, woman, medium-light skin tone}{1597}
\showcaseopenmojicolor{kiss- woman, woman}{1598}
\showcaseopenmojicolor{kiss}{1599}
\showcaseopenmojicolor{kissing cat}{1600}
\showcaseopenmojicolor{kissing face with closed eyes}{1601}
\showcaseopenmojicolor{kissing face with smiling eyes}{1602}
\showcaseopenmojicolor{kissing face}{1603}
\showcaseopenmojicolor{kitchen knife}{1604}
\showcaseopenmojicolor{kite}{1605}
\showcaseopenmojicolor{kiwi fruit}{1606}
\showcaseopenmojicolor{knee pain}{1607}
\showcaseopenmojicolor{knot}{1608}
\showcaseopenmojicolor{koala}{1609}
\showcaseopenmojicolor{kotlin}{1610}
\showcaseopenmojicolor{la rioja flag}{1611}
\showcaseopenmojicolor{lab coat}{1612}
\showcaseopenmojicolor{label}{1613}
\showcaseopenmojicolor{lacrosse}{1614}
\showcaseopenmojicolor{ladder}{1615}
\showcaseopenmojicolor{lady beetle}{1616}
\showcaseopenmojicolor{landslide}{1617}
\showcaseopenmojicolor{laptop}{1618}
\showcaseopenmojicolor{large blue diamond}{1619}
\showcaseopenmojicolor{large intestine}{1620}
\showcaseopenmojicolor{large orange diamond}{1621}
\showcaseopenmojicolor{last quarter moon face}{1622}
\showcaseopenmojicolor{last quarter moon}{1623}
\showcaseopenmojicolor{last track button}{1624}
\showcaseopenmojicolor{latin cross}{1625}
\showcaseopenmojicolor{latte macchiato}{1626}
\showcaseopenmojicolor{lawn mower}{1627}
\showcaseopenmojicolor{leaf fluttering in wind}{1628}
\showcaseopenmojicolor{leafy green}{1629}
\showcaseopenmojicolor{led}{1630}
\showcaseopenmojicolor{ledger}{1631}
\showcaseopenmojicolor{left arrow curving right}{1632}
\showcaseopenmojicolor{left arrow}{1633}
\showcaseopenmojicolor{left luggage}{1634}
\showcaseopenmojicolor{left right black arrow}{1635}
\showcaseopenmojicolor{left speech bubble}{1636}
\showcaseopenmojicolor{left-facing fist- dark skin tone}{1637}
\showcaseopenmojicolor{left-facing fist- light skin tone}{1638}
\showcaseopenmojicolor{left-facing fist- medium skin tone}{1639}
\showcaseopenmojicolor{left-facing fist- medium-dark skin tone}{1640}
\showcaseopenmojicolor{left-facing fist- medium-light skin tone}{1641}
\showcaseopenmojicolor{left-facing fist}{1642}
\showcaseopenmojicolor{left-right arrow}{1643}
\showcaseopenmojicolor{leftwards hand- dark skin tone}{1644}
\showcaseopenmojicolor{leftwards hand- light skin tone}{1645}
\showcaseopenmojicolor{leftwards hand- medium skin tone}{1646}
\showcaseopenmojicolor{leftwards hand- medium-dark skin tone}{1647}
\showcaseopenmojicolor{leftwards hand- medium-light skin tone}{1648}
\showcaseopenmojicolor{leftwards hand}{1649}
\showcaseopenmojicolor{leftwards pushing hand- dark skin tone}{1650}
\showcaseopenmojicolor{leftwards pushing hand- light skin tone}{1651}
\showcaseopenmojicolor{leftwards pushing hand- medium skin tone}{1652}
\showcaseopenmojicolor{leftwards pushing hand- medium-dark skin tone}{1653}
\showcaseopenmojicolor{leftwards pushing hand- medium-light skin tone}{1654}
\showcaseopenmojicolor{leftwards pushing hand}{1655}
\showcaseopenmojicolor{leg- dark skin tone}{1656}
\showcaseopenmojicolor{leg- light skin tone}{1657}
\showcaseopenmojicolor{leg- medium skin tone}{1658}
\showcaseopenmojicolor{leg- medium-dark skin tone}{1659}
\showcaseopenmojicolor{leg- medium-light skin tone}{1660}
\showcaseopenmojicolor{leg}{1661}
\showcaseopenmojicolor{lemon}{1662}
\showcaseopenmojicolor{lentils with spaetzle}{1663}
\showcaseopenmojicolor{leo}{1664}
\showcaseopenmojicolor{leopard}{1665}
\showcaseopenmojicolor{level slider}{1666}
\showcaseopenmojicolor{libra}{1667}
\showcaseopenmojicolor{light blue heart}{1668}
\showcaseopenmojicolor{light bulb}{1669}
\showcaseopenmojicolor{light rail}{1670}
\showcaseopenmojicolor{light skin tone}{1671}
\showcaseopenmojicolor{lighter}{1672}
\showcaseopenmojicolor{lighthouse of alexandria}{1673}
\showcaseopenmojicolor{lime}{1674}
\showcaseopenmojicolor{link}{1675}
\showcaseopenmojicolor{linked paperclips}{1676}
\showcaseopenmojicolor{linkedin}{1677}
\showcaseopenmojicolor{lion}{1678}
\showcaseopenmojicolor{lipstick}{1679}
\showcaseopenmojicolor{litter in bin sign}{1680}
\showcaseopenmojicolor{liver}{1681}
\showcaseopenmojicolor{lizard}{1682}
\showcaseopenmojicolor{llama}{1683}
\showcaseopenmojicolor{lobster}{1684}
\showcaseopenmojicolor{location indicator red}{1685}
\showcaseopenmojicolor{location indicator}{1686}
\showcaseopenmojicolor{locked with key}{1687}
\showcaseopenmojicolor{locked with pen}{1688}
\showcaseopenmojicolor{locked}{1689}
\showcaseopenmojicolor{locomotion}{1690}
\showcaseopenmojicolor{locomotive}{1691}
\showcaseopenmojicolor{lollipop}{1692}
\showcaseopenmojicolor{long drum}{1693}
\showcaseopenmojicolor{lotion bottle}{1694}
\showcaseopenmojicolor{lotus}{1695}
\showcaseopenmojicolor{loudly crying face}{1696}
\showcaseopenmojicolor{loudspeaker}{1697}
\showcaseopenmojicolor{love hotel}{1698}
\showcaseopenmojicolor{love letter}{1699}
\showcaseopenmojicolor{love-you gesture- dark skin tone}{1700}
\showcaseopenmojicolor{love-you gesture- light skin tone}{1701}
\showcaseopenmojicolor{love-you gesture- medium skin tone}{1702}
\showcaseopenmojicolor{love-you gesture- medium-dark skin tone}{1703}
\showcaseopenmojicolor{love-you gesture- medium-light skin tone}{1704}
\showcaseopenmojicolor{love-you gesture}{1705}
\showcaseopenmojicolor{low battery}{1706}
\showcaseopenmojicolor{luggage}{1707}
\showcaseopenmojicolor{lungs}{1708}
\showcaseopenmojicolor{lying face}{1709}
\showcaseopenmojicolor{macaw}{1710}
\showcaseopenmojicolor{madrid autonomous community flag}{1711}
\showcaseopenmojicolor{mage- dark skin tone}{1712}
\showcaseopenmojicolor{mage- light skin tone}{1713}
\showcaseopenmojicolor{mage- medium skin tone}{1714}
\showcaseopenmojicolor{mage- medium-dark skin tone}{1715}
\showcaseopenmojicolor{mage- medium-light skin tone}{1716}
\showcaseopenmojicolor{mage}{1717}
\showcaseopenmojicolor{magic wand}{1718}
\showcaseopenmojicolor{magnet}{1719}
\showcaseopenmojicolor{magnifying glass tilted left}{1720}
\showcaseopenmojicolor{magnifying glass tilted right}{1721}
\showcaseopenmojicolor{mahjong red dragon}{1722}
\showcaseopenmojicolor{male doctor}{1723}
\showcaseopenmojicolor{male nurse}{1724}
\showcaseopenmojicolor{male sign}{1725}
\showcaseopenmojicolor{mammoth}{1726}
\showcaseopenmojicolor{man artist- dark skin tone}{1727}
\showcaseopenmojicolor{man artist- light skin tone}{1728}
\showcaseopenmojicolor{man artist- medium skin tone}{1729}
\showcaseopenmojicolor{man artist- medium-dark skin tone}{1730}
\showcaseopenmojicolor{man artist- medium-light skin tone}{1731}
\showcaseopenmojicolor{man artist}{1732}
\showcaseopenmojicolor{man astronaut- dark skin tone}{1733}
\showcaseopenmojicolor{man astronaut- light skin tone}{1734}
\showcaseopenmojicolor{man astronaut- medium skin tone}{1735}
\showcaseopenmojicolor{man astronaut- medium-dark skin tone}{1736}
\showcaseopenmojicolor{man astronaut- medium-light skin tone}{1737}
\showcaseopenmojicolor{man astronaut}{1738}
\showcaseopenmojicolor{man barista}{1739}
\showcaseopenmojicolor{man biking- dark skin tone}{1740}
\showcaseopenmojicolor{man biking- light skin tone}{1741}
\showcaseopenmojicolor{man biking- medium skin tone}{1742}
\showcaseopenmojicolor{man biking- medium-dark skin tone}{1743}
\showcaseopenmojicolor{man biking- medium-light skin tone}{1744}
\showcaseopenmojicolor{man biking}{1745}
\showcaseopenmojicolor{man bouncing ball- dark skin tone}{1746}
\showcaseopenmojicolor{man bouncing ball- light skin tone}{1747}
\showcaseopenmojicolor{man bouncing ball- medium skin tone}{1748}
\showcaseopenmojicolor{man bouncing ball- medium-dark skin tone}{1749}
\showcaseopenmojicolor{man bouncing ball- medium-light skin tone}{1750}
\showcaseopenmojicolor{man bouncing ball}{1751}
\showcaseopenmojicolor{man bowing- dark skin tone}{1752}
\showcaseopenmojicolor{man bowing- light skin tone}{1753}
\showcaseopenmojicolor{man bowing- medium skin tone}{1754}
\showcaseopenmojicolor{man bowing- medium-dark skin tone}{1755}
\showcaseopenmojicolor{man bowing- medium-light skin tone}{1756}
\showcaseopenmojicolor{man bowing}{1757}
\showcaseopenmojicolor{man cartwheeling- dark skin tone}{1758}
\showcaseopenmojicolor{man cartwheeling- light skin tone}{1759}
\showcaseopenmojicolor{man cartwheeling- medium skin tone}{1760}
\showcaseopenmojicolor{man cartwheeling- medium-dark skin tone}{1761}
\showcaseopenmojicolor{man cartwheeling- medium-light skin tone}{1762}
\showcaseopenmojicolor{man cartwheeling}{1763}
\showcaseopenmojicolor{man climbing- dark skin tone}{1764}
\showcaseopenmojicolor{man climbing- light skin tone}{1765}
\showcaseopenmojicolor{man climbing- medium skin tone}{1766}
\showcaseopenmojicolor{man climbing- medium-dark skin tone}{1767}
\showcaseopenmojicolor{man climbing- medium-light skin tone}{1768}
\showcaseopenmojicolor{man climbing}{1769}
\showcaseopenmojicolor{man construction worker- dark skin tone}{1770}
\showcaseopenmojicolor{man construction worker- light skin tone}{1771}
\showcaseopenmojicolor{man construction worker- medium skin tone}{1772}
\showcaseopenmojicolor{man construction worker- medium-dark skin tone}{1773}
\showcaseopenmojicolor{man construction worker- medium-light skin tone}{1774}
\showcaseopenmojicolor{man construction worker}{1775}
\showcaseopenmojicolor{man cook- dark skin tone}{1776}
\showcaseopenmojicolor{man cook- light skin tone}{1777}
\showcaseopenmojicolor{man cook- medium skin tone}{1778}
\showcaseopenmojicolor{man cook- medium-dark skin tone}{1779}
\showcaseopenmojicolor{man cook- medium-light skin tone}{1780}
\showcaseopenmojicolor{man cook}{1781}
\showcaseopenmojicolor{man dancing- dark skin tone}{1782}
\showcaseopenmojicolor{man dancing- light skin tone}{1783}
\showcaseopenmojicolor{man dancing- medium skin tone}{1784}
\showcaseopenmojicolor{man dancing- medium-dark skin tone}{1785}
\showcaseopenmojicolor{man dancing- medium-light skin tone}{1786}
\showcaseopenmojicolor{man dancing}{1787}
\showcaseopenmojicolor{man detective- dark skin tone}{1788}
\showcaseopenmojicolor{man detective- light skin tone}{1789}
\showcaseopenmojicolor{man detective- medium skin tone}{1790}
\showcaseopenmojicolor{man detective- medium-dark skin tone}{1791}
\showcaseopenmojicolor{man detective- medium-light skin tone}{1792}
\showcaseopenmojicolor{man detective}{1793}
\showcaseopenmojicolor{man elf- dark skin tone}{1794}
\showcaseopenmojicolor{man elf- light skin tone}{1795}
\showcaseopenmojicolor{man elf- medium skin tone}{1796}
\showcaseopenmojicolor{man elf- medium-dark skin tone}{1797}
\showcaseopenmojicolor{man elf- medium-light skin tone}{1798}
\showcaseopenmojicolor{man elf}{1799}
\showcaseopenmojicolor{man facepalming- dark skin tone}{1800}
\showcaseopenmojicolor{man facepalming- light skin tone}{1801}
\showcaseopenmojicolor{man facepalming- medium skin tone}{1802}
\showcaseopenmojicolor{man facepalming- medium-dark skin tone}{1803}
\showcaseopenmojicolor{man facepalming- medium-light skin tone}{1804}
\showcaseopenmojicolor{man facepalming}{1805}
\showcaseopenmojicolor{man factory worker- dark skin tone}{1806}
\showcaseopenmojicolor{man factory worker- light skin tone}{1807}
\showcaseopenmojicolor{man factory worker- medium skin tone}{1808}
\showcaseopenmojicolor{man factory worker- medium-dark skin tone}{1809}
\showcaseopenmojicolor{man factory worker- medium-light skin tone}{1810}
\showcaseopenmojicolor{man factory worker}{1811}
\showcaseopenmojicolor{man fairy- dark skin tone}{1812}
\showcaseopenmojicolor{man fairy- light skin tone}{1813}
\showcaseopenmojicolor{man fairy- medium skin tone}{1814}
\showcaseopenmojicolor{man fairy- medium-dark skin tone}{1815}
\showcaseopenmojicolor{man fairy- medium-light skin tone}{1816}
\showcaseopenmojicolor{man fairy}{1817}
\showcaseopenmojicolor{man farmer- dark skin tone}{1818}
\showcaseopenmojicolor{man farmer- light skin tone}{1819}
\showcaseopenmojicolor{man farmer- medium skin tone}{1820}
\showcaseopenmojicolor{man farmer- medium-dark skin tone}{1821}
\showcaseopenmojicolor{man farmer- medium-light skin tone}{1822}
\showcaseopenmojicolor{man farmer}{1823}
\showcaseopenmojicolor{man feeding baby- dark skin tone}{1824}
\showcaseopenmojicolor{man feeding baby- light skin tone}{1825}
\showcaseopenmojicolor{man feeding baby- medium skin tone}{1826}
\showcaseopenmojicolor{man feeding baby- medium-dark skin tone}{1827}
\showcaseopenmojicolor{man feeding baby- medium-light skin tone}{1828}
\showcaseopenmojicolor{man feeding baby}{1829}
\showcaseopenmojicolor{man firefighter- dark skin tone}{1830}
\showcaseopenmojicolor{man firefighter- light skin tone}{1831}
\showcaseopenmojicolor{man firefighter- medium skin tone}{1832}
\showcaseopenmojicolor{man firefighter- medium-dark skin tone}{1833}
\showcaseopenmojicolor{man firefighter- medium-light skin tone}{1834}
\showcaseopenmojicolor{man firefighter}{1835}
\showcaseopenmojicolor{man frowning- dark skin tone}{1836}
\showcaseopenmojicolor{man frowning- light skin tone}{1837}
\showcaseopenmojicolor{man frowning- medium skin tone}{1838}
\showcaseopenmojicolor{man frowning- medium-dark skin tone}{1839}
\showcaseopenmojicolor{man frowning- medium-light skin tone}{1840}
\showcaseopenmojicolor{man frowning}{1841}
\showcaseopenmojicolor{man genie}{1842}
\showcaseopenmojicolor{man gesturing no- dark skin tone}{1843}
\showcaseopenmojicolor{man gesturing no- light skin tone}{1844}
\showcaseopenmojicolor{man gesturing no- medium skin tone}{1845}
\showcaseopenmojicolor{man gesturing no- medium-dark skin tone}{1846}
\showcaseopenmojicolor{man gesturing no- medium-light skin tone}{1847}
\showcaseopenmojicolor{man gesturing no}{1848}
\showcaseopenmojicolor{man gesturing ok- dark skin tone}{1849}
\showcaseopenmojicolor{man gesturing ok- light skin tone}{1850}
\showcaseopenmojicolor{man gesturing ok- medium skin tone}{1851}
\showcaseopenmojicolor{man gesturing ok- medium-dark skin tone}{1852}
\showcaseopenmojicolor{man gesturing ok- medium-light skin tone}{1853}
\showcaseopenmojicolor{man gesturing ok}{1854}
\showcaseopenmojicolor{man getting haircut- dark skin tone}{1855}
\showcaseopenmojicolor{man getting haircut- light skin tone}{1856}
\showcaseopenmojicolor{man getting haircut- medium skin tone}{1857}
\showcaseopenmojicolor{man getting haircut- medium-dark skin tone}{1858}
\showcaseopenmojicolor{man getting haircut- medium-light skin tone}{1859}
\showcaseopenmojicolor{man getting haircut}{1860}
\showcaseopenmojicolor{man getting massage- dark skin tone}{1861}
\showcaseopenmojicolor{man getting massage- light skin tone}{1862}
\showcaseopenmojicolor{man getting massage- medium skin tone}{1863}
\showcaseopenmojicolor{man getting massage- medium-dark skin tone}{1864}
\showcaseopenmojicolor{man getting massage- medium-light skin tone}{1865}
\showcaseopenmojicolor{man getting massage}{1866}
\showcaseopenmojicolor{man golfing- dark skin tone}{1867}
\showcaseopenmojicolor{man golfing- light skin tone}{1868}
\showcaseopenmojicolor{man golfing- medium skin tone}{1869}
\showcaseopenmojicolor{man golfing- medium-dark skin tone}{1870}
\showcaseopenmojicolor{man golfing- medium-light skin tone}{1871}
\showcaseopenmojicolor{man golfing}{1872}
\showcaseopenmojicolor{man guard- dark skin tone}{1873}
\showcaseopenmojicolor{man guard- light skin tone}{1874}
\showcaseopenmojicolor{man guard- medium skin tone}{1875}
\showcaseopenmojicolor{man guard- medium-dark skin tone}{1876}
\showcaseopenmojicolor{man guard- medium-light skin tone}{1877}
\showcaseopenmojicolor{man guard}{1878}
\showcaseopenmojicolor{man health worker- dark skin tone}{1879}
\showcaseopenmojicolor{man health worker- light skin tone}{1880}
\showcaseopenmojicolor{man health worker- medium skin tone}{1881}
\showcaseopenmojicolor{man health worker- medium-dark skin tone}{1882}
\showcaseopenmojicolor{man health worker- medium-light skin tone}{1883}
\showcaseopenmojicolor{man health worker}{1884}
\showcaseopenmojicolor{man in lotus position- dark skin tone}{1885}
\showcaseopenmojicolor{man in lotus position- light skin tone}{1886}
\showcaseopenmojicolor{man in lotus position- medium skin tone}{1887}
\showcaseopenmojicolor{man in lotus position- medium-dark skin tone}{1888}
\showcaseopenmojicolor{man in lotus position- medium-light skin tone}{1889}
\showcaseopenmojicolor{man in lotus position}{1890}
\showcaseopenmojicolor{man in manual wheelchair facing right}{1891}
\showcaseopenmojicolor{man in manual wheelchair- dark skin tone}{1892}
\showcaseopenmojicolor{man in manual wheelchair- light skin tone}{1893}
\showcaseopenmojicolor{man in manual wheelchair- medium skin tone}{1894}
\showcaseopenmojicolor{man in manual wheelchair- medium-dark skin tone}{1895}
\showcaseopenmojicolor{man in manual wheelchair- medium-light skin tone}{1896}
\showcaseopenmojicolor{man in manual wheelchair}{1897}
\showcaseopenmojicolor{man in motorized wheelchair facing right}{1898}
\showcaseopenmojicolor{man in motorized wheelchair- dark skin tone}{1899}
\showcaseopenmojicolor{man in motorized wheelchair- light skin tone}{1900}
\showcaseopenmojicolor{man in motorized wheelchair- medium skin tone}{1901}
\showcaseopenmojicolor{man in motorized wheelchair- medium-dark skin tone}{1902}
\showcaseopenmojicolor{man in motorized wheelchair- medium-light skin tone}{1903}
\showcaseopenmojicolor{man in motorized wheelchair}{1904}
\showcaseopenmojicolor{man in steamy room- dark skin tone}{1905}
\showcaseopenmojicolor{man in steamy room- light skin tone}{1906}
\showcaseopenmojicolor{man in steamy room- medium skin tone}{1907}
\showcaseopenmojicolor{man in steamy room- medium-dark skin tone}{1908}
\showcaseopenmojicolor{man in steamy room- medium-light skin tone}{1909}
\showcaseopenmojicolor{man in steamy room}{1910}
\showcaseopenmojicolor{man in tuxedo- dark skin tone}{1911}
\showcaseopenmojicolor{man in tuxedo- light skin tone}{1912}
\showcaseopenmojicolor{man in tuxedo- medium skin tone}{1913}
\showcaseopenmojicolor{man in tuxedo- medium-dark skin tone}{1914}
\showcaseopenmojicolor{man in tuxedo- medium-light skin tone}{1915}
\showcaseopenmojicolor{man in tuxedo}{1916}
\showcaseopenmojicolor{man judge- dark skin tone}{1917}
\showcaseopenmojicolor{man judge- light skin tone}{1918}
\showcaseopenmojicolor{man judge- medium skin tone}{1919}
\showcaseopenmojicolor{man judge- medium-dark skin tone}{1920}
\showcaseopenmojicolor{man judge- medium-light skin tone}{1921}
\showcaseopenmojicolor{man judge}{1922}
\showcaseopenmojicolor{man juggling- dark skin tone}{1923}
\showcaseopenmojicolor{man juggling- light skin tone}{1924}
\showcaseopenmojicolor{man juggling- medium skin tone}{1925}
\showcaseopenmojicolor{man juggling- medium-dark skin tone}{1926}
\showcaseopenmojicolor{man juggling- medium-light skin tone}{1927}
\showcaseopenmojicolor{man juggling}{1928}
\showcaseopenmojicolor{man kneeling facing right}{1929}
\showcaseopenmojicolor{man kneeling- dark skin tone}{1930}
\showcaseopenmojicolor{man kneeling- light skin tone}{1931}
\showcaseopenmojicolor{man kneeling- medium skin tone}{1932}
\showcaseopenmojicolor{man kneeling- medium-dark skin tone}{1933}
\showcaseopenmojicolor{man kneeling- medium-light skin tone}{1934}
\showcaseopenmojicolor{man kneeling}{1935}
\showcaseopenmojicolor{man lifting weights- dark skin tone}{1936}
\showcaseopenmojicolor{man lifting weights- light skin tone}{1937}
\showcaseopenmojicolor{man lifting weights- medium skin tone}{1938}
\showcaseopenmojicolor{man lifting weights- medium-dark skin tone}{1939}
\showcaseopenmojicolor{man lifting weights- medium-light skin tone}{1940}
\showcaseopenmojicolor{man lifting weights}{1941}
\showcaseopenmojicolor{man mage- dark skin tone}{1942}
\showcaseopenmojicolor{man mage- light skin tone}{1943}
\showcaseopenmojicolor{man mage- medium skin tone}{1944}
\showcaseopenmojicolor{man mage- medium-dark skin tone}{1945}
\showcaseopenmojicolor{man mage- medium-light skin tone}{1946}
\showcaseopenmojicolor{man mage}{1947}
\showcaseopenmojicolor{man mechanic- dark skin tone}{1948}
\showcaseopenmojicolor{man mechanic- light skin tone}{1949}
\showcaseopenmojicolor{man mechanic- medium skin tone}{1950}
\showcaseopenmojicolor{man mechanic- medium-dark skin tone}{1951}
\showcaseopenmojicolor{man mechanic- medium-light skin tone}{1952}
\showcaseopenmojicolor{man mechanic}{1953}
\showcaseopenmojicolor{man mountain biking- dark skin tone}{1954}
\showcaseopenmojicolor{man mountain biking- light skin tone}{1955}
\showcaseopenmojicolor{man mountain biking- medium skin tone}{1956}
\showcaseopenmojicolor{man mountain biking- medium-dark skin tone}{1957}
\showcaseopenmojicolor{man mountain biking- medium-light skin tone}{1958}
\showcaseopenmojicolor{man mountain biking}{1959}
\showcaseopenmojicolor{man office worker- dark skin tone}{1960}
\showcaseopenmojicolor{man office worker- light skin tone}{1961}
\showcaseopenmojicolor{man office worker- medium skin tone}{1962}
\showcaseopenmojicolor{man office worker- medium-dark skin tone}{1963}
\showcaseopenmojicolor{man office worker- medium-light skin tone}{1964}
\showcaseopenmojicolor{man office worker}{1965}
\showcaseopenmojicolor{man pilot- dark skin tone}{1966}
\showcaseopenmojicolor{man pilot- light skin tone}{1967}
\showcaseopenmojicolor{man pilot- medium skin tone}{1968}
\showcaseopenmojicolor{man pilot- medium-dark skin tone}{1969}
\showcaseopenmojicolor{man pilot- medium-light skin tone}{1970}
\showcaseopenmojicolor{man pilot}{1971}
\showcaseopenmojicolor{man playing handball- dark skin tone}{1972}
\showcaseopenmojicolor{man playing handball- light skin tone}{1973}
\showcaseopenmojicolor{man playing handball- medium skin tone}{1974}
\showcaseopenmojicolor{man playing handball- medium-dark skin tone}{1975}
\showcaseopenmojicolor{man playing handball- medium-light skin tone}{1976}
\showcaseopenmojicolor{man playing handball}{1977}
\showcaseopenmojicolor{man playing water polo- dark skin tone}{1978}
\showcaseopenmojicolor{man playing water polo- light skin tone}{1979}
\showcaseopenmojicolor{man playing water polo- medium skin tone}{1980}
\showcaseopenmojicolor{man playing water polo- medium-dark skin tone}{1981}
\showcaseopenmojicolor{man playing water polo- medium-light skin tone}{1982}
\showcaseopenmojicolor{man playing water polo}{1983}
\showcaseopenmojicolor{man police officer- dark skin tone}{1984}
\showcaseopenmojicolor{man police officer- light skin tone}{1985}
\showcaseopenmojicolor{man police officer- medium skin tone}{1986}
\showcaseopenmojicolor{man police officer- medium-dark skin tone}{1987}
\showcaseopenmojicolor{man police officer- medium-light skin tone}{1988}
\showcaseopenmojicolor{man police officer}{1989}
\showcaseopenmojicolor{man pouting- dark skin tone}{1990}
\showcaseopenmojicolor{man pouting- light skin tone}{1991}
\showcaseopenmojicolor{man pouting- medium skin tone}{1992}
\showcaseopenmojicolor{man pouting- medium-dark skin tone}{1993}
\showcaseopenmojicolor{man pouting- medium-light skin tone}{1994}
\showcaseopenmojicolor{man pouting}{1995}
\showcaseopenmojicolor{man raising hand- dark skin tone}{1996}
\showcaseopenmojicolor{man raising hand- light skin tone}{1997}
\showcaseopenmojicolor{man raising hand- medium skin tone}{1998}
\showcaseopenmojicolor{man raising hand- medium-dark skin tone}{1999}
\showcaseopenmojicolor{man raising hand- medium-light skin tone}{2000}
\showcaseopenmojicolor{man raising hand}{2001}
\showcaseopenmojicolor{man rowing boat- dark skin tone}{2002}
\showcaseopenmojicolor{man rowing boat- light skin tone}{2003}
\showcaseopenmojicolor{man rowing boat- medium skin tone}{2004}
\showcaseopenmojicolor{man rowing boat- medium-dark skin tone}{2005}
\showcaseopenmojicolor{man rowing boat- medium-light skin tone}{2006}
\showcaseopenmojicolor{man rowing boat}{2007}
\showcaseopenmojicolor{man running facing right}{2008}
\showcaseopenmojicolor{man running- dark skin tone}{2009}
\showcaseopenmojicolor{man running- light skin tone}{2010}
\showcaseopenmojicolor{man running- medium skin tone}{2011}
\showcaseopenmojicolor{man running- medium-dark skin tone}{2012}
\showcaseopenmojicolor{man running- medium-light skin tone}{2013}
\showcaseopenmojicolor{man running}{2014}
\showcaseopenmojicolor{man scientist- dark skin tone}{2015}
\showcaseopenmojicolor{man scientist- light skin tone}{2016}
\showcaseopenmojicolor{man scientist- medium skin tone}{2017}
\showcaseopenmojicolor{man scientist- medium-dark skin tone}{2018}
\showcaseopenmojicolor{man scientist- medium-light skin tone}{2019}
\showcaseopenmojicolor{man scientist}{2020}
\showcaseopenmojicolor{man shrugging- dark skin tone}{2021}
\showcaseopenmojicolor{man shrugging- light skin tone}{2022}
\showcaseopenmojicolor{man shrugging- medium skin tone}{2023}
\showcaseopenmojicolor{man shrugging- medium-dark skin tone}{2024}
\showcaseopenmojicolor{man shrugging- medium-light skin tone}{2025}
\showcaseopenmojicolor{man shrugging}{2026}
\showcaseopenmojicolor{man singer- dark skin tone}{2027}
\showcaseopenmojicolor{man singer- light skin tone}{2028}
\showcaseopenmojicolor{man singer- medium skin tone}{2029}
\showcaseopenmojicolor{man singer- medium-dark skin tone}{2030}
\showcaseopenmojicolor{man singer- medium-light skin tone}{2031}
\showcaseopenmojicolor{man singer}{2032}
\showcaseopenmojicolor{man sneezing into elbow}{2033}
\showcaseopenmojicolor{man standing- dark skin tone}{2034}
\showcaseopenmojicolor{man standing- light skin tone}{2035}
\showcaseopenmojicolor{man standing- medium skin tone}{2036}
\showcaseopenmojicolor{man standing- medium-dark skin tone}{2037}
\showcaseopenmojicolor{man standing- medium-light skin tone}{2038}
\showcaseopenmojicolor{man standing}{2039}
\showcaseopenmojicolor{man student- dark skin tone}{2040}
\showcaseopenmojicolor{man student- light skin tone}{2041}
\showcaseopenmojicolor{man student- medium skin tone}{2042}
\showcaseopenmojicolor{man student- medium-dark skin tone}{2043}
\showcaseopenmojicolor{man student- medium-light skin tone}{2044}
\showcaseopenmojicolor{man student}{2045}
\showcaseopenmojicolor{man superhero- dark skin tone}{2046}
\showcaseopenmojicolor{man superhero- light skin tone}{2047}
\showcaseopenmojicolor{man superhero- medium skin tone}{2048}
\showcaseopenmojicolor{man superhero- medium-dark skin tone}{2049}
\showcaseopenmojicolor{man superhero- medium-light skin tone}{2050}
\showcaseopenmojicolor{man superhero}{2051}
\showcaseopenmojicolor{man supervillain- dark skin tone}{2052}
\showcaseopenmojicolor{man supervillain- light skin tone}{2053}
\showcaseopenmojicolor{man supervillain- medium skin tone}{2054}
\showcaseopenmojicolor{man supervillain- medium-dark skin tone}{2055}
\showcaseopenmojicolor{man supervillain- medium-light skin tone}{2056}
\showcaseopenmojicolor{man supervillain}{2057}
\showcaseopenmojicolor{man surfing- dark skin tone}{2058}
\showcaseopenmojicolor{man surfing- light skin tone}{2059}
\showcaseopenmojicolor{man surfing- medium skin tone}{2060}
\showcaseopenmojicolor{man surfing- medium-dark skin tone}{2061}
\showcaseopenmojicolor{man surfing- medium-light skin tone}{2062}
\showcaseopenmojicolor{man surfing}{2063}
\showcaseopenmojicolor{man swimming- dark skin tone}{2064}
\showcaseopenmojicolor{man swimming- light skin tone}{2065}
\showcaseopenmojicolor{man swimming- medium skin tone}{2066}
\showcaseopenmojicolor{man swimming- medium-dark skin tone}{2067}
\showcaseopenmojicolor{man swimming- medium-light skin tone}{2068}
\showcaseopenmojicolor{man swimming}{2069}
\showcaseopenmojicolor{man teacher- dark skin tone}{2070}
\showcaseopenmojicolor{man teacher- light skin tone}{2071}
\showcaseopenmojicolor{man teacher- medium skin tone}{2072}
\showcaseopenmojicolor{man teacher- medium-dark skin tone}{2073}
\showcaseopenmojicolor{man teacher- medium-light skin tone}{2074}
\showcaseopenmojicolor{man teacher}{2075}
\showcaseopenmojicolor{man technologist- dark skin tone}{2076}
\showcaseopenmojicolor{man technologist- light skin tone}{2077}
\showcaseopenmojicolor{man technologist- medium skin tone}{2078}
\showcaseopenmojicolor{man technologist- medium-dark skin tone}{2079}
\showcaseopenmojicolor{man technologist- medium-light skin tone}{2080}
\showcaseopenmojicolor{man technologist}{2081}
\showcaseopenmojicolor{man tipping hand- dark skin tone}{2082}
\showcaseopenmojicolor{man tipping hand- light skin tone}{2083}
\showcaseopenmojicolor{man tipping hand- medium skin tone}{2084}
\showcaseopenmojicolor{man tipping hand- medium-dark skin tone}{2085}
\showcaseopenmojicolor{man tipping hand- medium-light skin tone}{2086}
\showcaseopenmojicolor{man tipping hand}{2087}
\showcaseopenmojicolor{man vampire- dark skin tone}{2088}
\showcaseopenmojicolor{man vampire- light skin tone}{2089}
\showcaseopenmojicolor{man vampire- medium skin tone}{2090}
\showcaseopenmojicolor{man vampire- medium-dark skin tone}{2091}
\showcaseopenmojicolor{man vampire- medium-light skin tone}{2092}
\showcaseopenmojicolor{man vampire}{2093}
\showcaseopenmojicolor{man walking facing right}{2094}
\showcaseopenmojicolor{man walking- dark skin tone}{2095}
\showcaseopenmojicolor{man walking- light skin tone}{2096}
\showcaseopenmojicolor{man walking- medium skin tone}{2097}
\showcaseopenmojicolor{man walking- medium-dark skin tone}{2098}
\showcaseopenmojicolor{man walking- medium-light skin tone}{2099}
\showcaseopenmojicolor{man walking}{2100}
\showcaseopenmojicolor{man wearing turban- dark skin tone}{2101}
\showcaseopenmojicolor{man wearing turban- light skin tone}{2102}
\showcaseopenmojicolor{man wearing turban- medium skin tone}{2103}
\showcaseopenmojicolor{man wearing turban- medium-dark skin tone}{2104}
\showcaseopenmojicolor{man wearing turban- medium-light skin tone}{2105}
\showcaseopenmojicolor{man wearing turban}{2106}
\showcaseopenmojicolor{man with medical mask}{2107}
\showcaseopenmojicolor{man with veil- dark skin tone}{2108}
\showcaseopenmojicolor{man with veil- light skin tone}{2109}
\showcaseopenmojicolor{man with veil- medium skin tone}{2110}
\showcaseopenmojicolor{man with veil- medium-dark skin tone}{2111}
\showcaseopenmojicolor{man with veil- medium-light skin tone}{2112}
\showcaseopenmojicolor{man with veil}{2113}
\showcaseopenmojicolor{man with white cane facing right}{2114}
\showcaseopenmojicolor{man with white cane- dark skin tone}{2115}
\showcaseopenmojicolor{man with white cane- light skin tone}{2116}
\showcaseopenmojicolor{man with white cane- medium skin tone}{2117}
\showcaseopenmojicolor{man with white cane- medium-dark skin tone}{2118}
\showcaseopenmojicolor{man with white cane- medium-light skin tone}{2119}
\showcaseopenmojicolor{man with white cane}{2120}
\showcaseopenmojicolor{man zombie}{2121}
\showcaseopenmojicolor{man- bald}{2122}
\showcaseopenmojicolor{man- beard}{2123}
\showcaseopenmojicolor{man- blond hair}{2124}
\showcaseopenmojicolor{man- curly hair}{2125}
\showcaseopenmojicolor{man- dark skin tone, bald}{2126}
\showcaseopenmojicolor{man- dark skin tone, beard}{2127}
\showcaseopenmojicolor{man- dark skin tone, blond hair}{2128}
\showcaseopenmojicolor{man- dark skin tone, curly hair}{2129}
\showcaseopenmojicolor{man- dark skin tone, red hair}{2130}
\showcaseopenmojicolor{man- dark skin tone, white hair}{2131}
\showcaseopenmojicolor{man- dark skin tone}{2132}
\showcaseopenmojicolor{man- light skin tone, bald}{2133}
\showcaseopenmojicolor{man- light skin tone, beard}{2134}
\showcaseopenmojicolor{man- light skin tone, blond hair}{2135}
\showcaseopenmojicolor{man- light skin tone, curly hair}{2136}
\showcaseopenmojicolor{man- light skin tone, red hair}{2137}
\showcaseopenmojicolor{man- light skin tone, white hair}{2138}
\showcaseopenmojicolor{man- light skin tone}{2139}
\showcaseopenmojicolor{man- medium skin tone, bald}{2140}
\showcaseopenmojicolor{man- medium skin tone, beard}{2141}
\showcaseopenmojicolor{man- medium skin tone, blond hair}{2142}
\showcaseopenmojicolor{man- medium skin tone, curly hair}{2143}
\showcaseopenmojicolor{man- medium skin tone, red hair}{2144}
\showcaseopenmojicolor{man- medium skin tone, white hair}{2145}
\showcaseopenmojicolor{man- medium skin tone}{2146}
\showcaseopenmojicolor{man- medium-dark skin tone, bald}{2147}
\showcaseopenmojicolor{man- medium-dark skin tone, beard}{2148}
\showcaseopenmojicolor{man- medium-dark skin tone, blond hair}{2149}
\showcaseopenmojicolor{man- medium-dark skin tone, curly hair}{2150}
\showcaseopenmojicolor{man- medium-dark skin tone, red hair}{2151}
\showcaseopenmojicolor{man- medium-dark skin tone, white hair}{2152}
\showcaseopenmojicolor{man- medium-dark skin tone}{2153}
\showcaseopenmojicolor{man- medium-light skin tone, bald}{2154}
\showcaseopenmojicolor{man- medium-light skin tone, beard}{2155}
\showcaseopenmojicolor{man- medium-light skin tone, blond hair}{2156}
\showcaseopenmojicolor{man- medium-light skin tone, curly hair}{2157}
\showcaseopenmojicolor{man- medium-light skin tone, red hair}{2158}
\showcaseopenmojicolor{man- medium-light skin tone, white hair}{2159}
\showcaseopenmojicolor{man- medium-light skin tone}{2160}
\showcaseopenmojicolor{man- red hair}{2161}
\showcaseopenmojicolor{man- white hair}{2162}
\showcaseopenmojicolor{man}{2163}
\showcaseopenmojicolor{mango}{2164}
\showcaseopenmojicolor{mantelpiece clock}{2165}
\showcaseopenmojicolor{manual wheelchair}{2166}
\showcaseopenmojicolor{man s shoe}{2167}
\showcaseopenmojicolor{map of japan}{2168}
\showcaseopenmojicolor{maple leaf}{2169}
\showcaseopenmojicolor{maracas}{2170}
\showcaseopenmojicolor{mark}{2171}
\showcaseopenmojicolor{markdown}{2172}
\showcaseopenmojicolor{martial arts uniform}{2173}
\showcaseopenmojicolor{mastodon}{2174}
\showcaseopenmojicolor{mate}{2175}
\showcaseopenmojicolor{maultasche}{2176}
\showcaseopenmojicolor{mausoleum at halicarnassus}{2177}
\showcaseopenmojicolor{meat consumption}{2178}
\showcaseopenmojicolor{meat on bone}{2179}
\showcaseopenmojicolor{mechanic- dark skin tone}{2180}
\showcaseopenmojicolor{mechanic- light skin tone}{2181}
\showcaseopenmojicolor{mechanic- medium skin tone}{2182}
\showcaseopenmojicolor{mechanic- medium-dark skin tone}{2183}
\showcaseopenmojicolor{mechanic- medium-light skin tone}{2184}
\showcaseopenmojicolor{mechanic}{2185}
\showcaseopenmojicolor{mechanical arm}{2186}
\showcaseopenmojicolor{mechanical leg}{2187}
\showcaseopenmojicolor{medical gloves}{2188}
\showcaseopenmojicolor{medical symbol}{2189}
\showcaseopenmojicolor{medication}{2190}
\showcaseopenmojicolor{medium skin tone}{2191}
\showcaseopenmojicolor{medium-dark skin tone}{2192}
\showcaseopenmojicolor{medium-light skin tone}{2193}
\showcaseopenmojicolor{megaphone}{2194}
\showcaseopenmojicolor{melilla flag}{2195}
\showcaseopenmojicolor{melon}{2196}
\showcaseopenmojicolor{melting face}{2197}
\showcaseopenmojicolor{memo}{2198}
\showcaseopenmojicolor{men holding hands- dark skin tone, light skin tone}{2199}
\showcaseopenmojicolor{men holding hands- dark skin tone, medium skin tone}{2200}
\showcaseopenmojicolor{men holding hands- dark skin tone, medium-dark skin tone}{2201}
\showcaseopenmojicolor{men holding hands- dark skin tone, medium-light skin tone}{2202}
\showcaseopenmojicolor{men holding hands- dark skin tone}{2203}
\showcaseopenmojicolor{men holding hands- light skin tone, dark skin tone}{2204}
\showcaseopenmojicolor{men holding hands- light skin tone, medium skin tone}{2205}
\showcaseopenmojicolor{men holding hands- light skin tone, medium-dark skin tone}{2206}
\showcaseopenmojicolor{men holding hands- light skin tone, medium-light skin tone}{2207}
\showcaseopenmojicolor{men holding hands- light skin tone}{2208}
\showcaseopenmojicolor{men holding hands- medium skin tone, dark skin tone}{2209}
\showcaseopenmojicolor{men holding hands- medium skin tone, light skin tone}{2210}
\showcaseopenmojicolor{men holding hands- medium skin tone, medium-dark skin tone}{2211}
\showcaseopenmojicolor{men holding hands- medium skin tone, medium-light skin tone}{2212}
\showcaseopenmojicolor{men holding hands- medium skin tone}{2213}
\showcaseopenmojicolor{men holding hands- medium-dark skin tone, dark skin tone}{2214}
\showcaseopenmojicolor{men holding hands- medium-dark skin tone, light skin tone}{2215}
\showcaseopenmojicolor{men holding hands- medium-dark skin tone, medium skin tone}{2216}
\showcaseopenmojicolor{men holding hands- medium-dark skin tone, medium-light skin tone}{2217}
\showcaseopenmojicolor{men holding hands- medium-dark skin tone}{2218}
\showcaseopenmojicolor{men holding hands- medium-light skin tone, dark skin tone}{2219}
\showcaseopenmojicolor{men holding hands- medium-light skin tone, light skin tone}{2220}
\showcaseopenmojicolor{men holding hands- medium-light skin tone, medium skin tone}{2221}
\showcaseopenmojicolor{men holding hands- medium-light skin tone, medium-dark skin tone}{2222}
\showcaseopenmojicolor{men holding hands- medium-light skin tone}{2223}
\showcaseopenmojicolor{men holding hands}{2224}
\showcaseopenmojicolor{men with bunny ears}{2225}
\showcaseopenmojicolor{men wrestling}{2226}
\showcaseopenmojicolor{mending heart}{2227}
\showcaseopenmojicolor{menorah}{2228}
\showcaseopenmojicolor{men s room}{2229}
\showcaseopenmojicolor{mermaid- dark skin tone}{2230}
\showcaseopenmojicolor{mermaid- light skin tone}{2231}
\showcaseopenmojicolor{mermaid- medium skin tone}{2232}
\showcaseopenmojicolor{mermaid- medium-dark skin tone}{2233}
\showcaseopenmojicolor{mermaid- medium-light skin tone}{2234}
\showcaseopenmojicolor{mermaid}{2235}
\showcaseopenmojicolor{merman- dark skin tone}{2236}
\showcaseopenmojicolor{merman- light skin tone}{2237}
\showcaseopenmojicolor{merman- medium skin tone}{2238}
\showcaseopenmojicolor{merman- medium-dark skin tone}{2239}
\showcaseopenmojicolor{merman- medium-light skin tone}{2240}
\showcaseopenmojicolor{merman}{2241}
\showcaseopenmojicolor{merperson- dark skin tone}{2242}
\showcaseopenmojicolor{merperson- light skin tone}{2243}
\showcaseopenmojicolor{merperson- medium skin tone}{2244}
\showcaseopenmojicolor{merperson- medium-dark skin tone}{2245}
\showcaseopenmojicolor{merperson- medium-light skin tone}{2246}
\showcaseopenmojicolor{merperson}{2247}
\showcaseopenmojicolor{metro}{2248}
\showcaseopenmojicolor{microbe}{2249}
\showcaseopenmojicolor{microphone}{2250}
\showcaseopenmojicolor{microscope}{2251}
\showcaseopenmojicolor{middle finger- dark skin tone}{2252}
\showcaseopenmojicolor{middle finger- light skin tone}{2253}
\showcaseopenmojicolor{middle finger- medium skin tone}{2254}
\showcaseopenmojicolor{middle finger- medium-dark skin tone}{2255}
\showcaseopenmojicolor{middle finger- medium-light skin tone}{2256}
\showcaseopenmojicolor{middle finger}{2257}
\showcaseopenmojicolor{military helmet}{2258}
\showcaseopenmojicolor{military medal}{2259}
\showcaseopenmojicolor{milk jug}{2260}
\showcaseopenmojicolor{milky way}{2261}
\showcaseopenmojicolor{minibus}{2262}
\showcaseopenmojicolor{minus}{2263}
\showcaseopenmojicolor{mirror ball}{2264}
\showcaseopenmojicolor{mirror}{2265}
\showcaseopenmojicolor{moai}{2266}
\showcaseopenmojicolor{mobile info}{2267}
\showcaseopenmojicolor{mobile message}{2268}
\showcaseopenmojicolor{mobile phone off}{2269}
\showcaseopenmojicolor{mobile phone with arrow}{2270}
\showcaseopenmojicolor{mobile phone}{2271}
\showcaseopenmojicolor{moka pot}{2272}
\showcaseopenmojicolor{money bag}{2273}
\showcaseopenmojicolor{money with wings}{2274}
\showcaseopenmojicolor{money-mouth face}{2275}
\showcaseopenmojicolor{monkey face}{2276}
\showcaseopenmojicolor{monkey}{2277}
\showcaseopenmojicolor{monorail}{2278}
\showcaseopenmojicolor{moon cake}{2279}
\showcaseopenmojicolor{moon viewing ceremony}{2280}
\showcaseopenmojicolor{moose}{2281}
\showcaseopenmojicolor{more information}{2282}
\showcaseopenmojicolor{mosque}{2283}
\showcaseopenmojicolor{mosquito}{2284}
\showcaseopenmojicolor{motor boat}{2285}
\showcaseopenmojicolor{motor scooter}{2286}
\showcaseopenmojicolor{motor}{2287}
\showcaseopenmojicolor{motorbike helmet}{2288}
\showcaseopenmojicolor{motorcycle}{2289}
\showcaseopenmojicolor{motorized wheelchair}{2290}
\showcaseopenmojicolor{motorway}{2291}
\showcaseopenmojicolor{mount fuji}{2292}
\showcaseopenmojicolor{mountain cableway}{2293}
\showcaseopenmojicolor{mountain railway}{2294}
\showcaseopenmojicolor{mountain}{2295}
\showcaseopenmojicolor{mouse face}{2296}
\showcaseopenmojicolor{mouse trap}{2297}
\showcaseopenmojicolor{mouse}{2298}
\showcaseopenmojicolor{mouth}{2299}
\showcaseopenmojicolor{move}{2300}
\showcaseopenmojicolor{movie camera}{2301}
\showcaseopenmojicolor{mrs}{2302}
\showcaseopenmojicolor{multiply}{2303}
\showcaseopenmojicolor{murcia flag}{2304}
\showcaseopenmojicolor{mushroom}{2305}
\showcaseopenmojicolor{musical keyboard}{2306}
\showcaseopenmojicolor{musical note}{2307}
\showcaseopenmojicolor{musical notes}{2308}
\showcaseopenmojicolor{musical score}{2309}
\showcaseopenmojicolor{musicbrainz}{2310}
\showcaseopenmojicolor{muted speaker}{2311}
\showcaseopenmojicolor{mx claus- dark skin tone}{2312}
\showcaseopenmojicolor{mx claus- light skin tone}{2313}
\showcaseopenmojicolor{mx claus- medium skin tone}{2314}
\showcaseopenmojicolor{mx claus- medium-dark skin tone}{2315}
\showcaseopenmojicolor{mx claus- medium-light skin tone}{2316}
\showcaseopenmojicolor{mx claus}{2317}
\showcaseopenmojicolor{nail and gear flag}{2318}
\showcaseopenmojicolor{nail polish- dark skin tone}{2319}
\showcaseopenmojicolor{nail polish- light skin tone}{2320}
\showcaseopenmojicolor{nail polish- medium skin tone}{2321}
\showcaseopenmojicolor{nail polish- medium-dark skin tone}{2322}
\showcaseopenmojicolor{nail polish- medium-light skin tone}{2323}
\showcaseopenmojicolor{nail polish}{2324}
\showcaseopenmojicolor{name badge}{2325}
\showcaseopenmojicolor{narwhal}{2326}
\showcaseopenmojicolor{national park}{2327}
\showcaseopenmojicolor{nauseated face}{2328}
\showcaseopenmojicolor{navarra chartered community}{2329}
\showcaseopenmojicolor{nazar amulet}{2330}
\showcaseopenmojicolor{necktie}{2331}
\showcaseopenmojicolor{nerd face}{2332}
\showcaseopenmojicolor{nest with eggs}{2333}
\showcaseopenmojicolor{nesting dolls}{2334}
\showcaseopenmojicolor{netscape navigator}{2335}
\showcaseopenmojicolor{neutral face}{2336}
\showcaseopenmojicolor{new button}{2337}
\showcaseopenmojicolor{new moon face}{2338}
\showcaseopenmojicolor{new moon}{2339}
\showcaseopenmojicolor{newspaper}{2340}
\showcaseopenmojicolor{next track button}{2341}
\showcaseopenmojicolor{ng button}{2342}
\showcaseopenmojicolor{night with stars}{2343}
\showcaseopenmojicolor{nine o clock}{2344}
\showcaseopenmojicolor{nine-thirty}{2345}
\showcaseopenmojicolor{ninja- dark skin tone}{2346}
\showcaseopenmojicolor{ninja- light skin tone}{2347}
\showcaseopenmojicolor{ninja- medium skin tone}{2348}
\showcaseopenmojicolor{ninja- medium-dark skin tone}{2349}
\showcaseopenmojicolor{ninja- medium-light skin tone}{2350}
\showcaseopenmojicolor{ninja}{2351}
\showcaseopenmojicolor{no bicycles}{2352}
\showcaseopenmojicolor{no entry}{2353}
\showcaseopenmojicolor{no handshaking}{2354}
\showcaseopenmojicolor{no littering}{2355}
\showcaseopenmojicolor{no mobile phones}{2356}
\showcaseopenmojicolor{no one under eighteen}{2357}
\showcaseopenmojicolor{no pedestrians}{2358}
\showcaseopenmojicolor{no smoking}{2359}
\showcaseopenmojicolor{no stencil}{2360}
\showcaseopenmojicolor{non-potable water}{2361}
\showcaseopenmojicolor{north}{2362}
\showcaseopenmojicolor{nose- dark skin tone}{2363}
\showcaseopenmojicolor{nose- light skin tone}{2364}
\showcaseopenmojicolor{nose- medium skin tone}{2365}
\showcaseopenmojicolor{nose- medium-dark skin tone}{2366}
\showcaseopenmojicolor{nose- medium-light skin tone}{2367}
\showcaseopenmojicolor{nose}{2368}
\showcaseopenmojicolor{notebook with decorative cover}{2369}
\showcaseopenmojicolor{notebook}{2370}
\showcaseopenmojicolor{nuclear power plant ruin}{2371}
\showcaseopenmojicolor{nuclear power plant}{2372}
\showcaseopenmojicolor{nuclear protection}{2373}
\showcaseopenmojicolor{nuclear worker man}{2374}
\showcaseopenmojicolor{nuclear worker woman}{2375}
\showcaseopenmojicolor{nut and bolt}{2376}
\showcaseopenmojicolor{o button (blood type)}{2377}
\showcaseopenmojicolor{octopus}{2378}
\showcaseopenmojicolor{oden}{2379}
\showcaseopenmojicolor{office building}{2380}
\showcaseopenmojicolor{office worker- dark skin tone}{2381}
\showcaseopenmojicolor{office worker- light skin tone}{2382}
\showcaseopenmojicolor{office worker- medium skin tone}{2383}
\showcaseopenmojicolor{office worker- medium-dark skin tone}{2384}
\showcaseopenmojicolor{office worker- medium-light skin tone}{2385}
\showcaseopenmojicolor{office worker}{2386}
\showcaseopenmojicolor{ogre}{2387}
\showcaseopenmojicolor{oil drum}{2388}
\showcaseopenmojicolor{oil spill}{2389}
\showcaseopenmojicolor{ok button}{2390}
\showcaseopenmojicolor{ok hand- dark skin tone}{2391}
\showcaseopenmojicolor{ok hand- light skin tone}{2392}
\showcaseopenmojicolor{ok hand- medium skin tone}{2393}
\showcaseopenmojicolor{ok hand- medium-dark skin tone}{2394}
\showcaseopenmojicolor{ok hand- medium-light skin tone}{2395}
\showcaseopenmojicolor{ok hand}{2396}
\showcaseopenmojicolor{ok stencil}{2397}
\showcaseopenmojicolor{old key}{2398}
\showcaseopenmojicolor{old man- dark skin tone}{2399}
\showcaseopenmojicolor{old man- light skin tone}{2400}
\showcaseopenmojicolor{old man- medium skin tone}{2401}
\showcaseopenmojicolor{old man- medium-dark skin tone}{2402}
\showcaseopenmojicolor{old man- medium-light skin tone}{2403}
\showcaseopenmojicolor{old man}{2404}
\showcaseopenmojicolor{old woman- dark skin tone}{2405}
\showcaseopenmojicolor{old woman- light skin tone}{2406}
\showcaseopenmojicolor{old woman- medium skin tone}{2407}
\showcaseopenmojicolor{old woman- medium-dark skin tone}{2408}
\showcaseopenmojicolor{old woman- medium-light skin tone}{2409}
\showcaseopenmojicolor{old woman}{2410}
\showcaseopenmojicolor{older person- dark skin tone}{2411}
\showcaseopenmojicolor{older person- light skin tone}{2412}
\showcaseopenmojicolor{older person- medium skin tone}{2413}
\showcaseopenmojicolor{older person- medium-dark skin tone}{2414}
\showcaseopenmojicolor{older person- medium-light skin tone}{2415}
\showcaseopenmojicolor{older person}{2416}
\showcaseopenmojicolor{olive}{2417}
\showcaseopenmojicolor{om}{2418}
\showcaseopenmojicolor{on! arrow}{2419}
\showcaseopenmojicolor{oncoming automobile}{2420}
\showcaseopenmojicolor{oncoming bus}{2421}
\showcaseopenmojicolor{oncoming fist- dark skin tone}{2422}
\showcaseopenmojicolor{oncoming fist- light skin tone}{2423}
\showcaseopenmojicolor{oncoming fist- medium skin tone}{2424}
\showcaseopenmojicolor{oncoming fist- medium-dark skin tone}{2425}
\showcaseopenmojicolor{oncoming fist- medium-light skin tone}{2426}
\showcaseopenmojicolor{oncoming fist}{2427}
\showcaseopenmojicolor{oncoming police car}{2428}
\showcaseopenmojicolor{oncoming taxi}{2429}
\showcaseopenmojicolor{one o clock}{2430}
\showcaseopenmojicolor{one-piece swimsuit}{2431}
\showcaseopenmojicolor{one-thirty}{2432}
\showcaseopenmojicolor{onion}{2433}
\showcaseopenmojicolor{open book}{2434}
\showcaseopenmojicolor{open file folder}{2435}
\showcaseopenmojicolor{open hands- dark skin tone}{2436}
\showcaseopenmojicolor{open hands- light skin tone}{2437}
\showcaseopenmojicolor{open hands- medium skin tone}{2438}
\showcaseopenmojicolor{open hands- medium-dark skin tone}{2439}
\showcaseopenmojicolor{open hands- medium-light skin tone}{2440}
\showcaseopenmojicolor{open hands}{2441}
\showcaseopenmojicolor{open mailbox with lowered flag}{2442}
\showcaseopenmojicolor{open mailbox with raised flag}{2443}
\showcaseopenmojicolor{openfoodfact}{2444}
\showcaseopenmojicolor{openstreetmap}{2445}
\showcaseopenmojicolor{opera}{2446}
\showcaseopenmojicolor{ophiuchus}{2447}
\showcaseopenmojicolor{optical disk}{2448}
\showcaseopenmojicolor{orange book}{2449}
\showcaseopenmojicolor{orange circle}{2450}
\showcaseopenmojicolor{orange flag}{2451}
\showcaseopenmojicolor{orange heart}{2452}
\showcaseopenmojicolor{orange hexagon}{2453}
\showcaseopenmojicolor{orange square}{2454}
\showcaseopenmojicolor{orangutan}{2455}
\showcaseopenmojicolor{orca}{2456}
\showcaseopenmojicolor{orthodox cross}{2457}
\showcaseopenmojicolor{otter}{2458}
\showcaseopenmojicolor{outbox tray}{2459}
\showcaseopenmojicolor{outlet}{2460}
\showcaseopenmojicolor{overlapping black squares}{2461}
\showcaseopenmojicolor{overlapping white and black squares}{2462}
\showcaseopenmojicolor{overlapping white squares}{2463}
\showcaseopenmojicolor{overview}{2464}
\showcaseopenmojicolor{owl}{2465}
\showcaseopenmojicolor{ox}{2466}
\showcaseopenmojicolor{oyster}{2467}
\showcaseopenmojicolor{p button}{2468}
\showcaseopenmojicolor{package}{2469}
\showcaseopenmojicolor{page facing up}{2470}
\showcaseopenmojicolor{page move}{2471}
\showcaseopenmojicolor{page with curl}{2472}
\showcaseopenmojicolor{pager}{2473}
\showcaseopenmojicolor{paintbrush}{2474}
\showcaseopenmojicolor{palm down hand- dark skin tone}{2475}
\showcaseopenmojicolor{palm down hand- light skin tone}{2476}
\showcaseopenmojicolor{palm down hand- medium skin tone}{2477}
\showcaseopenmojicolor{palm down hand- medium-dark skin tone}{2478}
\showcaseopenmojicolor{palm down hand- medium-light skin tone}{2479}
\showcaseopenmojicolor{palm down hand}{2480}
\showcaseopenmojicolor{palm tree}{2481}
\showcaseopenmojicolor{palm up hand- dark skin tone}{2482}
\showcaseopenmojicolor{palm up hand- light skin tone}{2483}
\showcaseopenmojicolor{palm up hand- medium skin tone}{2484}
\showcaseopenmojicolor{palm up hand- medium-dark skin tone}{2485}
\showcaseopenmojicolor{palm up hand- medium-light skin tone}{2486}
\showcaseopenmojicolor{palm up hand}{2487}
\showcaseopenmojicolor{palms up together- dark skin tone}{2488}
\showcaseopenmojicolor{palms up together- light skin tone}{2489}
\showcaseopenmojicolor{palms up together- medium skin tone}{2490}
\showcaseopenmojicolor{palms up together- medium-dark skin tone}{2491}
\showcaseopenmojicolor{palms up together- medium-light skin tone}{2492}
\showcaseopenmojicolor{palms up together}{2493}
\showcaseopenmojicolor{pancakes}{2494}
\showcaseopenmojicolor{panda}{2495}
\showcaseopenmojicolor{paperclip}{2496}
\showcaseopenmojicolor{parachute}{2497}
\showcaseopenmojicolor{parking garage}{2498}
\showcaseopenmojicolor{parrot}{2499}
\showcaseopenmojicolor{part alternation mark}{2500}
\showcaseopenmojicolor{party popper}{2501}
\showcaseopenmojicolor{partying face}{2502}
\showcaseopenmojicolor{passenger ship}{2503}
\showcaseopenmojicolor{passport control}{2504}
\showcaseopenmojicolor{patient clipboard}{2505}
\showcaseopenmojicolor{patient file}{2506}
\showcaseopenmojicolor{pause button}{2507}
\showcaseopenmojicolor{paw prints}{2508}
\showcaseopenmojicolor{pea pod}{2509}
\showcaseopenmojicolor{peace symbol}{2510}
\showcaseopenmojicolor{peach}{2511}
\showcaseopenmojicolor{peacock}{2512}
\showcaseopenmojicolor{peanuts}{2513}
\showcaseopenmojicolor{pear}{2514}
\showcaseopenmojicolor{peertube}{2515}
\showcaseopenmojicolor{pen}{2516}
\showcaseopenmojicolor{pencil}{2517}
\showcaseopenmojicolor{penguin}{2518}
\showcaseopenmojicolor{pensive face}{2519}
\showcaseopenmojicolor{people dialogue}{2520}
\showcaseopenmojicolor{people holding hands- dark skin tone, light skin tone}{2521}
\showcaseopenmojicolor{people holding hands- dark skin tone, medium skin tone}{2522}
\showcaseopenmojicolor{people holding hands- dark skin tone, medium-dark skin tone}{2523}
\showcaseopenmojicolor{people holding hands- dark skin tone, medium-light skin tone}{2524}
\showcaseopenmojicolor{people holding hands- dark skin tone}{2525}
\showcaseopenmojicolor{people holding hands- light skin tone, dark skin tone}{2526}
\showcaseopenmojicolor{people holding hands- light skin tone, medium skin tone}{2527}
\showcaseopenmojicolor{people holding hands- light skin tone, medium-dark skin tone}{2528}
\showcaseopenmojicolor{people holding hands- light skin tone, medium-light skin tone}{2529}
\showcaseopenmojicolor{people holding hands- light skin tone}{2530}
\showcaseopenmojicolor{people holding hands- medium skin tone, dark skin tone}{2531}
\showcaseopenmojicolor{people holding hands- medium skin tone, light skin tone}{2532}
\showcaseopenmojicolor{people holding hands- medium skin tone, medium-dark skin tone}{2533}
\showcaseopenmojicolor{people holding hands- medium skin tone, medium-light skin tone}{2534}
\showcaseopenmojicolor{people holding hands- medium skin tone}{2535}
\showcaseopenmojicolor{people holding hands- medium-dark skin tone, dark skin tone}{2536}
\showcaseopenmojicolor{people holding hands- medium-dark skin tone, light skin tone}{2537}
\showcaseopenmojicolor{people holding hands- medium-dark skin tone, medium skin tone}{2538}
\showcaseopenmojicolor{people holding hands- medium-dark skin tone, medium-light skin tone}{2539}
\showcaseopenmojicolor{people holding hands- medium-dark skin tone}{2540}
\showcaseopenmojicolor{people holding hands- medium-light skin tone, dark skin tone}{2541}
\showcaseopenmojicolor{people holding hands- medium-light skin tone, light skin tone}{2542}
\showcaseopenmojicolor{people holding hands- medium-light skin tone, medium skin tone}{2543}
\showcaseopenmojicolor{people holding hands- medium-light skin tone, medium-dark skin tone}{2544}
\showcaseopenmojicolor{people holding hands- medium-light skin tone}{2545}
\showcaseopenmojicolor{people holding hands}{2546}
\showcaseopenmojicolor{people hugging}{2547}
\showcaseopenmojicolor{people with bunny ears}{2548}
\showcaseopenmojicolor{people wrestling}{2549}
\showcaseopenmojicolor{performing arts}{2550}
\showcaseopenmojicolor{persevering face}{2551}
\showcaseopenmojicolor{person biking- dark skin tone}{2552}
\showcaseopenmojicolor{person biking- light skin tone}{2553}
\showcaseopenmojicolor{person biking- medium skin tone}{2554}
\showcaseopenmojicolor{person biking- medium-dark skin tone}{2555}
\showcaseopenmojicolor{person biking- medium-light skin tone}{2556}
\showcaseopenmojicolor{person biking}{2557}
\showcaseopenmojicolor{person bouncing ball- dark skin tone}{2558}
\showcaseopenmojicolor{person bouncing ball- light skin tone}{2559}
\showcaseopenmojicolor{person bouncing ball- medium skin tone}{2560}
\showcaseopenmojicolor{person bouncing ball- medium-dark skin tone}{2561}
\showcaseopenmojicolor{person bouncing ball- medium-light skin tone}{2562}
\showcaseopenmojicolor{person bouncing ball}{2563}
\showcaseopenmojicolor{person bowing- dark skin tone}{2564}
\showcaseopenmojicolor{person bowing- light skin tone}{2565}
\showcaseopenmojicolor{person bowing- medium skin tone}{2566}
\showcaseopenmojicolor{person bowing- medium-dark skin tone}{2567}
\showcaseopenmojicolor{person bowing- medium-light skin tone}{2568}
\showcaseopenmojicolor{person bowing}{2569}
\showcaseopenmojicolor{person cartwheeling- dark skin tone}{2570}
\showcaseopenmojicolor{person cartwheeling- light skin tone}{2571}
\showcaseopenmojicolor{person cartwheeling- medium skin tone}{2572}
\showcaseopenmojicolor{person cartwheeling- medium-dark skin tone}{2573}
\showcaseopenmojicolor{person cartwheeling- medium-light skin tone}{2574}
\showcaseopenmojicolor{person cartwheeling}{2575}
\showcaseopenmojicolor{person climbing- dark skin tone}{2576}
\showcaseopenmojicolor{person climbing- light skin tone}{2577}
\showcaseopenmojicolor{person climbing- medium skin tone}{2578}
\showcaseopenmojicolor{person climbing- medium-dark skin tone}{2579}
\showcaseopenmojicolor{person climbing- medium-light skin tone}{2580}
\showcaseopenmojicolor{person climbing}{2581}
\showcaseopenmojicolor{person facepalming- dark skin tone}{2582}
\showcaseopenmojicolor{person facepalming- light skin tone}{2583}
\showcaseopenmojicolor{person facepalming- medium skin tone}{2584}
\showcaseopenmojicolor{person facepalming- medium-dark skin tone}{2585}
\showcaseopenmojicolor{person facepalming- medium-light skin tone}{2586}
\showcaseopenmojicolor{person facepalming}{2587}
\showcaseopenmojicolor{person feeding baby- dark skin tone}{2588}
\showcaseopenmojicolor{person feeding baby- light skin tone}{2589}
\showcaseopenmojicolor{person feeding baby- medium skin tone}{2590}
\showcaseopenmojicolor{person feeding baby- medium-dark skin tone}{2591}
\showcaseopenmojicolor{person feeding baby- medium-light skin tone}{2592}
\showcaseopenmojicolor{person feeding baby}{2593}
\showcaseopenmojicolor{person fencing}{2594}
\showcaseopenmojicolor{person frowning- dark skin tone}{2595}
\showcaseopenmojicolor{person frowning- light skin tone}{2596}
\showcaseopenmojicolor{person frowning- medium skin tone}{2597}
\showcaseopenmojicolor{person frowning- medium-dark skin tone}{2598}
\showcaseopenmojicolor{person frowning- medium-light skin tone}{2599}
\showcaseopenmojicolor{person frowning}{2600}
\showcaseopenmojicolor{person gesturing no- dark skin tone}{2601}
\showcaseopenmojicolor{person gesturing no- light skin tone}{2602}
\showcaseopenmojicolor{person gesturing no- medium skin tone}{2603}
\showcaseopenmojicolor{person gesturing no- medium-dark skin tone}{2604}
\showcaseopenmojicolor{person gesturing no- medium-light skin tone}{2605}
\showcaseopenmojicolor{person gesturing no}{2606}
\showcaseopenmojicolor{person gesturing ok- dark skin tone}{2607}
\showcaseopenmojicolor{person gesturing ok- light skin tone}{2608}
\showcaseopenmojicolor{person gesturing ok- medium skin tone}{2609}
\showcaseopenmojicolor{person gesturing ok- medium-dark skin tone}{2610}
\showcaseopenmojicolor{person gesturing ok- medium-light skin tone}{2611}
\showcaseopenmojicolor{person gesturing ok}{2612}
\showcaseopenmojicolor{person getting haircut- dark skin tone}{2613}
\showcaseopenmojicolor{person getting haircut- light skin tone}{2614}
\showcaseopenmojicolor{person getting haircut- medium skin tone}{2615}
\showcaseopenmojicolor{person getting haircut- medium-dark skin tone}{2616}
\showcaseopenmojicolor{person getting haircut- medium-light skin tone}{2617}
\showcaseopenmojicolor{person getting haircut}{2618}
\showcaseopenmojicolor{person getting massage- dark skin tone}{2619}
\showcaseopenmojicolor{person getting massage- light skin tone}{2620}
\showcaseopenmojicolor{person getting massage- medium skin tone}{2621}
\showcaseopenmojicolor{person getting massage- medium-dark skin tone}{2622}
\showcaseopenmojicolor{person getting massage- medium-light skin tone}{2623}
\showcaseopenmojicolor{person getting massage}{2624}
\showcaseopenmojicolor{person golfing- dark skin tone}{2625}
\showcaseopenmojicolor{person golfing- light skin tone}{2626}
\showcaseopenmojicolor{person golfing- medium skin tone}{2627}
\showcaseopenmojicolor{person golfing- medium-dark skin tone}{2628}
\showcaseopenmojicolor{person golfing- medium-light skin tone}{2629}
\showcaseopenmojicolor{person golfing}{2630}
\showcaseopenmojicolor{person in bed- dark skin tone}{2631}
\showcaseopenmojicolor{person in bed- light skin tone}{2632}
\showcaseopenmojicolor{person in bed- medium skin tone}{2633}
\showcaseopenmojicolor{person in bed- medium-dark skin tone}{2634}
\showcaseopenmojicolor{person in bed- medium-light skin tone}{2635}
\showcaseopenmojicolor{person in bed}{2636}
\showcaseopenmojicolor{person in lotus position- dark skin tone}{2637}
\showcaseopenmojicolor{person in lotus position- light skin tone}{2638}
\showcaseopenmojicolor{person in lotus position- medium skin tone}{2639}
\showcaseopenmojicolor{person in lotus position- medium-dark skin tone}{2640}
\showcaseopenmojicolor{person in lotus position- medium-light skin tone}{2641}
\showcaseopenmojicolor{person in lotus position}{2642}
\showcaseopenmojicolor{person in manual wheelchair facing right}{2643}
\showcaseopenmojicolor{person in manual wheelchair- dark skin tone}{2644}
\showcaseopenmojicolor{person in manual wheelchair- light skin tone}{2645}
\showcaseopenmojicolor{person in manual wheelchair- medium skin tone}{2646}
\showcaseopenmojicolor{person in manual wheelchair- medium-dark skin tone}{2647}
\showcaseopenmojicolor{person in manual wheelchair- medium-light skin tone}{2648}
\showcaseopenmojicolor{person in manual wheelchair}{2649}
\showcaseopenmojicolor{person in motorized wheelchair facing right}{2650}
\showcaseopenmojicolor{person in motorized wheelchair- dark skin tone}{2651}
\showcaseopenmojicolor{person in motorized wheelchair- light skin tone}{2652}
\showcaseopenmojicolor{person in motorized wheelchair- medium skin tone}{2653}
\showcaseopenmojicolor{person in motorized wheelchair- medium-dark skin tone}{2654}
\showcaseopenmojicolor{person in motorized wheelchair- medium-light skin tone}{2655}
\showcaseopenmojicolor{person in motorized wheelchair}{2656}
\showcaseopenmojicolor{person in steamy room- dark skin tone}{2657}
\showcaseopenmojicolor{person in steamy room- light skin tone}{2658}
\showcaseopenmojicolor{person in steamy room- medium skin tone}{2659}
\showcaseopenmojicolor{person in steamy room- medium-dark skin tone}{2660}
\showcaseopenmojicolor{person in steamy room- medium-light skin tone}{2661}
\showcaseopenmojicolor{person in steamy room}{2662}
\showcaseopenmojicolor{person in suit levitating- dark skin tone}{2663}
\showcaseopenmojicolor{person in suit levitating- light skin tone}{2664}
\showcaseopenmojicolor{person in suit levitating- medium skin tone}{2665}
\showcaseopenmojicolor{person in suit levitating- medium-dark skin tone}{2666}
\showcaseopenmojicolor{person in suit levitating- medium-light skin tone}{2667}
\showcaseopenmojicolor{person in suit levitating}{2668}
\showcaseopenmojicolor{person in tuxedo- dark skin tone}{2669}
\showcaseopenmojicolor{person in tuxedo- light skin tone}{2670}
\showcaseopenmojicolor{person in tuxedo- medium skin tone}{2671}
\showcaseopenmojicolor{person in tuxedo- medium-dark skin tone}{2672}
\showcaseopenmojicolor{person in tuxedo- medium-light skin tone}{2673}
\showcaseopenmojicolor{person in tuxedo}{2674}
\showcaseopenmojicolor{person juggling- dark skin tone}{2675}
\showcaseopenmojicolor{person juggling- light skin tone}{2676}
\showcaseopenmojicolor{person juggling- medium skin tone}{2677}
\showcaseopenmojicolor{person juggling- medium-dark skin tone}{2678}
\showcaseopenmojicolor{person juggling- medium-light skin tone}{2679}
\showcaseopenmojicolor{person juggling}{2680}
\showcaseopenmojicolor{person kneeling facing right}{2681}
\showcaseopenmojicolor{person kneeling- dark skin tone}{2682}
\showcaseopenmojicolor{person kneeling- light skin tone}{2683}
\showcaseopenmojicolor{person kneeling- medium skin tone}{2684}
\showcaseopenmojicolor{person kneeling- medium-dark skin tone}{2685}
\showcaseopenmojicolor{person kneeling- medium-light skin tone}{2686}
\showcaseopenmojicolor{person kneeling}{2687}
\showcaseopenmojicolor{person lifting weights- dark skin tone}{2688}
\showcaseopenmojicolor{person lifting weights- light skin tone}{2689}
\showcaseopenmojicolor{person lifting weights- medium skin tone}{2690}
\showcaseopenmojicolor{person lifting weights- medium-dark skin tone}{2691}
\showcaseopenmojicolor{person lifting weights- medium-light skin tone}{2692}
\showcaseopenmojicolor{person lifting weights}{2693}
\showcaseopenmojicolor{person mountain biking- dark skin tone}{2694}
\showcaseopenmojicolor{person mountain biking- light skin tone}{2695}
\showcaseopenmojicolor{person mountain biking- medium skin tone}{2696}
\showcaseopenmojicolor{person mountain biking- medium-dark skin tone}{2697}
\showcaseopenmojicolor{person mountain biking- medium-light skin tone}{2698}
\showcaseopenmojicolor{person mountain biking}{2699}
\showcaseopenmojicolor{person playing handball- dark skin tone}{2700}
\showcaseopenmojicolor{person playing handball- light skin tone}{2701}
\showcaseopenmojicolor{person playing handball- medium skin tone}{2702}
\showcaseopenmojicolor{person playing handball- medium-dark skin tone}{2703}
\showcaseopenmojicolor{person playing handball- medium-light skin tone}{2704}
\showcaseopenmojicolor{person playing handball}{2705}
\showcaseopenmojicolor{person playing water polo- dark skin tone}{2706}
\showcaseopenmojicolor{person playing water polo- light skin tone}{2707}
\showcaseopenmojicolor{person playing water polo- medium skin tone}{2708}
\showcaseopenmojicolor{person playing water polo- medium-dark skin tone}{2709}
\showcaseopenmojicolor{person playing water polo- medium-light skin tone}{2710}
\showcaseopenmojicolor{person playing water polo}{2711}
\showcaseopenmojicolor{person pouting- dark skin tone}{2712}
\showcaseopenmojicolor{person pouting- light skin tone}{2713}
\showcaseopenmojicolor{person pouting- medium skin tone}{2714}
\showcaseopenmojicolor{person pouting- medium-dark skin tone}{2715}
\showcaseopenmojicolor{person pouting- medium-light skin tone}{2716}
\showcaseopenmojicolor{person pouting}{2717}
\showcaseopenmojicolor{person raising hand- dark skin tone}{2718}
\showcaseopenmojicolor{person raising hand- light skin tone}{2719}
\showcaseopenmojicolor{person raising hand- medium skin tone}{2720}
\showcaseopenmojicolor{person raising hand- medium-dark skin tone}{2721}
\showcaseopenmojicolor{person raising hand- medium-light skin tone}{2722}
\showcaseopenmojicolor{person raising hand}{2723}
\showcaseopenmojicolor{person rowing boat- dark skin tone}{2724}
\showcaseopenmojicolor{person rowing boat- light skin tone}{2725}
\showcaseopenmojicolor{person rowing boat- medium skin tone}{2726}
\showcaseopenmojicolor{person rowing boat- medium-dark skin tone}{2727}
\showcaseopenmojicolor{person rowing boat- medium-light skin tone}{2728}
\showcaseopenmojicolor{person rowing boat}{2729}
\showcaseopenmojicolor{person running facing right}{2730}
\showcaseopenmojicolor{person running- dark skin tone}{2731}
\showcaseopenmojicolor{person running- light skin tone}{2732}
\showcaseopenmojicolor{person running- medium skin tone}{2733}
\showcaseopenmojicolor{person running- medium-dark skin tone}{2734}
\showcaseopenmojicolor{person running- medium-light skin tone}{2735}
\showcaseopenmojicolor{person running}{2736}
\showcaseopenmojicolor{person shrugging- dark skin tone}{2737}
\showcaseopenmojicolor{person shrugging- light skin tone}{2738}
\showcaseopenmojicolor{person shrugging- medium skin tone}{2739}
\showcaseopenmojicolor{person shrugging- medium-dark skin tone}{2740}
\showcaseopenmojicolor{person shrugging- medium-light skin tone}{2741}
\showcaseopenmojicolor{person shrugging}{2742}
\showcaseopenmojicolor{person sneezing into elbow}{2743}
\showcaseopenmojicolor{person standing- dark skin tone}{2744}
\showcaseopenmojicolor{person standing- light skin tone}{2745}
\showcaseopenmojicolor{person standing- medium skin tone}{2746}
\showcaseopenmojicolor{person standing- medium-dark skin tone}{2747}
\showcaseopenmojicolor{person standing- medium-light skin tone}{2748}
\showcaseopenmojicolor{person standing}{2749}
\showcaseopenmojicolor{person surfing- dark skin tone}{2750}
\showcaseopenmojicolor{person surfing- light skin tone}{2751}
\showcaseopenmojicolor{person surfing- medium skin tone}{2752}
\showcaseopenmojicolor{person surfing- medium-dark skin tone}{2753}
\showcaseopenmojicolor{person surfing- medium-light skin tone}{2754}
\showcaseopenmojicolor{person surfing}{2755}
\showcaseopenmojicolor{person swimming- dark skin tone}{2756}
\showcaseopenmojicolor{person swimming- light skin tone}{2757}
\showcaseopenmojicolor{person swimming- medium skin tone}{2758}
\showcaseopenmojicolor{person swimming- medium-dark skin tone}{2759}
\showcaseopenmojicolor{person swimming- medium-light skin tone}{2760}
\showcaseopenmojicolor{person swimming}{2761}
\showcaseopenmojicolor{person taking bath- dark skin tone}{2762}
\showcaseopenmojicolor{person taking bath- light skin tone}{2763}
\showcaseopenmojicolor{person taking bath- medium skin tone}{2764}
\showcaseopenmojicolor{person taking bath- medium-dark skin tone}{2765}
\showcaseopenmojicolor{person taking bath- medium-light skin tone}{2766}
\showcaseopenmojicolor{person taking bath}{2767}
\showcaseopenmojicolor{person tipping hand- dark skin tone}{2768}
\showcaseopenmojicolor{person tipping hand- light skin tone}{2769}
\showcaseopenmojicolor{person tipping hand- medium skin tone}{2770}
\showcaseopenmojicolor{person tipping hand- medium-dark skin tone}{2771}
\showcaseopenmojicolor{person tipping hand- medium-light skin tone}{2772}
\showcaseopenmojicolor{person tipping hand}{2773}
\showcaseopenmojicolor{person walking facing right}{2774}
\showcaseopenmojicolor{person walking- dark skin tone}{2775}
\showcaseopenmojicolor{person walking- light skin tone}{2776}
\showcaseopenmojicolor{person walking- medium skin tone}{2777}
\showcaseopenmojicolor{person walking- medium-dark skin tone}{2778}
\showcaseopenmojicolor{person walking- medium-light skin tone}{2779}
\showcaseopenmojicolor{person walking}{2780}
\showcaseopenmojicolor{person wearing turban- dark skin tone}{2781}
\showcaseopenmojicolor{person wearing turban- light skin tone}{2782}
\showcaseopenmojicolor{person wearing turban- medium skin tone}{2783}
\showcaseopenmojicolor{person wearing turban- medium-dark skin tone}{2784}
\showcaseopenmojicolor{person wearing turban- medium-light skin tone}{2785}
\showcaseopenmojicolor{person wearing turban}{2786}
\showcaseopenmojicolor{person with crown- dark skin tone}{2787}
\showcaseopenmojicolor{person with crown- light skin tone}{2788}
\showcaseopenmojicolor{person with crown- medium skin tone}{2789}
\showcaseopenmojicolor{person with crown- medium-dark skin tone}{2790}
\showcaseopenmojicolor{person with crown- medium-light skin tone}{2791}
\showcaseopenmojicolor{person with crown}{2792}
\showcaseopenmojicolor{person with dog}{2793}
\showcaseopenmojicolor{person with medical mask}{2794}
\showcaseopenmojicolor{person with skullcap- dark skin tone}{2795}
\showcaseopenmojicolor{person with skullcap- light skin tone}{2796}
\showcaseopenmojicolor{person with skullcap- medium skin tone}{2797}
\showcaseopenmojicolor{person with skullcap- medium-dark skin tone}{2798}
\showcaseopenmojicolor{person with skullcap- medium-light skin tone}{2799}
\showcaseopenmojicolor{person with skullcap}{2800}
\showcaseopenmojicolor{person with veil- dark skin tone}{2801}
\showcaseopenmojicolor{person with veil- light skin tone}{2802}
\showcaseopenmojicolor{person with veil- medium skin tone}{2803}
\showcaseopenmojicolor{person with veil- medium-dark skin tone}{2804}
\showcaseopenmojicolor{person with veil- medium-light skin tone}{2805}
\showcaseopenmojicolor{person with veil}{2806}
\showcaseopenmojicolor{person with white cane facing right}{2807}
\showcaseopenmojicolor{person with white cane- dark skin tone}{2808}
\showcaseopenmojicolor{person with white cane- light skin tone}{2809}
\showcaseopenmojicolor{person with white cane- medium skin tone}{2810}
\showcaseopenmojicolor{person with white cane- medium-dark skin tone}{2811}
\showcaseopenmojicolor{person with white cane- medium-light skin tone}{2812}
\showcaseopenmojicolor{person with white cane}{2813}
\showcaseopenmojicolor{person- bald}{2814}
\showcaseopenmojicolor{person- beard}{2815}
\showcaseopenmojicolor{person- blond hair}{2816}
\showcaseopenmojicolor{person- curly hair}{2817}
\showcaseopenmojicolor{person- dark skin tone, bald}{2818}
\showcaseopenmojicolor{person- dark skin tone, beard}{2819}
\showcaseopenmojicolor{person- dark skin tone, blond hair}{2820}
\showcaseopenmojicolor{person- dark skin tone, curly hair}{2821}
\showcaseopenmojicolor{person- dark skin tone, red hair}{2822}
\showcaseopenmojicolor{person- dark skin tone, white hair}{2823}
\showcaseopenmojicolor{person- dark skin tone}{2824}
\showcaseopenmojicolor{person- light skin tone, bald}{2825}
\showcaseopenmojicolor{person- light skin tone, beard}{2826}
\showcaseopenmojicolor{person- light skin tone, blond hair}{2827}
\showcaseopenmojicolor{person- light skin tone, curly hair}{2828}
\showcaseopenmojicolor{person- light skin tone, red hair}{2829}
\showcaseopenmojicolor{person- light skin tone, white hair}{2830}
\showcaseopenmojicolor{person- light skin tone}{2831}
\showcaseopenmojicolor{person- medium skin tone, bald}{2832}
\showcaseopenmojicolor{person- medium skin tone, beard}{2833}
\showcaseopenmojicolor{person- medium skin tone, blond hair}{2834}
\showcaseopenmojicolor{person- medium skin tone, curly hair}{2835}
\showcaseopenmojicolor{person- medium skin tone, red hair}{2836}
\showcaseopenmojicolor{person- medium skin tone, white hair}{2837}
\showcaseopenmojicolor{person- medium skin tone}{2838}
\showcaseopenmojicolor{person- medium-dark skin tone, bald}{2839}
\showcaseopenmojicolor{person- medium-dark skin tone, beard}{2840}
\showcaseopenmojicolor{person- medium-dark skin tone, blond hair}{2841}
\showcaseopenmojicolor{person- medium-dark skin tone, curly hair}{2842}
\showcaseopenmojicolor{person- medium-dark skin tone, red hair}{2843}
\showcaseopenmojicolor{person- medium-dark skin tone, white hair}{2844}
\showcaseopenmojicolor{person- medium-dark skin tone}{2845}
\showcaseopenmojicolor{person- medium-light skin tone, bald}{2846}
\showcaseopenmojicolor{person- medium-light skin tone, beard}{2847}
\showcaseopenmojicolor{person- medium-light skin tone, blond hair}{2848}
\showcaseopenmojicolor{person- medium-light skin tone, curly hair}{2849}
\showcaseopenmojicolor{person- medium-light skin tone, red hair}{2850}
\showcaseopenmojicolor{person- medium-light skin tone, white hair}{2851}
\showcaseopenmojicolor{person- medium-light skin tone}{2852}
\showcaseopenmojicolor{person- red hair}{2853}
\showcaseopenmojicolor{person- white hair}{2854}
\showcaseopenmojicolor{person}{2855}
\showcaseopenmojicolor{petri dish}{2856}
\showcaseopenmojicolor{phoenix}{2857}
\showcaseopenmojicolor{pick}{2858}
\showcaseopenmojicolor{pickup truck}{2859}
\showcaseopenmojicolor{picture}{2860}
\showcaseopenmojicolor{pie}{2861}
\showcaseopenmojicolor{pig face}{2862}
\showcaseopenmojicolor{pig nose}{2863}
\showcaseopenmojicolor{pig}{2864}
\showcaseopenmojicolor{pigeon}{2865}
\showcaseopenmojicolor{pile of poo}{2866}
\showcaseopenmojicolor{pill}{2867}
\showcaseopenmojicolor{pills}{2868}
\showcaseopenmojicolor{pilot- dark skin tone}{2869}
\showcaseopenmojicolor{pilot- light skin tone}{2870}
\showcaseopenmojicolor{pilot- medium skin tone}{2871}
\showcaseopenmojicolor{pilot- medium-dark skin tone}{2872}
\showcaseopenmojicolor{pilot- medium-light skin tone}{2873}
\showcaseopenmojicolor{pilot}{2874}
\showcaseopenmojicolor{pinched fingers- dark skin tone}{2875}
\showcaseopenmojicolor{pinched fingers- light skin tone}{2876}
\showcaseopenmojicolor{pinched fingers- medium skin tone}{2877}
\showcaseopenmojicolor{pinched fingers- medium-dark skin tone}{2878}
\showcaseopenmojicolor{pinched fingers- medium-light skin tone}{2879}
\showcaseopenmojicolor{pinched fingers}{2880}
\showcaseopenmojicolor{pinching hand- dark skin tone}{2881}
\showcaseopenmojicolor{pinching hand- light skin tone}{2882}
\showcaseopenmojicolor{pinching hand- medium skin tone}{2883}
\showcaseopenmojicolor{pinching hand- medium-dark skin tone}{2884}
\showcaseopenmojicolor{pinching hand- medium-light skin tone}{2885}
\showcaseopenmojicolor{pinching hand}{2886}
\showcaseopenmojicolor{pine decoration}{2887}
\showcaseopenmojicolor{pineapple}{2888}
\showcaseopenmojicolor{ping pong}{2889}
\showcaseopenmojicolor{pink heart}{2890}
\showcaseopenmojicolor{pinterest}{2891}
\showcaseopenmojicolor{pirate flag}{2892}
\showcaseopenmojicolor{pisces}{2893}
\showcaseopenmojicolor{pixelfed}{2894}
\showcaseopenmojicolor{pizza}{2895}
\showcaseopenmojicolor{pinata}{2896}
\showcaseopenmojicolor{placard}{2897}
\showcaseopenmojicolor{place of worship}{2898}
\showcaseopenmojicolor{plaster}{2899}
\showcaseopenmojicolor{plastic bottle}{2900}
\showcaseopenmojicolor{play button}{2901}
\showcaseopenmojicolor{play or pause button}{2902}
\showcaseopenmojicolor{playground slide}{2903}
\showcaseopenmojicolor{pleading face}{2904}
\showcaseopenmojicolor{plunger}{2905}
\showcaseopenmojicolor{plus}{2906}
\showcaseopenmojicolor{polar bear}{2907}
\showcaseopenmojicolor{polar explorer man}{2908}
\showcaseopenmojicolor{polar explorer woman}{2909}
\showcaseopenmojicolor{polar explorer}{2910}
\showcaseopenmojicolor{polar research station}{2911}
\showcaseopenmojicolor{police car light}{2912}
\showcaseopenmojicolor{police car}{2913}
\showcaseopenmojicolor{police officer- dark skin tone}{2914}
\showcaseopenmojicolor{police officer- light skin tone}{2915}
\showcaseopenmojicolor{police officer- medium skin tone}{2916}
\showcaseopenmojicolor{police officer- medium-dark skin tone}{2917}
\showcaseopenmojicolor{police officer- medium-light skin tone}{2918}
\showcaseopenmojicolor{police officer}{2919}
\showcaseopenmojicolor{pomegranate}{2920}
\showcaseopenmojicolor{poodle}{2921}
\showcaseopenmojicolor{pool 8 ball}{2922}
\showcaseopenmojicolor{popcorn}{2923}
\showcaseopenmojicolor{poppy}{2924}
\showcaseopenmojicolor{porpoise}{2925}
\showcaseopenmojicolor{post office}{2926}
\showcaseopenmojicolor{postal horn}{2927}
\showcaseopenmojicolor{postbox}{2928}
\showcaseopenmojicolor{pot of food}{2929}
\showcaseopenmojicolor{potable water}{2930}
\showcaseopenmojicolor{potato}{2931}
\showcaseopenmojicolor{potentiometer}{2932}
\showcaseopenmojicolor{potted plant}{2933}
\showcaseopenmojicolor{poultry leg}{2934}
\showcaseopenmojicolor{pound banknote}{2935}
\showcaseopenmojicolor{pouring liquid}{2936}
\showcaseopenmojicolor{pouting cat}{2937}
\showcaseopenmojicolor{power on symbol}{2938}
\showcaseopenmojicolor{power on-off symbol}{2939}
\showcaseopenmojicolor{power sleep symbol}{2940}
\showcaseopenmojicolor{power symbol}{2941}
\showcaseopenmojicolor{prayer beads}{2942}
\showcaseopenmojicolor{pregnant man- dark skin tone}{2943}
\showcaseopenmojicolor{pregnant man- light skin tone}{2944}
\showcaseopenmojicolor{pregnant man- medium skin tone}{2945}
\showcaseopenmojicolor{pregnant man- medium-dark skin tone}{2946}
\showcaseopenmojicolor{pregnant man- medium-light skin tone}{2947}
\showcaseopenmojicolor{pregnant man}{2948}
\showcaseopenmojicolor{pregnant person- dark skin tone}{2949}
\showcaseopenmojicolor{pregnant person- light skin tone}{2950}
\showcaseopenmojicolor{pregnant person- medium skin tone}{2951}
\showcaseopenmojicolor{pregnant person- medium-dark skin tone}{2952}
\showcaseopenmojicolor{pregnant person- medium-light skin tone}{2953}
\showcaseopenmojicolor{pregnant person}{2954}
\showcaseopenmojicolor{pregnant woman- dark skin tone}{2955}
\showcaseopenmojicolor{pregnant woman- light skin tone}{2956}
\showcaseopenmojicolor{pregnant woman- medium skin tone}{2957}
\showcaseopenmojicolor{pregnant woman- medium-dark skin tone}{2958}
\showcaseopenmojicolor{pregnant woman- medium-light skin tone}{2959}
\showcaseopenmojicolor{pregnant woman}{2960}
\showcaseopenmojicolor{pretzel}{2961}
\showcaseopenmojicolor{prince- dark skin tone}{2962}
\showcaseopenmojicolor{prince- light skin tone}{2963}
\showcaseopenmojicolor{prince- medium skin tone}{2964}
\showcaseopenmojicolor{prince- medium-dark skin tone}{2965}
\showcaseopenmojicolor{prince- medium-light skin tone}{2966}
\showcaseopenmojicolor{prince}{2967}
\showcaseopenmojicolor{princess- dark skin tone}{2968}
\showcaseopenmojicolor{princess- light skin tone}{2969}
\showcaseopenmojicolor{princess- medium skin tone}{2970}
\showcaseopenmojicolor{princess- medium-dark skin tone}{2971}
\showcaseopenmojicolor{princess- medium-light skin tone}{2972}
\showcaseopenmojicolor{princess}{2973}
\showcaseopenmojicolor{printer}{2974}
\showcaseopenmojicolor{prohibited}{2975}
\showcaseopenmojicolor{purple circle}{2976}
\showcaseopenmojicolor{purple flag}{2977}
\showcaseopenmojicolor{purple heart}{2978}
\showcaseopenmojicolor{purple hexagon}{2979}
\showcaseopenmojicolor{purple square}{2980}
\showcaseopenmojicolor{purse}{2981}
\showcaseopenmojicolor{pushpin}{2982}
\showcaseopenmojicolor{puzzle piece}{2983}
\showcaseopenmojicolor{qr code}{2984}
\showcaseopenmojicolor{quarantine}{2985}
\showcaseopenmojicolor{quebec flag}{2986}
\showcaseopenmojicolor{rabbit face}{2987}
\showcaseopenmojicolor{rabbit}{2988}
\showcaseopenmojicolor{raccoon}{2989}
\showcaseopenmojicolor{racing car}{2990}
\showcaseopenmojicolor{radio button}{2991}
\showcaseopenmojicolor{radio}{2992}
\showcaseopenmojicolor{radioactive waste}{2993}
\showcaseopenmojicolor{radioactive}{2994}
\showcaseopenmojicolor{railway car}{2995}
\showcaseopenmojicolor{railway track}{2996}
\showcaseopenmojicolor{rainbow flag}{2997}
\showcaseopenmojicolor{rainbow hexagon}{2998}
\showcaseopenmojicolor{rainbow}{2999}
\showcaseopenmojicolor{raised back of hand- dark skin tone}{3000}
\showcaseopenmojicolor{raised back of hand- light skin tone}{3001}
\showcaseopenmojicolor{raised back of hand- medium skin tone}{3002}
\showcaseopenmojicolor{raised back of hand- medium-dark skin tone}{3003}
\showcaseopenmojicolor{raised back of hand- medium-light skin tone}{3004}
\showcaseopenmojicolor{raised back of hand}{3005}
\showcaseopenmojicolor{raised fist- dark skin tone}{3006}
\showcaseopenmojicolor{raised fist- light skin tone}{3007}
\showcaseopenmojicolor{raised fist- medium skin tone}{3008}
\showcaseopenmojicolor{raised fist- medium-dark skin tone}{3009}
\showcaseopenmojicolor{raised fist- medium-light skin tone}{3010}
\showcaseopenmojicolor{raised fist}{3011}
\showcaseopenmojicolor{raised hand- dark skin tone}{3012}
\showcaseopenmojicolor{raised hand- light skin tone}{3013}
\showcaseopenmojicolor{raised hand- medium skin tone}{3014}
\showcaseopenmojicolor{raised hand- medium-dark skin tone}{3015}
\showcaseopenmojicolor{raised hand- medium-light skin tone}{3016}
\showcaseopenmojicolor{raised hand}{3017}
\showcaseopenmojicolor{raising hands- dark skin tone}{3018}
\showcaseopenmojicolor{raising hands- light skin tone}{3019}
\showcaseopenmojicolor{raising hands- medium skin tone}{3020}
\showcaseopenmojicolor{raising hands- medium-dark skin tone}{3021}
\showcaseopenmojicolor{raising hands- medium-light skin tone}{3022}
\showcaseopenmojicolor{raising hands}{3023}
\showcaseopenmojicolor{ram}{3024}
\showcaseopenmojicolor{raspberry pi}{3025}
\showcaseopenmojicolor{rat}{3026}
\showcaseopenmojicolor{razor}{3027}
\showcaseopenmojicolor{receipt}{3028}
\showcaseopenmojicolor{record button}{3029}
\showcaseopenmojicolor{recycling symbol}{3030}
\showcaseopenmojicolor{red and black flag}{3031}
\showcaseopenmojicolor{red apple}{3032}
\showcaseopenmojicolor{red circle}{3033}
\showcaseopenmojicolor{red envelope}{3034}
\showcaseopenmojicolor{red exclamation mark}{3035}
\showcaseopenmojicolor{red eye}{3036}
\showcaseopenmojicolor{red flag}{3037}
\showcaseopenmojicolor{red hair}{3038}
\showcaseopenmojicolor{red heart}{3039}
\showcaseopenmojicolor{red hexagon}{3040}
\showcaseopenmojicolor{red paper lantern}{3041}
\showcaseopenmojicolor{red question mark}{3042}
\showcaseopenmojicolor{red square}{3043}
\showcaseopenmojicolor{red triangle pointed down}{3044}
\showcaseopenmojicolor{red triangle pointed up}{3045}
\showcaseopenmojicolor{reddit}{3046}
\showcaseopenmojicolor{regional indicator a}{3047}
\showcaseopenmojicolor{regional indicator b}{3048}
\showcaseopenmojicolor{regional indicator c}{3049}
\showcaseopenmojicolor{regional indicator d}{3050}
\showcaseopenmojicolor{regional indicator e}{3051}
\showcaseopenmojicolor{regional indicator f}{3052}
\showcaseopenmojicolor{regional indicator g}{3053}
\showcaseopenmojicolor{regional indicator h}{3054}
\showcaseopenmojicolor{regional indicator i}{3055}
\showcaseopenmojicolor{regional indicator j}{3056}
\showcaseopenmojicolor{regional indicator k}{3057}
\showcaseopenmojicolor{regional indicator l}{3058}
\showcaseopenmojicolor{regional indicator m}{3059}
\showcaseopenmojicolor{regional indicator n}{3060}
\showcaseopenmojicolor{regional indicator o}{3061}
\showcaseopenmojicolor{regional indicator p}{3062}
\showcaseopenmojicolor{regional indicator q}{3063}
\showcaseopenmojicolor{regional indicator r}{3064}
\showcaseopenmojicolor{regional indicator s}{3065}
\showcaseopenmojicolor{regional indicator t}{3066}
\showcaseopenmojicolor{regional indicator u}{3067}
\showcaseopenmojicolor{regional indicator v}{3068}
\showcaseopenmojicolor{regional indicator w}{3069}
\showcaseopenmojicolor{regional indicator x}{3070}
\showcaseopenmojicolor{regional indicator y}{3071}
\showcaseopenmojicolor{regional indicator z}{3072}
\showcaseopenmojicolor{registered}{3073}
\showcaseopenmojicolor{relieved face}{3074}
\showcaseopenmojicolor{reminder ribbon}{3075}
\showcaseopenmojicolor{repeat button}{3076}
\showcaseopenmojicolor{repeat single button}{3077}
\showcaseopenmojicolor{rescue worker s helmet}{3078}
\showcaseopenmojicolor{restroom}{3079}
\showcaseopenmojicolor{return}{3080}
\showcaseopenmojicolor{reusable bag}{3081}
\showcaseopenmojicolor{reverse button}{3082}
\showcaseopenmojicolor{revolving hearts}{3083}
\showcaseopenmojicolor{rhinoceros}{3084}
\showcaseopenmojicolor{ribbon}{3085}
\showcaseopenmojicolor{rice ball}{3086}
\showcaseopenmojicolor{rice cracker}{3087}
\showcaseopenmojicolor{right anger bubble}{3088}
\showcaseopenmojicolor{right arrow curving down}{3089}
\showcaseopenmojicolor{right arrow curving left}{3090}
\showcaseopenmojicolor{right arrow curving up}{3091}
\showcaseopenmojicolor{right arrow}{3092}
\showcaseopenmojicolor{right-facing fist- dark skin tone}{3093}
\showcaseopenmojicolor{right-facing fist- light skin tone}{3094}
\showcaseopenmojicolor{right-facing fist- medium skin tone}{3095}
\showcaseopenmojicolor{right-facing fist- medium-dark skin tone}{3096}
\showcaseopenmojicolor{right-facing fist- medium-light skin tone}{3097}
\showcaseopenmojicolor{right-facing fist}{3098}
\showcaseopenmojicolor{rightwards hand- dark skin tone}{3099}
\showcaseopenmojicolor{rightwards hand- light skin tone}{3100}
\showcaseopenmojicolor{rightwards hand- medium skin tone}{3101}
\showcaseopenmojicolor{rightwards hand- medium-dark skin tone}{3102}
\showcaseopenmojicolor{rightwards hand- medium-light skin tone}{3103}
\showcaseopenmojicolor{rightwards hand}{3104}
\showcaseopenmojicolor{rightwards pushing hand- dark skin tone}{3105}
\showcaseopenmojicolor{rightwards pushing hand- light skin tone}{3106}
\showcaseopenmojicolor{rightwards pushing hand- medium skin tone}{3107}
\showcaseopenmojicolor{rightwards pushing hand- medium-dark skin tone}{3108}
\showcaseopenmojicolor{rightwards pushing hand- medium-light skin tone}{3109}
\showcaseopenmojicolor{rightwards pushing hand}{3110}
\showcaseopenmojicolor{ring buoy}{3111}
\showcaseopenmojicolor{ring}{3112}
\showcaseopenmojicolor{ringed planet}{3113}
\showcaseopenmojicolor{roasted coffee bean}{3114}
\showcaseopenmojicolor{roasted sweet potato}{3115}
\showcaseopenmojicolor{robot}{3116}
\showcaseopenmojicolor{rock}{3117}
\showcaseopenmojicolor{rocket}{3118}
\showcaseopenmojicolor{roll of paper}{3119}
\showcaseopenmojicolor{rolled-up newspaper}{3120}
\showcaseopenmojicolor{roller coaster}{3121}
\showcaseopenmojicolor{roller skate}{3122}
\showcaseopenmojicolor{rolling on the floor laughing}{3123}
\showcaseopenmojicolor{rooster}{3124}
\showcaseopenmojicolor{rose}{3125}
\showcaseopenmojicolor{rosette}{3126}
\showcaseopenmojicolor{round pushpin}{3127}
\showcaseopenmojicolor{rounded symbol for cai}{3128}
\showcaseopenmojicolor{rounded symbol for fu}{3129}
\showcaseopenmojicolor{rounded symbol for lu}{3130}
\showcaseopenmojicolor{rounded symbol for shou}{3131}
\showcaseopenmojicolor{rounded symbol for shuangxi}{3132}
\showcaseopenmojicolor{rounded symbol for xi}{3133}
\showcaseopenmojicolor{ruby}{3134}
\showcaseopenmojicolor{rugby football}{3135}
\showcaseopenmojicolor{running shirt}{3136}
\showcaseopenmojicolor{running shoe}{3137}
\showcaseopenmojicolor{sad but relieved face}{3138}
\showcaseopenmojicolor{safari}{3139}
\showcaseopenmojicolor{safety pin}{3140}
\showcaseopenmojicolor{safety vest}{3141}
\showcaseopenmojicolor{safety}{3142}
\showcaseopenmojicolor{sagittarius}{3143}
\showcaseopenmojicolor{sailboat}{3144}
\showcaseopenmojicolor{sake}{3145}
\showcaseopenmojicolor{saline drip}{3146}
\showcaseopenmojicolor{salt}{3147}
\showcaseopenmojicolor{saluting face}{3148}
\showcaseopenmojicolor{sandwich}{3149}
\showcaseopenmojicolor{sanitizer spray}{3150}
\showcaseopenmojicolor{santa claus- dark skin tone}{3151}
\showcaseopenmojicolor{santa claus- light skin tone}{3152}
\showcaseopenmojicolor{santa claus- medium skin tone}{3153}
\showcaseopenmojicolor{santa claus- medium-dark skin tone}{3154}
\showcaseopenmojicolor{santa claus- medium-light skin tone}{3155}
\showcaseopenmojicolor{santa claus}{3156}
\showcaseopenmojicolor{sari}{3157}
\showcaseopenmojicolor{satellite antenna}{3158}
\showcaseopenmojicolor{satellite}{3159}
\showcaseopenmojicolor{sauropod}{3160}
\showcaseopenmojicolor{save}{3161}
\showcaseopenmojicolor{saw}{3162}
\showcaseopenmojicolor{saxophone}{3163}
\showcaseopenmojicolor{scale}{3164}
\showcaseopenmojicolor{scales}{3165}
\showcaseopenmojicolor{scarf}{3166}
\showcaseopenmojicolor{school}{3167}
\showcaseopenmojicolor{schwabisch gmund forum gold und silber}{3168}
\showcaseopenmojicolor{schwabisch gmund fanfknopfturm}{3169}
\showcaseopenmojicolor{schwabisch gmund ratshaus}{3170}
\showcaseopenmojicolor{scientist- dark skin tone}{3171}
\showcaseopenmojicolor{scientist- light skin tone}{3172}
\showcaseopenmojicolor{scientist- medium skin tone}{3173}
\showcaseopenmojicolor{scientist- medium-dark skin tone}{3174}
\showcaseopenmojicolor{scientist- medium-light skin tone}{3175}
\showcaseopenmojicolor{scientist}{3176}
\showcaseopenmojicolor{scissors}{3177}
\showcaseopenmojicolor{scorpio}{3178}
\showcaseopenmojicolor{scorpion}{3179}
\showcaseopenmojicolor{screwdriver}{3180}
\showcaseopenmojicolor{scroll horizontal}{3181}
\showcaseopenmojicolor{scroll}{3182}
\showcaseopenmojicolor{sea level rise}{3183}
\showcaseopenmojicolor{seal}{3184}
\showcaseopenmojicolor{seat}{3185}
\showcaseopenmojicolor{see-no-evil monkey}{3186}
\showcaseopenmojicolor{seedling}{3187}
\showcaseopenmojicolor{selfie- dark skin tone}{3188}
\showcaseopenmojicolor{selfie- light skin tone}{3189}
\showcaseopenmojicolor{selfie- medium skin tone}{3190}
\showcaseopenmojicolor{selfie- medium-dark skin tone}{3191}
\showcaseopenmojicolor{selfie- medium-light skin tone}{3192}
\showcaseopenmojicolor{selfie}{3193}
\showcaseopenmojicolor{service dog}{3194}
\showcaseopenmojicolor{service mark}{3195}
\showcaseopenmojicolor{seven o clock}{3196}
\showcaseopenmojicolor{seven-thirty}{3197}
\showcaseopenmojicolor{sewing needle}{3198}
\showcaseopenmojicolor{shaking face}{3199}
\showcaseopenmojicolor{shallow pan of food}{3200}
\showcaseopenmojicolor{shamrock}{3201}
\showcaseopenmojicolor{share}{3202}
\showcaseopenmojicolor{shark}{3203}
\showcaseopenmojicolor{shaved ice}{3204}
\showcaseopenmojicolor{sheaf of rice}{3205}
\showcaseopenmojicolor{shelter}{3206}
\showcaseopenmojicolor{shield}{3207}
\showcaseopenmojicolor{shinto shrine}{3208}
\showcaseopenmojicolor{ship}{3209}
\showcaseopenmojicolor{shooting star}{3210}
\showcaseopenmojicolor{shopping bags}{3211}
\showcaseopenmojicolor{shopping cart}{3212}
\showcaseopenmojicolor{shortcake}{3213}
\showcaseopenmojicolor{shorts}{3214}
\showcaseopenmojicolor{shower}{3215}
\showcaseopenmojicolor{shrimp}{3216}
\showcaseopenmojicolor{shuffle tracks button}{3217}
\showcaseopenmojicolor{shushing face}{3218}
\showcaseopenmojicolor{sign of the horns- dark skin tone}{3219}
\showcaseopenmojicolor{sign of the horns- light skin tone}{3220}
\showcaseopenmojicolor{sign of the horns- medium skin tone}{3221}
\showcaseopenmojicolor{sign of the horns- medium-dark skin tone}{3222}
\showcaseopenmojicolor{sign of the horns- medium-light skin tone}{3223}
\showcaseopenmojicolor{sign of the horns}{3224}
\showcaseopenmojicolor{signal}{3225}
\showcaseopenmojicolor{signpost}{3226}
\showcaseopenmojicolor{simple}{3227}
\showcaseopenmojicolor{singer- dark skin tone}{3228}
\showcaseopenmojicolor{singer- light skin tone}{3229}
\showcaseopenmojicolor{singer- medium skin tone}{3230}
\showcaseopenmojicolor{singer- medium-dark skin tone}{3231}
\showcaseopenmojicolor{singer- medium-light skin tone}{3232}
\showcaseopenmojicolor{singer}{3233}
\showcaseopenmojicolor{six o clock}{3234}
\showcaseopenmojicolor{six-thirty}{3235}
\showcaseopenmojicolor{skateboard}{3236}
\showcaseopenmojicolor{skier}{3237}
\showcaseopenmojicolor{skis}{3238}
\showcaseopenmojicolor{skull and crossbones}{3239}
\showcaseopenmojicolor{skull}{3240}
\showcaseopenmojicolor{skunk}{3241}
\showcaseopenmojicolor{sled}{3242}
\showcaseopenmojicolor{sleeping face}{3243}
\showcaseopenmojicolor{sleepy face}{3244}
\showcaseopenmojicolor{slightly frowning face}{3245}
\showcaseopenmojicolor{slightly smiling face}{3246}
\showcaseopenmojicolor{slot machine}{3247}
\showcaseopenmojicolor{sloth}{3248}
\showcaseopenmojicolor{small airplane}{3249}
\showcaseopenmojicolor{small blue diamond}{3250}
\showcaseopenmojicolor{small orange diamond}{3251}
\showcaseopenmojicolor{smartwatch}{3252}
\showcaseopenmojicolor{smiling cat with heart-eyes}{3253}
\showcaseopenmojicolor{smiling face with halo}{3254}
\showcaseopenmojicolor{smiling face with heart-eyes}{3255}
\showcaseopenmojicolor{smiling face with hearts}{3256}
\showcaseopenmojicolor{smiling face with horns}{3257}
\showcaseopenmojicolor{smiling face with open hands}{3258}
\showcaseopenmojicolor{smiling face with smiling eyes}{3259}
\showcaseopenmojicolor{smiling face with sunglasses}{3260}
\showcaseopenmojicolor{smiling face with tear}{3261}
\showcaseopenmojicolor{smiling face}{3262}
\showcaseopenmojicolor{smirking face}{3263}
\showcaseopenmojicolor{snail}{3264}
\showcaseopenmojicolor{snake}{3265}
\showcaseopenmojicolor{sneezing face}{3266}
\showcaseopenmojicolor{snow-capped mountain}{3267}
\showcaseopenmojicolor{snowboarder- dark skin tone}{3268}
\showcaseopenmojicolor{snowboarder- light skin tone}{3269}
\showcaseopenmojicolor{snowboarder- medium skin tone}{3270}
\showcaseopenmojicolor{snowboarder- medium-dark skin tone}{3271}
\showcaseopenmojicolor{snowboarder- medium-light skin tone}{3272}
\showcaseopenmojicolor{snowboarder}{3273}
\showcaseopenmojicolor{snowflake}{3274}
\showcaseopenmojicolor{snowman without snow}{3275}
\showcaseopenmojicolor{snowman}{3276}
\showcaseopenmojicolor{soap}{3277}
\showcaseopenmojicolor{soccer ball}{3278}
\showcaseopenmojicolor{social distancing}{3279}
\showcaseopenmojicolor{socks}{3280}
\showcaseopenmojicolor{soft ice cream}{3281}
\showcaseopenmojicolor{softball}{3282}
\showcaseopenmojicolor{solar cell}{3283}
\showcaseopenmojicolor{solar energy}{3284}
\showcaseopenmojicolor{soon arrow}{3285}
\showcaseopenmojicolor{sort}{3286}
\showcaseopenmojicolor{sos button}{3287}
\showcaseopenmojicolor{sos stencil}{3288}
\showcaseopenmojicolor{sound recording copyright}{3289}
\showcaseopenmojicolor{space shuttle}{3290}
\showcaseopenmojicolor{spade suit}{3291}
\showcaseopenmojicolor{spade}{3292}
\showcaseopenmojicolor{spaghetti}{3293}
\showcaseopenmojicolor{sparkle}{3294}
\showcaseopenmojicolor{sparkler}{3295}
\showcaseopenmojicolor{sparkles}{3296}
\showcaseopenmojicolor{sparkling heart}{3297}
\showcaseopenmojicolor{speak-no-evil monkey}{3298}
\showcaseopenmojicolor{speaker high volume}{3299}
\showcaseopenmojicolor{speaker low volume}{3300}
\showcaseopenmojicolor{speaker medium volume}{3301}
\showcaseopenmojicolor{speaking head}{3302}
\showcaseopenmojicolor{speech balloon}{3303}
\showcaseopenmojicolor{speedboat}{3304}
\showcaseopenmojicolor{spider web}{3305}
\showcaseopenmojicolor{spider}{3306}
\showcaseopenmojicolor{spiral calendar}{3307}
\showcaseopenmojicolor{spiral notepad}{3308}
\showcaseopenmojicolor{spiral shell}{3309}
\showcaseopenmojicolor{sponge}{3310}
\showcaseopenmojicolor{spoon}{3311}
\showcaseopenmojicolor{sport utility vehicle}{3312}
\showcaseopenmojicolor{sports medal}{3313}
\showcaseopenmojicolor{spouting whale}{3314}
\showcaseopenmojicolor{spouting-orca}{3315}
\showcaseopenmojicolor{spaetzlepresse}{3316}
\showcaseopenmojicolor{square with left half black}{3317}
\showcaseopenmojicolor{square with lower right diagonal black}{3318}
\showcaseopenmojicolor{square with right half black}{3319}
\showcaseopenmojicolor{square with upper left diagonal black}{3320}
\showcaseopenmojicolor{squid}{3321}
\showcaseopenmojicolor{squinting face with tongue}{3322}
\showcaseopenmojicolor{stadium}{3323}
\showcaseopenmojicolor{stairway}{3324}
\showcaseopenmojicolor{star and crescent}{3325}
\showcaseopenmojicolor{star of david}{3326}
\showcaseopenmojicolor{star with left half black}{3327}
\showcaseopenmojicolor{star with right half black}{3328}
\showcaseopenmojicolor{star-struck}{3329}
\showcaseopenmojicolor{star}{3330}
\showcaseopenmojicolor{station}{3331}
\showcaseopenmojicolor{statue of liberty}{3332}
\showcaseopenmojicolor{statue of zeus at olympia}{3333}
\showcaseopenmojicolor{steaming bowl}{3334}
\showcaseopenmojicolor{stethoscope}{3335}
\showcaseopenmojicolor{stick figure leaning left}{3336}
\showcaseopenmojicolor{stick figure leaning right}{3337}
\showcaseopenmojicolor{stick figure with arms raised}{3338}
\showcaseopenmojicolor{stick figure with dress and arms raised}{3339}
\showcaseopenmojicolor{stick figure with dress leaning left}{3340}
\showcaseopenmojicolor{stick figure with dress leaning right}{3341}
\showcaseopenmojicolor{stick figure with dress}{3342}
\showcaseopenmojicolor{stick figure}{3343}
\showcaseopenmojicolor{stomach}{3344}
\showcaseopenmojicolor{stop button}{3345}
\showcaseopenmojicolor{stop sign}{3346}
\showcaseopenmojicolor{stopwatch}{3347}
\showcaseopenmojicolor{straight ruler}{3348}
\showcaseopenmojicolor{strawberry}{3349}
\showcaseopenmojicolor{stretcher}{3350}
\showcaseopenmojicolor{structural fire}{3351}
\showcaseopenmojicolor{student- dark skin tone}{3352}
\showcaseopenmojicolor{student- light skin tone}{3353}
\showcaseopenmojicolor{student- medium skin tone}{3354}
\showcaseopenmojicolor{student- medium-dark skin tone}{3355}
\showcaseopenmojicolor{student- medium-light skin tone}{3356}
\showcaseopenmojicolor{student}{3357}
\showcaseopenmojicolor{studio microphone}{3358}
\showcaseopenmojicolor{stuffed flatbread}{3359}
\showcaseopenmojicolor{stuttgart fernsehturm}{3360}
\showcaseopenmojicolor{sun behind cloud}{3361}
\showcaseopenmojicolor{sun behind large cloud}{3362}
\showcaseopenmojicolor{sun behind rain cloud}{3363}
\showcaseopenmojicolor{sun behind small cloud}{3364}
\showcaseopenmojicolor{sun with face}{3365}
\showcaseopenmojicolor{sun}{3366}
\showcaseopenmojicolor{sunflower}{3367}
\showcaseopenmojicolor{sunglasses}{3368}
\showcaseopenmojicolor{sunrise over mountains}{3369}
\showcaseopenmojicolor{sunrise}{3370}
\showcaseopenmojicolor{sunset}{3371}
\showcaseopenmojicolor{superhero- dark skin tone}{3372}
\showcaseopenmojicolor{superhero- light skin tone}{3373}
\showcaseopenmojicolor{superhero- medium skin tone}{3374}
\showcaseopenmojicolor{superhero- medium-dark skin tone}{3375}
\showcaseopenmojicolor{superhero- medium-light skin tone}{3376}
\showcaseopenmojicolor{superhero}{3377}
\showcaseopenmojicolor{supervillain- dark skin tone}{3378}
\showcaseopenmojicolor{supervillain- light skin tone}{3379}
\showcaseopenmojicolor{supervillain- medium skin tone}{3380}
\showcaseopenmojicolor{supervillain- medium-dark skin tone}{3381}
\showcaseopenmojicolor{supervillain- medium-light skin tone}{3382}
\showcaseopenmojicolor{supervillain}{3383}
\showcaseopenmojicolor{surveillance}{3384}
\showcaseopenmojicolor{sushi}{3385}
\showcaseopenmojicolor{suspension railway}{3386}
\showcaseopenmojicolor{svg}{3387}
\showcaseopenmojicolor{swab pliers}{3388}
\showcaseopenmojicolor{swan}{3389}
\showcaseopenmojicolor{sweat droplets}{3390}
\showcaseopenmojicolor{swipe down}{3391}
\showcaseopenmojicolor{swipe left}{3392}
\showcaseopenmojicolor{swipe right}{3393}
\showcaseopenmojicolor{swipe up}{3394}
\showcaseopenmojicolor{swipe}{3395}
\showcaseopenmojicolor{switch}{3396}
\showcaseopenmojicolor{synagogue}{3397}
\showcaseopenmojicolor{syringe}{3398}
\showcaseopenmojicolor{t-rex}{3399}
\showcaseopenmojicolor{t-shirt}{3400}
\showcaseopenmojicolor{tablet}{3401}
\showcaseopenmojicolor{taco}{3402}
\showcaseopenmojicolor{takeout box}{3403}
\showcaseopenmojicolor{tamale}{3404}
\showcaseopenmojicolor{tanabata tree}{3405}
\showcaseopenmojicolor{tangerine}{3406}
\showcaseopenmojicolor{tap}{3407}
\showcaseopenmojicolor{tardis}{3408}
\showcaseopenmojicolor{taurus}{3409}
\showcaseopenmojicolor{taxi}{3410}
\showcaseopenmojicolor{teacher- dark skin tone}{3411}
\showcaseopenmojicolor{teacher- light skin tone}{3412}
\showcaseopenmojicolor{teacher- medium skin tone}{3413}
\showcaseopenmojicolor{teacher- medium-dark skin tone}{3414}
\showcaseopenmojicolor{teacher- medium-light skin tone}{3415}
\showcaseopenmojicolor{teacher}{3416}
\showcaseopenmojicolor{teacup without handle}{3417}
\showcaseopenmojicolor{teapot}{3418}
\showcaseopenmojicolor{tear-off calendar}{3419}
\showcaseopenmojicolor{technologist- dark skin tone}{3420}
\showcaseopenmojicolor{technologist- light skin tone}{3421}
\showcaseopenmojicolor{technologist- medium skin tone}{3422}
\showcaseopenmojicolor{technologist- medium-dark skin tone}{3423}
\showcaseopenmojicolor{technologist- medium-light skin tone}{3424}
\showcaseopenmojicolor{technologist}{3425}
\showcaseopenmojicolor{teddy bear}{3426}
\showcaseopenmojicolor{telephone receiver}{3427}
\showcaseopenmojicolor{telephone}{3428}
\showcaseopenmojicolor{telescope}{3429}
\showcaseopenmojicolor{television}{3430}
\showcaseopenmojicolor{temperature taking}{3431}
\showcaseopenmojicolor{temple of artemis at ephesus}{3432}
\showcaseopenmojicolor{ten o clock}{3433}
\showcaseopenmojicolor{ten-thirty}{3434}
\showcaseopenmojicolor{tennis}{3435}
\showcaseopenmojicolor{tent}{3436}
\showcaseopenmojicolor{test tube}{3437}
\showcaseopenmojicolor{texas flag}{3438}
\showcaseopenmojicolor{thermometer}{3439}
\showcaseopenmojicolor{thinking face}{3440}
\showcaseopenmojicolor{thong sandal}{3441}
\showcaseopenmojicolor{thought balloon}{3442}
\showcaseopenmojicolor{thread}{3443}
\showcaseopenmojicolor{three finger operation}{3444}
\showcaseopenmojicolor{three o clock}{3445}
\showcaseopenmojicolor{three-thirty}{3446}
\showcaseopenmojicolor{thumbs down- dark skin tone}{3447}
\showcaseopenmojicolor{thumbs down- light skin tone}{3448}
\showcaseopenmojicolor{thumbs down- medium skin tone}{3449}
\showcaseopenmojicolor{thumbs down- medium-dark skin tone}{3450}
\showcaseopenmojicolor{thumbs down- medium-light skin tone}{3451}
\showcaseopenmojicolor{thumbs down}{3452}
\showcaseopenmojicolor{thumbs up- dark skin tone}{3453}
\showcaseopenmojicolor{thumbs up- light skin tone}{3454}
\showcaseopenmojicolor{thumbs up- medium skin tone}{3455}
\showcaseopenmojicolor{thumbs up- medium-dark skin tone}{3456}
\showcaseopenmojicolor{thumbs up- medium-light skin tone}{3457}
\showcaseopenmojicolor{thumbs up}{3458}
\showcaseopenmojicolor{ticket}{3459}
\showcaseopenmojicolor{tiger face}{3460}
\showcaseopenmojicolor{tiger}{3461}
\showcaseopenmojicolor{timer clock}{3462}
\showcaseopenmojicolor{timer}{3463}
\showcaseopenmojicolor{tired face}{3464}
\showcaseopenmojicolor{toggle button state b}{3465}
\showcaseopenmojicolor{toggle button}{3466}
\showcaseopenmojicolor{toilet}{3467}
\showcaseopenmojicolor{tokyo tower}{3468}
\showcaseopenmojicolor{tomato}{3469}
\showcaseopenmojicolor{tongue}{3470}
\showcaseopenmojicolor{toolbox}{3471}
\showcaseopenmojicolor{tooth}{3472}
\showcaseopenmojicolor{toothbrush}{3473}
\showcaseopenmojicolor{top arrow}{3474}
\showcaseopenmojicolor{top hat}{3475}
\showcaseopenmojicolor{tornado}{3476}
\showcaseopenmojicolor{town}{3477}
\showcaseopenmojicolor{trackball}{3478}
\showcaseopenmojicolor{tractor}{3479}
\showcaseopenmojicolor{trade mark}{3480}
\showcaseopenmojicolor{train}{3481}
\showcaseopenmojicolor{tram car}{3482}
\showcaseopenmojicolor{tram}{3483}
\showcaseopenmojicolor{transgender flag}{3484}
\showcaseopenmojicolor{transgender symbol}{3485}
\showcaseopenmojicolor{transmission}{3486}
\showcaseopenmojicolor{triangular flag}{3487}
\showcaseopenmojicolor{triangular ruler}{3488}
\showcaseopenmojicolor{trident emblem}{3489}
\showcaseopenmojicolor{troll}{3490}
\showcaseopenmojicolor{trolleybus}{3491}
\showcaseopenmojicolor{trophy}{3492}
\showcaseopenmojicolor{tropical drink}{3493}
\showcaseopenmojicolor{tropical fish}{3494}
\showcaseopenmojicolor{trowel}{3495}
\showcaseopenmojicolor{true south (antarctica) flag}{3496}
\showcaseopenmojicolor{trump}{3497}
\showcaseopenmojicolor{trumpet}{3498}
\showcaseopenmojicolor{tsunami}{3499}
\showcaseopenmojicolor{tulip}{3500}
\showcaseopenmojicolor{tumbler glass}{3501}
\showcaseopenmojicolor{turkey}{3502}
\showcaseopenmojicolor{turtle}{3503}
\showcaseopenmojicolor{twelve o clock}{3504}
\showcaseopenmojicolor{twelve-thirty}{3505}
\showcaseopenmojicolor{twitter}{3506}
\showcaseopenmojicolor{two hearts}{3507}
\showcaseopenmojicolor{two o clock}{3508}
\showcaseopenmojicolor{two-hump camel}{3509}
\showcaseopenmojicolor{two-thirty}{3510}
\showcaseopenmojicolor{typescript}{3511}
\showcaseopenmojicolor{ubuntu}{3512}
\showcaseopenmojicolor{umbrella on ground}{3513}
\showcaseopenmojicolor{umbrella with rain drops}{3514}
\showcaseopenmojicolor{umbrella}{3515}
\showcaseopenmojicolor{unamused face}{3516}
\showcaseopenmojicolor{unicorn}{3517}
\showcaseopenmojicolor{united federation of planets flag (star trek)}{3518}
\showcaseopenmojicolor{unlocked}{3519}
\showcaseopenmojicolor{up arrow}{3520}
\showcaseopenmojicolor{up down black arrow}{3521}
\showcaseopenmojicolor{up! button}{3522}
\showcaseopenmojicolor{up-down arrow}{3523}
\showcaseopenmojicolor{up-left arrow}{3524}
\showcaseopenmojicolor{up-pointing triangle with left half black}{3525}
\showcaseopenmojicolor{up-pointing triangle with right half black}{3526}
\showcaseopenmojicolor{up-right arrow}{3527}
\showcaseopenmojicolor{upload}{3528}
\showcaseopenmojicolor{upside-down face}{3529}
\showcaseopenmojicolor{upwards button}{3530}
\showcaseopenmojicolor{valencian community flag}{3531}
\showcaseopenmojicolor{vampire- dark skin tone}{3532}
\showcaseopenmojicolor{vampire- light skin tone}{3533}
\showcaseopenmojicolor{vampire- medium skin tone}{3534}
\showcaseopenmojicolor{vampire- medium-dark skin tone}{3535}
\showcaseopenmojicolor{vampire- medium-light skin tone}{3536}
\showcaseopenmojicolor{vampire}{3537}
\showcaseopenmojicolor{vertical traffic light}{3538}
\showcaseopenmojicolor{vibration mode}{3539}
\showcaseopenmojicolor{victory hand- dark skin tone}{3540}
\showcaseopenmojicolor{victory hand- light skin tone}{3541}
\showcaseopenmojicolor{victory hand- medium skin tone}{3542}
\showcaseopenmojicolor{victory hand- medium-dark skin tone}{3543}
\showcaseopenmojicolor{victory hand- medium-light skin tone}{3544}
\showcaseopenmojicolor{victory hand}{3545}
\showcaseopenmojicolor{video camera}{3546}
\showcaseopenmojicolor{video game}{3547}
\showcaseopenmojicolor{videocassette}{3548}
\showcaseopenmojicolor{viennese coffee house}{3549}
\showcaseopenmojicolor{violin}{3550}
\showcaseopenmojicolor{virgo}{3551}
\showcaseopenmojicolor{virtual reality}{3552}
\showcaseopenmojicolor{volcano ashes}{3553}
\showcaseopenmojicolor{volcano eruption}{3554}
\showcaseopenmojicolor{volcano}{3555}
\showcaseopenmojicolor{volleyball}{3556}
\showcaseopenmojicolor{vs button}{3557}
\showcaseopenmojicolor{vulcan salute- dark skin tone}{3558}
\showcaseopenmojicolor{vulcan salute- light skin tone}{3559}
\showcaseopenmojicolor{vulcan salute- medium skin tone}{3560}
\showcaseopenmojicolor{vulcan salute- medium-dark skin tone}{3561}
\showcaseopenmojicolor{vulcan salute- medium-light skin tone}{3562}
\showcaseopenmojicolor{vulcan salute}{3563}
\showcaseopenmojicolor{waffle}{3564}
\showcaseopenmojicolor{waning crescent moon}{3565}
\showcaseopenmojicolor{waning gibbous moon}{3566}
\showcaseopenmojicolor{warning fire}{3567}
\showcaseopenmojicolor{warning strip right}{3568}
\showcaseopenmojicolor{warning strip}{3569}
\showcaseopenmojicolor{warning tsunami}{3570}
\showcaseopenmojicolor{warning volcano}{3571}
\showcaseopenmojicolor{warning}{3572}
\showcaseopenmojicolor{wash hands}{3573}
\showcaseopenmojicolor{washing machine}{3574}
\showcaseopenmojicolor{washington d}{3575}
\showcaseopenmojicolor{wastebasket}{3576}
\showcaseopenmojicolor{watch}{3577}
\showcaseopenmojicolor{water buffalo}{3578}
\showcaseopenmojicolor{water closet}{3579}
\showcaseopenmojicolor{water pistol}{3580}
\showcaseopenmojicolor{water wave}{3581}
\showcaseopenmojicolor{watermelon}{3582}
\showcaseopenmojicolor{waving hand- dark skin tone}{3583}
\showcaseopenmojicolor{waving hand- light skin tone}{3584}
\showcaseopenmojicolor{waving hand- medium skin tone}{3585}
\showcaseopenmojicolor{waving hand- medium-dark skin tone}{3586}
\showcaseopenmojicolor{waving hand- medium-light skin tone}{3587}
\showcaseopenmojicolor{waving hand}{3588}
\showcaseopenmojicolor{wavy dash}{3589}
\showcaseopenmojicolor{waxing crescent moon}{3590}
\showcaseopenmojicolor{waxing gibbous moon}{3591}
\showcaseopenmojicolor{weary cat}{3592}
\showcaseopenmojicolor{weary face}{3593}
\showcaseopenmojicolor{web syndication}{3594}
\showcaseopenmojicolor{webassembly}{3595}
\showcaseopenmojicolor{wedding}{3596}
\showcaseopenmojicolor{whale}{3597}
\showcaseopenmojicolor{wheel chair}{3598}
\showcaseopenmojicolor{wheel of dharma}{3599}
\showcaseopenmojicolor{wheel}{3600}
\showcaseopenmojicolor{wheelbarrow}{3601}
\showcaseopenmojicolor{wheelchair symbol}{3602}
\showcaseopenmojicolor{white cane}{3603}
\showcaseopenmojicolor{white circle}{3604}
\showcaseopenmojicolor{white exclamation mark}{3605}
\showcaseopenmojicolor{white flag}{3606}
\showcaseopenmojicolor{white flower}{3607}
\showcaseopenmojicolor{white hair}{3608}
\showcaseopenmojicolor{white heart}{3609}
\showcaseopenmojicolor{white hexagon}{3610}
\showcaseopenmojicolor{white large square}{3611}
\showcaseopenmojicolor{white medium square}{3612}
\showcaseopenmojicolor{white medium-small square}{3613}
\showcaseopenmojicolor{white pentagon}{3614}
\showcaseopenmojicolor{white question mark}{3615}
\showcaseopenmojicolor{white rectangle}{3616}
\showcaseopenmojicolor{white small square}{3617}
\showcaseopenmojicolor{white square button}{3618}
\showcaseopenmojicolor{white square}{3619}
\showcaseopenmojicolor{white vertical ellipse}{3620}
\showcaseopenmojicolor{wifi}{3621}
\showcaseopenmojicolor{wikidata}{3622}
\showcaseopenmojicolor{wild fire}{3623}
\showcaseopenmojicolor{wilted flower}{3624}
\showcaseopenmojicolor{wind chime}{3625}
\showcaseopenmojicolor{wind energy}{3626}
\showcaseopenmojicolor{wind face}{3627}
\showcaseopenmojicolor{window}{3628}
\showcaseopenmojicolor{windows}{3629}
\showcaseopenmojicolor{windsurfing}{3630}
\showcaseopenmojicolor{wine glass}{3631}
\showcaseopenmojicolor{wing}{3632}
\showcaseopenmojicolor{winking face with tongue}{3633}
\showcaseopenmojicolor{winking face}{3634}
\showcaseopenmojicolor{winrar}{3635}
\showcaseopenmojicolor{wire}{3636}
\showcaseopenmojicolor{wireframes}{3637}
\showcaseopenmojicolor{wireless}{3638}
\showcaseopenmojicolor{wolf}{3639}
\showcaseopenmojicolor{woman and man holding hands- dark skin tone, light skin tone}{3640}
\showcaseopenmojicolor{woman and man holding hands- dark skin tone, medium skin tone}{3641}
\showcaseopenmojicolor{woman and man holding hands- dark skin tone, medium-dark skin tone}{3642}
\showcaseopenmojicolor{woman and man holding hands- dark skin tone, medium-light skin tone}{3643}
\showcaseopenmojicolor{woman and man holding hands- dark skin tone}{3644}
\showcaseopenmojicolor{woman and man holding hands- light skin tone, dark skin tone}{3645}
\showcaseopenmojicolor{woman and man holding hands- light skin tone, medium skin tone}{3646}
\showcaseopenmojicolor{woman and man holding hands- light skin tone, medium-dark skin tone}{3647}
\showcaseopenmojicolor{woman and man holding hands- light skin tone, medium-light skin tone}{3648}
\showcaseopenmojicolor{woman and man holding hands- light skin tone}{3649}
\showcaseopenmojicolor{woman and man holding hands- medium skin tone, dark skin tone}{3650}
\showcaseopenmojicolor{woman and man holding hands- medium skin tone, light skin tone}{3651}
\showcaseopenmojicolor{woman and man holding hands- medium skin tone, medium-dark skin tone}{3652}
\showcaseopenmojicolor{woman and man holding hands- medium skin tone, medium-light skin tone}{3653}
\showcaseopenmojicolor{woman and man holding hands- medium skin tone}{3654}
\showcaseopenmojicolor{woman and man holding hands- medium-dark skin tone, dark skin tone}{3655}
\showcaseopenmojicolor{woman and man holding hands- medium-dark skin tone, light skin tone}{3656}
\showcaseopenmojicolor{woman and man holding hands- medium-dark skin tone, medium skin tone}{3657}
\showcaseopenmojicolor{woman and man holding hands- medium-dark skin tone, medium-light skin tone}{3658}
\showcaseopenmojicolor{woman and man holding hands- medium-dark skin tone}{3659}
\showcaseopenmojicolor{woman and man holding hands- medium-light skin tone, dark skin tone}{3660}
\showcaseopenmojicolor{woman and man holding hands- medium-light skin tone, light skin tone}{3661}
\showcaseopenmojicolor{woman and man holding hands- medium-light skin tone, medium skin tone}{3662}
\showcaseopenmojicolor{woman and man holding hands- medium-light skin tone, medium-dark skin tone}{3663}
\showcaseopenmojicolor{woman and man holding hands- medium-light skin tone}{3664}
\showcaseopenmojicolor{woman and man holding hands}{3665}
\showcaseopenmojicolor{woman artist- dark skin tone}{3666}
\showcaseopenmojicolor{woman artist- light skin tone}{3667}
\showcaseopenmojicolor{woman artist- medium skin tone}{3668}
\showcaseopenmojicolor{woman artist- medium-dark skin tone}{3669}
\showcaseopenmojicolor{woman artist- medium-light skin tone}{3670}
\showcaseopenmojicolor{woman artist}{3671}
\showcaseopenmojicolor{woman astronaut- dark skin tone}{3672}
\showcaseopenmojicolor{woman astronaut- light skin tone}{3673}
\showcaseopenmojicolor{woman astronaut- medium skin tone}{3674}
\showcaseopenmojicolor{woman astronaut- medium-dark skin tone}{3675}
\showcaseopenmojicolor{woman astronaut- medium-light skin tone}{3676}
\showcaseopenmojicolor{woman astronaut}{3677}
\showcaseopenmojicolor{woman barista}{3678}
\showcaseopenmojicolor{woman biking- dark skin tone}{3679}
\showcaseopenmojicolor{woman biking- light skin tone}{3680}
\showcaseopenmojicolor{woman biking- medium skin tone}{3681}
\showcaseopenmojicolor{woman biking- medium-dark skin tone}{3682}
\showcaseopenmojicolor{woman biking- medium-light skin tone}{3683}
\showcaseopenmojicolor{woman biking}{3684}
\showcaseopenmojicolor{woman bouncing ball- dark skin tone}{3685}
\showcaseopenmojicolor{woman bouncing ball- light skin tone}{3686}
\showcaseopenmojicolor{woman bouncing ball- medium skin tone}{3687}
\showcaseopenmojicolor{woman bouncing ball- medium-dark skin tone}{3688}
\showcaseopenmojicolor{woman bouncing ball- medium-light skin tone}{3689}
\showcaseopenmojicolor{woman bouncing ball}{3690}
\showcaseopenmojicolor{woman bowing- dark skin tone}{3691}
\showcaseopenmojicolor{woman bowing- light skin tone}{3692}
\showcaseopenmojicolor{woman bowing- medium skin tone}{3693}
\showcaseopenmojicolor{woman bowing- medium-dark skin tone}{3694}
\showcaseopenmojicolor{woman bowing- medium-light skin tone}{3695}
\showcaseopenmojicolor{woman bowing}{3696}
\showcaseopenmojicolor{woman cartwheeling- dark skin tone}{3697}
\showcaseopenmojicolor{woman cartwheeling- light skin tone}{3698}
\showcaseopenmojicolor{woman cartwheeling- medium skin tone}{3699}
\showcaseopenmojicolor{woman cartwheeling- medium-dark skin tone}{3700}
\showcaseopenmojicolor{woman cartwheeling- medium-light skin tone}{3701}
\showcaseopenmojicolor{woman cartwheeling}{3702}
\showcaseopenmojicolor{woman climbing- dark skin tone}{3703}
\showcaseopenmojicolor{woman climbing- light skin tone}{3704}
\showcaseopenmojicolor{woman climbing- medium skin tone}{3705}
\showcaseopenmojicolor{woman climbing- medium-dark skin tone}{3706}
\showcaseopenmojicolor{woman climbing- medium-light skin tone}{3707}
\showcaseopenmojicolor{woman climbing}{3708}
\showcaseopenmojicolor{woman construction worker- dark skin tone}{3709}
\showcaseopenmojicolor{woman construction worker- light skin tone}{3710}
\showcaseopenmojicolor{woman construction worker- medium skin tone}{3711}
\showcaseopenmojicolor{woman construction worker- medium-dark skin tone}{3712}
\showcaseopenmojicolor{woman construction worker- medium-light skin tone}{3713}
\showcaseopenmojicolor{woman construction worker}{3714}
\showcaseopenmojicolor{woman cook- dark skin tone}{3715}
\showcaseopenmojicolor{woman cook- light skin tone}{3716}
\showcaseopenmojicolor{woman cook- medium skin tone}{3717}
\showcaseopenmojicolor{woman cook- medium-dark skin tone}{3718}
\showcaseopenmojicolor{woman cook- medium-light skin tone}{3719}
\showcaseopenmojicolor{woman cook}{3720}
\showcaseopenmojicolor{woman dancing- dark skin tone}{3721}
\showcaseopenmojicolor{woman dancing- light skin tone}{3722}
\showcaseopenmojicolor{woman dancing- medium skin tone}{3723}
\showcaseopenmojicolor{woman dancing- medium-dark skin tone}{3724}
\showcaseopenmojicolor{woman dancing- medium-light skin tone}{3725}
\showcaseopenmojicolor{woman dancing}{3726}
\showcaseopenmojicolor{woman detective- dark skin tone}{3727}
\showcaseopenmojicolor{woman detective- light skin tone}{3728}
\showcaseopenmojicolor{woman detective- medium skin tone}{3729}
\showcaseopenmojicolor{woman detective- medium-dark skin tone}{3730}
\showcaseopenmojicolor{woman detective- medium-light skin tone}{3731}
\showcaseopenmojicolor{woman detective}{3732}
\showcaseopenmojicolor{woman elf- dark skin tone}{3733}
\showcaseopenmojicolor{woman elf- light skin tone}{3734}
\showcaseopenmojicolor{woman elf- medium skin tone}{3735}
\showcaseopenmojicolor{woman elf- medium-dark skin tone}{3736}
\showcaseopenmojicolor{woman elf- medium-light skin tone}{3737}
\showcaseopenmojicolor{woman elf}{3738}
\showcaseopenmojicolor{woman facepalming- dark skin tone}{3739}
\showcaseopenmojicolor{woman facepalming- light skin tone}{3740}
\showcaseopenmojicolor{woman facepalming- medium skin tone}{3741}
\showcaseopenmojicolor{woman facepalming- medium-dark skin tone}{3742}
\showcaseopenmojicolor{woman facepalming- medium-light skin tone}{3743}
\showcaseopenmojicolor{woman facepalming}{3744}
\showcaseopenmojicolor{woman factory worker- dark skin tone}{3745}
\showcaseopenmojicolor{woman factory worker- light skin tone}{3746}
\showcaseopenmojicolor{woman factory worker- medium skin tone}{3747}
\showcaseopenmojicolor{woman factory worker- medium-dark skin tone}{3748}
\showcaseopenmojicolor{woman factory worker- medium-light skin tone}{3749}
\showcaseopenmojicolor{woman factory worker}{3750}
\showcaseopenmojicolor{woman fairy- dark skin tone}{3751}
\showcaseopenmojicolor{woman fairy- light skin tone}{3752}
\showcaseopenmojicolor{woman fairy- medium skin tone}{3753}
\showcaseopenmojicolor{woman fairy- medium-dark skin tone}{3754}
\showcaseopenmojicolor{woman fairy- medium-light skin tone}{3755}
\showcaseopenmojicolor{woman fairy}{3756}
\showcaseopenmojicolor{woman farmer- dark skin tone}{3757}
\showcaseopenmojicolor{woman farmer- light skin tone}{3758}
\showcaseopenmojicolor{woman farmer- medium skin tone}{3759}
\showcaseopenmojicolor{woman farmer- medium-dark skin tone}{3760}
\showcaseopenmojicolor{woman farmer- medium-light skin tone}{3761}
\showcaseopenmojicolor{woman farmer}{3762}
\showcaseopenmojicolor{woman feeding baby- dark skin tone}{3763}
\showcaseopenmojicolor{woman feeding baby- light skin tone}{3764}
\showcaseopenmojicolor{woman feeding baby- medium skin tone}{3765}
\showcaseopenmojicolor{woman feeding baby- medium-dark skin tone}{3766}
\showcaseopenmojicolor{woman feeding baby- medium-light skin tone}{3767}
\showcaseopenmojicolor{woman feeding baby}{3768}
\showcaseopenmojicolor{woman firefighter- dark skin tone}{3769}
\showcaseopenmojicolor{woman firefighter- light skin tone}{3770}
\showcaseopenmojicolor{woman firefighter- medium skin tone}{3771}
\showcaseopenmojicolor{woman firefighter- medium-dark skin tone}{3772}
\showcaseopenmojicolor{woman firefighter- medium-light skin tone}{3773}
\showcaseopenmojicolor{woman firefighter}{3774}
\showcaseopenmojicolor{woman frowning- dark skin tone}{3775}
\showcaseopenmojicolor{woman frowning- light skin tone}{3776}
\showcaseopenmojicolor{woman frowning- medium skin tone}{3777}
\showcaseopenmojicolor{woman frowning- medium-dark skin tone}{3778}
\showcaseopenmojicolor{woman frowning- medium-light skin tone}{3779}
\showcaseopenmojicolor{woman frowning}{3780}
\showcaseopenmojicolor{woman genie}{3781}
\showcaseopenmojicolor{woman gesturing no- dark skin tone}{3782}
\showcaseopenmojicolor{woman gesturing no- light skin tone}{3783}
\showcaseopenmojicolor{woman gesturing no- medium skin tone}{3784}
\showcaseopenmojicolor{woman gesturing no- medium-dark skin tone}{3785}
\showcaseopenmojicolor{woman gesturing no- medium-light skin tone}{3786}
\showcaseopenmojicolor{woman gesturing no}{3787}
\showcaseopenmojicolor{woman gesturing ok- dark skin tone}{3788}
\showcaseopenmojicolor{woman gesturing ok- light skin tone}{3789}
\showcaseopenmojicolor{woman gesturing ok- medium skin tone}{3790}
\showcaseopenmojicolor{woman gesturing ok- medium-dark skin tone}{3791}
\showcaseopenmojicolor{woman gesturing ok- medium-light skin tone}{3792}
\showcaseopenmojicolor{woman gesturing ok}{3793}
\showcaseopenmojicolor{woman getting haircut- dark skin tone}{3794}
\showcaseopenmojicolor{woman getting haircut- light skin tone}{3795}
\showcaseopenmojicolor{woman getting haircut- medium skin tone}{3796}
\showcaseopenmojicolor{woman getting haircut- medium-dark skin tone}{3797}
\showcaseopenmojicolor{woman getting haircut- medium-light skin tone}{3798}
\showcaseopenmojicolor{woman getting haircut}{3799}
\showcaseopenmojicolor{woman getting massage- dark skin tone}{3800}
\showcaseopenmojicolor{woman getting massage- light skin tone}{3801}
\showcaseopenmojicolor{woman getting massage- medium skin tone}{3802}
\showcaseopenmojicolor{woman getting massage- medium-dark skin tone}{3803}
\showcaseopenmojicolor{woman getting massage- medium-light skin tone}{3804}
\showcaseopenmojicolor{woman getting massage}{3805}
\showcaseopenmojicolor{woman golfing- dark skin tone}{3806}
\showcaseopenmojicolor{woman golfing- light skin tone}{3807}
\showcaseopenmojicolor{woman golfing- medium skin tone}{3808}
\showcaseopenmojicolor{woman golfing- medium-dark skin tone}{3809}
\showcaseopenmojicolor{woman golfing- medium-light skin tone}{3810}
\showcaseopenmojicolor{woman golfing}{3811}
\showcaseopenmojicolor{woman guard- dark skin tone}{3812}
\showcaseopenmojicolor{woman guard- light skin tone}{3813}
\showcaseopenmojicolor{woman guard- medium skin tone}{3814}
\showcaseopenmojicolor{woman guard- medium-dark skin tone}{3815}
\showcaseopenmojicolor{woman guard- medium-light skin tone}{3816}
\showcaseopenmojicolor{woman guard}{3817}
\showcaseopenmojicolor{woman health worker- dark skin tone}{3818}
\showcaseopenmojicolor{woman health worker- light skin tone}{3819}
\showcaseopenmojicolor{woman health worker- medium skin tone}{3820}
\showcaseopenmojicolor{woman health worker- medium-dark skin tone}{3821}
\showcaseopenmojicolor{woman health worker- medium-light skin tone}{3822}
\showcaseopenmojicolor{woman health worker}{3823}
\showcaseopenmojicolor{woman in lotus position- dark skin tone}{3824}
\showcaseopenmojicolor{woman in lotus position- light skin tone}{3825}
\showcaseopenmojicolor{woman in lotus position- medium skin tone}{3826}
\showcaseopenmojicolor{woman in lotus position- medium-dark skin tone}{3827}
\showcaseopenmojicolor{woman in lotus position- medium-light skin tone}{3828}
\showcaseopenmojicolor{woman in lotus position}{3829}
\showcaseopenmojicolor{woman in manual wheelchair facing right}{3830}
\showcaseopenmojicolor{woman in manual wheelchair- dark skin tone}{3831}
\showcaseopenmojicolor{woman in manual wheelchair- light skin tone}{3832}
\showcaseopenmojicolor{woman in manual wheelchair- medium skin tone}{3833}
\showcaseopenmojicolor{woman in manual wheelchair- medium-dark skin tone}{3834}
\showcaseopenmojicolor{woman in manual wheelchair- medium-light skin tone}{3835}
\showcaseopenmojicolor{woman in manual wheelchair}{3836}
\showcaseopenmojicolor{woman in motorized wheelchair facing right}{3837}
\showcaseopenmojicolor{woman in motorized wheelchair- dark skin tone}{3838}
\showcaseopenmojicolor{woman in motorized wheelchair- light skin tone}{3839}
\showcaseopenmojicolor{woman in motorized wheelchair- medium skin tone}{3840}
\showcaseopenmojicolor{woman in motorized wheelchair- medium-dark skin tone}{3841}
\showcaseopenmojicolor{woman in motorized wheelchair- medium-light skin tone}{3842}
\showcaseopenmojicolor{woman in motorized wheelchair}{3843}
\showcaseopenmojicolor{woman in steamy room- dark skin tone}{3844}
\showcaseopenmojicolor{woman in steamy room- light skin tone}{3845}
\showcaseopenmojicolor{woman in steamy room- medium skin tone}{3846}
\showcaseopenmojicolor{woman in steamy room- medium-dark skin tone}{3847}
\showcaseopenmojicolor{woman in steamy room- medium-light skin tone}{3848}
\showcaseopenmojicolor{woman in steamy room}{3849}
\showcaseopenmojicolor{woman in tuxedo- dark skin tone}{3850}
\showcaseopenmojicolor{woman in tuxedo- light skin tone}{3851}
\showcaseopenmojicolor{woman in tuxedo- medium skin tone}{3852}
\showcaseopenmojicolor{woman in tuxedo- medium-dark skin tone}{3853}
\showcaseopenmojicolor{woman in tuxedo- medium-light skin tone}{3854}
\showcaseopenmojicolor{woman in tuxedo}{3855}
\showcaseopenmojicolor{woman judge- dark skin tone}{3856}
\showcaseopenmojicolor{woman judge- light skin tone}{3857}
\showcaseopenmojicolor{woman judge- medium skin tone}{3858}
\showcaseopenmojicolor{woman judge- medium-dark skin tone}{3859}
\showcaseopenmojicolor{woman judge- medium-light skin tone}{3860}
\showcaseopenmojicolor{woman judge}{3861}
\showcaseopenmojicolor{woman juggling- dark skin tone}{3862}
\showcaseopenmojicolor{woman juggling- light skin tone}{3863}
\showcaseopenmojicolor{woman juggling- medium skin tone}{3864}
\showcaseopenmojicolor{woman juggling- medium-dark skin tone}{3865}
\showcaseopenmojicolor{woman juggling- medium-light skin tone}{3866}
\showcaseopenmojicolor{woman juggling}{3867}
\showcaseopenmojicolor{woman kneeling facing right}{3868}
\showcaseopenmojicolor{woman kneeling- dark skin tone}{3869}
\showcaseopenmojicolor{woman kneeling- light skin tone}{3870}
\showcaseopenmojicolor{woman kneeling- medium skin tone}{3871}
\showcaseopenmojicolor{woman kneeling- medium-dark skin tone}{3872}
\showcaseopenmojicolor{woman kneeling- medium-light skin tone}{3873}
\showcaseopenmojicolor{woman kneeling}{3874}
\showcaseopenmojicolor{woman lifting weights- dark skin tone}{3875}
\showcaseopenmojicolor{woman lifting weights- light skin tone}{3876}
\showcaseopenmojicolor{woman lifting weights- medium skin tone}{3877}
\showcaseopenmojicolor{woman lifting weights- medium-dark skin tone}{3878}
\showcaseopenmojicolor{woman lifting weights- medium-light skin tone}{3879}
\showcaseopenmojicolor{woman lifting weights}{3880}
\showcaseopenmojicolor{woman mage- dark skin tone}{3881}
\showcaseopenmojicolor{woman mage- light skin tone}{3882}
\showcaseopenmojicolor{woman mage- medium skin tone}{3883}
\showcaseopenmojicolor{woman mage- medium-dark skin tone}{3884}
\showcaseopenmojicolor{woman mage- medium-light skin tone}{3885}
\showcaseopenmojicolor{woman mage}{3886}
\showcaseopenmojicolor{woman mechanic- dark skin tone}{3887}
\showcaseopenmojicolor{woman mechanic- light skin tone}{3888}
\showcaseopenmojicolor{woman mechanic- medium skin tone}{3889}
\showcaseopenmojicolor{woman mechanic- medium-dark skin tone}{3890}
\showcaseopenmojicolor{woman mechanic- medium-light skin tone}{3891}
\showcaseopenmojicolor{woman mechanic}{3892}
\showcaseopenmojicolor{woman mountain biking- dark skin tone}{3893}
\showcaseopenmojicolor{woman mountain biking- light skin tone}{3894}
\showcaseopenmojicolor{woman mountain biking- medium skin tone}{3895}
\showcaseopenmojicolor{woman mountain biking- medium-dark skin tone}{3896}
\showcaseopenmojicolor{woman mountain biking- medium-light skin tone}{3897}
\showcaseopenmojicolor{woman mountain biking}{3898}
\showcaseopenmojicolor{woman office worker- dark skin tone}{3899}
\showcaseopenmojicolor{woman office worker- light skin tone}{3900}
\showcaseopenmojicolor{woman office worker- medium skin tone}{3901}
\showcaseopenmojicolor{woman office worker- medium-dark skin tone}{3902}
\showcaseopenmojicolor{woman office worker- medium-light skin tone}{3903}
\showcaseopenmojicolor{woman office worker}{3904}
\showcaseopenmojicolor{woman pilot- dark skin tone}{3905}
\showcaseopenmojicolor{woman pilot- light skin tone}{3906}
\showcaseopenmojicolor{woman pilot- medium skin tone}{3907}
\showcaseopenmojicolor{woman pilot- medium-dark skin tone}{3908}
\showcaseopenmojicolor{woman pilot- medium-light skin tone}{3909}
\showcaseopenmojicolor{woman pilot}{3910}
\showcaseopenmojicolor{woman playing handball- dark skin tone}{3911}
\showcaseopenmojicolor{woman playing handball- light skin tone}{3912}
\showcaseopenmojicolor{woman playing handball- medium skin tone}{3913}
\showcaseopenmojicolor{woman playing handball- medium-dark skin tone}{3914}
\showcaseopenmojicolor{woman playing handball- medium-light skin tone}{3915}
\showcaseopenmojicolor{woman playing handball}{3916}
\showcaseopenmojicolor{woman playing water polo- dark skin tone}{3917}
\showcaseopenmojicolor{woman playing water polo- light skin tone}{3918}
\showcaseopenmojicolor{woman playing water polo- medium skin tone}{3919}
\showcaseopenmojicolor{woman playing water polo- medium-dark skin tone}{3920}
\showcaseopenmojicolor{woman playing water polo- medium-light skin tone}{3921}
\showcaseopenmojicolor{woman playing water polo}{3922}
\showcaseopenmojicolor{woman police officer- dark skin tone}{3923}
\showcaseopenmojicolor{woman police officer- light skin tone}{3924}
\showcaseopenmojicolor{woman police officer- medium skin tone}{3925}
\showcaseopenmojicolor{woman police officer- medium-dark skin tone}{3926}
\showcaseopenmojicolor{woman police officer- medium-light skin tone}{3927}
\showcaseopenmojicolor{woman police officer}{3928}
\showcaseopenmojicolor{woman pouting- dark skin tone}{3929}
\showcaseopenmojicolor{woman pouting- light skin tone}{3930}
\showcaseopenmojicolor{woman pouting- medium skin tone}{3931}
\showcaseopenmojicolor{woman pouting- medium-dark skin tone}{3932}
\showcaseopenmojicolor{woman pouting- medium-light skin tone}{3933}
\showcaseopenmojicolor{woman pouting}{3934}
\showcaseopenmojicolor{woman raising hand- dark skin tone}{3935}
\showcaseopenmojicolor{woman raising hand- light skin tone}{3936}
\showcaseopenmojicolor{woman raising hand- medium skin tone}{3937}
\showcaseopenmojicolor{woman raising hand- medium-dark skin tone}{3938}
\showcaseopenmojicolor{woman raising hand- medium-light skin tone}{3939}
\showcaseopenmojicolor{woman raising hand}{3940}
\showcaseopenmojicolor{woman rowing boat- dark skin tone}{3941}
\showcaseopenmojicolor{woman rowing boat- light skin tone}{3942}
\showcaseopenmojicolor{woman rowing boat- medium skin tone}{3943}
\showcaseopenmojicolor{woman rowing boat- medium-dark skin tone}{3944}
\showcaseopenmojicolor{woman rowing boat- medium-light skin tone}{3945}
\showcaseopenmojicolor{woman rowing boat}{3946}
\showcaseopenmojicolor{woman running facing right}{3947}
\showcaseopenmojicolor{woman running- dark skin tone}{3948}
\showcaseopenmojicolor{woman running- light skin tone}{3949}
\showcaseopenmojicolor{woman running- medium skin tone}{3950}
\showcaseopenmojicolor{woman running- medium-dark skin tone}{3951}
\showcaseopenmojicolor{woman running- medium-light skin tone}{3952}
\showcaseopenmojicolor{woman running}{3953}
\showcaseopenmojicolor{woman scientist- dark skin tone}{3954}
\showcaseopenmojicolor{woman scientist- light skin tone}{3955}
\showcaseopenmojicolor{woman scientist- medium skin tone}{3956}
\showcaseopenmojicolor{woman scientist- medium-dark skin tone}{3957}
\showcaseopenmojicolor{woman scientist- medium-light skin tone}{3958}
\showcaseopenmojicolor{woman scientist}{3959}
\showcaseopenmojicolor{woman shrugging- dark skin tone}{3960}
\showcaseopenmojicolor{woman shrugging- light skin tone}{3961}
\showcaseopenmojicolor{woman shrugging- medium skin tone}{3962}
\showcaseopenmojicolor{woman shrugging- medium-dark skin tone}{3963}
\showcaseopenmojicolor{woman shrugging- medium-light skin tone}{3964}
\showcaseopenmojicolor{woman shrugging}{3965}
\showcaseopenmojicolor{woman singer- dark skin tone}{3966}
\showcaseopenmojicolor{woman singer- light skin tone}{3967}
\showcaseopenmojicolor{woman singer- medium skin tone}{3968}
\showcaseopenmojicolor{woman singer- medium-dark skin tone}{3969}
\showcaseopenmojicolor{woman singer- medium-light skin tone}{3970}
\showcaseopenmojicolor{woman singer}{3971}
\showcaseopenmojicolor{woman sneezing into elbow}{3972}
\showcaseopenmojicolor{woman standing- dark skin tone}{3973}
\showcaseopenmojicolor{woman standing- light skin tone}{3974}
\showcaseopenmojicolor{woman standing- medium skin tone}{3975}
\showcaseopenmojicolor{woman standing- medium-dark skin tone}{3976}
\showcaseopenmojicolor{woman standing- medium-light skin tone}{3977}
\showcaseopenmojicolor{woman standing}{3978}
\showcaseopenmojicolor{woman student- dark skin tone}{3979}
\showcaseopenmojicolor{woman student- light skin tone}{3980}
\showcaseopenmojicolor{woman student- medium skin tone}{3981}
\showcaseopenmojicolor{woman student- medium-dark skin tone}{3982}
\showcaseopenmojicolor{woman student- medium-light skin tone}{3983}
\showcaseopenmojicolor{woman student}{3984}
\showcaseopenmojicolor{woman superhero- dark skin tone}{3985}
\showcaseopenmojicolor{woman superhero- light skin tone}{3986}
\showcaseopenmojicolor{woman superhero- medium skin tone}{3987}
\showcaseopenmojicolor{woman superhero- medium-dark skin tone}{3988}
\showcaseopenmojicolor{woman superhero- medium-light skin tone}{3989}
\showcaseopenmojicolor{woman superhero}{3990}
\showcaseopenmojicolor{woman supervillain- dark skin tone}{3991}
\showcaseopenmojicolor{woman supervillain- light skin tone}{3992}
\showcaseopenmojicolor{woman supervillain- medium skin tone}{3993}
\showcaseopenmojicolor{woman supervillain- medium-dark skin tone}{3994}
\showcaseopenmojicolor{woman supervillain- medium-light skin tone}{3995}
\showcaseopenmojicolor{woman supervillain}{3996}
\showcaseopenmojicolor{woman surfing- dark skin tone}{3997}
\showcaseopenmojicolor{woman surfing- light skin tone}{3998}
\showcaseopenmojicolor{woman surfing- medium skin tone}{3999}
\showcaseopenmojicolor{woman surfing- medium-dark skin tone}{4000}
\showcaseopenmojicolor{woman surfing- medium-light skin tone}{4001}
\showcaseopenmojicolor{woman surfing}{4002}
\showcaseopenmojicolor{woman swimming- dark skin tone}{4003}
\showcaseopenmojicolor{woman swimming- light skin tone}{4004}
\showcaseopenmojicolor{woman swimming- medium skin tone}{4005}
\showcaseopenmojicolor{woman swimming- medium-dark skin tone}{4006}
\showcaseopenmojicolor{woman swimming- medium-light skin tone}{4007}
\showcaseopenmojicolor{woman swimming}{4008}
\showcaseopenmojicolor{woman teacher- dark skin tone}{4009}
\showcaseopenmojicolor{woman teacher- light skin tone}{4010}
\showcaseopenmojicolor{woman teacher- medium skin tone}{4011}
\showcaseopenmojicolor{woman teacher- medium-dark skin tone}{4012}
\showcaseopenmojicolor{woman teacher- medium-light skin tone}{4013}
\showcaseopenmojicolor{woman teacher}{4014}
\showcaseopenmojicolor{woman technologist- dark skin tone}{4015}
\showcaseopenmojicolor{woman technologist- light skin tone}{4016}
\showcaseopenmojicolor{woman technologist- medium skin tone}{4017}
\showcaseopenmojicolor{woman technologist- medium-dark skin tone}{4018}
\showcaseopenmojicolor{woman technologist- medium-light skin tone}{4019}
\showcaseopenmojicolor{woman technologist}{4020}
\showcaseopenmojicolor{woman tipping hand- dark skin tone}{4021}
\showcaseopenmojicolor{woman tipping hand- light skin tone}{4022}
\showcaseopenmojicolor{woman tipping hand- medium skin tone}{4023}
\showcaseopenmojicolor{woman tipping hand- medium-dark skin tone}{4024}
\showcaseopenmojicolor{woman tipping hand- medium-light skin tone}{4025}
\showcaseopenmojicolor{woman tipping hand}{4026}
\showcaseopenmojicolor{woman vampire- dark skin tone}{4027}
\showcaseopenmojicolor{woman vampire- light skin tone}{4028}
\showcaseopenmojicolor{woman vampire- medium skin tone}{4029}
\showcaseopenmojicolor{woman vampire- medium-dark skin tone}{4030}
\showcaseopenmojicolor{woman vampire- medium-light skin tone}{4031}
\showcaseopenmojicolor{woman vampire}{4032}
\showcaseopenmojicolor{woman walking facing right}{4033}
\showcaseopenmojicolor{woman walking- dark skin tone}{4034}
\showcaseopenmojicolor{woman walking- light skin tone}{4035}
\showcaseopenmojicolor{woman walking- medium skin tone}{4036}
\showcaseopenmojicolor{woman walking- medium-dark skin tone}{4037}
\showcaseopenmojicolor{woman walking- medium-light skin tone}{4038}
\showcaseopenmojicolor{woman walking}{4039}
\showcaseopenmojicolor{woman wearing turban- dark skin tone}{4040}
\showcaseopenmojicolor{woman wearing turban- light skin tone}{4041}
\showcaseopenmojicolor{woman wearing turban- medium skin tone}{4042}
\showcaseopenmojicolor{woman wearing turban- medium-dark skin tone}{4043}
\showcaseopenmojicolor{woman wearing turban- medium-light skin tone}{4044}
\showcaseopenmojicolor{woman wearing turban}{4045}
\showcaseopenmojicolor{woman with headscarf- dark skin tone}{4046}
\showcaseopenmojicolor{woman with headscarf- light skin tone}{4047}
\showcaseopenmojicolor{woman with headscarf- medium skin tone}{4048}
\showcaseopenmojicolor{woman with headscarf- medium-dark skin tone}{4049}
\showcaseopenmojicolor{woman with headscarf- medium-light skin tone}{4050}
\showcaseopenmojicolor{woman with headscarf}{4051}
\showcaseopenmojicolor{woman with medical mask}{4052}
\showcaseopenmojicolor{woman with veil- dark skin tone}{4053}
\showcaseopenmojicolor{woman with veil- light skin tone}{4054}
\showcaseopenmojicolor{woman with veil- medium skin tone}{4055}
\showcaseopenmojicolor{woman with veil- medium-dark skin tone}{4056}
\showcaseopenmojicolor{woman with veil- medium-light skin tone}{4057}
\showcaseopenmojicolor{woman with veil}{4058}
\showcaseopenmojicolor{woman with white cane facing right}{4059}
\showcaseopenmojicolor{woman with white cane- dark skin tone}{4060}
\showcaseopenmojicolor{woman with white cane- light skin tone}{4061}
\showcaseopenmojicolor{woman with white cane- medium skin tone}{4062}
\showcaseopenmojicolor{woman with white cane- medium-dark skin tone}{4063}
\showcaseopenmojicolor{woman with white cane- medium-light skin tone}{4064}
\showcaseopenmojicolor{woman with white cane}{4065}
\showcaseopenmojicolor{woman zombie}{4066}
\showcaseopenmojicolor{woman- bald}{4067}
\showcaseopenmojicolor{woman- beard}{4068}
\showcaseopenmojicolor{woman- blond hair}{4069}
\showcaseopenmojicolor{woman- curly hair}{4070}
\showcaseopenmojicolor{woman- dark skin tone, bald}{4071}
\showcaseopenmojicolor{woman- dark skin tone, beard}{4072}
\showcaseopenmojicolor{woman- dark skin tone, blond hair}{4073}
\showcaseopenmojicolor{woman- dark skin tone, curly hair}{4074}
\showcaseopenmojicolor{woman- dark skin tone, red hair}{4075}
\showcaseopenmojicolor{woman- dark skin tone, white hair}{4076}
\showcaseopenmojicolor{woman- dark skin tone}{4077}
\showcaseopenmojicolor{woman- light skin tone, bald}{4078}
\showcaseopenmojicolor{woman- light skin tone, beard}{4079}
\showcaseopenmojicolor{woman- light skin tone, blond hair}{4080}
\showcaseopenmojicolor{woman- light skin tone, curly hair}{4081}
\showcaseopenmojicolor{woman- light skin tone, red hair}{4082}
\showcaseopenmojicolor{woman- light skin tone, white hair}{4083}
\showcaseopenmojicolor{woman- light skin tone}{4084}
\showcaseopenmojicolor{woman- medium skin tone, bald}{4085}
\showcaseopenmojicolor{woman- medium skin tone, beard}{4086}
\showcaseopenmojicolor{woman- medium skin tone, blond hair}{4087}
\showcaseopenmojicolor{woman- medium skin tone, curly hair}{4088}
\showcaseopenmojicolor{woman- medium skin tone, red hair}{4089}
\showcaseopenmojicolor{woman- medium skin tone, white hair}{4090}
\showcaseopenmojicolor{woman- medium skin tone}{4091}
\showcaseopenmojicolor{woman- medium-dark skin tone, bald}{4092}
\showcaseopenmojicolor{woman- medium-dark skin tone, beard}{4093}
\showcaseopenmojicolor{woman- medium-dark skin tone, blond hair}{4094}
\showcaseopenmojicolor{woman- medium-dark skin tone, curly hair}{4095}
\showcaseopenmojicolor{woman- medium-dark skin tone, red hair}{4096}
\showcaseopenmojicolor{woman- medium-dark skin tone, white hair}{4097}
\showcaseopenmojicolor{woman- medium-dark skin tone}{4098}
\showcaseopenmojicolor{woman- medium-light skin tone, bald}{4099}
\showcaseopenmojicolor{woman- medium-light skin tone, beard}{4100}
\showcaseopenmojicolor{woman- medium-light skin tone, blond hair}{4101}
\showcaseopenmojicolor{woman- medium-light skin tone, curly hair}{4102}
\showcaseopenmojicolor{woman- medium-light skin tone, red hair}{4103}
\showcaseopenmojicolor{woman- medium-light skin tone, white hair}{4104}
\showcaseopenmojicolor{woman- medium-light skin tone}{4105}
\showcaseopenmojicolor{woman- red hair}{4106}
\showcaseopenmojicolor{woman- white hair}{4107}
\showcaseopenmojicolor{woman}{4108}
\showcaseopenmojicolor{woman s boot}{4109}
\showcaseopenmojicolor{woman s clothes}{4110}
\showcaseopenmojicolor{woman s hat}{4111}
\showcaseopenmojicolor{woman s sandal}{4112}
\showcaseopenmojicolor{women holding hands- dark skin tone, light skin tone}{4113}
\showcaseopenmojicolor{women holding hands- dark skin tone, medium skin tone}{4114}
\showcaseopenmojicolor{women holding hands- dark skin tone, medium-dark skin tone}{4115}
\showcaseopenmojicolor{women holding hands- dark skin tone, medium-light skin tone}{4116}
\showcaseopenmojicolor{women holding hands- dark skin tone}{4117}
\showcaseopenmojicolor{women holding hands- light skin tone, dark skin tone}{4118}
\showcaseopenmojicolor{women holding hands- light skin tone, medium skin tone}{4119}
\showcaseopenmojicolor{women holding hands- light skin tone, medium-dark skin tone}{4120}
\showcaseopenmojicolor{women holding hands- light skin tone, medium-light skin tone}{4121}
\showcaseopenmojicolor{women holding hands- light skin tone}{4122}
\showcaseopenmojicolor{women holding hands- medium skin tone, dark skin tone}{4123}
\showcaseopenmojicolor{women holding hands- medium skin tone, light skin tone}{4124}
\showcaseopenmojicolor{women holding hands- medium skin tone, medium-dark skin tone}{4125}
\showcaseopenmojicolor{women holding hands- medium skin tone, medium-light skin tone}{4126}
\showcaseopenmojicolor{women holding hands- medium skin tone}{4127}
\showcaseopenmojicolor{women holding hands- medium-dark skin tone, dark skin tone}{4128}
\showcaseopenmojicolor{women holding hands- medium-dark skin tone, light skin tone}{4129}
\showcaseopenmojicolor{women holding hands- medium-dark skin tone, medium skin tone}{4130}
\showcaseopenmojicolor{women holding hands- medium-dark skin tone, medium-light skin tone}{4131}
\showcaseopenmojicolor{women holding hands- medium-dark skin tone}{4132}
\showcaseopenmojicolor{women holding hands- medium-light skin tone, dark skin tone}{4133}
\showcaseopenmojicolor{women holding hands- medium-light skin tone, light skin tone}{4134}
\showcaseopenmojicolor{women holding hands- medium-light skin tone, medium skin tone}{4135}
\showcaseopenmojicolor{women holding hands- medium-light skin tone, medium-dark skin tone}{4136}
\showcaseopenmojicolor{women holding hands- medium-light skin tone}{4137}
\showcaseopenmojicolor{women holding hands}{4138}
\showcaseopenmojicolor{women with bunny ears}{4139}
\showcaseopenmojicolor{women wrestling}{4140}
\showcaseopenmojicolor{women s room}{4141}
\showcaseopenmojicolor{wood}{4142}
\showcaseopenmojicolor{woozy face}{4143}
\showcaseopenmojicolor{world map}{4144}
\showcaseopenmojicolor{worm}{4145}
\showcaseopenmojicolor{worried face}{4146}
\showcaseopenmojicolor{wrapped gift}{4147}
\showcaseopenmojicolor{wrench}{4148}
\showcaseopenmojicolor{writing hand- dark skin tone}{4149}
\showcaseopenmojicolor{writing hand- light skin tone}{4150}
\showcaseopenmojicolor{writing hand- medium skin tone}{4151}
\showcaseopenmojicolor{writing hand- medium-dark skin tone}{4152}
\showcaseopenmojicolor{writing hand- medium-light skin tone}{4153}
\showcaseopenmojicolor{writing hand}{4154}
\showcaseopenmojicolor{x-ray}{4155}
\showcaseopenmojicolor{yarn}{4156}
\showcaseopenmojicolor{yawning face}{4157}
\showcaseopenmojicolor{yellow circle}{4158}
\showcaseopenmojicolor{yellow flag}{4159}
\showcaseopenmojicolor{yellow heart}{4160}
\showcaseopenmojicolor{yellow hexagon}{4161}
\showcaseopenmojicolor{yellow square}{4162}
\showcaseopenmojicolor{yen banknote}{4163}
\showcaseopenmojicolor{yin yang}{4164}
\showcaseopenmojicolor{yo-yo}{4165}
\showcaseopenmojicolor{youtube}{4166}
\showcaseopenmojicolor{zany face}{4167}
\showcaseopenmojicolor{zebra}{4168}
\showcaseopenmojicolor{zipper-mouth face}{4169}
\showcaseopenmojicolor{zombie}{4170}
\showcaseopenmojicolor{zzz}{4171}

\end{openmojicase}

\pagebreak

\subsection{Black version (4171 icons)}

The \textsf{PDF} file is \texttt{openmoji-black-all.pdf}

Direct command is: {\ttfamily\textbackslash openmoji[black]\{<Name>\}}.

\textit{Some are duplicates, due to the tints of the colored versions, or due to flags, it's not a bug\ldots}

\begin{openmojicase}
\showcaseopenmojiblack{1st place medal}{1}
\showcaseopenmojiblack{2nd place medal}{2}
\showcaseopenmojiblack{3rd place medal}{3}
\showcaseopenmojiblack{a button (blood type)}{4}
\showcaseopenmojiblack{ab button (blood type)}{5}
\showcaseopenmojiblack{abacus}{6}
\showcaseopenmojiblack{accordion}{7}
\showcaseopenmojiblack{add button}{8}
\showcaseopenmojiblack{add contact}{9}
\showcaseopenmojiblack{adhesive bandage}{10}
\showcaseopenmojiblack{admission tickets}{11}
\showcaseopenmojiblack{aerial tramway}{12}
\showcaseopenmojiblack{airplane arrival}{13}
\showcaseopenmojiblack{airplane departure}{14}
\showcaseopenmojiblack{airplane}{15}
\showcaseopenmojiblack{alabama flag}{16}
\showcaseopenmojiblack{alarm clock}{17}
\showcaseopenmojiblack{alembic}{18}
\showcaseopenmojiblack{alien monster}{19}
\showcaseopenmojiblack{alien}{20}
\showcaseopenmojiblack{ambulance}{21}
\showcaseopenmojiblack{american football}{22}
\showcaseopenmojiblack{amphora}{23}
\showcaseopenmojiblack{anatomical heart}{24}
\showcaseopenmojiblack{anchor}{25}
\showcaseopenmojiblack{andalusia flag}{26}
\showcaseopenmojiblack{android}{27}
\showcaseopenmojiblack{anger symbol}{28}
\showcaseopenmojiblack{angry face with horns}{29}
\showcaseopenmojiblack{angry face}{30}
\showcaseopenmojiblack{anguished face}{31}
\showcaseopenmojiblack{annoyed face with tongue}{32}
\showcaseopenmojiblack{ant}{33}
\showcaseopenmojiblack{antenna bars}{34}
\showcaseopenmojiblack{anticlockwise triangle-headed top u-shaped arrow}{35}
\showcaseopenmojiblack{anxious face with sweat}{36}
\showcaseopenmojiblack{apple}{37}
\showcaseopenmojiblack{aquarius}{38}
\showcaseopenmojiblack{aragon flag}{39}
\showcaseopenmojiblack{archive}{40}
\showcaseopenmojiblack{arduino}{41}
\showcaseopenmojiblack{aries}{42}
\showcaseopenmojiblack{armchair}{43}
\showcaseopenmojiblack{arrow turn right}{44}
\showcaseopenmojiblack{articulated lorry}{45}
\showcaseopenmojiblack{artist palette}{46}
\showcaseopenmojiblack{artist- dark skin tone}{47}
\showcaseopenmojiblack{artist- light skin tone}{48}
\showcaseopenmojiblack{artist- medium skin tone}{49}
\showcaseopenmojiblack{artist- medium-dark skin tone}{50}
\showcaseopenmojiblack{artist- medium-light skin tone}{51}
\showcaseopenmojiblack{artist}{52}
\showcaseopenmojiblack{artstation}{53}
\showcaseopenmojiblack{assembly group}{54}
\showcaseopenmojiblack{assembly point}{55}
\showcaseopenmojiblack{astonished face}{56}
\showcaseopenmojiblack{astronaut- dark skin tone}{57}
\showcaseopenmojiblack{astronaut- light skin tone}{58}
\showcaseopenmojiblack{astronaut- medium skin tone}{59}
\showcaseopenmojiblack{astronaut- medium-dark skin tone}{60}
\showcaseopenmojiblack{astronaut- medium-light skin tone}{61}
\showcaseopenmojiblack{astronaut}{62}
\showcaseopenmojiblack{asturian flag}{63}
\showcaseopenmojiblack{atm sign}{64}
\showcaseopenmojiblack{atom bomb}{65}
\showcaseopenmojiblack{atom symbol}{66}
\showcaseopenmojiblack{augmented reality}{67}
\showcaseopenmojiblack{authority building}{68}
\showcaseopenmojiblack{authority instruction}{69}
\showcaseopenmojiblack{authority}{70}
\showcaseopenmojiblack{auto rickshaw}{71}
\showcaseopenmojiblack{automobile}{72}
\showcaseopenmojiblack{autonomous car}{73}
\showcaseopenmojiblack{avalanche}{74}
\showcaseopenmojiblack{avocado}{75}
\showcaseopenmojiblack{axe}{76}
\showcaseopenmojiblack{b button (blood type)}{77}
\showcaseopenmojiblack{baby angel- dark skin tone}{78}
\showcaseopenmojiblack{baby angel- light skin tone}{79}
\showcaseopenmojiblack{baby angel- medium skin tone}{80}
\showcaseopenmojiblack{baby angel- medium-dark skin tone}{81}
\showcaseopenmojiblack{baby angel- medium-light skin tone}{82}
\showcaseopenmojiblack{baby angel}{83}
\showcaseopenmojiblack{baby bottle}{84}
\showcaseopenmojiblack{baby chick}{85}
\showcaseopenmojiblack{baby symbol}{86}
\showcaseopenmojiblack{baby- dark skin tone}{87}
\showcaseopenmojiblack{baby- light skin tone}{88}
\showcaseopenmojiblack{baby- medium skin tone}{89}
\showcaseopenmojiblack{baby- medium-dark skin tone}{90}
\showcaseopenmojiblack{baby- medium-light skin tone}{91}
\showcaseopenmojiblack{baby}{92}
\showcaseopenmojiblack{back arrow}{93}
\showcaseopenmojiblack{backache}{94}
\showcaseopenmojiblack{backhand index pointing down- dark skin tone}{95}
\showcaseopenmojiblack{backhand index pointing down- light skin tone}{96}
\showcaseopenmojiblack{backhand index pointing down- medium skin tone}{97}
\showcaseopenmojiblack{backhand index pointing down- medium-dark skin tone}{98}
\showcaseopenmojiblack{backhand index pointing down- medium-light skin tone}{99}
\showcaseopenmojiblack{backhand index pointing down}{100}
\showcaseopenmojiblack{backhand index pointing left- dark skin tone}{101}
\showcaseopenmojiblack{backhand index pointing left- light skin tone}{102}
\showcaseopenmojiblack{backhand index pointing left- medium skin tone}{103}
\showcaseopenmojiblack{backhand index pointing left- medium-dark skin tone}{104}
\showcaseopenmojiblack{backhand index pointing left- medium-light skin tone}{105}
\showcaseopenmojiblack{backhand index pointing left}{106}
\showcaseopenmojiblack{backhand index pointing right- dark skin tone}{107}
\showcaseopenmojiblack{backhand index pointing right- light skin tone}{108}
\showcaseopenmojiblack{backhand index pointing right- medium skin tone}{109}
\showcaseopenmojiblack{backhand index pointing right- medium-dark skin tone}{110}
\showcaseopenmojiblack{backhand index pointing right- medium-light skin tone}{111}
\showcaseopenmojiblack{backhand index pointing right}{112}
\showcaseopenmojiblack{backhand index pointing up- dark skin tone}{113}
\showcaseopenmojiblack{backhand index pointing up- light skin tone}{114}
\showcaseopenmojiblack{backhand index pointing up- medium skin tone}{115}
\showcaseopenmojiblack{backhand index pointing up- medium-dark skin tone}{116}
\showcaseopenmojiblack{backhand index pointing up- medium-light skin tone}{117}
\showcaseopenmojiblack{backhand index pointing up}{118}
\showcaseopenmojiblack{backpack}{119}
\showcaseopenmojiblack{bacon}{120}
\showcaseopenmojiblack{badger}{121}
\showcaseopenmojiblack{badminton}{122}
\showcaseopenmojiblack{bagel}{123}
\showcaseopenmojiblack{baggage claim}{124}
\showcaseopenmojiblack{baguette bread}{125}
\showcaseopenmojiblack{balance scale}{126}
\showcaseopenmojiblack{bald}{127}
\showcaseopenmojiblack{balearic islands flag}{128}
\showcaseopenmojiblack{ballet shoes}{129}
\showcaseopenmojiblack{balloon}{130}
\showcaseopenmojiblack{ballot box with ballot}{131}
\showcaseopenmojiblack{banana}{132}
\showcaseopenmojiblack{bandage change}{133}
\showcaseopenmojiblack{bandage scissors}{134}
\showcaseopenmojiblack{banjo}{135}
\showcaseopenmojiblack{bank}{136}
\showcaseopenmojiblack{bar chart}{137}
\showcaseopenmojiblack{barber pole}{138}
\showcaseopenmojiblack{barcode}{139}
\showcaseopenmojiblack{barista}{140}
\showcaseopenmojiblack{baseball}{141}
\showcaseopenmojiblack{basket}{142}
\showcaseopenmojiblack{basketball}{143}
\showcaseopenmojiblack{basque flag}{144}
\showcaseopenmojiblack{bat}{145}
\showcaseopenmojiblack{bathtub}{146}
\showcaseopenmojiblack{battery}{147}
\showcaseopenmojiblack{bavaria flag}{148}
\showcaseopenmojiblack{beach with umbrella}{149}
\showcaseopenmojiblack{beaming face with smiling eyes}{150}
\showcaseopenmojiblack{beans}{151}
\showcaseopenmojiblack{bear}{152}
\showcaseopenmojiblack{beating heart}{153}
\showcaseopenmojiblack{beaver}{154}
\showcaseopenmojiblack{bed linen}{155}
\showcaseopenmojiblack{bed}{156}
\showcaseopenmojiblack{beer mug}{157}
\showcaseopenmojiblack{beetle}{158}
\showcaseopenmojiblack{bell pepper}{159}
\showcaseopenmojiblack{bell with slash}{160}
\showcaseopenmojiblack{bell}{161}
\showcaseopenmojiblack{bellhop bell}{162}
\showcaseopenmojiblack{beluga}{163}
\showcaseopenmojiblack{bento box}{164}
\showcaseopenmojiblack{berber flag}{165}
\showcaseopenmojiblack{berlin flag}{166}
\showcaseopenmojiblack{beverage box}{167}
\showcaseopenmojiblack{bicycle}{168}
\showcaseopenmojiblack{bikini}{169}
\showcaseopenmojiblack{billed cap}{170}
\showcaseopenmojiblack{biohazard}{171}
\showcaseopenmojiblack{bird}{172}
\showcaseopenmojiblack{birthday cake}{173}
\showcaseopenmojiblack{bison}{174}
\showcaseopenmojiblack{biting lip}{175}
\showcaseopenmojiblack{black bird}{176}
\showcaseopenmojiblack{black cat}{177}
\showcaseopenmojiblack{black circle}{178}
\showcaseopenmojiblack{black flag}{179}
\showcaseopenmojiblack{black heart}{180}
\showcaseopenmojiblack{black hexagon}{181}
\showcaseopenmojiblack{black hole}{182}
\showcaseopenmojiblack{black large circle}{183}
\showcaseopenmojiblack{black large square}{184}
\showcaseopenmojiblack{black medium square}{185}
\showcaseopenmojiblack{black medium-small square}{186}
\showcaseopenmojiblack{black nib}{187}
\showcaseopenmojiblack{black octagon}{188}
\showcaseopenmojiblack{black pentagon}{189}
\showcaseopenmojiblack{black rectangle}{190}
\showcaseopenmojiblack{black small square}{191}
\showcaseopenmojiblack{black square button}{192}
\showcaseopenmojiblack{black star}{193}
\showcaseopenmojiblack{black vertical ellipse}{194}
\showcaseopenmojiblack{black vertical rectangle}{195}
\showcaseopenmojiblack{blood transfusion}{196}
\showcaseopenmojiblack{blossom}{197}
\showcaseopenmojiblack{blowfish}{198}
\showcaseopenmojiblack{blue book}{199}
\showcaseopenmojiblack{blue circle}{200}
\showcaseopenmojiblack{blue flag}{201}
\showcaseopenmojiblack{blue heart}{202}
\showcaseopenmojiblack{blue hexagon}{203}
\showcaseopenmojiblack{blue square}{204}
\showcaseopenmojiblack{blueberries}{205}
\showcaseopenmojiblack{boar}{206}
\showcaseopenmojiblack{bomb}{207}
\showcaseopenmojiblack{bone}{208}
\showcaseopenmojiblack{bookmark tabs}{209}
\showcaseopenmojiblack{bookmark}{210}
\showcaseopenmojiblack{books}{211}
\showcaseopenmojiblack{boomerang}{212}
\showcaseopenmojiblack{bottle with popping cork}{213}
\showcaseopenmojiblack{boule bread}{214}
\showcaseopenmojiblack{bouquet}{215}
\showcaseopenmojiblack{bow and arrow}{216}
\showcaseopenmojiblack{bowl with spoon}{217}
\showcaseopenmojiblack{bowling}{218}
\showcaseopenmojiblack{boxing glove}{219}
\showcaseopenmojiblack{boy- dark skin tone}{220}
\showcaseopenmojiblack{boy- light skin tone}{221}
\showcaseopenmojiblack{boy- medium skin tone}{222}
\showcaseopenmojiblack{boy- medium-dark skin tone}{223}
\showcaseopenmojiblack{boy- medium-light skin tone}{224}
\showcaseopenmojiblack{boy}{225}
\showcaseopenmojiblack{brain}{226}
\showcaseopenmojiblack{bread}{227}
\showcaseopenmojiblack{breast-feeding- dark skin tone}{228}
\showcaseopenmojiblack{breast-feeding- light skin tone}{229}
\showcaseopenmojiblack{breast-feeding- medium skin tone}{230}
\showcaseopenmojiblack{breast-feeding- medium-dark skin tone}{231}
\showcaseopenmojiblack{breast-feeding- medium-light skin tone}{232}
\showcaseopenmojiblack{breast-feeding}{233}
\showcaseopenmojiblack{bretagne flag}{234}
\showcaseopenmojiblack{brick}{235}
\showcaseopenmojiblack{bridge at night}{236}
\showcaseopenmojiblack{briefcase}{237}
\showcaseopenmojiblack{briefs}{238}
\showcaseopenmojiblack{bright button}{239}
\showcaseopenmojiblack{broccoli}{240}
\showcaseopenmojiblack{broken chain}{241}
\showcaseopenmojiblack{broken heart}{242}
\showcaseopenmojiblack{broom}{243}
\showcaseopenmojiblack{brown circle}{244}
\showcaseopenmojiblack{brown flag}{245}
\showcaseopenmojiblack{brown heart}{246}
\showcaseopenmojiblack{brown hexagon}{247}
\showcaseopenmojiblack{brown mushroom}{248}
\showcaseopenmojiblack{brown square}{249}
\showcaseopenmojiblack{browncoat flag}{250}
\showcaseopenmojiblack{bubble tea}{251}
\showcaseopenmojiblack{bubbles}{252}
\showcaseopenmojiblack{bucket}{253}
\showcaseopenmojiblack{bug}{254}
\showcaseopenmojiblack{building construction}{255}
\showcaseopenmojiblack{bullet train}{256}
\showcaseopenmojiblack{bullseye}{257}
\showcaseopenmojiblack{burrito}{258}
\showcaseopenmojiblack{bus stop}{259}
\showcaseopenmojiblack{bus}{260}
\showcaseopenmojiblack{bust in silhouette}{261}
\showcaseopenmojiblack{busts in silhouette}{262}
\showcaseopenmojiblack{butter}{263}
\showcaseopenmojiblack{butterfly}{264}
\showcaseopenmojiblack{c}{265}
\showcaseopenmojiblack{cable}{266}
\showcaseopenmojiblack{cactus}{267}
\showcaseopenmojiblack{cafeteria}{268}
\showcaseopenmojiblack{cake}{269}
\showcaseopenmojiblack{calendar}{270}
\showcaseopenmojiblack{california flag}{271}
\showcaseopenmojiblack{call me hand- dark skin tone}{272}
\showcaseopenmojiblack{call me hand- light skin tone}{273}
\showcaseopenmojiblack{call me hand- medium skin tone}{274}
\showcaseopenmojiblack{call me hand- medium-dark skin tone}{275}
\showcaseopenmojiblack{call me hand- medium-light skin tone}{276}
\showcaseopenmojiblack{call me hand}{277}
\showcaseopenmojiblack{camel}{278}
\showcaseopenmojiblack{camera with flash}{279}
\showcaseopenmojiblack{camera}{280}
\showcaseopenmojiblack{camping}{281}
\showcaseopenmojiblack{canary islands flag}{282}
\showcaseopenmojiblack{cancer}{283}
\showcaseopenmojiblack{candle}{284}
\showcaseopenmojiblack{candy}{285}
\showcaseopenmojiblack{canned food}{286}
\showcaseopenmojiblack{canoe}{287}
\showcaseopenmojiblack{cantabria flag}{288}
\showcaseopenmojiblack{capricorn}{289}
\showcaseopenmojiblack{card file box}{290}
\showcaseopenmojiblack{card index dividers}{291}
\showcaseopenmojiblack{card index}{292}
\showcaseopenmojiblack{carousel horse}{293}
\showcaseopenmojiblack{carp streamer}{294}
\showcaseopenmojiblack{carpentry saw}{295}
\showcaseopenmojiblack{carrot}{296}
\showcaseopenmojiblack{castile and leon flag}{297}
\showcaseopenmojiblack{castile-la mancha flag}{298}
\showcaseopenmojiblack{castle}{299}
\showcaseopenmojiblack{cat face}{300}
\showcaseopenmojiblack{cat with tears of joy}{301}
\showcaseopenmojiblack{cat with wry smile}{302}
\showcaseopenmojiblack{cat}{303}
\showcaseopenmojiblack{catalonia flag}{304}
\showcaseopenmojiblack{ceuta flag}{305}
\showcaseopenmojiblack{chains}{306}
\showcaseopenmojiblack{chair}{307}
\showcaseopenmojiblack{champignon brown}{308}
\showcaseopenmojiblack{champignon white}{309}
\showcaseopenmojiblack{charge plug}{310}
\showcaseopenmojiblack{chart decreasing}{311}
\showcaseopenmojiblack{chart increasing with yen}{312}
\showcaseopenmojiblack{chart increasing}{313}
\showcaseopenmojiblack{chats}{314}
\showcaseopenmojiblack{check box with check}{315}
\showcaseopenmojiblack{check mark button}{316}
\showcaseopenmojiblack{check mark}{317}
\showcaseopenmojiblack{cheese wedge}{318}
\showcaseopenmojiblack{chequered flag}{319}
\showcaseopenmojiblack{cherries}{320}
\showcaseopenmojiblack{cherry blossom}{321}
\showcaseopenmojiblack{chess pawn}{322}
\showcaseopenmojiblack{chestnut}{323}
\showcaseopenmojiblack{chicken}{324}
\showcaseopenmojiblack{child- dark skin tone}{325}
\showcaseopenmojiblack{child- light skin tone}{326}
\showcaseopenmojiblack{child- medium skin tone}{327}
\showcaseopenmojiblack{child- medium-dark skin tone}{328}
\showcaseopenmojiblack{child- medium-light skin tone}{329}
\showcaseopenmojiblack{child}{330}
\showcaseopenmojiblack{children crossing}{331}
\showcaseopenmojiblack{chipmunk}{332}
\showcaseopenmojiblack{chocolate bar}{333}
\showcaseopenmojiblack{chopsticks}{334}
\showcaseopenmojiblack{christmas tree}{335}
\showcaseopenmojiblack{chrome canary}{336}
\showcaseopenmojiblack{chrome}{337}
\showcaseopenmojiblack{chromium}{338}
\showcaseopenmojiblack{church}{339}
\showcaseopenmojiblack{cigarette}{340}
\showcaseopenmojiblack{cinema}{341}
\showcaseopenmojiblack{circle with left half black}{342}
\showcaseopenmojiblack{circle with right half black}{343}
\showcaseopenmojiblack{circled anticlockwise arrow}{344}
\showcaseopenmojiblack{circled c with overlaid backslash}{345}
\showcaseopenmojiblack{circled cc}{346}
\showcaseopenmojiblack{circled dollar sign with overlaid backslash}{347}
\showcaseopenmojiblack{circled equals}{348}
\showcaseopenmojiblack{circled human figure}{349}
\showcaseopenmojiblack{circled m}{350}
\showcaseopenmojiblack{circled zero with slash}{351}
\showcaseopenmojiblack{circuit}{352}
\showcaseopenmojiblack{circus tent}{353}
\showcaseopenmojiblack{cityscape at dusk}{354}
\showcaseopenmojiblack{cityscape}{355}
\showcaseopenmojiblack{cl button}{356}
\showcaseopenmojiblack{clamp}{357}
\showcaseopenmojiblack{clapper board}{358}
\showcaseopenmojiblack{clapping hands- dark skin tone}{359}
\showcaseopenmojiblack{clapping hands- light skin tone}{360}
\showcaseopenmojiblack{clapping hands- medium skin tone}{361}
\showcaseopenmojiblack{clapping hands- medium-dark skin tone}{362}
\showcaseopenmojiblack{clapping hands- medium-light skin tone}{363}
\showcaseopenmojiblack{clapping hands}{364}
\showcaseopenmojiblack{classical building}{365}
\showcaseopenmojiblack{clinical thermometer}{366}
\showcaseopenmojiblack{clinking beer mugs}{367}
\showcaseopenmojiblack{clinking glasses}{368}
\showcaseopenmojiblack{clipboard}{369}
\showcaseopenmojiblack{clockwise vertical arrows}{370}
\showcaseopenmojiblack{close}{371}
\showcaseopenmojiblack{closed book}{372}
\showcaseopenmojiblack{closed mailbox with lowered flag}{373}
\showcaseopenmojiblack{closed mailbox with raised flag}{374}
\showcaseopenmojiblack{closed umbrella}{375}
\showcaseopenmojiblack{cloud with lightning and rain}{376}
\showcaseopenmojiblack{cloud with lightning}{377}
\showcaseopenmojiblack{cloud with rain}{378}
\showcaseopenmojiblack{cloud with snow}{379}
\showcaseopenmojiblack{cloud}{380}
\showcaseopenmojiblack{clown face}{381}
\showcaseopenmojiblack{club suit}{382}
\showcaseopenmojiblack{clutch bag}{383}
\showcaseopenmojiblack{coat}{384}
\showcaseopenmojiblack{cockroach}{385}
\showcaseopenmojiblack{cocktail glass}{386}
\showcaseopenmojiblack{coconut}{387}
\showcaseopenmojiblack{code editor}{388}
\showcaseopenmojiblack{coffee grinder}{389}
\showcaseopenmojiblack{coffin}{390}
\showcaseopenmojiblack{coin}{391}
\showcaseopenmojiblack{cold face}{392}
\showcaseopenmojiblack{collaboration}{393}
\showcaseopenmojiblack{collision}{394}
\showcaseopenmojiblack{colorado flag}{395}
\showcaseopenmojiblack{colossus of rhodes}{396}
\showcaseopenmojiblack{comet}{397}
\showcaseopenmojiblack{comment}{398}
\showcaseopenmojiblack{compass}{399}
\showcaseopenmojiblack{compose}{400}
\showcaseopenmojiblack{computer disk}{401}
\showcaseopenmojiblack{computer mouse}{402}
\showcaseopenmojiblack{confetti ball}{403}
\showcaseopenmojiblack{confounded face}{404}
\showcaseopenmojiblack{confused face}{405}
\showcaseopenmojiblack{construction worker- dark skin tone}{406}
\showcaseopenmojiblack{construction worker- light skin tone}{407}
\showcaseopenmojiblack{construction worker- medium skin tone}{408}
\showcaseopenmojiblack{construction worker- medium-dark skin tone}{409}
\showcaseopenmojiblack{construction worker- medium-light skin tone}{410}
\showcaseopenmojiblack{construction worker}{411}
\showcaseopenmojiblack{construction}{412}
\showcaseopenmojiblack{contacts}{413}
\showcaseopenmojiblack{control knobs}{414}
\showcaseopenmojiblack{convenience store}{415}
\showcaseopenmojiblack{cook- dark skin tone}{416}
\showcaseopenmojiblack{cook- light skin tone}{417}
\showcaseopenmojiblack{cook- medium skin tone}{418}
\showcaseopenmojiblack{cook- medium-dark skin tone}{419}
\showcaseopenmojiblack{cook- medium-light skin tone}{420}
\showcaseopenmojiblack{cook}{421}
\showcaseopenmojiblack{cooked rice}{422}
\showcaseopenmojiblack{cookie}{423}
\showcaseopenmojiblack{cooking}{424}
\showcaseopenmojiblack{cool button}{425}
\showcaseopenmojiblack{copy}{426}
\showcaseopenmojiblack{copyleft symbol}{427}
\showcaseopenmojiblack{copyright}{428}
\showcaseopenmojiblack{coral}{429}
\showcaseopenmojiblack{couch and lamp}{430}
\showcaseopenmojiblack{counterclockwise arrows button}{431}
\showcaseopenmojiblack{couple with heart- dark skin tone}{432}
\showcaseopenmojiblack{couple with heart- light skin tone}{433}
\showcaseopenmojiblack{couple with heart- man, man, dark skin tone, light skin tone}{434}
\showcaseopenmojiblack{couple with heart- man, man, dark skin tone, medium skin tone}{435}
\showcaseopenmojiblack{couple with heart- man, man, dark skin tone, medium-dark skin tone}{436}
\showcaseopenmojiblack{couple with heart- man, man, dark skin tone, medium-light skin tone}{437}
\showcaseopenmojiblack{couple with heart- man, man, dark skin tone}{438}
\showcaseopenmojiblack{couple with heart- man, man, light skin tone, dark skin tone}{439}
\showcaseopenmojiblack{couple with heart- man, man, light skin tone, medium skin tone}{440}
\showcaseopenmojiblack{couple with heart- man, man, light skin tone, medium-dark skin tone}{441}
\showcaseopenmojiblack{couple with heart- man, man, light skin tone, medium-light skin tone}{442}
\showcaseopenmojiblack{couple with heart- man, man, light skin tone}{443}
\showcaseopenmojiblack{couple with heart- man, man, medium skin tone, dark skin tone}{444}
\showcaseopenmojiblack{couple with heart- man, man, medium skin tone, light skin tone}{445}
\showcaseopenmojiblack{couple with heart- man, man, medium skin tone, medium-dark skin tone}{446}
\showcaseopenmojiblack{couple with heart- man, man, medium skin tone, medium-light skin tone}{447}
\showcaseopenmojiblack{couple with heart- man, man, medium skin tone}{448}
\showcaseopenmojiblack{couple with heart- man, man, medium-dark skin tone, dark skin tone}{449}
\showcaseopenmojiblack{couple with heart- man, man, medium-dark skin tone, light skin tone}{450}
\showcaseopenmojiblack{couple with heart- man, man, medium-dark skin tone, medium skin tone}{451}
\showcaseopenmojiblack{couple with heart- man, man, medium-dark skin tone, medium-light skin tone}{452}
\showcaseopenmojiblack{couple with heart- man, man, medium-dark skin tone}{453}
\showcaseopenmojiblack{couple with heart- man, man, medium-light skin tone, dark skin tone}{454}
\showcaseopenmojiblack{couple with heart- man, man, medium-light skin tone, light skin tone}{455}
\showcaseopenmojiblack{couple with heart- man, man, medium-light skin tone, medium skin tone}{456}
\showcaseopenmojiblack{couple with heart- man, man, medium-light skin tone, medium-dark skin tone}{457}
\showcaseopenmojiblack{couple with heart- man, man, medium-light skin tone}{458}
\showcaseopenmojiblack{couple with heart- man, man}{459}
\showcaseopenmojiblack{couple with heart- medium skin tone}{460}
\showcaseopenmojiblack{couple with heart- medium-dark skin tone}{461}
\showcaseopenmojiblack{couple with heart- medium-light skin tone}{462}
\showcaseopenmojiblack{couple with heart- person, person, dark skin tone, light skin tone}{463}
\showcaseopenmojiblack{couple with heart- person, person, dark skin tone, medium skin tone}{464}
\showcaseopenmojiblack{couple with heart- person, person, dark skin tone, medium-dark skin tone}{465}
\showcaseopenmojiblack{couple with heart- person, person, dark skin tone, medium-light skin tone}{466}
\showcaseopenmojiblack{couple with heart- person, person, light skin tone, dark skin tone}{467}
\showcaseopenmojiblack{couple with heart- person, person, light skin tone, medium skin tone}{468}
\showcaseopenmojiblack{couple with heart- person, person, light skin tone, medium-dark skin tone}{469}
\showcaseopenmojiblack{couple with heart- person, person, light skin tone, medium-light skin tone}{470}
\showcaseopenmojiblack{couple with heart- person, person, medium skin tone, dark skin tone}{471}
\showcaseopenmojiblack{couple with heart- person, person, medium skin tone, light skin tone}{472}
\showcaseopenmojiblack{couple with heart- person, person, medium skin tone, medium-dark skin tone}{473}
\showcaseopenmojiblack{couple with heart- person, person, medium skin tone, medium-light skin tone}{474}
\showcaseopenmojiblack{couple with heart- person, person, medium-dark skin tone, dark skin tone}{475}
\showcaseopenmojiblack{couple with heart- person, person, medium-dark skin tone, light skin tone}{476}
\showcaseopenmojiblack{couple with heart- person, person, medium-dark skin tone, medium skin tone}{477}
\showcaseopenmojiblack{couple with heart- person, person, medium-dark skin tone, medium-light skin tone}{478}
\showcaseopenmojiblack{couple with heart- person, person, medium-light skin tone, dark skin tone}{479}
\showcaseopenmojiblack{couple with heart- person, person, medium-light skin tone, light skin tone}{480}
\showcaseopenmojiblack{couple with heart- person, person, medium-light skin tone, medium skin tone}{481}
\showcaseopenmojiblack{couple with heart- person, person, medium-light skin tone, medium-dark skin tone}{482}
\showcaseopenmojiblack{couple with heart- woman, man, dark skin tone, light skin tone}{483}
\showcaseopenmojiblack{couple with heart- woman, man, dark skin tone, medium skin tone}{484}
\showcaseopenmojiblack{couple with heart- woman, man, dark skin tone, medium-dark skin tone}{485}
\showcaseopenmojiblack{couple with heart- woman, man, dark skin tone, medium-light skin tone}{486}
\showcaseopenmojiblack{couple with heart- woman, man, dark skin tone}{487}
\showcaseopenmojiblack{couple with heart- woman, man, light skin tone, dark skin tone}{488}
\showcaseopenmojiblack{couple with heart- woman, man, light skin tone, medium skin tone}{489}
\showcaseopenmojiblack{couple with heart- woman, man, light skin tone, medium-dark skin tone}{490}
\showcaseopenmojiblack{couple with heart- woman, man, light skin tone, medium-light skin tone}{491}
\showcaseopenmojiblack{couple with heart- woman, man, light skin tone}{492}
\showcaseopenmojiblack{couple with heart- woman, man, medium skin tone, dark skin tone}{493}
\showcaseopenmojiblack{couple with heart- woman, man, medium skin tone, light skin tone}{494}
\showcaseopenmojiblack{couple with heart- woman, man, medium skin tone, medium-dark skin tone}{495}
\showcaseopenmojiblack{couple with heart- woman, man, medium skin tone, medium-light skin tone}{496}
\showcaseopenmojiblack{couple with heart- woman, man, medium skin tone}{497}
\showcaseopenmojiblack{couple with heart- woman, man, medium-dark skin tone, dark skin tone}{498}
\showcaseopenmojiblack{couple with heart- woman, man, medium-dark skin tone, light skin tone}{499}
\showcaseopenmojiblack{couple with heart- woman, man, medium-dark skin tone, medium skin tone}{500}
\showcaseopenmojiblack{couple with heart- woman, man, medium-dark skin tone, medium-light skin tone}{501}
\showcaseopenmojiblack{couple with heart- woman, man, medium-dark skin tone}{502}
\showcaseopenmojiblack{couple with heart- woman, man, medium-light skin tone, dark skin tone}{503}
\showcaseopenmojiblack{couple with heart- woman, man, medium-light skin tone, light skin tone}{504}
\showcaseopenmojiblack{couple with heart- woman, man, medium-light skin tone, medium skin tone}{505}
\showcaseopenmojiblack{couple with heart- woman, man, medium-light skin tone, medium-dark skin tone}{506}
\showcaseopenmojiblack{couple with heart- woman, man, medium-light skin tone}{507}
\showcaseopenmojiblack{couple with heart- woman, man}{508}
\showcaseopenmojiblack{couple with heart- woman, woman, dark skin tone, light skin tone}{509}
\showcaseopenmojiblack{couple with heart- woman, woman, dark skin tone, medium skin tone}{510}
\showcaseopenmojiblack{couple with heart- woman, woman, dark skin tone, medium-dark skin tone}{511}
\showcaseopenmojiblack{couple with heart- woman, woman, dark skin tone, medium-light skin tone}{512}
\showcaseopenmojiblack{couple with heart- woman, woman, dark skin tone}{513}
\showcaseopenmojiblack{couple with heart- woman, woman, light skin tone, dark skin tone}{514}
\showcaseopenmojiblack{couple with heart- woman, woman, light skin tone, medium skin tone}{515}
\showcaseopenmojiblack{couple with heart- woman, woman, light skin tone, medium-dark skin tone}{516}
\showcaseopenmojiblack{couple with heart- woman, woman, light skin tone, medium-light skin tone}{517}
\showcaseopenmojiblack{couple with heart- woman, woman, light skin tone}{518}
\showcaseopenmojiblack{couple with heart- woman, woman, medium skin tone, dark skin tone}{519}
\showcaseopenmojiblack{couple with heart- woman, woman, medium skin tone, light skin tone}{520}
\showcaseopenmojiblack{couple with heart- woman, woman, medium skin tone, medium-dark skin tone}{521}
\showcaseopenmojiblack{couple with heart- woman, woman, medium skin tone, medium-light skin tone}{522}
\showcaseopenmojiblack{couple with heart- woman, woman, medium skin tone}{523}
\showcaseopenmojiblack{couple with heart- woman, woman, medium-dark skin tone, dark skin tone}{524}
\showcaseopenmojiblack{couple with heart- woman, woman, medium-dark skin tone, light skin tone}{525}
\showcaseopenmojiblack{couple with heart- woman, woman, medium-dark skin tone, medium skin tone}{526}
\showcaseopenmojiblack{couple with heart- woman, woman, medium-dark skin tone, medium-light skin tone}{527}
\showcaseopenmojiblack{couple with heart- woman, woman, medium-dark skin tone}{528}
\showcaseopenmojiblack{couple with heart- woman, woman, medium-light skin tone, dark skin tone}{529}
\showcaseopenmojiblack{couple with heart- woman, woman, medium-light skin tone, light skin tone}{530}
\showcaseopenmojiblack{couple with heart- woman, woman, medium-light skin tone, medium skin tone}{531}
\showcaseopenmojiblack{couple with heart- woman, woman, medium-light skin tone, medium-dark skin tone}{532}
\showcaseopenmojiblack{couple with heart- woman, woman, medium-light skin tone}{533}
\showcaseopenmojiblack{couple with heart- woman, woman}{534}
\showcaseopenmojiblack{couple with heart}{535}
\showcaseopenmojiblack{cow face}{536}
\showcaseopenmojiblack{cow}{537}
\showcaseopenmojiblack{cowboy hat face}{538}
\showcaseopenmojiblack{cplusplus}{539}
\showcaseopenmojiblack{crab}{540}
\showcaseopenmojiblack{crayon}{541}
\showcaseopenmojiblack{credit card}{542}
\showcaseopenmojiblack{crescent moon}{543}
\showcaseopenmojiblack{cricket game}{544}
\showcaseopenmojiblack{cricket}{545}
\showcaseopenmojiblack{crocodile}{546}
\showcaseopenmojiblack{croissant}{547}
\showcaseopenmojiblack{cross mark button}{548}
\showcaseopenmojiblack{cross mark}{549}
\showcaseopenmojiblack{crossed fingers- dark skin tone}{550}
\showcaseopenmojiblack{crossed fingers- light skin tone}{551}
\showcaseopenmojiblack{crossed fingers- medium skin tone}{552}
\showcaseopenmojiblack{crossed fingers- medium-dark skin tone}{553}
\showcaseopenmojiblack{crossed fingers- medium-light skin tone}{554}
\showcaseopenmojiblack{crossed fingers}{555}
\showcaseopenmojiblack{crossed flags}{556}
\showcaseopenmojiblack{crossed swords}{557}
\showcaseopenmojiblack{crown}{558}
\showcaseopenmojiblack{crutch}{559}
\showcaseopenmojiblack{crutches}{560}
\showcaseopenmojiblack{crying cat}{561}
\showcaseopenmojiblack{crying face}{562}
\showcaseopenmojiblack{crystal ball}{563}
\showcaseopenmojiblack{csharp}{564}
\showcaseopenmojiblack{ct scan}{565}
\showcaseopenmojiblack{cucumber}{566}
\showcaseopenmojiblack{cup with straw}{567}
\showcaseopenmojiblack{cupcake}{568}
\showcaseopenmojiblack{curling stone}{569}
\showcaseopenmojiblack{curly hair}{570}
\showcaseopenmojiblack{curly loop}{571}
\showcaseopenmojiblack{currency exchange}{572}
\showcaseopenmojiblack{curry rice}{573}
\showcaseopenmojiblack{cursor}{574}
\showcaseopenmojiblack{custard}{575}
\showcaseopenmojiblack{customs}{576}
\showcaseopenmojiblack{cut of meat}{577}
\showcaseopenmojiblack{cyclone}{578}
\showcaseopenmojiblack{dagger}{579}
\showcaseopenmojiblack{dango}{580}
\showcaseopenmojiblack{dark skin tone}{581}
\showcaseopenmojiblack{dashing away}{582}
\showcaseopenmojiblack{deaf man- dark skin tone}{583}
\showcaseopenmojiblack{deaf man- light skin tone}{584}
\showcaseopenmojiblack{deaf man- medium skin tone}{585}
\showcaseopenmojiblack{deaf man- medium-dark skin tone}{586}
\showcaseopenmojiblack{deaf man- medium-light skin tone}{587}
\showcaseopenmojiblack{deaf man}{588}
\showcaseopenmojiblack{deaf person- dark skin tone}{589}
\showcaseopenmojiblack{deaf person- light skin tone}{590}
\showcaseopenmojiblack{deaf person- medium skin tone}{591}
\showcaseopenmojiblack{deaf person- medium-dark skin tone}{592}
\showcaseopenmojiblack{deaf person- medium-light skin tone}{593}
\showcaseopenmojiblack{deaf person}{594}
\showcaseopenmojiblack{deaf woman- dark skin tone}{595}
\showcaseopenmojiblack{deaf woman- light skin tone}{596}
\showcaseopenmojiblack{deaf woman- medium skin tone}{597}
\showcaseopenmojiblack{deaf woman- medium-dark skin tone}{598}
\showcaseopenmojiblack{deaf woman- medium-light skin tone}{599}
\showcaseopenmojiblack{deaf woman}{600}
\showcaseopenmojiblack{deciduous tree}{601}
\showcaseopenmojiblack{deep blue flag}{602}
\showcaseopenmojiblack{deep brown flag}{603}
\showcaseopenmojiblack{deep green flag}{604}
\showcaseopenmojiblack{deep orange flag}{605}
\showcaseopenmojiblack{deep purple flag}{606}
\showcaseopenmojiblack{deep red flag}{607}
\showcaseopenmojiblack{deep yellow flag}{608}
\showcaseopenmojiblack{deer}{609}
\showcaseopenmojiblack{dejected face}{610}
\showcaseopenmojiblack{delete}{611}
\showcaseopenmojiblack{delivery truck}{612}
\showcaseopenmojiblack{department store}{613}
\showcaseopenmojiblack{derelict house}{614}
\showcaseopenmojiblack{desert island}{615}
\showcaseopenmojiblack{desert}{616}
\showcaseopenmojiblack{desktop computer}{617}
\showcaseopenmojiblack{details}{618}
\showcaseopenmojiblack{detective- dark skin tone}{619}
\showcaseopenmojiblack{detective- light skin tone}{620}
\showcaseopenmojiblack{detective- medium skin tone}{621}
\showcaseopenmojiblack{detective- medium-dark skin tone}{622}
\showcaseopenmojiblack{detective- medium-light skin tone}{623}
\showcaseopenmojiblack{detective}{624}
\showcaseopenmojiblack{diamond suit}{625}
\showcaseopenmojiblack{diamond with a dot}{626}
\showcaseopenmojiblack{dim button}{627}
\showcaseopenmojiblack{disappointed face}{628}
\showcaseopenmojiblack{discord}{629}
\showcaseopenmojiblack{disguised face}{630}
\showcaseopenmojiblack{disinfect surface}{631}
\showcaseopenmojiblack{divide}{632}
\showcaseopenmojiblack{diving mask}{633}
\showcaseopenmojiblack{diya lamp}{634}
\showcaseopenmojiblack{dizzy}{635}
\showcaseopenmojiblack{dj man}{636}
\showcaseopenmojiblack{dj woman}{637}
\showcaseopenmojiblack{dj}{638}
\showcaseopenmojiblack{dna}{639}
\showcaseopenmojiblack{dodo}{640}
\showcaseopenmojiblack{doe}{641}
\showcaseopenmojiblack{dog face}{642}
\showcaseopenmojiblack{dog}{643}
\showcaseopenmojiblack{dollar banknote}{644}
\showcaseopenmojiblack{dolphin}{645}
\showcaseopenmojiblack{donkey}{646}
\showcaseopenmojiblack{door}{647}
\showcaseopenmojiblack{dotnet}{648}
\showcaseopenmojiblack{dotted line face}{649}
\showcaseopenmojiblack{dotted six-pointed star}{650}
\showcaseopenmojiblack{double curly loop}{651}
\showcaseopenmojiblack{double exclamation mark}{652}
\showcaseopenmojiblack{double tap}{653}
\showcaseopenmojiblack{doughnut}{654}
\showcaseopenmojiblack{dove}{655}
\showcaseopenmojiblack{down arrow}{656}
\showcaseopenmojiblack{down-left arrow}{657}
\showcaseopenmojiblack{down-right arrow}{658}
\showcaseopenmojiblack{downcast face with sweat}{659}
\showcaseopenmojiblack{download}{660}
\showcaseopenmojiblack{downwards button}{661}
\showcaseopenmojiblack{dragon face}{662}
\showcaseopenmojiblack{dragon}{663}
\showcaseopenmojiblack{dress}{664}
\showcaseopenmojiblack{drip coffee maker}{665}
\showcaseopenmojiblack{drone}{666}
\showcaseopenmojiblack{drooling face}{667}
\showcaseopenmojiblack{drop cover hold}{668}
\showcaseopenmojiblack{drop of blood}{669}
\showcaseopenmojiblack{droplet}{670}
\showcaseopenmojiblack{drum}{671}
\showcaseopenmojiblack{drunk person}{672}
\showcaseopenmojiblack{duck}{673}
\showcaseopenmojiblack{dumpling}{674}
\showcaseopenmojiblack{duplicate}{675}
\showcaseopenmojiblack{dvd}{676}
\showcaseopenmojiblack{e-mail}{677}
\showcaseopenmojiblack{eagle}{678}
\showcaseopenmojiblack{ear of corn}{679}
\showcaseopenmojiblack{ear with hearing aid- dark skin tone}{680}
\showcaseopenmojiblack{ear with hearing aid- light skin tone}{681}
\showcaseopenmojiblack{ear with hearing aid- medium skin tone}{682}
\showcaseopenmojiblack{ear with hearing aid- medium-dark skin tone}{683}
\showcaseopenmojiblack{ear with hearing aid- medium-light skin tone}{684}
\showcaseopenmojiblack{ear with hearing aid}{685}
\showcaseopenmojiblack{ear- dark skin tone}{686}
\showcaseopenmojiblack{ear- light skin tone}{687}
\showcaseopenmojiblack{ear- medium skin tone}{688}
\showcaseopenmojiblack{ear- medium-dark skin tone}{689}
\showcaseopenmojiblack{ear- medium-light skin tone}{690}
\showcaseopenmojiblack{ear}{691}
\showcaseopenmojiblack{earache}{692}
\showcaseopenmojiblack{earthquake}{693}
\showcaseopenmojiblack{ecg waves}{694}
\showcaseopenmojiblack{edge}{695}
\showcaseopenmojiblack{edit}{696}
\showcaseopenmojiblack{egg}{697}
\showcaseopenmojiblack{eggplant}{698}
\showcaseopenmojiblack{eiffel tower}{699}
\showcaseopenmojiblack{eight o clock}{700}
\showcaseopenmojiblack{eight-pointed star}{701}
\showcaseopenmojiblack{eight-spoked asterisk}{702}
\showcaseopenmojiblack{eight-thirty}{703}
\showcaseopenmojiblack{eject button}{704}
\showcaseopenmojiblack{electric coffee percolator}{705}
\showcaseopenmojiblack{electric plug red}{706}
\showcaseopenmojiblack{electric plug}{707}
\showcaseopenmojiblack{element}{708}
\showcaseopenmojiblack{elephant}{709}
\showcaseopenmojiblack{elevator}{710}
\showcaseopenmojiblack{eleven o clock}{711}
\showcaseopenmojiblack{eleven-thirty}{712}
\showcaseopenmojiblack{elf- dark skin tone}{713}
\showcaseopenmojiblack{elf- light skin tone}{714}
\showcaseopenmojiblack{elf- medium skin tone}{715}
\showcaseopenmojiblack{elf- medium-dark skin tone}{716}
\showcaseopenmojiblack{elf- medium-light skin tone}{717}
\showcaseopenmojiblack{elf}{718}
\showcaseopenmojiblack{emergency exit door}{719}
\showcaseopenmojiblack{emergency exit}{720}
\showcaseopenmojiblack{empty nest}{721}
\showcaseopenmojiblack{end arrow}{722}
\showcaseopenmojiblack{enraged face}{723}
\showcaseopenmojiblack{envelope with arrow}{724}
\showcaseopenmojiblack{envelope}{725}
\showcaseopenmojiblack{esperanto flag}{726}
\showcaseopenmojiblack{espresso machine}{727}
\showcaseopenmojiblack{euro banknote}{728}
\showcaseopenmojiblack{european name badge}{729}
\showcaseopenmojiblack{evacuate downstairs}{730}
\showcaseopenmojiblack{evacuate fire}{731}
\showcaseopenmojiblack{evacuate to shelter}{732}
\showcaseopenmojiblack{evacuate vertical}{733}
\showcaseopenmojiblack{evacuate}{734}
\showcaseopenmojiblack{evergreen tree}{735}
\showcaseopenmojiblack{ewe}{736}
\showcaseopenmojiblack{exclamation question mark}{737}
\showcaseopenmojiblack{exhaust gases car}{738}
\showcaseopenmojiblack{exhaust gases factory}{739}
\showcaseopenmojiblack{exhausted face}{740}
\showcaseopenmojiblack{exit}{741}
\showcaseopenmojiblack{exploding head}{742}
\showcaseopenmojiblack{expressionless face}{743}
\showcaseopenmojiblack{extremadura flag}{744}
\showcaseopenmojiblack{eye in speech bubble}{745}
\showcaseopenmojiblack{eye pain}{746}
\showcaseopenmojiblack{eye}{747}
\showcaseopenmojiblack{eyes}{748}
\showcaseopenmojiblack{face blowing a kiss}{749}
\showcaseopenmojiblack{face exhaling}{750}
\showcaseopenmojiblack{face holding back tears}{751}
\showcaseopenmojiblack{face in clouds}{752}
\showcaseopenmojiblack{face savoring food}{753}
\showcaseopenmojiblack{face screaming in fear}{754}
\showcaseopenmojiblack{face vomiting}{755}
\showcaseopenmojiblack{face with crossed-out eyes}{756}
\showcaseopenmojiblack{face with diagonal mouth}{757}
\showcaseopenmojiblack{face with hand over mouth}{758}
\showcaseopenmojiblack{face with head-bandage}{759}
\showcaseopenmojiblack{face with medical mask}{760}
\showcaseopenmojiblack{face with monocle}{761}
\showcaseopenmojiblack{face with open eyes and hand over mouth}{762}
\showcaseopenmojiblack{face with open mouth}{763}
\showcaseopenmojiblack{face with peeking eye}{764}
\showcaseopenmojiblack{face with raised eyebrow}{765}
\showcaseopenmojiblack{face with rolling eyes}{766}
\showcaseopenmojiblack{face with spiral eyes}{767}
\showcaseopenmojiblack{face with steam from nose}{768}
\showcaseopenmojiblack{face with symbols on mouth}{769}
\showcaseopenmojiblack{face with tears of joy}{770}
\showcaseopenmojiblack{face with thermometer}{771}
\showcaseopenmojiblack{face with tongue}{772}
\showcaseopenmojiblack{face without mouth}{773}
\showcaseopenmojiblack{facebook}{774}
\showcaseopenmojiblack{factory worker- dark skin tone}{775}
\showcaseopenmojiblack{factory worker- light skin tone}{776}
\showcaseopenmojiblack{factory worker- medium skin tone}{777}
\showcaseopenmojiblack{factory worker- medium-dark skin tone}{778}
\showcaseopenmojiblack{factory worker- medium-light skin tone}{779}
\showcaseopenmojiblack{factory worker}{780}
\showcaseopenmojiblack{factory}{781}
\showcaseopenmojiblack{fairy- dark skin tone}{782}
\showcaseopenmojiblack{fairy- light skin tone}{783}
\showcaseopenmojiblack{fairy- medium skin tone}{784}
\showcaseopenmojiblack{fairy- medium-dark skin tone}{785}
\showcaseopenmojiblack{fairy- medium-light skin tone}{786}
\showcaseopenmojiblack{fairy}{787}
\showcaseopenmojiblack{falafel}{788}
\showcaseopenmojiblack{fallen leaf}{789}
\showcaseopenmojiblack{family- adult, adult, child, child}{790}
\showcaseopenmojiblack{family- adult, adult, child}{791}
\showcaseopenmojiblack{family- adult, child, child}{792}
\showcaseopenmojiblack{family- adult, child}{793}
\showcaseopenmojiblack{family- man, boy, boy}{794}
\showcaseopenmojiblack{family- man, boy}{795}
\showcaseopenmojiblack{family- man, girl, boy}{796}
\showcaseopenmojiblack{family- man, girl, girl}{797}
\showcaseopenmojiblack{family- man, girl}{798}
\showcaseopenmojiblack{family- man, man, boy, boy}{799}
\showcaseopenmojiblack{family- man, man, boy}{800}
\showcaseopenmojiblack{family- man, man, girl, boy}{801}
\showcaseopenmojiblack{family- man, man, girl, girl}{802}
\showcaseopenmojiblack{family- man, man, girl}{803}
\showcaseopenmojiblack{family- man, woman, boy, boy}{804}
\showcaseopenmojiblack{family- man, woman, boy}{805}
\showcaseopenmojiblack{family- man, woman, girl, boy}{806}
\showcaseopenmojiblack{family- man, woman, girl, girl}{807}
\showcaseopenmojiblack{family- man, woman, girl}{808}
\showcaseopenmojiblack{family- woman, boy, boy}{809}
\showcaseopenmojiblack{family- woman, boy}{810}
\showcaseopenmojiblack{family- woman, girl, boy}{811}
\showcaseopenmojiblack{family- woman, girl, girl}{812}
\showcaseopenmojiblack{family- woman, girl}{813}
\showcaseopenmojiblack{family- woman, woman, boy, boy}{814}
\showcaseopenmojiblack{family- woman, woman, boy}{815}
\showcaseopenmojiblack{family- woman, woman, girl, boy}{816}
\showcaseopenmojiblack{family- woman, woman, girl, girl}{817}
\showcaseopenmojiblack{family- woman, woman, girl}{818}
\showcaseopenmojiblack{family}{819}
\showcaseopenmojiblack{farmer- dark skin tone}{820}
\showcaseopenmojiblack{farmer- light skin tone}{821}
\showcaseopenmojiblack{farmer- medium skin tone}{822}
\showcaseopenmojiblack{farmer- medium-dark skin tone}{823}
\showcaseopenmojiblack{farmer- medium-light skin tone}{824}
\showcaseopenmojiblack{farmer}{825}
\showcaseopenmojiblack{fast down button}{826}
\showcaseopenmojiblack{fast reverse button}{827}
\showcaseopenmojiblack{fast up button}{828}
\showcaseopenmojiblack{fast-forward button}{829}
\showcaseopenmojiblack{fax machine}{830}
\showcaseopenmojiblack{fearful face}{831}
\showcaseopenmojiblack{feather}{832}
\showcaseopenmojiblack{female doctor}{833}
\showcaseopenmojiblack{female nurse}{834}
\showcaseopenmojiblack{female sign}{835}
\showcaseopenmojiblack{ferris wheel}{836}
\showcaseopenmojiblack{ferry}{837}
\showcaseopenmojiblack{field hockey}{838}
\showcaseopenmojiblack{file cabinet}{839}
\showcaseopenmojiblack{file folder}{840}
\showcaseopenmojiblack{film frames}{841}
\showcaseopenmojiblack{film projector}{842}
\showcaseopenmojiblack{filter}{843}
\showcaseopenmojiblack{finger pushing button}{844}
\showcaseopenmojiblack{fire engine}{845}
\showcaseopenmojiblack{fire extinguisher}{846}
\showcaseopenmojiblack{fire}{847}
\showcaseopenmojiblack{firecracker}{848}
\showcaseopenmojiblack{firefighter- dark skin tone}{849}
\showcaseopenmojiblack{firefighter- light skin tone}{850}
\showcaseopenmojiblack{firefighter- medium skin tone}{851}
\showcaseopenmojiblack{firefighter- medium-dark skin tone}{852}
\showcaseopenmojiblack{firefighter- medium-light skin tone}{853}
\showcaseopenmojiblack{firefighter}{854}
\showcaseopenmojiblack{firefox developer}{855}
\showcaseopenmojiblack{firefox nightly}{856}
\showcaseopenmojiblack{firefox}{857}
\showcaseopenmojiblack{fireworks}{858}
\showcaseopenmojiblack{first aid bag}{859}
\showcaseopenmojiblack{first aid kit}{860}
\showcaseopenmojiblack{first aid}{861}
\showcaseopenmojiblack{first quarter moon face}{862}
\showcaseopenmojiblack{first quarter moon}{863}
\showcaseopenmojiblack{fish cake with swirl}{864}
\showcaseopenmojiblack{fish}{865}
\showcaseopenmojiblack{fisheye}{866}
\showcaseopenmojiblack{fishing pole}{867}
\showcaseopenmojiblack{five o clock}{868}
\showcaseopenmojiblack{five-thirty}{869}
\showcaseopenmojiblack{flag in hole}{870}
\showcaseopenmojiblack{flag- afghanistan}{871}
\showcaseopenmojiblack{flag- albania}{872}
\showcaseopenmojiblack{flag- algeria}{873}
\showcaseopenmojiblack{flag- american samoa}{874}
\showcaseopenmojiblack{flag- andorra}{875}
\showcaseopenmojiblack{flag- angola}{876}
\showcaseopenmojiblack{flag- anguilla}{877}
\showcaseopenmojiblack{flag- antarctica}{878}
\showcaseopenmojiblack{flag- antigua and barbuda}{879}
\showcaseopenmojiblack{flag- argentina}{880}
\showcaseopenmojiblack{flag- armenia}{881}
\showcaseopenmojiblack{flag- aruba}{882}
\showcaseopenmojiblack{flag- ascension island}{883}
\showcaseopenmojiblack{flag- australia}{884}
\showcaseopenmojiblack{flag- austria}{885}
\showcaseopenmojiblack{flag- azerbaijan}{886}
\showcaseopenmojiblack{flag- bahamas}{887}
\showcaseopenmojiblack{flag- bahrain}{888}
\showcaseopenmojiblack{flag- bangladesh}{889}
\showcaseopenmojiblack{flag- barbados}{890}
\showcaseopenmojiblack{flag- belarus}{891}
\showcaseopenmojiblack{flag- belgium}{892}
\showcaseopenmojiblack{flag- belize}{893}
\showcaseopenmojiblack{flag- benin}{894}
\showcaseopenmojiblack{flag- bermuda}{895}
\showcaseopenmojiblack{flag- bhutan}{896}
\showcaseopenmojiblack{flag- bolivia}{897}
\showcaseopenmojiblack{flag- bosnia and herzegovina}{898}
\showcaseopenmojiblack{flag- botswana}{899}
\showcaseopenmojiblack{flag- bouvet island}{900}
\showcaseopenmojiblack{flag- brazil}{901}
\showcaseopenmojiblack{flag- british indian ocean territory}{902}
\showcaseopenmojiblack{flag- british virgin islands}{903}
\showcaseopenmojiblack{flag- brunei}{904}
\showcaseopenmojiblack{flag- bulgaria}{905}
\showcaseopenmojiblack{flag- burkina faso}{906}
\showcaseopenmojiblack{flag- burundi}{907}
\showcaseopenmojiblack{flag- cambodia}{908}
\showcaseopenmojiblack{flag- cameroon}{909}
\showcaseopenmojiblack{flag- canada}{910}
\showcaseopenmojiblack{flag- canary islands}{911}
\showcaseopenmojiblack{flag- cape verde}{912}
\showcaseopenmojiblack{flag- caribbean netherlands}{913}
\showcaseopenmojiblack{flag- cayman islands}{914}
\showcaseopenmojiblack{flag- central african republic}{915}
\showcaseopenmojiblack{flag- ceuta and melilla}{916}
\showcaseopenmojiblack{flag- chad}{917}
\showcaseopenmojiblack{flag- chile}{918}
\showcaseopenmojiblack{flag- china}{919}
\showcaseopenmojiblack{flag- christmas island}{920}
\showcaseopenmojiblack{flag- clipperton island}{921}
\showcaseopenmojiblack{flag- cocos (keeling) islands}{922}
\showcaseopenmojiblack{flag- colombia}{923}
\showcaseopenmojiblack{flag- comoros}{924}
\showcaseopenmojiblack{flag- congo - brazzaville}{925}
\showcaseopenmojiblack{flag- congo - kinshasa}{926}
\showcaseopenmojiblack{flag- cook islands}{927}
\showcaseopenmojiblack{flag- costa rica}{928}
\showcaseopenmojiblack{flag- croatia}{929}
\showcaseopenmojiblack{flag- cuba}{930}
\showcaseopenmojiblack{flag- curacao}{931}
\showcaseopenmojiblack{flag- cyprus}{932}
\showcaseopenmojiblack{flag- czechia}{933}
\showcaseopenmojiblack{flag- cote d ivoire}{934}
\showcaseopenmojiblack{flag- denmark}{935}
\showcaseopenmojiblack{flag- diego garcia}{936}
\showcaseopenmojiblack{flag- djibouti}{937}
\showcaseopenmojiblack{flag- dominica}{938}
\showcaseopenmojiblack{flag- dominican republic}{939}
\showcaseopenmojiblack{flag- ecuador}{940}
\showcaseopenmojiblack{flag- egypt}{941}
\showcaseopenmojiblack{flag- el salvador}{942}
\showcaseopenmojiblack{flag- england}{943}
\showcaseopenmojiblack{flag- equatorial guinea}{944}
\showcaseopenmojiblack{flag- eritrea}{945}
\showcaseopenmojiblack{flag- estonia}{946}
\showcaseopenmojiblack{flag- eswatini}{947}
\showcaseopenmojiblack{flag- ethiopia}{948}
\showcaseopenmojiblack{flag- european union}{949}
\showcaseopenmojiblack{flag- falkland islands}{950}
\showcaseopenmojiblack{flag- faroe islands}{951}
\showcaseopenmojiblack{flag- fiji}{952}
\showcaseopenmojiblack{flag- finland}{953}
\showcaseopenmojiblack{flag- france}{954}
\showcaseopenmojiblack{flag- french guiana}{955}
\showcaseopenmojiblack{flag- french polynesia}{956}
\showcaseopenmojiblack{flag- french southern territories}{957}
\showcaseopenmojiblack{flag- gabon}{958}
\showcaseopenmojiblack{flag- gambia}{959}
\showcaseopenmojiblack{flag- georgia}{960}
\showcaseopenmojiblack{flag- germany}{961}
\showcaseopenmojiblack{flag- ghana}{962}
\showcaseopenmojiblack{flag- gibraltar}{963}
\showcaseopenmojiblack{flag- greece}{964}
\showcaseopenmojiblack{flag- greenland}{965}
\showcaseopenmojiblack{flag- grenada}{966}
\showcaseopenmojiblack{flag- guadeloupe}{967}
\showcaseopenmojiblack{flag- guam}{968}
\showcaseopenmojiblack{flag- guatemala}{969}
\showcaseopenmojiblack{flag- guernsey}{970}
\showcaseopenmojiblack{flag- guinea-bissau}{971}
\showcaseopenmojiblack{flag- guinea}{972}
\showcaseopenmojiblack{flag- guyana}{973}
\showcaseopenmojiblack{flag- haiti}{974}
\showcaseopenmojiblack{flag- heard and mcdonald islands}{975}
\showcaseopenmojiblack{flag- honduras}{976}
\showcaseopenmojiblack{flag- hong kong sar china}{977}
\showcaseopenmojiblack{flag- hungary}{978}
\showcaseopenmojiblack{flag- iceland}{979}
\showcaseopenmojiblack{flag- india}{980}
\showcaseopenmojiblack{flag- indonesia}{981}
\showcaseopenmojiblack{flag- iran}{982}
\showcaseopenmojiblack{flag- iraq}{983}
\showcaseopenmojiblack{flag- ireland}{984}
\showcaseopenmojiblack{flag- isle of man}{985}
\showcaseopenmojiblack{flag- israel}{986}
\showcaseopenmojiblack{flag- italy}{987}
\showcaseopenmojiblack{flag- jamaica}{988}
\showcaseopenmojiblack{flag- japan}{989}
\showcaseopenmojiblack{flag- jersey}{990}
\showcaseopenmojiblack{flag- jordan}{991}
\showcaseopenmojiblack{flag- kazakhstan}{992}
\showcaseopenmojiblack{flag- kenya}{993}
\showcaseopenmojiblack{flag- kiribati}{994}
\showcaseopenmojiblack{flag- kosovo}{995}
\showcaseopenmojiblack{flag- kuwait}{996}
\showcaseopenmojiblack{flag- kyrgyzstan}{997}
\showcaseopenmojiblack{flag- laos}{998}
\showcaseopenmojiblack{flag- latvia}{999}
\showcaseopenmojiblack{flag- lebanon}{1000}
\showcaseopenmojiblack{flag- lesotho}{1001}
\showcaseopenmojiblack{flag- liberia}{1002}
\showcaseopenmojiblack{flag- libya}{1003}
\showcaseopenmojiblack{flag- liechtenstein}{1004}
\showcaseopenmojiblack{flag- lithuania}{1005}
\showcaseopenmojiblack{flag- luxembourg}{1006}
\showcaseopenmojiblack{flag- macao sar china}{1007}
\showcaseopenmojiblack{flag- madagascar}{1008}
\showcaseopenmojiblack{flag- malawi}{1009}
\showcaseopenmojiblack{flag- malaysia}{1010}
\showcaseopenmojiblack{flag- maldives}{1011}
\showcaseopenmojiblack{flag- mali}{1012}
\showcaseopenmojiblack{flag- malta}{1013}
\showcaseopenmojiblack{flag- marshall islands}{1014}
\showcaseopenmojiblack{flag- martinique}{1015}
\showcaseopenmojiblack{flag- mauritania}{1016}
\showcaseopenmojiblack{flag- mauritius}{1017}
\showcaseopenmojiblack{flag- mayotte}{1018}
\showcaseopenmojiblack{flag- mexico}{1019}
\showcaseopenmojiblack{flag- micronesia}{1020}
\showcaseopenmojiblack{flag- moldova}{1021}
\showcaseopenmojiblack{flag- monaco}{1022}
\showcaseopenmojiblack{flag- mongolia}{1023}
\showcaseopenmojiblack{flag- montenegro}{1024}
\showcaseopenmojiblack{flag- montserrat}{1025}
\showcaseopenmojiblack{flag- morocco}{1026}
\showcaseopenmojiblack{flag- mozambique}{1027}
\showcaseopenmojiblack{flag- myanmar (burma)}{1028}
\showcaseopenmojiblack{flag- namibia}{1029}
\showcaseopenmojiblack{flag- nauru}{1030}
\showcaseopenmojiblack{flag- nepal}{1031}
\showcaseopenmojiblack{flag- netherlands}{1032}
\showcaseopenmojiblack{flag- new caledonia}{1033}
\showcaseopenmojiblack{flag- new zealand}{1034}
\showcaseopenmojiblack{flag- nicaragua}{1035}
\showcaseopenmojiblack{flag- niger}{1036}
\showcaseopenmojiblack{flag- nigeria}{1037}
\showcaseopenmojiblack{flag- niue}{1038}
\showcaseopenmojiblack{flag- norfolk island}{1039}
\showcaseopenmojiblack{flag- north korea}{1040}
\showcaseopenmojiblack{flag- north macedonia}{1041}
\showcaseopenmojiblack{flag- northern mariana islands}{1042}
\showcaseopenmojiblack{flag- norway}{1043}
\showcaseopenmojiblack{flag- oman}{1044}
\showcaseopenmojiblack{flag- pakistan}{1045}
\showcaseopenmojiblack{flag- palau}{1046}
\showcaseopenmojiblack{flag- palestinian territories}{1047}
\showcaseopenmojiblack{flag- panama}{1048}
\showcaseopenmojiblack{flag- papua new guinea}{1049}
\showcaseopenmojiblack{flag- paraguay}{1050}
\showcaseopenmojiblack{flag- peru}{1051}
\showcaseopenmojiblack{flag- philippines}{1052}
\showcaseopenmojiblack{flag- pitcairn islands}{1053}
\showcaseopenmojiblack{flag- poland}{1054}
\showcaseopenmojiblack{flag- portugal}{1055}
\showcaseopenmojiblack{flag- puerto rico}{1056}
\showcaseopenmojiblack{flag- qatar}{1057}
\showcaseopenmojiblack{flag- romania}{1058}
\showcaseopenmojiblack{flag- russia}{1059}
\showcaseopenmojiblack{flag- rwanda}{1060}
\showcaseopenmojiblack{flag- reunion}{1061}
\showcaseopenmojiblack{flag- samoa}{1062}
\showcaseopenmojiblack{flag- san marino}{1063}
\showcaseopenmojiblack{flag- saudi arabia}{1064}
\showcaseopenmojiblack{flag- scotland}{1065}
\showcaseopenmojiblack{flag- senegal}{1066}
\showcaseopenmojiblack{flag- serbia}{1067}
\showcaseopenmojiblack{flag- seychelles}{1068}
\showcaseopenmojiblack{flag- sierra leone}{1069}
\showcaseopenmojiblack{flag- singapore}{1070}
\showcaseopenmojiblack{flag- sint maarten}{1071}
\showcaseopenmojiblack{flag- slovakia}{1072}
\showcaseopenmojiblack{flag- slovenia}{1073}
\showcaseopenmojiblack{flag- solomon islands}{1074}
\showcaseopenmojiblack{flag- somalia}{1075}
\showcaseopenmojiblack{flag- south africa}{1076}
\showcaseopenmojiblack{flag- south georgia and south sandwich islands}{1077}
\showcaseopenmojiblack{flag- south korea}{1078}
\showcaseopenmojiblack{flag- south sudan}{1079}
\showcaseopenmojiblack{flag- spain}{1080}
\showcaseopenmojiblack{flag- sri lanka}{1081}
\showcaseopenmojiblack{flag- st}{1082}
\showcaseopenmojiblack{flag- sudan}{1083}
\showcaseopenmojiblack{flag- suriname}{1084}
\showcaseopenmojiblack{flag- svalbard and jan mayen}{1085}
\showcaseopenmojiblack{flag- sweden}{1086}
\showcaseopenmojiblack{flag- switzerland}{1087}
\showcaseopenmojiblack{flag- syria}{1088}
\showcaseopenmojiblack{flag- sao toma and principe}{1089}
\showcaseopenmojiblack{flag- taiwan}{1090}
\showcaseopenmojiblack{flag- tajikistan}{1091}
\showcaseopenmojiblack{flag- tanzania}{1092}
\showcaseopenmojiblack{flag- thailand}{1093}
\showcaseopenmojiblack{flag- timor-leste}{1094}
\showcaseopenmojiblack{flag- togo}{1095}
\showcaseopenmojiblack{flag- tokelau}{1096}
\showcaseopenmojiblack{flag- tonga}{1097}
\showcaseopenmojiblack{flag- trinidad and tobago}{1098}
\showcaseopenmojiblack{flag- tristan da cunha}{1099}
\showcaseopenmojiblack{flag- tunisia}{1100}
\showcaseopenmojiblack{flag- turkiye}{1101}
\showcaseopenmojiblack{flag- turkmenistan}{1102}
\showcaseopenmojiblack{flag- turks and caicos islands}{1103}
\showcaseopenmojiblack{flag- tuvalu}{1104}
\showcaseopenmojiblack{flag- u}{1105}
\showcaseopenmojiblack{flag- uganda}{1106}
\showcaseopenmojiblack{flag- ukraine}{1107}
\showcaseopenmojiblack{flag- united arab emirates}{1108}
\showcaseopenmojiblack{flag- united kingdom}{1109}
\showcaseopenmojiblack{flag- united nations}{1110}
\showcaseopenmojiblack{flag- united states}{1111}
\showcaseopenmojiblack{flag- uruguay}{1112}
\showcaseopenmojiblack{flag- uzbekistan}{1113}
\showcaseopenmojiblack{flag- vanuatu}{1114}
\showcaseopenmojiblack{flag- vatican city}{1115}
\showcaseopenmojiblack{flag- venezuela}{1116}
\showcaseopenmojiblack{flag- vietnam}{1117}
\showcaseopenmojiblack{flag- wales}{1118}
\showcaseopenmojiblack{flag- wallis and futuna}{1119}
\showcaseopenmojiblack{flag- western sahara}{1120}
\showcaseopenmojiblack{flag- yemen}{1121}
\showcaseopenmojiblack{flag- zambia}{1122}
\showcaseopenmojiblack{flag- zimbabwe}{1123}
\showcaseopenmojiblack{flag- aland islands}{1124}
\showcaseopenmojiblack{flagged building}{1125}
\showcaseopenmojiblack{flagged point}{1126}
\showcaseopenmojiblack{flamingo}{1127}
\showcaseopenmojiblack{flashlight}{1128}
\showcaseopenmojiblack{flat shoe}{1129}
\showcaseopenmojiblack{flatbread}{1130}
\showcaseopenmojiblack{fleur-de-lis}{1131}
\showcaseopenmojiblack{flexed biceps- dark skin tone}{1132}
\showcaseopenmojiblack{flexed biceps- light skin tone}{1133}
\showcaseopenmojiblack{flexed biceps- medium skin tone}{1134}
\showcaseopenmojiblack{flexed biceps- medium-dark skin tone}{1135}
\showcaseopenmojiblack{flexed biceps- medium-light skin tone}{1136}
\showcaseopenmojiblack{flexed biceps}{1137}
\showcaseopenmojiblack{floating ice broken}{1138}
\showcaseopenmojiblack{floating ice}{1139}
\showcaseopenmojiblack{flood}{1140}
\showcaseopenmojiblack{floppy disk}{1141}
\showcaseopenmojiblack{flower playing cards}{1142}
\showcaseopenmojiblack{flushed face}{1143}
\showcaseopenmojiblack{flute}{1144}
\showcaseopenmojiblack{fly}{1145}
\showcaseopenmojiblack{flying disc}{1146}
\showcaseopenmojiblack{flying saucer}{1147}
\showcaseopenmojiblack{fog}{1148}
\showcaseopenmojiblack{foggy mountain}{1149}
\showcaseopenmojiblack{foggy}{1150}
\showcaseopenmojiblack{folded hands- dark skin tone}{1151}
\showcaseopenmojiblack{folded hands- light skin tone}{1152}
\showcaseopenmojiblack{folded hands- medium skin tone}{1153}
\showcaseopenmojiblack{folded hands- medium-dark skin tone}{1154}
\showcaseopenmojiblack{folded hands- medium-light skin tone}{1155}
\showcaseopenmojiblack{folded hands}{1156}
\showcaseopenmojiblack{folding hand fan}{1157}
\showcaseopenmojiblack{fondue}{1158}
\showcaseopenmojiblack{foot- dark skin tone}{1159}
\showcaseopenmojiblack{foot- light skin tone}{1160}
\showcaseopenmojiblack{foot- medium skin tone}{1161}
\showcaseopenmojiblack{foot- medium-dark skin tone}{1162}
\showcaseopenmojiblack{foot- medium-light skin tone}{1163}
\showcaseopenmojiblack{foot}{1164}
\showcaseopenmojiblack{footprints}{1165}
\showcaseopenmojiblack{forceps}{1166}
\showcaseopenmojiblack{fork and knife with plate}{1167}
\showcaseopenmojiblack{fork and knife}{1168}
\showcaseopenmojiblack{fortune cookie}{1169}
\showcaseopenmojiblack{forward}{1170}
\showcaseopenmojiblack{fountain pen}{1171}
\showcaseopenmojiblack{fountain}{1172}
\showcaseopenmojiblack{four leaf clover}{1173}
\showcaseopenmojiblack{four o clock}{1174}
\showcaseopenmojiblack{four-thirty}{1175}
\showcaseopenmojiblack{fox}{1176}
\showcaseopenmojiblack{fracture leg}{1177}
\showcaseopenmojiblack{framed picture}{1178}
\showcaseopenmojiblack{free button}{1179}
\showcaseopenmojiblack{french fries}{1180}
\showcaseopenmojiblack{french press}{1181}
\showcaseopenmojiblack{fried shrimp}{1182}
\showcaseopenmojiblack{frog}{1183}
\showcaseopenmojiblack{front-facing baby chick}{1184}
\showcaseopenmojiblack{frowning face with open mouth}{1185}
\showcaseopenmojiblack{frowning face}{1186}
\showcaseopenmojiblack{fuel pump}{1187}
\showcaseopenmojiblack{full moon face}{1188}
\showcaseopenmojiblack{full moon}{1189}
\showcaseopenmojiblack{funeral urn}{1190}
\showcaseopenmojiblack{galicia flag}{1191}
\showcaseopenmojiblack{game die}{1192}
\showcaseopenmojiblack{gardener man}{1193}
\showcaseopenmojiblack{gardener woman}{1194}
\showcaseopenmojiblack{gardening gloves}{1195}
\showcaseopenmojiblack{garlic}{1196}
\showcaseopenmojiblack{gear}{1197}
\showcaseopenmojiblack{geiger counter}{1198}
\showcaseopenmojiblack{gem stone}{1199}
\showcaseopenmojiblack{gemini}{1200}
\showcaseopenmojiblack{genie}{1201}
\showcaseopenmojiblack{ghost}{1202}
\showcaseopenmojiblack{ginger root}{1203}
\showcaseopenmojiblack{giraffe}{1204}
\showcaseopenmojiblack{girl- dark skin tone}{1205}
\showcaseopenmojiblack{girl- light skin tone}{1206}
\showcaseopenmojiblack{girl- medium skin tone}{1207}
\showcaseopenmojiblack{girl- medium-dark skin tone}{1208}
\showcaseopenmojiblack{girl- medium-light skin tone}{1209}
\showcaseopenmojiblack{girl}{1210}
\showcaseopenmojiblack{github}{1211}
\showcaseopenmojiblack{gitlab}{1212}
\showcaseopenmojiblack{glass bottle}{1213}
\showcaseopenmojiblack{glass of milk}{1214}
\showcaseopenmojiblack{glasses}{1215}
\showcaseopenmojiblack{globe showing americas}{1216}
\showcaseopenmojiblack{globe showing asia-australia}{1217}
\showcaseopenmojiblack{globe showing europe-africa}{1218}
\showcaseopenmojiblack{globe with meridians}{1219}
\showcaseopenmojiblack{gloves}{1220}
\showcaseopenmojiblack{glowing star}{1221}
\showcaseopenmojiblack{goal net}{1222}
\showcaseopenmojiblack{goat}{1223}
\showcaseopenmojiblack{goblin}{1224}
\showcaseopenmojiblack{goggles}{1225}
\showcaseopenmojiblack{golang}{1226}
\showcaseopenmojiblack{goldfish}{1227}
\showcaseopenmojiblack{goose}{1228}
\showcaseopenmojiblack{gorilla}{1229}
\showcaseopenmojiblack{gps}{1230}
\showcaseopenmojiblack{graduation cap}{1231}
\showcaseopenmojiblack{grapes}{1232}
\showcaseopenmojiblack{great pyramid of giza}{1233}
\showcaseopenmojiblack{green apple}{1234}
\showcaseopenmojiblack{green book}{1235}
\showcaseopenmojiblack{green circle}{1236}
\showcaseopenmojiblack{green flag}{1237}
\showcaseopenmojiblack{green heart}{1238}
\showcaseopenmojiblack{green hexagon}{1239}
\showcaseopenmojiblack{green salad}{1240}
\showcaseopenmojiblack{green square}{1241}
\showcaseopenmojiblack{greta thunberg}{1242}
\showcaseopenmojiblack{grey heart}{1243}
\showcaseopenmojiblack{grimacing face}{1244}
\showcaseopenmojiblack{grinning cat with smiling eyes}{1245}
\showcaseopenmojiblack{grinning cat}{1246}
\showcaseopenmojiblack{grinning face with big eyes}{1247}
\showcaseopenmojiblack{grinning face with smiling eyes}{1248}
\showcaseopenmojiblack{grinning face with sweat}{1249}
\showcaseopenmojiblack{grinning face}{1250}
\showcaseopenmojiblack{grinning squinting face}{1251}
\showcaseopenmojiblack{growing heart}{1252}
\showcaseopenmojiblack{guard- dark skin tone}{1253}
\showcaseopenmojiblack{guard- light skin tone}{1254}
\showcaseopenmojiblack{guard- medium skin tone}{1255}
\showcaseopenmojiblack{guard- medium-dark skin tone}{1256}
\showcaseopenmojiblack{guard- medium-light skin tone}{1257}
\showcaseopenmojiblack{guard}{1258}
\showcaseopenmojiblack{guide dog}{1259}
\showcaseopenmojiblack{guitar}{1260}
\showcaseopenmojiblack{guy fawkes mask}{1261}
\showcaseopenmojiblack{hacker cat}{1262}
\showcaseopenmojiblack{hair pick}{1263}
\showcaseopenmojiblack{hal 9000}{1264}
\showcaseopenmojiblack{half orange fruit}{1265}
\showcaseopenmojiblack{hamburger menu}{1266}
\showcaseopenmojiblack{hamburger}{1267}
\showcaseopenmojiblack{hammer and pick}{1268}
\showcaseopenmojiblack{hammer and wrench}{1269}
\showcaseopenmojiblack{hammer}{1270}
\showcaseopenmojiblack{hamsa}{1271}
\showcaseopenmojiblack{hamster}{1272}
\showcaseopenmojiblack{hand with fingers splayed- dark skin tone}{1273}
\showcaseopenmojiblack{hand with fingers splayed- light skin tone}{1274}
\showcaseopenmojiblack{hand with fingers splayed- medium skin tone}{1275}
\showcaseopenmojiblack{hand with fingers splayed- medium-dark skin tone}{1276}
\showcaseopenmojiblack{hand with fingers splayed- medium-light skin tone}{1277}
\showcaseopenmojiblack{hand with fingers splayed}{1278}
\showcaseopenmojiblack{hand with index finger and thumb crossed- dark skin tone}{1279}
\showcaseopenmojiblack{hand with index finger and thumb crossed- light skin tone}{1280}
\showcaseopenmojiblack{hand with index finger and thumb crossed- medium skin tone}{1281}
\showcaseopenmojiblack{hand with index finger and thumb crossed- medium-dark skin tone}{1282}
\showcaseopenmojiblack{hand with index finger and thumb crossed- medium-light skin tone}{1283}
\showcaseopenmojiblack{hand with index finger and thumb crossed}{1284}
\showcaseopenmojiblack{handbag}{1285}
\showcaseopenmojiblack{handshake- dark skin tone, light skin tone}{1286}
\showcaseopenmojiblack{handshake- dark skin tone, medium skin tone}{1287}
\showcaseopenmojiblack{handshake- dark skin tone, medium-dark skin tone}{1288}
\showcaseopenmojiblack{handshake- dark skin tone, medium-light skin tone}{1289}
\showcaseopenmojiblack{handshake- dark skin tone}{1290}
\showcaseopenmojiblack{handshake- light skin tone, dark skin tone}{1291}
\showcaseopenmojiblack{handshake- light skin tone, medium skin tone}{1292}
\showcaseopenmojiblack{handshake- light skin tone, medium-dark skin tone}{1293}
\showcaseopenmojiblack{handshake- light skin tone, medium-light skin tone}{1294}
\showcaseopenmojiblack{handshake- light skin tone}{1295}
\showcaseopenmojiblack{handshake- medium skin tone, dark skin tone}{1296}
\showcaseopenmojiblack{handshake- medium skin tone, light skin tone}{1297}
\showcaseopenmojiblack{handshake- medium skin tone, medium-dark skin tone}{1298}
\showcaseopenmojiblack{handshake- medium skin tone, medium-light skin tone}{1299}
\showcaseopenmojiblack{handshake- medium skin tone}{1300}
\showcaseopenmojiblack{handshake- medium-dark skin tone, dark skin tone}{1301}
\showcaseopenmojiblack{handshake- medium-dark skin tone, light skin tone}{1302}
\showcaseopenmojiblack{handshake- medium-dark skin tone, medium skin tone}{1303}
\showcaseopenmojiblack{handshake- medium-dark skin tone, medium-light skin tone}{1304}
\showcaseopenmojiblack{handshake- medium-dark skin tone}{1305}
\showcaseopenmojiblack{handshake- medium-light skin tone, dark skin tone}{1306}
\showcaseopenmojiblack{handshake- medium-light skin tone, light skin tone}{1307}
\showcaseopenmojiblack{handshake- medium-light skin tone, medium skin tone}{1308}
\showcaseopenmojiblack{handshake- medium-light skin tone, medium-dark skin tone}{1309}
\showcaseopenmojiblack{handshake- medium-light skin tone}{1310}
\showcaseopenmojiblack{handshake}{1311}
\showcaseopenmojiblack{hanging gardens of babylon}{1312}
\showcaseopenmojiblack{hatching chick}{1313}
\showcaseopenmojiblack{head shaking horizontally}{1314}
\showcaseopenmojiblack{head shaking vertically}{1315}
\showcaseopenmojiblack{headache}{1316}
\showcaseopenmojiblack{headphone}{1317}
\showcaseopenmojiblack{headstone}{1318}
\showcaseopenmojiblack{health worker- dark skin tone}{1319}
\showcaseopenmojiblack{health worker- light skin tone}{1320}
\showcaseopenmojiblack{health worker- medium skin tone}{1321}
\showcaseopenmojiblack{health worker- medium-dark skin tone}{1322}
\showcaseopenmojiblack{health worker- medium-light skin tone}{1323}
\showcaseopenmojiblack{health worker}{1324}
\showcaseopenmojiblack{hear-no-evil monkey}{1325}
\showcaseopenmojiblack{heart decoration}{1326}
\showcaseopenmojiblack{heart exclamation}{1327}
\showcaseopenmojiblack{heart hands- dark skin tone}{1328}
\showcaseopenmojiblack{heart hands- light skin tone}{1329}
\showcaseopenmojiblack{heart hands- medium skin tone}{1330}
\showcaseopenmojiblack{heart hands- medium-dark skin tone}{1331}
\showcaseopenmojiblack{heart hands- medium-light skin tone}{1332}
\showcaseopenmojiblack{heart hands}{1333}
\showcaseopenmojiblack{heart on fire}{1334}
\showcaseopenmojiblack{heart suit}{1335}
\showcaseopenmojiblack{heart with arrow}{1336}
\showcaseopenmojiblack{heart with ribbon}{1337}
\showcaseopenmojiblack{heavy circle}{1338}
\showcaseopenmojiblack{heavy dollar sign}{1339}
\showcaseopenmojiblack{heavy equals sign}{1340}
\showcaseopenmojiblack{hedgehog}{1341}
\showcaseopenmojiblack{helicopter}{1342}
\showcaseopenmojiblack{help others}{1343}
\showcaseopenmojiblack{herb}{1344}
\showcaseopenmojiblack{hibiscus}{1345}
\showcaseopenmojiblack{high voltage}{1346}
\showcaseopenmojiblack{high-heeled shoe}{1347}
\showcaseopenmojiblack{high-speed train}{1348}
\showcaseopenmojiblack{hiking boot}{1349}
\showcaseopenmojiblack{hindu temple}{1350}
\showcaseopenmojiblack{hippopotamus}{1351}
\showcaseopenmojiblack{hold}{1352}
\showcaseopenmojiblack{hole}{1353}
\showcaseopenmojiblack{hollow red circle}{1354}
\showcaseopenmojiblack{home button}{1355}
\showcaseopenmojiblack{honey pot}{1356}
\showcaseopenmojiblack{honeybee}{1357}
\showcaseopenmojiblack{hook}{1358}
\showcaseopenmojiblack{horizontal black hexagon}{1359}
\showcaseopenmojiblack{horizontal black octagon}{1360}
\showcaseopenmojiblack{horizontal traffic light}{1361}
\showcaseopenmojiblack{horse face}{1362}
\showcaseopenmojiblack{horse jumping hurdle}{1363}
\showcaseopenmojiblack{horse racing- dark skin tone}{1364}
\showcaseopenmojiblack{horse racing- light skin tone}{1365}
\showcaseopenmojiblack{horse racing- medium skin tone}{1366}
\showcaseopenmojiblack{horse racing- medium-dark skin tone}{1367}
\showcaseopenmojiblack{horse racing- medium-light skin tone}{1368}
\showcaseopenmojiblack{horse racing}{1369}
\showcaseopenmojiblack{horse riding}{1370}
\showcaseopenmojiblack{horse}{1371}
\showcaseopenmojiblack{hospital}{1372}
\showcaseopenmojiblack{hot beverage}{1373}
\showcaseopenmojiblack{hot dog}{1374}
\showcaseopenmojiblack{hot face}{1375}
\showcaseopenmojiblack{hot pepper}{1376}
\showcaseopenmojiblack{hot springs}{1377}
\showcaseopenmojiblack{hot-water bottle}{1378}
\showcaseopenmojiblack{hotel}{1379}
\showcaseopenmojiblack{hourglass done}{1380}
\showcaseopenmojiblack{hourglass not done}{1381}
\showcaseopenmojiblack{house with garden}{1382}
\showcaseopenmojiblack{house}{1383}
\showcaseopenmojiblack{houses}{1384}
\showcaseopenmojiblack{hundred points}{1385}
\showcaseopenmojiblack{hushed face}{1386}
\showcaseopenmojiblack{hut}{1387}
\showcaseopenmojiblack{hyacinth}{1388}
\showcaseopenmojiblack{hyphen-minus}{1389}
\showcaseopenmojiblack{ibeacon}{1390}
\showcaseopenmojiblack{ice core sample}{1391}
\showcaseopenmojiblack{ice cream}{1392}
\showcaseopenmojiblack{ice hockey}{1393}
\showcaseopenmojiblack{ice shelf melting}{1394}
\showcaseopenmojiblack{ice shelf}{1395}
\showcaseopenmojiblack{ice skate}{1396}
\showcaseopenmojiblack{ice}{1397}
\showcaseopenmojiblack{iceberg}{1398}
\showcaseopenmojiblack{id button}{1399}
\showcaseopenmojiblack{identification card}{1400}
\showcaseopenmojiblack{inaturalist}{1401}
\showcaseopenmojiblack{inbox tray}{1402}
\showcaseopenmojiblack{inbox}{1403}
\showcaseopenmojiblack{incoming envelope}{1404}
\showcaseopenmojiblack{incredulous face}{1405}
\showcaseopenmojiblack{index pointing at the viewer- dark skin tone}{1406}
\showcaseopenmojiblack{index pointing at the viewer- light skin tone}{1407}
\showcaseopenmojiblack{index pointing at the viewer- medium skin tone}{1408}
\showcaseopenmojiblack{index pointing at the viewer- medium-dark skin tone}{1409}
\showcaseopenmojiblack{index pointing at the viewer- medium-light skin tone}{1410}
\showcaseopenmojiblack{index pointing at the viewer}{1411}
\showcaseopenmojiblack{index pointing up- dark skin tone}{1412}
\showcaseopenmojiblack{index pointing up- light skin tone}{1413}
\showcaseopenmojiblack{index pointing up- medium skin tone}{1414}
\showcaseopenmojiblack{index pointing up- medium-dark skin tone}{1415}
\showcaseopenmojiblack{index pointing up- medium-light skin tone}{1416}
\showcaseopenmojiblack{index pointing up}{1417}
\showcaseopenmojiblack{infinity}{1418}
\showcaseopenmojiblack{information}{1419}
\showcaseopenmojiblack{input latin letters}{1420}
\showcaseopenmojiblack{input latin lowercase}{1421}
\showcaseopenmojiblack{input latin uppercase}{1422}
\showcaseopenmojiblack{input numbers}{1423}
\showcaseopenmojiblack{input symbols}{1424}
\showcaseopenmojiblack{instagram}{1425}
\showcaseopenmojiblack{internet explorer}{1426}
\showcaseopenmojiblack{interview}{1427}
\showcaseopenmojiblack{intestine}{1428}
\showcaseopenmojiblack{intricate}{1429}
\showcaseopenmojiblack{jack-o-lantern}{1430}
\showcaseopenmojiblack{japanese acceptable button}{1431}
\showcaseopenmojiblack{japanese application button}{1432}
\showcaseopenmojiblack{japanese bargain button}{1433}
\showcaseopenmojiblack{japanese castle}{1434}
\showcaseopenmojiblack{japanese congratulations button}{1435}
\showcaseopenmojiblack{japanese discount button}{1436}
\showcaseopenmojiblack{japanese dolls}{1437}
\showcaseopenmojiblack{japanese free of charge button}{1438}
\showcaseopenmojiblack{japanese here button}{1439}
\showcaseopenmojiblack{japanese monthly amount button}{1440}
\showcaseopenmojiblack{japanese no vacancy button}{1441}
\showcaseopenmojiblack{japanese not free of charge button}{1442}
\showcaseopenmojiblack{japanese open for business button}{1443}
\showcaseopenmojiblack{japanese passing grade button}{1444}
\showcaseopenmojiblack{japanese post office}{1445}
\showcaseopenmojiblack{japanese prohibited button}{1446}
\showcaseopenmojiblack{japanese reserved button}{1447}
\showcaseopenmojiblack{japanese secret button}{1448}
\showcaseopenmojiblack{japanese service charge button}{1449}
\showcaseopenmojiblack{japanese symbol for beginner}{1450}
\showcaseopenmojiblack{japanese vacancy button}{1451}
\showcaseopenmojiblack{jar with blue content}{1452}
\showcaseopenmojiblack{jar with brown content}{1453}
\showcaseopenmojiblack{jar with green content}{1454}
\showcaseopenmojiblack{jar with orange content}{1455}
\showcaseopenmojiblack{jar with purple content}{1456}
\showcaseopenmojiblack{jar with red content}{1457}
\showcaseopenmojiblack{jar with yellow content}{1458}
\showcaseopenmojiblack{jar}{1459}
\showcaseopenmojiblack{javascript}{1460}
\showcaseopenmojiblack{jeans}{1461}
\showcaseopenmojiblack{jellyfin}{1462}
\showcaseopenmojiblack{jellyfish}{1463}
\showcaseopenmojiblack{joint pain}{1464}
\showcaseopenmojiblack{joker}{1465}
\showcaseopenmojiblack{joystick}{1466}
\showcaseopenmojiblack{judge- dark skin tone}{1467}
\showcaseopenmojiblack{judge- light skin tone}{1468}
\showcaseopenmojiblack{judge- medium skin tone}{1469}
\showcaseopenmojiblack{judge- medium-dark skin tone}{1470}
\showcaseopenmojiblack{judge- medium-light skin tone}{1471}
\showcaseopenmojiblack{judge}{1472}
\showcaseopenmojiblack{kaaba}{1473}
\showcaseopenmojiblack{kangaroo}{1474}
\showcaseopenmojiblack{kehrwoche}{1475}
\showcaseopenmojiblack{key}{1476}
\showcaseopenmojiblack{keyboard}{1477}
\showcaseopenmojiblack{keycap- hashtag}{1478}
\showcaseopenmojiblack{keycap- -}{1479}
\showcaseopenmojiblack{keycap- 0}{1480}
\showcaseopenmojiblack{keycap- 1}{1481}
\showcaseopenmojiblack{keycap- 2}{1482}
\showcaseopenmojiblack{keycap- 3}{1483}
\showcaseopenmojiblack{keycap- 4}{1484}
\showcaseopenmojiblack{keycap- 5}{1485}
\showcaseopenmojiblack{keycap- 6}{1486}
\showcaseopenmojiblack{keycap- 7}{1487}
\showcaseopenmojiblack{keycap- 8}{1488}
\showcaseopenmojiblack{keycap- 9}{1489}
\showcaseopenmojiblack{keycap- 10}{1490}
\showcaseopenmojiblack{khanda}{1491}
\showcaseopenmojiblack{kick scooter}{1492}
\showcaseopenmojiblack{kidney}{1493}
\showcaseopenmojiblack{kimono}{1494}
\showcaseopenmojiblack{kiss mark}{1495}
\showcaseopenmojiblack{kiss- dark skin tone}{1496}
\showcaseopenmojiblack{kiss- light skin tone}{1497}
\showcaseopenmojiblack{kiss- man, man, dark skin tone, light skin tone}{1498}
\showcaseopenmojiblack{kiss- man, man, dark skin tone, medium skin tone}{1499}
\showcaseopenmojiblack{kiss- man, man, dark skin tone, medium-dark skin tone}{1500}
\showcaseopenmojiblack{kiss- man, man, dark skin tone, medium-light skin tone}{1501}
\showcaseopenmojiblack{kiss- man, man, dark skin tone}{1502}
\showcaseopenmojiblack{kiss- man, man, light skin tone, dark skin tone}{1503}
\showcaseopenmojiblack{kiss- man, man, light skin tone, medium skin tone}{1504}
\showcaseopenmojiblack{kiss- man, man, light skin tone, medium-dark skin tone}{1505}
\showcaseopenmojiblack{kiss- man, man, light skin tone, medium-light skin tone}{1506}
\showcaseopenmojiblack{kiss- man, man, light skin tone}{1507}
\showcaseopenmojiblack{kiss- man, man, medium skin tone, dark skin tone}{1508}
\showcaseopenmojiblack{kiss- man, man, medium skin tone, light skin tone}{1509}
\showcaseopenmojiblack{kiss- man, man, medium skin tone, medium-dark skin tone}{1510}
\showcaseopenmojiblack{kiss- man, man, medium skin tone, medium-light skin tone}{1511}
\showcaseopenmojiblack{kiss- man, man, medium skin tone}{1512}
\showcaseopenmojiblack{kiss- man, man, medium-dark skin tone, dark skin tone}{1513}
\showcaseopenmojiblack{kiss- man, man, medium-dark skin tone, light skin tone}{1514}
\showcaseopenmojiblack{kiss- man, man, medium-dark skin tone, medium skin tone}{1515}
\showcaseopenmojiblack{kiss- man, man, medium-dark skin tone, medium-light skin tone}{1516}
\showcaseopenmojiblack{kiss- man, man, medium-dark skin tone}{1517}
\showcaseopenmojiblack{kiss- man, man, medium-light skin tone, dark skin tone}{1518}
\showcaseopenmojiblack{kiss- man, man, medium-light skin tone, light skin tone}{1519}
\showcaseopenmojiblack{kiss- man, man, medium-light skin tone, medium skin tone}{1520}
\showcaseopenmojiblack{kiss- man, man, medium-light skin tone, medium-dark skin tone}{1521}
\showcaseopenmojiblack{kiss- man, man, medium-light skin tone}{1522}
\showcaseopenmojiblack{kiss- man, man}{1523}
\showcaseopenmojiblack{kiss- medium skin tone}{1524}
\showcaseopenmojiblack{kiss- medium-dark skin tone}{1525}
\showcaseopenmojiblack{kiss- medium-light skin tone}{1526}
\showcaseopenmojiblack{kiss- person, person, dark skin tone, light skin tone}{1527}
\showcaseopenmojiblack{kiss- person, person, dark skin tone, medium skin tone}{1528}
\showcaseopenmojiblack{kiss- person, person, dark skin tone, medium-dark skin tone}{1529}
\showcaseopenmojiblack{kiss- person, person, dark skin tone, medium-light skin tone}{1530}
\showcaseopenmojiblack{kiss- person, person, light skin tone, dark skin tone}{1531}
\showcaseopenmojiblack{kiss- person, person, light skin tone, medium skin tone}{1532}
\showcaseopenmojiblack{kiss- person, person, light skin tone, medium-dark skin tone}{1533}
\showcaseopenmojiblack{kiss- person, person, light skin tone, medium-light skin tone}{1534}
\showcaseopenmojiblack{kiss- person, person, medium skin tone, dark skin tone}{1535}
\showcaseopenmojiblack{kiss- person, person, medium skin tone, light skin tone}{1536}
\showcaseopenmojiblack{kiss- person, person, medium skin tone, medium-dark skin tone}{1537}
\showcaseopenmojiblack{kiss- person, person, medium skin tone, medium-light skin tone}{1538}
\showcaseopenmojiblack{kiss- person, person, medium-dark skin tone, dark skin tone}{1539}
\showcaseopenmojiblack{kiss- person, person, medium-dark skin tone, light skin tone}{1540}
\showcaseopenmojiblack{kiss- person, person, medium-dark skin tone, medium skin tone}{1541}
\showcaseopenmojiblack{kiss- person, person, medium-dark skin tone, medium-light skin tone}{1542}
\showcaseopenmojiblack{kiss- person, person, medium-light skin tone, dark skin tone}{1543}
\showcaseopenmojiblack{kiss- person, person, medium-light skin tone, light skin tone}{1544}
\showcaseopenmojiblack{kiss- person, person, medium-light skin tone, medium skin tone}{1545}
\showcaseopenmojiblack{kiss- person, person, medium-light skin tone, medium-dark skin tone}{1546}
\showcaseopenmojiblack{kiss- woman, man, dark skin tone, light skin tone}{1547}
\showcaseopenmojiblack{kiss- woman, man, dark skin tone, medium skin tone}{1548}
\showcaseopenmojiblack{kiss- woman, man, dark skin tone, medium-dark skin tone}{1549}
\showcaseopenmojiblack{kiss- woman, man, dark skin tone, medium-light skin tone}{1550}
\showcaseopenmojiblack{kiss- woman, man, dark skin tone}{1551}
\showcaseopenmojiblack{kiss- woman, man, light skin tone, dark skin tone}{1552}
\showcaseopenmojiblack{kiss- woman, man, light skin tone, medium skin tone}{1553}
\showcaseopenmojiblack{kiss- woman, man, light skin tone, medium-dark skin tone}{1554}
\showcaseopenmojiblack{kiss- woman, man, light skin tone, medium-light skin tone}{1555}
\showcaseopenmojiblack{kiss- woman, man, light skin tone}{1556}
\showcaseopenmojiblack{kiss- woman, man, medium skin tone, dark skin tone}{1557}
\showcaseopenmojiblack{kiss- woman, man, medium skin tone, light skin tone}{1558}
\showcaseopenmojiblack{kiss- woman, man, medium skin tone, medium-dark skin tone}{1559}
\showcaseopenmojiblack{kiss- woman, man, medium skin tone, medium-light skin tone}{1560}
\showcaseopenmojiblack{kiss- woman, man, medium skin tone}{1561}
\showcaseopenmojiblack{kiss- woman, man, medium-dark skin tone, dark skin tone}{1562}
\showcaseopenmojiblack{kiss- woman, man, medium-dark skin tone, light skin tone}{1563}
\showcaseopenmojiblack{kiss- woman, man, medium-dark skin tone, medium skin tone}{1564}
\showcaseopenmojiblack{kiss- woman, man, medium-dark skin tone, medium-light skin tone}{1565}
\showcaseopenmojiblack{kiss- woman, man, medium-dark skin tone}{1566}
\showcaseopenmojiblack{kiss- woman, man, medium-light skin tone, dark skin tone}{1567}
\showcaseopenmojiblack{kiss- woman, man, medium-light skin tone, light skin tone}{1568}
\showcaseopenmojiblack{kiss- woman, man, medium-light skin tone, medium skin tone}{1569}
\showcaseopenmojiblack{kiss- woman, man, medium-light skin tone, medium-dark skin tone}{1570}
\showcaseopenmojiblack{kiss- woman, man, medium-light skin tone}{1571}
\showcaseopenmojiblack{kiss- woman, man}{1572}
\showcaseopenmojiblack{kiss- woman, woman, dark skin tone, light skin tone}{1573}
\showcaseopenmojiblack{kiss- woman, woman, dark skin tone, medium skin tone}{1574}
\showcaseopenmojiblack{kiss- woman, woman, dark skin tone, medium-dark skin tone}{1575}
\showcaseopenmojiblack{kiss- woman, woman, dark skin tone, medium-light skin tone}{1576}
\showcaseopenmojiblack{kiss- woman, woman, dark skin tone}{1577}
\showcaseopenmojiblack{kiss- woman, woman, light skin tone, dark skin tone}{1578}
\showcaseopenmojiblack{kiss- woman, woman, light skin tone, medium skin tone}{1579}
\showcaseopenmojiblack{kiss- woman, woman, light skin tone, medium-dark skin tone}{1580}
\showcaseopenmojiblack{kiss- woman, woman, light skin tone, medium-light skin tone}{1581}
\showcaseopenmojiblack{kiss- woman, woman, light skin tone}{1582}
\showcaseopenmojiblack{kiss- woman, woman, medium skin tone, dark skin tone}{1583}
\showcaseopenmojiblack{kiss- woman, woman, medium skin tone, light skin tone}{1584}
\showcaseopenmojiblack{kiss- woman, woman, medium skin tone, medium-dark skin tone}{1585}
\showcaseopenmojiblack{kiss- woman, woman, medium skin tone, medium-light skin tone}{1586}
\showcaseopenmojiblack{kiss- woman, woman, medium skin tone}{1587}
\showcaseopenmojiblack{kiss- woman, woman, medium-dark skin tone, dark skin tone}{1588}
\showcaseopenmojiblack{kiss- woman, woman, medium-dark skin tone, light skin tone}{1589}
\showcaseopenmojiblack{kiss- woman, woman, medium-dark skin tone, medium skin tone}{1590}
\showcaseopenmojiblack{kiss- woman, woman, medium-dark skin tone, medium-light skin tone}{1591}
\showcaseopenmojiblack{kiss- woman, woman, medium-dark skin tone}{1592}
\showcaseopenmojiblack{kiss- woman, woman, medium-light skin tone, dark skin tone}{1593}
\showcaseopenmojiblack{kiss- woman, woman, medium-light skin tone, light skin tone}{1594}
\showcaseopenmojiblack{kiss- woman, woman, medium-light skin tone, medium skin tone}{1595}
\showcaseopenmojiblack{kiss- woman, woman, medium-light skin tone, medium-dark skin tone}{1596}
\showcaseopenmojiblack{kiss- woman, woman, medium-light skin tone}{1597}
\showcaseopenmojiblack{kiss- woman, woman}{1598}
\showcaseopenmojiblack{kiss}{1599}
\showcaseopenmojiblack{kissing cat}{1600}
\showcaseopenmojiblack{kissing face with closed eyes}{1601}
\showcaseopenmojiblack{kissing face with smiling eyes}{1602}
\showcaseopenmojiblack{kissing face}{1603}
\showcaseopenmojiblack{kitchen knife}{1604}
\showcaseopenmojiblack{kite}{1605}
\showcaseopenmojiblack{kiwi fruit}{1606}
\showcaseopenmojiblack{knee pain}{1607}
\showcaseopenmojiblack{knot}{1608}
\showcaseopenmojiblack{koala}{1609}
\showcaseopenmojiblack{kotlin}{1610}
\showcaseopenmojiblack{la rioja flag}{1611}
\showcaseopenmojiblack{lab coat}{1612}
\showcaseopenmojiblack{label}{1613}
\showcaseopenmojiblack{lacrosse}{1614}
\showcaseopenmojiblack{ladder}{1615}
\showcaseopenmojiblack{lady beetle}{1616}
\showcaseopenmojiblack{landslide}{1617}
\showcaseopenmojiblack{laptop}{1618}
\showcaseopenmojiblack{large blue diamond}{1619}
\showcaseopenmojiblack{large intestine}{1620}
\showcaseopenmojiblack{large orange diamond}{1621}
\showcaseopenmojiblack{last quarter moon face}{1622}
\showcaseopenmojiblack{last quarter moon}{1623}
\showcaseopenmojiblack{last track button}{1624}
\showcaseopenmojiblack{latin cross}{1625}
\showcaseopenmojiblack{latte macchiato}{1626}
\showcaseopenmojiblack{lawn mower}{1627}
\showcaseopenmojiblack{leaf fluttering in wind}{1628}
\showcaseopenmojiblack{leafy green}{1629}
\showcaseopenmojiblack{led}{1630}
\showcaseopenmojiblack{ledger}{1631}
\showcaseopenmojiblack{left arrow curving right}{1632}
\showcaseopenmojiblack{left arrow}{1633}
\showcaseopenmojiblack{left luggage}{1634}
\showcaseopenmojiblack{left right black arrow}{1635}
\showcaseopenmojiblack{left speech bubble}{1636}
\showcaseopenmojiblack{left-facing fist- dark skin tone}{1637}
\showcaseopenmojiblack{left-facing fist- light skin tone}{1638}
\showcaseopenmojiblack{left-facing fist- medium skin tone}{1639}
\showcaseopenmojiblack{left-facing fist- medium-dark skin tone}{1640}
\showcaseopenmojiblack{left-facing fist- medium-light skin tone}{1641}
\showcaseopenmojiblack{left-facing fist}{1642}
\showcaseopenmojiblack{left-right arrow}{1643}
\showcaseopenmojiblack{leftwards hand- dark skin tone}{1644}
\showcaseopenmojiblack{leftwards hand- light skin tone}{1645}
\showcaseopenmojiblack{leftwards hand- medium skin tone}{1646}
\showcaseopenmojiblack{leftwards hand- medium-dark skin tone}{1647}
\showcaseopenmojiblack{leftwards hand- medium-light skin tone}{1648}
\showcaseopenmojiblack{leftwards hand}{1649}
\showcaseopenmojiblack{leftwards pushing hand- dark skin tone}{1650}
\showcaseopenmojiblack{leftwards pushing hand- light skin tone}{1651}
\showcaseopenmojiblack{leftwards pushing hand- medium skin tone}{1652}
\showcaseopenmojiblack{leftwards pushing hand- medium-dark skin tone}{1653}
\showcaseopenmojiblack{leftwards pushing hand- medium-light skin tone}{1654}
\showcaseopenmojiblack{leftwards pushing hand}{1655}
\showcaseopenmojiblack{leg- dark skin tone}{1656}
\showcaseopenmojiblack{leg- light skin tone}{1657}
\showcaseopenmojiblack{leg- medium skin tone}{1658}
\showcaseopenmojiblack{leg- medium-dark skin tone}{1659}
\showcaseopenmojiblack{leg- medium-light skin tone}{1660}
\showcaseopenmojiblack{leg}{1661}
\showcaseopenmojiblack{lemon}{1662}
\showcaseopenmojiblack{lentils with spaetzle}{1663}
\showcaseopenmojiblack{leo}{1664}
\showcaseopenmojiblack{leopard}{1665}
\showcaseopenmojiblack{level slider}{1666}
\showcaseopenmojiblack{libra}{1667}
\showcaseopenmojiblack{light blue heart}{1668}
\showcaseopenmojiblack{light bulb}{1669}
\showcaseopenmojiblack{light rail}{1670}
\showcaseopenmojiblack{light skin tone}{1671}
\showcaseopenmojiblack{lighter}{1672}
\showcaseopenmojiblack{lighthouse of alexandria}{1673}
\showcaseopenmojiblack{lime}{1674}
\showcaseopenmojiblack{link}{1675}
\showcaseopenmojiblack{linked paperclips}{1676}
\showcaseopenmojiblack{linkedin}{1677}
\showcaseopenmojiblack{lion}{1678}
\showcaseopenmojiblack{lipstick}{1679}
\showcaseopenmojiblack{litter in bin sign}{1680}
\showcaseopenmojiblack{liver}{1681}
\showcaseopenmojiblack{lizard}{1682}
\showcaseopenmojiblack{llama}{1683}
\showcaseopenmojiblack{lobster}{1684}
\showcaseopenmojiblack{location indicator red}{1685}
\showcaseopenmojiblack{location indicator}{1686}
\showcaseopenmojiblack{locked with key}{1687}
\showcaseopenmojiblack{locked with pen}{1688}
\showcaseopenmojiblack{locked}{1689}
\showcaseopenmojiblack{locomotion}{1690}
\showcaseopenmojiblack{locomotive}{1691}
\showcaseopenmojiblack{lollipop}{1692}
\showcaseopenmojiblack{long drum}{1693}
\showcaseopenmojiblack{lotion bottle}{1694}
\showcaseopenmojiblack{lotus}{1695}
\showcaseopenmojiblack{loudly crying face}{1696}
\showcaseopenmojiblack{loudspeaker}{1697}
\showcaseopenmojiblack{love hotel}{1698}
\showcaseopenmojiblack{love letter}{1699}
\showcaseopenmojiblack{love-you gesture- dark skin tone}{1700}
\showcaseopenmojiblack{love-you gesture- light skin tone}{1701}
\showcaseopenmojiblack{love-you gesture- medium skin tone}{1702}
\showcaseopenmojiblack{love-you gesture- medium-dark skin tone}{1703}
\showcaseopenmojiblack{love-you gesture- medium-light skin tone}{1704}
\showcaseopenmojiblack{love-you gesture}{1705}
\showcaseopenmojiblack{low battery}{1706}
\showcaseopenmojiblack{luggage}{1707}
\showcaseopenmojiblack{lungs}{1708}
\showcaseopenmojiblack{lying face}{1709}
\showcaseopenmojiblack{macaw}{1710}
\showcaseopenmojiblack{madrid autonomous community flag}{1711}
\showcaseopenmojiblack{mage- dark skin tone}{1712}
\showcaseopenmojiblack{mage- light skin tone}{1713}
\showcaseopenmojiblack{mage- medium skin tone}{1714}
\showcaseopenmojiblack{mage- medium-dark skin tone}{1715}
\showcaseopenmojiblack{mage- medium-light skin tone}{1716}
\showcaseopenmojiblack{mage}{1717}
\showcaseopenmojiblack{magic wand}{1718}
\showcaseopenmojiblack{magnet}{1719}
\showcaseopenmojiblack{magnifying glass tilted left}{1720}
\showcaseopenmojiblack{magnifying glass tilted right}{1721}
\showcaseopenmojiblack{mahjong red dragon}{1722}
\showcaseopenmojiblack{male doctor}{1723}
\showcaseopenmojiblack{male nurse}{1724}
\showcaseopenmojiblack{male sign}{1725}
\showcaseopenmojiblack{mammoth}{1726}
\showcaseopenmojiblack{man artist- dark skin tone}{1727}
\showcaseopenmojiblack{man artist- light skin tone}{1728}
\showcaseopenmojiblack{man artist- medium skin tone}{1729}
\showcaseopenmojiblack{man artist- medium-dark skin tone}{1730}
\showcaseopenmojiblack{man artist- medium-light skin tone}{1731}
\showcaseopenmojiblack{man artist}{1732}
\showcaseopenmojiblack{man astronaut- dark skin tone}{1733}
\showcaseopenmojiblack{man astronaut- light skin tone}{1734}
\showcaseopenmojiblack{man astronaut- medium skin tone}{1735}
\showcaseopenmojiblack{man astronaut- medium-dark skin tone}{1736}
\showcaseopenmojiblack{man astronaut- medium-light skin tone}{1737}
\showcaseopenmojiblack{man astronaut}{1738}
\showcaseopenmojiblack{man barista}{1739}
\showcaseopenmojiblack{man biking- dark skin tone}{1740}
\showcaseopenmojiblack{man biking- light skin tone}{1741}
\showcaseopenmojiblack{man biking- medium skin tone}{1742}
\showcaseopenmojiblack{man biking- medium-dark skin tone}{1743}
\showcaseopenmojiblack{man biking- medium-light skin tone}{1744}
\showcaseopenmojiblack{man biking}{1745}
\showcaseopenmojiblack{man bouncing ball- dark skin tone}{1746}
\showcaseopenmojiblack{man bouncing ball- light skin tone}{1747}
\showcaseopenmojiblack{man bouncing ball- medium skin tone}{1748}
\showcaseopenmojiblack{man bouncing ball- medium-dark skin tone}{1749}
\showcaseopenmojiblack{man bouncing ball- medium-light skin tone}{1750}
\showcaseopenmojiblack{man bouncing ball}{1751}
\showcaseopenmojiblack{man bowing- dark skin tone}{1752}
\showcaseopenmojiblack{man bowing- light skin tone}{1753}
\showcaseopenmojiblack{man bowing- medium skin tone}{1754}
\showcaseopenmojiblack{man bowing- medium-dark skin tone}{1755}
\showcaseopenmojiblack{man bowing- medium-light skin tone}{1756}
\showcaseopenmojiblack{man bowing}{1757}
\showcaseopenmojiblack{man cartwheeling- dark skin tone}{1758}
\showcaseopenmojiblack{man cartwheeling- light skin tone}{1759}
\showcaseopenmojiblack{man cartwheeling- medium skin tone}{1760}
\showcaseopenmojiblack{man cartwheeling- medium-dark skin tone}{1761}
\showcaseopenmojiblack{man cartwheeling- medium-light skin tone}{1762}
\showcaseopenmojiblack{man cartwheeling}{1763}
\showcaseopenmojiblack{man climbing- dark skin tone}{1764}
\showcaseopenmojiblack{man climbing- light skin tone}{1765}
\showcaseopenmojiblack{man climbing- medium skin tone}{1766}
\showcaseopenmojiblack{man climbing- medium-dark skin tone}{1767}
\showcaseopenmojiblack{man climbing- medium-light skin tone}{1768}
\showcaseopenmojiblack{man climbing}{1769}
\showcaseopenmojiblack{man construction worker- dark skin tone}{1770}
\showcaseopenmojiblack{man construction worker- light skin tone}{1771}
\showcaseopenmojiblack{man construction worker- medium skin tone}{1772}
\showcaseopenmojiblack{man construction worker- medium-dark skin tone}{1773}
\showcaseopenmojiblack{man construction worker- medium-light skin tone}{1774}
\showcaseopenmojiblack{man construction worker}{1775}
\showcaseopenmojiblack{man cook- dark skin tone}{1776}
\showcaseopenmojiblack{man cook- light skin tone}{1777}
\showcaseopenmojiblack{man cook- medium skin tone}{1778}
\showcaseopenmojiblack{man cook- medium-dark skin tone}{1779}
\showcaseopenmojiblack{man cook- medium-light skin tone}{1780}
\showcaseopenmojiblack{man cook}{1781}
\showcaseopenmojiblack{man dancing- dark skin tone}{1782}
\showcaseopenmojiblack{man dancing- light skin tone}{1783}
\showcaseopenmojiblack{man dancing- medium skin tone}{1784}
\showcaseopenmojiblack{man dancing- medium-dark skin tone}{1785}
\showcaseopenmojiblack{man dancing- medium-light skin tone}{1786}
\showcaseopenmojiblack{man dancing}{1787}
\showcaseopenmojiblack{man detective- dark skin tone}{1788}
\showcaseopenmojiblack{man detective- light skin tone}{1789}
\showcaseopenmojiblack{man detective- medium skin tone}{1790}
\showcaseopenmojiblack{man detective- medium-dark skin tone}{1791}
\showcaseopenmojiblack{man detective- medium-light skin tone}{1792}
\showcaseopenmojiblack{man detective}{1793}
\showcaseopenmojiblack{man elf- dark skin tone}{1794}
\showcaseopenmojiblack{man elf- light skin tone}{1795}
\showcaseopenmojiblack{man elf- medium skin tone}{1796}
\showcaseopenmojiblack{man elf- medium-dark skin tone}{1797}
\showcaseopenmojiblack{man elf- medium-light skin tone}{1798}
\showcaseopenmojiblack{man elf}{1799}
\showcaseopenmojiblack{man facepalming- dark skin tone}{1800}
\showcaseopenmojiblack{man facepalming- light skin tone}{1801}
\showcaseopenmojiblack{man facepalming- medium skin tone}{1802}
\showcaseopenmojiblack{man facepalming- medium-dark skin tone}{1803}
\showcaseopenmojiblack{man facepalming- medium-light skin tone}{1804}
\showcaseopenmojiblack{man facepalming}{1805}
\showcaseopenmojiblack{man factory worker- dark skin tone}{1806}
\showcaseopenmojiblack{man factory worker- light skin tone}{1807}
\showcaseopenmojiblack{man factory worker- medium skin tone}{1808}
\showcaseopenmojiblack{man factory worker- medium-dark skin tone}{1809}
\showcaseopenmojiblack{man factory worker- medium-light skin tone}{1810}
\showcaseopenmojiblack{man factory worker}{1811}
\showcaseopenmojiblack{man fairy- dark skin tone}{1812}
\showcaseopenmojiblack{man fairy- light skin tone}{1813}
\showcaseopenmojiblack{man fairy- medium skin tone}{1814}
\showcaseopenmojiblack{man fairy- medium-dark skin tone}{1815}
\showcaseopenmojiblack{man fairy- medium-light skin tone}{1816}
\showcaseopenmojiblack{man fairy}{1817}
\showcaseopenmojiblack{man farmer- dark skin tone}{1818}
\showcaseopenmojiblack{man farmer- light skin tone}{1819}
\showcaseopenmojiblack{man farmer- medium skin tone}{1820}
\showcaseopenmojiblack{man farmer- medium-dark skin tone}{1821}
\showcaseopenmojiblack{man farmer- medium-light skin tone}{1822}
\showcaseopenmojiblack{man farmer}{1823}
\showcaseopenmojiblack{man feeding baby- dark skin tone}{1824}
\showcaseopenmojiblack{man feeding baby- light skin tone}{1825}
\showcaseopenmojiblack{man feeding baby- medium skin tone}{1826}
\showcaseopenmojiblack{man feeding baby- medium-dark skin tone}{1827}
\showcaseopenmojiblack{man feeding baby- medium-light skin tone}{1828}
\showcaseopenmojiblack{man feeding baby}{1829}
\showcaseopenmojiblack{man firefighter- dark skin tone}{1830}
\showcaseopenmojiblack{man firefighter- light skin tone}{1831}
\showcaseopenmojiblack{man firefighter- medium skin tone}{1832}
\showcaseopenmojiblack{man firefighter- medium-dark skin tone}{1833}
\showcaseopenmojiblack{man firefighter- medium-light skin tone}{1834}
\showcaseopenmojiblack{man firefighter}{1835}
\showcaseopenmojiblack{man frowning- dark skin tone}{1836}
\showcaseopenmojiblack{man frowning- light skin tone}{1837}
\showcaseopenmojiblack{man frowning- medium skin tone}{1838}
\showcaseopenmojiblack{man frowning- medium-dark skin tone}{1839}
\showcaseopenmojiblack{man frowning- medium-light skin tone}{1840}
\showcaseopenmojiblack{man frowning}{1841}
\showcaseopenmojiblack{man genie}{1842}
\showcaseopenmojiblack{man gesturing no- dark skin tone}{1843}
\showcaseopenmojiblack{man gesturing no- light skin tone}{1844}
\showcaseopenmojiblack{man gesturing no- medium skin tone}{1845}
\showcaseopenmojiblack{man gesturing no- medium-dark skin tone}{1846}
\showcaseopenmojiblack{man gesturing no- medium-light skin tone}{1847}
\showcaseopenmojiblack{man gesturing no}{1848}
\showcaseopenmojiblack{man gesturing ok- dark skin tone}{1849}
\showcaseopenmojiblack{man gesturing ok- light skin tone}{1850}
\showcaseopenmojiblack{man gesturing ok- medium skin tone}{1851}
\showcaseopenmojiblack{man gesturing ok- medium-dark skin tone}{1852}
\showcaseopenmojiblack{man gesturing ok- medium-light skin tone}{1853}
\showcaseopenmojiblack{man gesturing ok}{1854}
\showcaseopenmojiblack{man getting haircut- dark skin tone}{1855}
\showcaseopenmojiblack{man getting haircut- light skin tone}{1856}
\showcaseopenmojiblack{man getting haircut- medium skin tone}{1857}
\showcaseopenmojiblack{man getting haircut- medium-dark skin tone}{1858}
\showcaseopenmojiblack{man getting haircut- medium-light skin tone}{1859}
\showcaseopenmojiblack{man getting haircut}{1860}
\showcaseopenmojiblack{man getting massage- dark skin tone}{1861}
\showcaseopenmojiblack{man getting massage- light skin tone}{1862}
\showcaseopenmojiblack{man getting massage- medium skin tone}{1863}
\showcaseopenmojiblack{man getting massage- medium-dark skin tone}{1864}
\showcaseopenmojiblack{man getting massage- medium-light skin tone}{1865}
\showcaseopenmojiblack{man getting massage}{1866}
\showcaseopenmojiblack{man golfing- dark skin tone}{1867}
\showcaseopenmojiblack{man golfing- light skin tone}{1868}
\showcaseopenmojiblack{man golfing- medium skin tone}{1869}
\showcaseopenmojiblack{man golfing- medium-dark skin tone}{1870}
\showcaseopenmojiblack{man golfing- medium-light skin tone}{1871}
\showcaseopenmojiblack{man golfing}{1872}
\showcaseopenmojiblack{man guard- dark skin tone}{1873}
\showcaseopenmojiblack{man guard- light skin tone}{1874}
\showcaseopenmojiblack{man guard- medium skin tone}{1875}
\showcaseopenmojiblack{man guard- medium-dark skin tone}{1876}
\showcaseopenmojiblack{man guard- medium-light skin tone}{1877}
\showcaseopenmojiblack{man guard}{1878}
\showcaseopenmojiblack{man health worker- dark skin tone}{1879}
\showcaseopenmojiblack{man health worker- light skin tone}{1880}
\showcaseopenmojiblack{man health worker- medium skin tone}{1881}
\showcaseopenmojiblack{man health worker- medium-dark skin tone}{1882}
\showcaseopenmojiblack{man health worker- medium-light skin tone}{1883}
\showcaseopenmojiblack{man health worker}{1884}
\showcaseopenmojiblack{man in lotus position- dark skin tone}{1885}
\showcaseopenmojiblack{man in lotus position- light skin tone}{1886}
\showcaseopenmojiblack{man in lotus position- medium skin tone}{1887}
\showcaseopenmojiblack{man in lotus position- medium-dark skin tone}{1888}
\showcaseopenmojiblack{man in lotus position- medium-light skin tone}{1889}
\showcaseopenmojiblack{man in lotus position}{1890}
\showcaseopenmojiblack{man in manual wheelchair facing right}{1891}
\showcaseopenmojiblack{man in manual wheelchair- dark skin tone}{1892}
\showcaseopenmojiblack{man in manual wheelchair- light skin tone}{1893}
\showcaseopenmojiblack{man in manual wheelchair- medium skin tone}{1894}
\showcaseopenmojiblack{man in manual wheelchair- medium-dark skin tone}{1895}
\showcaseopenmojiblack{man in manual wheelchair- medium-light skin tone}{1896}
\showcaseopenmojiblack{man in manual wheelchair}{1897}
\showcaseopenmojiblack{man in motorized wheelchair facing right}{1898}
\showcaseopenmojiblack{man in motorized wheelchair- dark skin tone}{1899}
\showcaseopenmojiblack{man in motorized wheelchair- light skin tone}{1900}
\showcaseopenmojiblack{man in motorized wheelchair- medium skin tone}{1901}
\showcaseopenmojiblack{man in motorized wheelchair- medium-dark skin tone}{1902}
\showcaseopenmojiblack{man in motorized wheelchair- medium-light skin tone}{1903}
\showcaseopenmojiblack{man in motorized wheelchair}{1904}
\showcaseopenmojiblack{man in steamy room- dark skin tone}{1905}
\showcaseopenmojiblack{man in steamy room- light skin tone}{1906}
\showcaseopenmojiblack{man in steamy room- medium skin tone}{1907}
\showcaseopenmojiblack{man in steamy room- medium-dark skin tone}{1908}
\showcaseopenmojiblack{man in steamy room- medium-light skin tone}{1909}
\showcaseopenmojiblack{man in steamy room}{1910}
\showcaseopenmojiblack{man in tuxedo- dark skin tone}{1911}
\showcaseopenmojiblack{man in tuxedo- light skin tone}{1912}
\showcaseopenmojiblack{man in tuxedo- medium skin tone}{1913}
\showcaseopenmojiblack{man in tuxedo- medium-dark skin tone}{1914}
\showcaseopenmojiblack{man in tuxedo- medium-light skin tone}{1915}
\showcaseopenmojiblack{man in tuxedo}{1916}
\showcaseopenmojiblack{man judge- dark skin tone}{1917}
\showcaseopenmojiblack{man judge- light skin tone}{1918}
\showcaseopenmojiblack{man judge- medium skin tone}{1919}
\showcaseopenmojiblack{man judge- medium-dark skin tone}{1920}
\showcaseopenmojiblack{man judge- medium-light skin tone}{1921}
\showcaseopenmojiblack{man judge}{1922}
\showcaseopenmojiblack{man juggling- dark skin tone}{1923}
\showcaseopenmojiblack{man juggling- light skin tone}{1924}
\showcaseopenmojiblack{man juggling- medium skin tone}{1925}
\showcaseopenmojiblack{man juggling- medium-dark skin tone}{1926}
\showcaseopenmojiblack{man juggling- medium-light skin tone}{1927}
\showcaseopenmojiblack{man juggling}{1928}
\showcaseopenmojiblack{man kneeling facing right}{1929}
\showcaseopenmojiblack{man kneeling- dark skin tone}{1930}
\showcaseopenmojiblack{man kneeling- light skin tone}{1931}
\showcaseopenmojiblack{man kneeling- medium skin tone}{1932}
\showcaseopenmojiblack{man kneeling- medium-dark skin tone}{1933}
\showcaseopenmojiblack{man kneeling- medium-light skin tone}{1934}
\showcaseopenmojiblack{man kneeling}{1935}
\showcaseopenmojiblack{man lifting weights- dark skin tone}{1936}
\showcaseopenmojiblack{man lifting weights- light skin tone}{1937}
\showcaseopenmojiblack{man lifting weights- medium skin tone}{1938}
\showcaseopenmojiblack{man lifting weights- medium-dark skin tone}{1939}
\showcaseopenmojiblack{man lifting weights- medium-light skin tone}{1940}
\showcaseopenmojiblack{man lifting weights}{1941}
\showcaseopenmojiblack{man mage- dark skin tone}{1942}
\showcaseopenmojiblack{man mage- light skin tone}{1943}
\showcaseopenmojiblack{man mage- medium skin tone}{1944}
\showcaseopenmojiblack{man mage- medium-dark skin tone}{1945}
\showcaseopenmojiblack{man mage- medium-light skin tone}{1946}
\showcaseopenmojiblack{man mage}{1947}
\showcaseopenmojiblack{man mechanic- dark skin tone}{1948}
\showcaseopenmojiblack{man mechanic- light skin tone}{1949}
\showcaseopenmojiblack{man mechanic- medium skin tone}{1950}
\showcaseopenmojiblack{man mechanic- medium-dark skin tone}{1951}
\showcaseopenmojiblack{man mechanic- medium-light skin tone}{1952}
\showcaseopenmojiblack{man mechanic}{1953}
\showcaseopenmojiblack{man mountain biking- dark skin tone}{1954}
\showcaseopenmojiblack{man mountain biking- light skin tone}{1955}
\showcaseopenmojiblack{man mountain biking- medium skin tone}{1956}
\showcaseopenmojiblack{man mountain biking- medium-dark skin tone}{1957}
\showcaseopenmojiblack{man mountain biking- medium-light skin tone}{1958}
\showcaseopenmojiblack{man mountain biking}{1959}
\showcaseopenmojiblack{man office worker- dark skin tone}{1960}
\showcaseopenmojiblack{man office worker- light skin tone}{1961}
\showcaseopenmojiblack{man office worker- medium skin tone}{1962}
\showcaseopenmojiblack{man office worker- medium-dark skin tone}{1963}
\showcaseopenmojiblack{man office worker- medium-light skin tone}{1964}
\showcaseopenmojiblack{man office worker}{1965}
\showcaseopenmojiblack{man pilot- dark skin tone}{1966}
\showcaseopenmojiblack{man pilot- light skin tone}{1967}
\showcaseopenmojiblack{man pilot- medium skin tone}{1968}
\showcaseopenmojiblack{man pilot- medium-dark skin tone}{1969}
\showcaseopenmojiblack{man pilot- medium-light skin tone}{1970}
\showcaseopenmojiblack{man pilot}{1971}
\showcaseopenmojiblack{man playing handball- dark skin tone}{1972}
\showcaseopenmojiblack{man playing handball- light skin tone}{1973}
\showcaseopenmojiblack{man playing handball- medium skin tone}{1974}
\showcaseopenmojiblack{man playing handball- medium-dark skin tone}{1975}
\showcaseopenmojiblack{man playing handball- medium-light skin tone}{1976}
\showcaseopenmojiblack{man playing handball}{1977}
\showcaseopenmojiblack{man playing water polo- dark skin tone}{1978}
\showcaseopenmojiblack{man playing water polo- light skin tone}{1979}
\showcaseopenmojiblack{man playing water polo- medium skin tone}{1980}
\showcaseopenmojiblack{man playing water polo- medium-dark skin tone}{1981}
\showcaseopenmojiblack{man playing water polo- medium-light skin tone}{1982}
\showcaseopenmojiblack{man playing water polo}{1983}
\showcaseopenmojiblack{man police officer- dark skin tone}{1984}
\showcaseopenmojiblack{man police officer- light skin tone}{1985}
\showcaseopenmojiblack{man police officer- medium skin tone}{1986}
\showcaseopenmojiblack{man police officer- medium-dark skin tone}{1987}
\showcaseopenmojiblack{man police officer- medium-light skin tone}{1988}
\showcaseopenmojiblack{man police officer}{1989}
\showcaseopenmojiblack{man pouting- dark skin tone}{1990}
\showcaseopenmojiblack{man pouting- light skin tone}{1991}
\showcaseopenmojiblack{man pouting- medium skin tone}{1992}
\showcaseopenmojiblack{man pouting- medium-dark skin tone}{1993}
\showcaseopenmojiblack{man pouting- medium-light skin tone}{1994}
\showcaseopenmojiblack{man pouting}{1995}
\showcaseopenmojiblack{man raising hand- dark skin tone}{1996}
\showcaseopenmojiblack{man raising hand- light skin tone}{1997}
\showcaseopenmojiblack{man raising hand- medium skin tone}{1998}
\showcaseopenmojiblack{man raising hand- medium-dark skin tone}{1999}
\showcaseopenmojiblack{man raising hand- medium-light skin tone}{2000}
\showcaseopenmojiblack{man raising hand}{2001}
\showcaseopenmojiblack{man rowing boat- dark skin tone}{2002}
\showcaseopenmojiblack{man rowing boat- light skin tone}{2003}
\showcaseopenmojiblack{man rowing boat- medium skin tone}{2004}
\showcaseopenmojiblack{man rowing boat- medium-dark skin tone}{2005}
\showcaseopenmojiblack{man rowing boat- medium-light skin tone}{2006}
\showcaseopenmojiblack{man rowing boat}{2007}
\showcaseopenmojiblack{man running facing right}{2008}
\showcaseopenmojiblack{man running- dark skin tone}{2009}
\showcaseopenmojiblack{man running- light skin tone}{2010}
\showcaseopenmojiblack{man running- medium skin tone}{2011}
\showcaseopenmojiblack{man running- medium-dark skin tone}{2012}
\showcaseopenmojiblack{man running- medium-light skin tone}{2013}
\showcaseopenmojiblack{man running}{2014}
\showcaseopenmojiblack{man s shoe}{2015}
\showcaseopenmojiblack{man scientist- dark skin tone}{2016}
\showcaseopenmojiblack{man scientist- light skin tone}{2017}
\showcaseopenmojiblack{man scientist- medium skin tone}{2018}
\showcaseopenmojiblack{man scientist- medium-dark skin tone}{2019}
\showcaseopenmojiblack{man scientist- medium-light skin tone}{2020}
\showcaseopenmojiblack{man scientist}{2021}
\showcaseopenmojiblack{man shrugging- dark skin tone}{2022}
\showcaseopenmojiblack{man shrugging- light skin tone}{2023}
\showcaseopenmojiblack{man shrugging- medium skin tone}{2024}
\showcaseopenmojiblack{man shrugging- medium-dark skin tone}{2025}
\showcaseopenmojiblack{man shrugging- medium-light skin tone}{2026}
\showcaseopenmojiblack{man shrugging}{2027}
\showcaseopenmojiblack{man singer- dark skin tone}{2028}
\showcaseopenmojiblack{man singer- light skin tone}{2029}
\showcaseopenmojiblack{man singer- medium skin tone}{2030}
\showcaseopenmojiblack{man singer- medium-dark skin tone}{2031}
\showcaseopenmojiblack{man singer- medium-light skin tone}{2032}
\showcaseopenmojiblack{man singer}{2033}
\showcaseopenmojiblack{man sneezing into elbow}{2034}
\showcaseopenmojiblack{man standing- dark skin tone}{2035}
\showcaseopenmojiblack{man standing- light skin tone}{2036}
\showcaseopenmojiblack{man standing- medium skin tone}{2037}
\showcaseopenmojiblack{man standing- medium-dark skin tone}{2038}
\showcaseopenmojiblack{man standing- medium-light skin tone}{2039}
\showcaseopenmojiblack{man standing}{2040}
\showcaseopenmojiblack{man student- dark skin tone}{2041}
\showcaseopenmojiblack{man student- light skin tone}{2042}
\showcaseopenmojiblack{man student- medium skin tone}{2043}
\showcaseopenmojiblack{man student- medium-dark skin tone}{2044}
\showcaseopenmojiblack{man student- medium-light skin tone}{2045}
\showcaseopenmojiblack{man student}{2046}
\showcaseopenmojiblack{man superhero- dark skin tone}{2047}
\showcaseopenmojiblack{man superhero- light skin tone}{2048}
\showcaseopenmojiblack{man superhero- medium skin tone}{2049}
\showcaseopenmojiblack{man superhero- medium-dark skin tone}{2050}
\showcaseopenmojiblack{man superhero- medium-light skin tone}{2051}
\showcaseopenmojiblack{man superhero}{2052}
\showcaseopenmojiblack{man supervillain- dark skin tone}{2053}
\showcaseopenmojiblack{man supervillain- light skin tone}{2054}
\showcaseopenmojiblack{man supervillain- medium skin tone}{2055}
\showcaseopenmojiblack{man supervillain- medium-dark skin tone}{2056}
\showcaseopenmojiblack{man supervillain- medium-light skin tone}{2057}
\showcaseopenmojiblack{man supervillain}{2058}
\showcaseopenmojiblack{man surfing- dark skin tone}{2059}
\showcaseopenmojiblack{man surfing- light skin tone}{2060}
\showcaseopenmojiblack{man surfing- medium skin tone}{2061}
\showcaseopenmojiblack{man surfing- medium-dark skin tone}{2062}
\showcaseopenmojiblack{man surfing- medium-light skin tone}{2063}
\showcaseopenmojiblack{man surfing}{2064}
\showcaseopenmojiblack{man swimming- dark skin tone}{2065}
\showcaseopenmojiblack{man swimming- light skin tone}{2066}
\showcaseopenmojiblack{man swimming- medium skin tone}{2067}
\showcaseopenmojiblack{man swimming- medium-dark skin tone}{2068}
\showcaseopenmojiblack{man swimming- medium-light skin tone}{2069}
\showcaseopenmojiblack{man swimming}{2070}
\showcaseopenmojiblack{man teacher- dark skin tone}{2071}
\showcaseopenmojiblack{man teacher- light skin tone}{2072}
\showcaseopenmojiblack{man teacher- medium skin tone}{2073}
\showcaseopenmojiblack{man teacher- medium-dark skin tone}{2074}
\showcaseopenmojiblack{man teacher- medium-light skin tone}{2075}
\showcaseopenmojiblack{man teacher}{2076}
\showcaseopenmojiblack{man technologist- dark skin tone}{2077}
\showcaseopenmojiblack{man technologist- light skin tone}{2078}
\showcaseopenmojiblack{man technologist- medium skin tone}{2079}
\showcaseopenmojiblack{man technologist- medium-dark skin tone}{2080}
\showcaseopenmojiblack{man technologist- medium-light skin tone}{2081}
\showcaseopenmojiblack{man technologist}{2082}
\showcaseopenmojiblack{man tipping hand- dark skin tone}{2083}
\showcaseopenmojiblack{man tipping hand- light skin tone}{2084}
\showcaseopenmojiblack{man tipping hand- medium skin tone}{2085}
\showcaseopenmojiblack{man tipping hand- medium-dark skin tone}{2086}
\showcaseopenmojiblack{man tipping hand- medium-light skin tone}{2087}
\showcaseopenmojiblack{man tipping hand}{2088}
\showcaseopenmojiblack{man vampire- dark skin tone}{2089}
\showcaseopenmojiblack{man vampire- light skin tone}{2090}
\showcaseopenmojiblack{man vampire- medium skin tone}{2091}
\showcaseopenmojiblack{man vampire- medium-dark skin tone}{2092}
\showcaseopenmojiblack{man vampire- medium-light skin tone}{2093}
\showcaseopenmojiblack{man vampire}{2094}
\showcaseopenmojiblack{man walking facing right}{2095}
\showcaseopenmojiblack{man walking- dark skin tone}{2096}
\showcaseopenmojiblack{man walking- light skin tone}{2097}
\showcaseopenmojiblack{man walking- medium skin tone}{2098}
\showcaseopenmojiblack{man walking- medium-dark skin tone}{2099}
\showcaseopenmojiblack{man walking- medium-light skin tone}{2100}
\showcaseopenmojiblack{man walking}{2101}
\showcaseopenmojiblack{man wearing turban- dark skin tone}{2102}
\showcaseopenmojiblack{man wearing turban- light skin tone}{2103}
\showcaseopenmojiblack{man wearing turban- medium skin tone}{2104}
\showcaseopenmojiblack{man wearing turban- medium-dark skin tone}{2105}
\showcaseopenmojiblack{man wearing turban- medium-light skin tone}{2106}
\showcaseopenmojiblack{man wearing turban}{2107}
\showcaseopenmojiblack{man with medical mask}{2108}
\showcaseopenmojiblack{man with veil- dark skin tone}{2109}
\showcaseopenmojiblack{man with veil- light skin tone}{2110}
\showcaseopenmojiblack{man with veil- medium skin tone}{2111}
\showcaseopenmojiblack{man with veil- medium-dark skin tone}{2112}
\showcaseopenmojiblack{man with veil- medium-light skin tone}{2113}
\showcaseopenmojiblack{man with veil}{2114}
\showcaseopenmojiblack{man with white cane facing right}{2115}
\showcaseopenmojiblack{man with white cane- dark skin tone}{2116}
\showcaseopenmojiblack{man with white cane- light skin tone}{2117}
\showcaseopenmojiblack{man with white cane- medium skin tone}{2118}
\showcaseopenmojiblack{man with white cane- medium-dark skin tone}{2119}
\showcaseopenmojiblack{man with white cane- medium-light skin tone}{2120}
\showcaseopenmojiblack{man with white cane}{2121}
\showcaseopenmojiblack{man zombie}{2122}
\showcaseopenmojiblack{man- bald}{2123}
\showcaseopenmojiblack{man- beard}{2124}
\showcaseopenmojiblack{man- blond hair}{2125}
\showcaseopenmojiblack{man- curly hair}{2126}
\showcaseopenmojiblack{man- dark skin tone, bald}{2127}
\showcaseopenmojiblack{man- dark skin tone, beard}{2128}
\showcaseopenmojiblack{man- dark skin tone, blond hair}{2129}
\showcaseopenmojiblack{man- dark skin tone, curly hair}{2130}
\showcaseopenmojiblack{man- dark skin tone, red hair}{2131}
\showcaseopenmojiblack{man- dark skin tone, white hair}{2132}
\showcaseopenmojiblack{man- dark skin tone}{2133}
\showcaseopenmojiblack{man- light skin tone, bald}{2134}
\showcaseopenmojiblack{man- light skin tone, beard}{2135}
\showcaseopenmojiblack{man- light skin tone, blond hair}{2136}
\showcaseopenmojiblack{man- light skin tone, curly hair}{2137}
\showcaseopenmojiblack{man- light skin tone, red hair}{2138}
\showcaseopenmojiblack{man- light skin tone, white hair}{2139}
\showcaseopenmojiblack{man- light skin tone}{2140}
\showcaseopenmojiblack{man- medium skin tone, bald}{2141}
\showcaseopenmojiblack{man- medium skin tone, beard}{2142}
\showcaseopenmojiblack{man- medium skin tone, blond hair}{2143}
\showcaseopenmojiblack{man- medium skin tone, curly hair}{2144}
\showcaseopenmojiblack{man- medium skin tone, red hair}{2145}
\showcaseopenmojiblack{man- medium skin tone, white hair}{2146}
\showcaseopenmojiblack{man- medium skin tone}{2147}
\showcaseopenmojiblack{man- medium-dark skin tone, bald}{2148}
\showcaseopenmojiblack{man- medium-dark skin tone, beard}{2149}
\showcaseopenmojiblack{man- medium-dark skin tone, blond hair}{2150}
\showcaseopenmojiblack{man- medium-dark skin tone, curly hair}{2151}
\showcaseopenmojiblack{man- medium-dark skin tone, red hair}{2152}
\showcaseopenmojiblack{man- medium-dark skin tone, white hair}{2153}
\showcaseopenmojiblack{man- medium-dark skin tone}{2154}
\showcaseopenmojiblack{man- medium-light skin tone, bald}{2155}
\showcaseopenmojiblack{man- medium-light skin tone, beard}{2156}
\showcaseopenmojiblack{man- medium-light skin tone, blond hair}{2157}
\showcaseopenmojiblack{man- medium-light skin tone, curly hair}{2158}
\showcaseopenmojiblack{man- medium-light skin tone, red hair}{2159}
\showcaseopenmojiblack{man- medium-light skin tone, white hair}{2160}
\showcaseopenmojiblack{man- medium-light skin tone}{2161}
\showcaseopenmojiblack{man- red hair}{2162}
\showcaseopenmojiblack{man- white hair}{2163}
\showcaseopenmojiblack{man}{2164}
\showcaseopenmojiblack{mango}{2165}
\showcaseopenmojiblack{mantelpiece clock}{2166}
\showcaseopenmojiblack{manual wheelchair}{2167}
\showcaseopenmojiblack{map of japan}{2168}
\showcaseopenmojiblack{maple leaf}{2169}
\showcaseopenmojiblack{maracas}{2170}
\showcaseopenmojiblack{mark}{2171}
\showcaseopenmojiblack{markdown}{2172}
\showcaseopenmojiblack{martial arts uniform}{2173}
\showcaseopenmojiblack{mastodon}{2174}
\showcaseopenmojiblack{mate}{2175}
\showcaseopenmojiblack{maultasche}{2176}
\showcaseopenmojiblack{mausoleum at halicarnassus}{2177}
\showcaseopenmojiblack{meat consumption}{2178}
\showcaseopenmojiblack{meat on bone}{2179}
\showcaseopenmojiblack{mechanic- dark skin tone}{2180}
\showcaseopenmojiblack{mechanic- light skin tone}{2181}
\showcaseopenmojiblack{mechanic- medium skin tone}{2182}
\showcaseopenmojiblack{mechanic- medium-dark skin tone}{2183}
\showcaseopenmojiblack{mechanic- medium-light skin tone}{2184}
\showcaseopenmojiblack{mechanic}{2185}
\showcaseopenmojiblack{mechanical arm}{2186}
\showcaseopenmojiblack{mechanical leg}{2187}
\showcaseopenmojiblack{medical gloves}{2188}
\showcaseopenmojiblack{medical symbol}{2189}
\showcaseopenmojiblack{medication}{2190}
\showcaseopenmojiblack{medium skin tone}{2191}
\showcaseopenmojiblack{medium-dark skin tone}{2192}
\showcaseopenmojiblack{medium-light skin tone}{2193}
\showcaseopenmojiblack{megaphone}{2194}
\showcaseopenmojiblack{melilla flag}{2195}
\showcaseopenmojiblack{melon}{2196}
\showcaseopenmojiblack{melting face}{2197}
\showcaseopenmojiblack{memo}{2198}
\showcaseopenmojiblack{men holding hands- dark skin tone, light skin tone}{2199}
\showcaseopenmojiblack{men holding hands- dark skin tone, medium skin tone}{2200}
\showcaseopenmojiblack{men holding hands- dark skin tone, medium-dark skin tone}{2201}
\showcaseopenmojiblack{men holding hands- dark skin tone, medium-light skin tone}{2202}
\showcaseopenmojiblack{men holding hands- dark skin tone}{2203}
\showcaseopenmojiblack{men holding hands- light skin tone, dark skin tone}{2204}
\showcaseopenmojiblack{men holding hands- light skin tone, medium skin tone}{2205}
\showcaseopenmojiblack{men holding hands- light skin tone, medium-dark skin tone}{2206}
\showcaseopenmojiblack{men holding hands- light skin tone, medium-light skin tone}{2207}
\showcaseopenmojiblack{men holding hands- light skin tone}{2208}
\showcaseopenmojiblack{men holding hands- medium skin tone, dark skin tone}{2209}
\showcaseopenmojiblack{men holding hands- medium skin tone, light skin tone}{2210}
\showcaseopenmojiblack{men holding hands- medium skin tone, medium-dark skin tone}{2211}
\showcaseopenmojiblack{men holding hands- medium skin tone, medium-light skin tone}{2212}
\showcaseopenmojiblack{men holding hands- medium skin tone}{2213}
\showcaseopenmojiblack{men holding hands- medium-dark skin tone, dark skin tone}{2214}
\showcaseopenmojiblack{men holding hands- medium-dark skin tone, light skin tone}{2215}
\showcaseopenmojiblack{men holding hands- medium-dark skin tone, medium skin tone}{2216}
\showcaseopenmojiblack{men holding hands- medium-dark skin tone, medium-light skin tone}{2217}
\showcaseopenmojiblack{men holding hands- medium-dark skin tone}{2218}
\showcaseopenmojiblack{men holding hands- medium-light skin tone, dark skin tone}{2219}
\showcaseopenmojiblack{men holding hands- medium-light skin tone, light skin tone}{2220}
\showcaseopenmojiblack{men holding hands- medium-light skin tone, medium skin tone}{2221}
\showcaseopenmojiblack{men holding hands- medium-light skin tone, medium-dark skin tone}{2222}
\showcaseopenmojiblack{men holding hands- medium-light skin tone}{2223}
\showcaseopenmojiblack{men holding hands}{2224}
\showcaseopenmojiblack{men s room}{2225}
\showcaseopenmojiblack{men with bunny ears}{2226}
\showcaseopenmojiblack{men wrestling}{2227}
\showcaseopenmojiblack{mending heart}{2228}
\showcaseopenmojiblack{menorah}{2229}
\showcaseopenmojiblack{mermaid- dark skin tone}{2230}
\showcaseopenmojiblack{mermaid- light skin tone}{2231}
\showcaseopenmojiblack{mermaid- medium skin tone}{2232}
\showcaseopenmojiblack{mermaid- medium-dark skin tone}{2233}
\showcaseopenmojiblack{mermaid- medium-light skin tone}{2234}
\showcaseopenmojiblack{mermaid}{2235}
\showcaseopenmojiblack{merman- dark skin tone}{2236}
\showcaseopenmojiblack{merman- light skin tone}{2237}
\showcaseopenmojiblack{merman- medium skin tone}{2238}
\showcaseopenmojiblack{merman- medium-dark skin tone}{2239}
\showcaseopenmojiblack{merman- medium-light skin tone}{2240}
\showcaseopenmojiblack{merman}{2241}
\showcaseopenmojiblack{merperson- dark skin tone}{2242}
\showcaseopenmojiblack{merperson- light skin tone}{2243}
\showcaseopenmojiblack{merperson- medium skin tone}{2244}
\showcaseopenmojiblack{merperson- medium-dark skin tone}{2245}
\showcaseopenmojiblack{merperson- medium-light skin tone}{2246}
\showcaseopenmojiblack{merperson}{2247}
\showcaseopenmojiblack{metro}{2248}
\showcaseopenmojiblack{microbe}{2249}
\showcaseopenmojiblack{microphone}{2250}
\showcaseopenmojiblack{microscope}{2251}
\showcaseopenmojiblack{middle finger- dark skin tone}{2252}
\showcaseopenmojiblack{middle finger- light skin tone}{2253}
\showcaseopenmojiblack{middle finger- medium skin tone}{2254}
\showcaseopenmojiblack{middle finger- medium-dark skin tone}{2255}
\showcaseopenmojiblack{middle finger- medium-light skin tone}{2256}
\showcaseopenmojiblack{middle finger}{2257}
\showcaseopenmojiblack{military helmet}{2258}
\showcaseopenmojiblack{military medal}{2259}
\showcaseopenmojiblack{milk jug}{2260}
\showcaseopenmojiblack{milky way}{2261}
\showcaseopenmojiblack{minibus}{2262}
\showcaseopenmojiblack{minus}{2263}
\showcaseopenmojiblack{mirror ball}{2264}
\showcaseopenmojiblack{mirror}{2265}
\showcaseopenmojiblack{moai}{2266}
\showcaseopenmojiblack{mobile info}{2267}
\showcaseopenmojiblack{mobile message}{2268}
\showcaseopenmojiblack{mobile phone off}{2269}
\showcaseopenmojiblack{mobile phone with arrow}{2270}
\showcaseopenmojiblack{mobile phone}{2271}
\showcaseopenmojiblack{moka pot}{2272}
\showcaseopenmojiblack{money bag}{2273}
\showcaseopenmojiblack{money with wings}{2274}
\showcaseopenmojiblack{money-mouth face}{2275}
\showcaseopenmojiblack{monkey face}{2276}
\showcaseopenmojiblack{monkey}{2277}
\showcaseopenmojiblack{monorail}{2278}
\showcaseopenmojiblack{moon cake}{2279}
\showcaseopenmojiblack{moon viewing ceremony}{2280}
\showcaseopenmojiblack{moose}{2281}
\showcaseopenmojiblack{more information}{2282}
\showcaseopenmojiblack{mosque}{2283}
\showcaseopenmojiblack{mosquito}{2284}
\showcaseopenmojiblack{motor boat}{2285}
\showcaseopenmojiblack{motor scooter}{2286}
\showcaseopenmojiblack{motor}{2287}
\showcaseopenmojiblack{motorbike helmet}{2288}
\showcaseopenmojiblack{motorcycle}{2289}
\showcaseopenmojiblack{motorized wheelchair}{2290}
\showcaseopenmojiblack{motorway}{2291}
\showcaseopenmojiblack{mount fuji}{2292}
\showcaseopenmojiblack{mountain cableway}{2293}
\showcaseopenmojiblack{mountain railway}{2294}
\showcaseopenmojiblack{mountain}{2295}
\showcaseopenmojiblack{mouse face}{2296}
\showcaseopenmojiblack{mouse trap}{2297}
\showcaseopenmojiblack{mouse}{2298}
\showcaseopenmojiblack{mouth}{2299}
\showcaseopenmojiblack{move}{2300}
\showcaseopenmojiblack{movie camera}{2301}
\showcaseopenmojiblack{mrs}{2302}
\showcaseopenmojiblack{multiply}{2303}
\showcaseopenmojiblack{murcia flag}{2304}
\showcaseopenmojiblack{mushroom}{2305}
\showcaseopenmojiblack{musical keyboard}{2306}
\showcaseopenmojiblack{musical note}{2307}
\showcaseopenmojiblack{musical notes}{2308}
\showcaseopenmojiblack{musical score}{2309}
\showcaseopenmojiblack{musicbrainz}{2310}
\showcaseopenmojiblack{muted speaker}{2311}
\showcaseopenmojiblack{mx claus- dark skin tone}{2312}
\showcaseopenmojiblack{mx claus- light skin tone}{2313}
\showcaseopenmojiblack{mx claus- medium skin tone}{2314}
\showcaseopenmojiblack{mx claus- medium-dark skin tone}{2315}
\showcaseopenmojiblack{mx claus- medium-light skin tone}{2316}
\showcaseopenmojiblack{mx claus}{2317}
\showcaseopenmojiblack{nail and gear flag}{2318}
\showcaseopenmojiblack{nail polish- dark skin tone}{2319}
\showcaseopenmojiblack{nail polish- light skin tone}{2320}
\showcaseopenmojiblack{nail polish- medium skin tone}{2321}
\showcaseopenmojiblack{nail polish- medium-dark skin tone}{2322}
\showcaseopenmojiblack{nail polish- medium-light skin tone}{2323}
\showcaseopenmojiblack{nail polish}{2324}
\showcaseopenmojiblack{name badge}{2325}
\showcaseopenmojiblack{narwhal}{2326}
\showcaseopenmojiblack{national park}{2327}
\showcaseopenmojiblack{nauseated face}{2328}
\showcaseopenmojiblack{navarra chartered community}{2329}
\showcaseopenmojiblack{nazar amulet}{2330}
\showcaseopenmojiblack{necktie}{2331}
\showcaseopenmojiblack{nerd face}{2332}
\showcaseopenmojiblack{nest with eggs}{2333}
\showcaseopenmojiblack{nesting dolls}{2334}
\showcaseopenmojiblack{netscape navigator}{2335}
\showcaseopenmojiblack{neutral face}{2336}
\showcaseopenmojiblack{new button}{2337}
\showcaseopenmojiblack{new moon face}{2338}
\showcaseopenmojiblack{new moon}{2339}
\showcaseopenmojiblack{newspaper}{2340}
\showcaseopenmojiblack{next track button}{2341}
\showcaseopenmojiblack{ng button}{2342}
\showcaseopenmojiblack{night with stars}{2343}
\showcaseopenmojiblack{nine o clock}{2344}
\showcaseopenmojiblack{nine-thirty}{2345}
\showcaseopenmojiblack{ninja- dark skin tone}{2346}
\showcaseopenmojiblack{ninja- light skin tone}{2347}
\showcaseopenmojiblack{ninja- medium skin tone}{2348}
\showcaseopenmojiblack{ninja- medium-dark skin tone}{2349}
\showcaseopenmojiblack{ninja- medium-light skin tone}{2350}
\showcaseopenmojiblack{ninja}{2351}
\showcaseopenmojiblack{no bicycles}{2352}
\showcaseopenmojiblack{no entry}{2353}
\showcaseopenmojiblack{no handshaking}{2354}
\showcaseopenmojiblack{no littering}{2355}
\showcaseopenmojiblack{no mobile phones}{2356}
\showcaseopenmojiblack{no one under eighteen}{2357}
\showcaseopenmojiblack{no pedestrians}{2358}
\showcaseopenmojiblack{no smoking}{2359}
\showcaseopenmojiblack{no stencil}{2360}
\showcaseopenmojiblack{non-potable water}{2361}
\showcaseopenmojiblack{north}{2362}
\showcaseopenmojiblack{nose- dark skin tone}{2363}
\showcaseopenmojiblack{nose- light skin tone}{2364}
\showcaseopenmojiblack{nose- medium skin tone}{2365}
\showcaseopenmojiblack{nose- medium-dark skin tone}{2366}
\showcaseopenmojiblack{nose- medium-light skin tone}{2367}
\showcaseopenmojiblack{nose}{2368}
\showcaseopenmojiblack{notebook with decorative cover}{2369}
\showcaseopenmojiblack{notebook}{2370}
\showcaseopenmojiblack{nuclear power plant ruin}{2371}
\showcaseopenmojiblack{nuclear power plant}{2372}
\showcaseopenmojiblack{nuclear protection}{2373}
\showcaseopenmojiblack{nuclear worker man}{2374}
\showcaseopenmojiblack{nuclear worker woman}{2375}
\showcaseopenmojiblack{nut and bolt}{2376}
\showcaseopenmojiblack{o button (blood type)}{2377}
\showcaseopenmojiblack{octopus}{2378}
\showcaseopenmojiblack{oden}{2379}
\showcaseopenmojiblack{office building}{2380}
\showcaseopenmojiblack{office worker- dark skin tone}{2381}
\showcaseopenmojiblack{office worker- light skin tone}{2382}
\showcaseopenmojiblack{office worker- medium skin tone}{2383}
\showcaseopenmojiblack{office worker- medium-dark skin tone}{2384}
\showcaseopenmojiblack{office worker- medium-light skin tone}{2385}
\showcaseopenmojiblack{office worker}{2386}
\showcaseopenmojiblack{ogre}{2387}
\showcaseopenmojiblack{oil drum}{2388}
\showcaseopenmojiblack{oil spill}{2389}
\showcaseopenmojiblack{ok button}{2390}
\showcaseopenmojiblack{ok hand- dark skin tone}{2391}
\showcaseopenmojiblack{ok hand- light skin tone}{2392}
\showcaseopenmojiblack{ok hand- medium skin tone}{2393}
\showcaseopenmojiblack{ok hand- medium-dark skin tone}{2394}
\showcaseopenmojiblack{ok hand- medium-light skin tone}{2395}
\showcaseopenmojiblack{ok hand}{2396}
\showcaseopenmojiblack{ok stencil}{2397}
\showcaseopenmojiblack{old key}{2398}
\showcaseopenmojiblack{old man- dark skin tone}{2399}
\showcaseopenmojiblack{old man- light skin tone}{2400}
\showcaseopenmojiblack{old man- medium skin tone}{2401}
\showcaseopenmojiblack{old man- medium-dark skin tone}{2402}
\showcaseopenmojiblack{old man- medium-light skin tone}{2403}
\showcaseopenmojiblack{old man}{2404}
\showcaseopenmojiblack{old woman- dark skin tone}{2405}
\showcaseopenmojiblack{old woman- light skin tone}{2406}
\showcaseopenmojiblack{old woman- medium skin tone}{2407}
\showcaseopenmojiblack{old woman- medium-dark skin tone}{2408}
\showcaseopenmojiblack{old woman- medium-light skin tone}{2409}
\showcaseopenmojiblack{old woman}{2410}
\showcaseopenmojiblack{older person- dark skin tone}{2411}
\showcaseopenmojiblack{older person- light skin tone}{2412}
\showcaseopenmojiblack{older person- medium skin tone}{2413}
\showcaseopenmojiblack{older person- medium-dark skin tone}{2414}
\showcaseopenmojiblack{older person- medium-light skin tone}{2415}
\showcaseopenmojiblack{older person}{2416}
\showcaseopenmojiblack{olive}{2417}
\showcaseopenmojiblack{om}{2418}
\showcaseopenmojiblack{on! arrow}{2419}
\showcaseopenmojiblack{oncoming automobile}{2420}
\showcaseopenmojiblack{oncoming bus}{2421}
\showcaseopenmojiblack{oncoming fist- dark skin tone}{2422}
\showcaseopenmojiblack{oncoming fist- light skin tone}{2423}
\showcaseopenmojiblack{oncoming fist- medium skin tone}{2424}
\showcaseopenmojiblack{oncoming fist- medium-dark skin tone}{2425}
\showcaseopenmojiblack{oncoming fist- medium-light skin tone}{2426}
\showcaseopenmojiblack{oncoming fist}{2427}
\showcaseopenmojiblack{oncoming police car}{2428}
\showcaseopenmojiblack{oncoming taxi}{2429}
\showcaseopenmojiblack{one o clock}{2430}
\showcaseopenmojiblack{one-piece swimsuit}{2431}
\showcaseopenmojiblack{one-thirty}{2432}
\showcaseopenmojiblack{onion}{2433}
\showcaseopenmojiblack{open book}{2434}
\showcaseopenmojiblack{open file folder}{2435}
\showcaseopenmojiblack{open hands- dark skin tone}{2436}
\showcaseopenmojiblack{open hands- light skin tone}{2437}
\showcaseopenmojiblack{open hands- medium skin tone}{2438}
\showcaseopenmojiblack{open hands- medium-dark skin tone}{2439}
\showcaseopenmojiblack{open hands- medium-light skin tone}{2440}
\showcaseopenmojiblack{open hands}{2441}
\showcaseopenmojiblack{open mailbox with lowered flag}{2442}
\showcaseopenmojiblack{open mailbox with raised flag}{2443}
\showcaseopenmojiblack{openfoodfact}{2444}
\showcaseopenmojiblack{openstreetmap}{2445}
\showcaseopenmojiblack{opera}{2446}
\showcaseopenmojiblack{ophiuchus}{2447}
\showcaseopenmojiblack{optical disk}{2448}
\showcaseopenmojiblack{orange book}{2449}
\showcaseopenmojiblack{orange circle}{2450}
\showcaseopenmojiblack{orange flag}{2451}
\showcaseopenmojiblack{orange heart}{2452}
\showcaseopenmojiblack{orange hexagon}{2453}
\showcaseopenmojiblack{orange square}{2454}
\showcaseopenmojiblack{orangutan}{2455}
\showcaseopenmojiblack{orca}{2456}
\showcaseopenmojiblack{orthodox cross}{2457}
\showcaseopenmojiblack{otter}{2458}
\showcaseopenmojiblack{outbox tray}{2459}
\showcaseopenmojiblack{outlet}{2460}
\showcaseopenmojiblack{overlapping black squares}{2461}
\showcaseopenmojiblack{overlapping white and black squares}{2462}
\showcaseopenmojiblack{overlapping white squares}{2463}
\showcaseopenmojiblack{overview}{2464}
\showcaseopenmojiblack{owl}{2465}
\showcaseopenmojiblack{ox}{2466}
\showcaseopenmojiblack{oyster}{2467}
\showcaseopenmojiblack{p button}{2468}
\showcaseopenmojiblack{package}{2469}
\showcaseopenmojiblack{page facing up}{2470}
\showcaseopenmojiblack{page move}{2471}
\showcaseopenmojiblack{page with curl}{2472}
\showcaseopenmojiblack{pager}{2473}
\showcaseopenmojiblack{paintbrush}{2474}
\showcaseopenmojiblack{palm down hand- dark skin tone}{2475}
\showcaseopenmojiblack{palm down hand- light skin tone}{2476}
\showcaseopenmojiblack{palm down hand- medium skin tone}{2477}
\showcaseopenmojiblack{palm down hand- medium-dark skin tone}{2478}
\showcaseopenmojiblack{palm down hand- medium-light skin tone}{2479}
\showcaseopenmojiblack{palm down hand}{2480}
\showcaseopenmojiblack{palm tree}{2481}
\showcaseopenmojiblack{palm up hand- dark skin tone}{2482}
\showcaseopenmojiblack{palm up hand- light skin tone}{2483}
\showcaseopenmojiblack{palm up hand- medium skin tone}{2484}
\showcaseopenmojiblack{palm up hand- medium-dark skin tone}{2485}
\showcaseopenmojiblack{palm up hand- medium-light skin tone}{2486}
\showcaseopenmojiblack{palm up hand}{2487}
\showcaseopenmojiblack{palms up together- dark skin tone}{2488}
\showcaseopenmojiblack{palms up together- light skin tone}{2489}
\showcaseopenmojiblack{palms up together- medium skin tone}{2490}
\showcaseopenmojiblack{palms up together- medium-dark skin tone}{2491}
\showcaseopenmojiblack{palms up together- medium-light skin tone}{2492}
\showcaseopenmojiblack{palms up together}{2493}
\showcaseopenmojiblack{pancakes}{2494}
\showcaseopenmojiblack{panda}{2495}
\showcaseopenmojiblack{paperclip}{2496}
\showcaseopenmojiblack{parachute}{2497}
\showcaseopenmojiblack{parking garage}{2498}
\showcaseopenmojiblack{parrot}{2499}
\showcaseopenmojiblack{part alternation mark}{2500}
\showcaseopenmojiblack{party popper}{2501}
\showcaseopenmojiblack{partying face}{2502}
\showcaseopenmojiblack{passenger ship}{2503}
\showcaseopenmojiblack{passport control}{2504}
\showcaseopenmojiblack{patient clipboard}{2505}
\showcaseopenmojiblack{patient file}{2506}
\showcaseopenmojiblack{pause button}{2507}
\showcaseopenmojiblack{paw prints}{2508}
\showcaseopenmojiblack{pea pod}{2509}
\showcaseopenmojiblack{peace symbol}{2510}
\showcaseopenmojiblack{peach}{2511}
\showcaseopenmojiblack{peacock}{2512}
\showcaseopenmojiblack{peanuts}{2513}
\showcaseopenmojiblack{pear}{2514}
\showcaseopenmojiblack{peertube}{2515}
\showcaseopenmojiblack{pen}{2516}
\showcaseopenmojiblack{pencil}{2517}
\showcaseopenmojiblack{penguin}{2518}
\showcaseopenmojiblack{pensive face}{2519}
\showcaseopenmojiblack{people dialogue}{2520}
\showcaseopenmojiblack{people holding hands- dark skin tone, light skin tone}{2521}
\showcaseopenmojiblack{people holding hands- dark skin tone, medium skin tone}{2522}
\showcaseopenmojiblack{people holding hands- dark skin tone, medium-dark skin tone}{2523}
\showcaseopenmojiblack{people holding hands- dark skin tone, medium-light skin tone}{2524}
\showcaseopenmojiblack{people holding hands- dark skin tone}{2525}
\showcaseopenmojiblack{people holding hands- light skin tone, dark skin tone}{2526}
\showcaseopenmojiblack{people holding hands- light skin tone, medium skin tone}{2527}
\showcaseopenmojiblack{people holding hands- light skin tone, medium-dark skin tone}{2528}
\showcaseopenmojiblack{people holding hands- light skin tone, medium-light skin tone}{2529}
\showcaseopenmojiblack{people holding hands- light skin tone}{2530}
\showcaseopenmojiblack{people holding hands- medium skin tone, dark skin tone}{2531}
\showcaseopenmojiblack{people holding hands- medium skin tone, light skin tone}{2532}
\showcaseopenmojiblack{people holding hands- medium skin tone, medium-dark skin tone}{2533}
\showcaseopenmojiblack{people holding hands- medium skin tone, medium-light skin tone}{2534}
\showcaseopenmojiblack{people holding hands- medium skin tone}{2535}
\showcaseopenmojiblack{people holding hands- medium-dark skin tone, dark skin tone}{2536}
\showcaseopenmojiblack{people holding hands- medium-dark skin tone, light skin tone}{2537}
\showcaseopenmojiblack{people holding hands- medium-dark skin tone, medium skin tone}{2538}
\showcaseopenmojiblack{people holding hands- medium-dark skin tone, medium-light skin tone}{2539}
\showcaseopenmojiblack{people holding hands- medium-dark skin tone}{2540}
\showcaseopenmojiblack{people holding hands- medium-light skin tone, dark skin tone}{2541}
\showcaseopenmojiblack{people holding hands- medium-light skin tone, light skin tone}{2542}
\showcaseopenmojiblack{people holding hands- medium-light skin tone, medium skin tone}{2543}
\showcaseopenmojiblack{people holding hands- medium-light skin tone, medium-dark skin tone}{2544}
\showcaseopenmojiblack{people holding hands- medium-light skin tone}{2545}
\showcaseopenmojiblack{people holding hands}{2546}
\showcaseopenmojiblack{people hugging}{2547}
\showcaseopenmojiblack{people with bunny ears}{2548}
\showcaseopenmojiblack{people wrestling}{2549}
\showcaseopenmojiblack{performing arts}{2550}
\showcaseopenmojiblack{persevering face}{2551}
\showcaseopenmojiblack{person biking- dark skin tone}{2552}
\showcaseopenmojiblack{person biking- light skin tone}{2553}
\showcaseopenmojiblack{person biking- medium skin tone}{2554}
\showcaseopenmojiblack{person biking- medium-dark skin tone}{2555}
\showcaseopenmojiblack{person biking- medium-light skin tone}{2556}
\showcaseopenmojiblack{person biking}{2557}
\showcaseopenmojiblack{person bouncing ball- dark skin tone}{2558}
\showcaseopenmojiblack{person bouncing ball- light skin tone}{2559}
\showcaseopenmojiblack{person bouncing ball- medium skin tone}{2560}
\showcaseopenmojiblack{person bouncing ball- medium-dark skin tone}{2561}
\showcaseopenmojiblack{person bouncing ball- medium-light skin tone}{2562}
\showcaseopenmojiblack{person bouncing ball}{2563}
\showcaseopenmojiblack{person bowing- dark skin tone}{2564}
\showcaseopenmojiblack{person bowing- light skin tone}{2565}
\showcaseopenmojiblack{person bowing- medium skin tone}{2566}
\showcaseopenmojiblack{person bowing- medium-dark skin tone}{2567}
\showcaseopenmojiblack{person bowing- medium-light skin tone}{2568}
\showcaseopenmojiblack{person bowing}{2569}
\showcaseopenmojiblack{person cartwheeling- dark skin tone}{2570}
\showcaseopenmojiblack{person cartwheeling- light skin tone}{2571}
\showcaseopenmojiblack{person cartwheeling- medium skin tone}{2572}
\showcaseopenmojiblack{person cartwheeling- medium-dark skin tone}{2573}
\showcaseopenmojiblack{person cartwheeling- medium-light skin tone}{2574}
\showcaseopenmojiblack{person cartwheeling}{2575}
\showcaseopenmojiblack{person climbing- dark skin tone}{2576}
\showcaseopenmojiblack{person climbing- light skin tone}{2577}
\showcaseopenmojiblack{person climbing- medium skin tone}{2578}
\showcaseopenmojiblack{person climbing- medium-dark skin tone}{2579}
\showcaseopenmojiblack{person climbing- medium-light skin tone}{2580}
\showcaseopenmojiblack{person climbing}{2581}
\showcaseopenmojiblack{person facepalming- dark skin tone}{2582}
\showcaseopenmojiblack{person facepalming- light skin tone}{2583}
\showcaseopenmojiblack{person facepalming- medium skin tone}{2584}
\showcaseopenmojiblack{person facepalming- medium-dark skin tone}{2585}
\showcaseopenmojiblack{person facepalming- medium-light skin tone}{2586}
\showcaseopenmojiblack{person facepalming}{2587}
\showcaseopenmojiblack{person feeding baby- dark skin tone}{2588}
\showcaseopenmojiblack{person feeding baby- light skin tone}{2589}
\showcaseopenmojiblack{person feeding baby- medium skin tone}{2590}
\showcaseopenmojiblack{person feeding baby- medium-dark skin tone}{2591}
\showcaseopenmojiblack{person feeding baby- medium-light skin tone}{2592}
\showcaseopenmojiblack{person feeding baby}{2593}
\showcaseopenmojiblack{person fencing}{2594}
\showcaseopenmojiblack{person frowning- dark skin tone}{2595}
\showcaseopenmojiblack{person frowning- light skin tone}{2596}
\showcaseopenmojiblack{person frowning- medium skin tone}{2597}
\showcaseopenmojiblack{person frowning- medium-dark skin tone}{2598}
\showcaseopenmojiblack{person frowning- medium-light skin tone}{2599}
\showcaseopenmojiblack{person frowning}{2600}
\showcaseopenmojiblack{person gesturing no- dark skin tone}{2601}
\showcaseopenmojiblack{person gesturing no- light skin tone}{2602}
\showcaseopenmojiblack{person gesturing no- medium skin tone}{2603}
\showcaseopenmojiblack{person gesturing no- medium-dark skin tone}{2604}
\showcaseopenmojiblack{person gesturing no- medium-light skin tone}{2605}
\showcaseopenmojiblack{person gesturing no}{2606}
\showcaseopenmojiblack{person gesturing ok- dark skin tone}{2607}
\showcaseopenmojiblack{person gesturing ok- light skin tone}{2608}
\showcaseopenmojiblack{person gesturing ok- medium skin tone}{2609}
\showcaseopenmojiblack{person gesturing ok- medium-dark skin tone}{2610}
\showcaseopenmojiblack{person gesturing ok- medium-light skin tone}{2611}
\showcaseopenmojiblack{person gesturing ok}{2612}
\showcaseopenmojiblack{person getting haircut- dark skin tone}{2613}
\showcaseopenmojiblack{person getting haircut- light skin tone}{2614}
\showcaseopenmojiblack{person getting haircut- medium skin tone}{2615}
\showcaseopenmojiblack{person getting haircut- medium-dark skin tone}{2616}
\showcaseopenmojiblack{person getting haircut- medium-light skin tone}{2617}
\showcaseopenmojiblack{person getting haircut}{2618}
\showcaseopenmojiblack{person getting massage- dark skin tone}{2619}
\showcaseopenmojiblack{person getting massage- light skin tone}{2620}
\showcaseopenmojiblack{person getting massage- medium skin tone}{2621}
\showcaseopenmojiblack{person getting massage- medium-dark skin tone}{2622}
\showcaseopenmojiblack{person getting massage- medium-light skin tone}{2623}
\showcaseopenmojiblack{person getting massage}{2624}
\showcaseopenmojiblack{person golfing- dark skin tone}{2625}
\showcaseopenmojiblack{person golfing- light skin tone}{2626}
\showcaseopenmojiblack{person golfing- medium skin tone}{2627}
\showcaseopenmojiblack{person golfing- medium-dark skin tone}{2628}
\showcaseopenmojiblack{person golfing- medium-light skin tone}{2629}
\showcaseopenmojiblack{person golfing}{2630}
\showcaseopenmojiblack{person in bed- dark skin tone}{2631}
\showcaseopenmojiblack{person in bed- light skin tone}{2632}
\showcaseopenmojiblack{person in bed- medium skin tone}{2633}
\showcaseopenmojiblack{person in bed- medium-dark skin tone}{2634}
\showcaseopenmojiblack{person in bed- medium-light skin tone}{2635}
\showcaseopenmojiblack{person in bed}{2636}
\showcaseopenmojiblack{person in lotus position- dark skin tone}{2637}
\showcaseopenmojiblack{person in lotus position- light skin tone}{2638}
\showcaseopenmojiblack{person in lotus position- medium skin tone}{2639}
\showcaseopenmojiblack{person in lotus position- medium-dark skin tone}{2640}
\showcaseopenmojiblack{person in lotus position- medium-light skin tone}{2641}
\showcaseopenmojiblack{person in lotus position}{2642}
\showcaseopenmojiblack{person in manual wheelchair facing right}{2643}
\showcaseopenmojiblack{person in manual wheelchair- dark skin tone}{2644}
\showcaseopenmojiblack{person in manual wheelchair- light skin tone}{2645}
\showcaseopenmojiblack{person in manual wheelchair- medium skin tone}{2646}
\showcaseopenmojiblack{person in manual wheelchair- medium-dark skin tone}{2647}
\showcaseopenmojiblack{person in manual wheelchair- medium-light skin tone}{2648}
\showcaseopenmojiblack{person in manual wheelchair}{2649}
\showcaseopenmojiblack{person in motorized wheelchair facing right}{2650}
\showcaseopenmojiblack{person in motorized wheelchair- dark skin tone}{2651}
\showcaseopenmojiblack{person in motorized wheelchair- light skin tone}{2652}
\showcaseopenmojiblack{person in motorized wheelchair- medium skin tone}{2653}
\showcaseopenmojiblack{person in motorized wheelchair- medium-dark skin tone}{2654}
\showcaseopenmojiblack{person in motorized wheelchair- medium-light skin tone}{2655}
\showcaseopenmojiblack{person in motorized wheelchair}{2656}
\showcaseopenmojiblack{person in steamy room- dark skin tone}{2657}
\showcaseopenmojiblack{person in steamy room- light skin tone}{2658}
\showcaseopenmojiblack{person in steamy room- medium skin tone}{2659}
\showcaseopenmojiblack{person in steamy room- medium-dark skin tone}{2660}
\showcaseopenmojiblack{person in steamy room- medium-light skin tone}{2661}
\showcaseopenmojiblack{person in steamy room}{2662}
\showcaseopenmojiblack{person in suit levitating- dark skin tone}{2663}
\showcaseopenmojiblack{person in suit levitating- light skin tone}{2664}
\showcaseopenmojiblack{person in suit levitating- medium skin tone}{2665}
\showcaseopenmojiblack{person in suit levitating- medium-dark skin tone}{2666}
\showcaseopenmojiblack{person in suit levitating- medium-light skin tone}{2667}
\showcaseopenmojiblack{person in suit levitating}{2668}
\showcaseopenmojiblack{person in tuxedo- dark skin tone}{2669}
\showcaseopenmojiblack{person in tuxedo- light skin tone}{2670}
\showcaseopenmojiblack{person in tuxedo- medium skin tone}{2671}
\showcaseopenmojiblack{person in tuxedo- medium-dark skin tone}{2672}
\showcaseopenmojiblack{person in tuxedo- medium-light skin tone}{2673}
\showcaseopenmojiblack{person in tuxedo}{2674}
\showcaseopenmojiblack{person juggling- dark skin tone}{2675}
\showcaseopenmojiblack{person juggling- light skin tone}{2676}
\showcaseopenmojiblack{person juggling- medium skin tone}{2677}
\showcaseopenmojiblack{person juggling- medium-dark skin tone}{2678}
\showcaseopenmojiblack{person juggling- medium-light skin tone}{2679}
\showcaseopenmojiblack{person juggling}{2680}
\showcaseopenmojiblack{person kneeling facing right}{2681}
\showcaseopenmojiblack{person kneeling- dark skin tone}{2682}
\showcaseopenmojiblack{person kneeling- light skin tone}{2683}
\showcaseopenmojiblack{person kneeling- medium skin tone}{2684}
\showcaseopenmojiblack{person kneeling- medium-dark skin tone}{2685}
\showcaseopenmojiblack{person kneeling- medium-light skin tone}{2686}
\showcaseopenmojiblack{person kneeling}{2687}
\showcaseopenmojiblack{person lifting weights- dark skin tone}{2688}
\showcaseopenmojiblack{person lifting weights- light skin tone}{2689}
\showcaseopenmojiblack{person lifting weights- medium skin tone}{2690}
\showcaseopenmojiblack{person lifting weights- medium-dark skin tone}{2691}
\showcaseopenmojiblack{person lifting weights- medium-light skin tone}{2692}
\showcaseopenmojiblack{person lifting weights}{2693}
\showcaseopenmojiblack{person mountain biking- dark skin tone}{2694}
\showcaseopenmojiblack{person mountain biking- light skin tone}{2695}
\showcaseopenmojiblack{person mountain biking- medium skin tone}{2696}
\showcaseopenmojiblack{person mountain biking- medium-dark skin tone}{2697}
\showcaseopenmojiblack{person mountain biking- medium-light skin tone}{2698}
\showcaseopenmojiblack{person mountain biking}{2699}
\showcaseopenmojiblack{person playing handball- dark skin tone}{2700}
\showcaseopenmojiblack{person playing handball- light skin tone}{2701}
\showcaseopenmojiblack{person playing handball- medium skin tone}{2702}
\showcaseopenmojiblack{person playing handball- medium-dark skin tone}{2703}
\showcaseopenmojiblack{person playing handball- medium-light skin tone}{2704}
\showcaseopenmojiblack{person playing handball}{2705}
\showcaseopenmojiblack{person playing water polo- dark skin tone}{2706}
\showcaseopenmojiblack{person playing water polo- light skin tone}{2707}
\showcaseopenmojiblack{person playing water polo- medium skin tone}{2708}
\showcaseopenmojiblack{person playing water polo- medium-dark skin tone}{2709}
\showcaseopenmojiblack{person playing water polo- medium-light skin tone}{2710}
\showcaseopenmojiblack{person playing water polo}{2711}
\showcaseopenmojiblack{person pouting- dark skin tone}{2712}
\showcaseopenmojiblack{person pouting- light skin tone}{2713}
\showcaseopenmojiblack{person pouting- medium skin tone}{2714}
\showcaseopenmojiblack{person pouting- medium-dark skin tone}{2715}
\showcaseopenmojiblack{person pouting- medium-light skin tone}{2716}
\showcaseopenmojiblack{person pouting}{2717}
\showcaseopenmojiblack{person raising hand- dark skin tone}{2718}
\showcaseopenmojiblack{person raising hand- light skin tone}{2719}
\showcaseopenmojiblack{person raising hand- medium skin tone}{2720}
\showcaseopenmojiblack{person raising hand- medium-dark skin tone}{2721}
\showcaseopenmojiblack{person raising hand- medium-light skin tone}{2722}
\showcaseopenmojiblack{person raising hand}{2723}
\showcaseopenmojiblack{person rowing boat- dark skin tone}{2724}
\showcaseopenmojiblack{person rowing boat- light skin tone}{2725}
\showcaseopenmojiblack{person rowing boat- medium skin tone}{2726}
\showcaseopenmojiblack{person rowing boat- medium-dark skin tone}{2727}
\showcaseopenmojiblack{person rowing boat- medium-light skin tone}{2728}
\showcaseopenmojiblack{person rowing boat}{2729}
\showcaseopenmojiblack{person running facing right}{2730}
\showcaseopenmojiblack{person running- dark skin tone}{2731}
\showcaseopenmojiblack{person running- light skin tone}{2732}
\showcaseopenmojiblack{person running- medium skin tone}{2733}
\showcaseopenmojiblack{person running- medium-dark skin tone}{2734}
\showcaseopenmojiblack{person running- medium-light skin tone}{2735}
\showcaseopenmojiblack{person running}{2736}
\showcaseopenmojiblack{person shrugging- dark skin tone}{2737}
\showcaseopenmojiblack{person shrugging- light skin tone}{2738}
\showcaseopenmojiblack{person shrugging- medium skin tone}{2739}
\showcaseopenmojiblack{person shrugging- medium-dark skin tone}{2740}
\showcaseopenmojiblack{person shrugging- medium-light skin tone}{2741}
\showcaseopenmojiblack{person shrugging}{2742}
\showcaseopenmojiblack{person sneezing into elbow}{2743}
\showcaseopenmojiblack{person standing- dark skin tone}{2744}
\showcaseopenmojiblack{person standing- light skin tone}{2745}
\showcaseopenmojiblack{person standing- medium skin tone}{2746}
\showcaseopenmojiblack{person standing- medium-dark skin tone}{2747}
\showcaseopenmojiblack{person standing- medium-light skin tone}{2748}
\showcaseopenmojiblack{person standing}{2749}
\showcaseopenmojiblack{person surfing- dark skin tone}{2750}
\showcaseopenmojiblack{person surfing- light skin tone}{2751}
\showcaseopenmojiblack{person surfing- medium skin tone}{2752}
\showcaseopenmojiblack{person surfing- medium-dark skin tone}{2753}
\showcaseopenmojiblack{person surfing- medium-light skin tone}{2754}
\showcaseopenmojiblack{person surfing}{2755}
\showcaseopenmojiblack{person swimming- dark skin tone}{2756}
\showcaseopenmojiblack{person swimming- light skin tone}{2757}
\showcaseopenmojiblack{person swimming- medium skin tone}{2758}
\showcaseopenmojiblack{person swimming- medium-dark skin tone}{2759}
\showcaseopenmojiblack{person swimming- medium-light skin tone}{2760}
\showcaseopenmojiblack{person swimming}{2761}
\showcaseopenmojiblack{person taking bath- dark skin tone}{2762}
\showcaseopenmojiblack{person taking bath- light skin tone}{2763}
\showcaseopenmojiblack{person taking bath- medium skin tone}{2764}
\showcaseopenmojiblack{person taking bath- medium-dark skin tone}{2765}
\showcaseopenmojiblack{person taking bath- medium-light skin tone}{2766}
\showcaseopenmojiblack{person taking bath}{2767}
\showcaseopenmojiblack{person tipping hand- dark skin tone}{2768}
\showcaseopenmojiblack{person tipping hand- light skin tone}{2769}
\showcaseopenmojiblack{person tipping hand- medium skin tone}{2770}
\showcaseopenmojiblack{person tipping hand- medium-dark skin tone}{2771}
\showcaseopenmojiblack{person tipping hand- medium-light skin tone}{2772}
\showcaseopenmojiblack{person tipping hand}{2773}
\showcaseopenmojiblack{person walking facing right}{2774}
\showcaseopenmojiblack{person walking- dark skin tone}{2775}
\showcaseopenmojiblack{person walking- light skin tone}{2776}
\showcaseopenmojiblack{person walking- medium skin tone}{2777}
\showcaseopenmojiblack{person walking- medium-dark skin tone}{2778}
\showcaseopenmojiblack{person walking- medium-light skin tone}{2779}
\showcaseopenmojiblack{person walking}{2780}
\showcaseopenmojiblack{person wearing turban- dark skin tone}{2781}
\showcaseopenmojiblack{person wearing turban- light skin tone}{2782}
\showcaseopenmojiblack{person wearing turban- medium skin tone}{2783}
\showcaseopenmojiblack{person wearing turban- medium-dark skin tone}{2784}
\showcaseopenmojiblack{person wearing turban- medium-light skin tone}{2785}
\showcaseopenmojiblack{person wearing turban}{2786}
\showcaseopenmojiblack{person with crown- dark skin tone}{2787}
\showcaseopenmojiblack{person with crown- light skin tone}{2788}
\showcaseopenmojiblack{person with crown- medium skin tone}{2789}
\showcaseopenmojiblack{person with crown- medium-dark skin tone}{2790}
\showcaseopenmojiblack{person with crown- medium-light skin tone}{2791}
\showcaseopenmojiblack{person with crown}{2792}
\showcaseopenmojiblack{person with dog}{2793}
\showcaseopenmojiblack{person with medical mask}{2794}
\showcaseopenmojiblack{person with skullcap- dark skin tone}{2795}
\showcaseopenmojiblack{person with skullcap- light skin tone}{2796}
\showcaseopenmojiblack{person with skullcap- medium skin tone}{2797}
\showcaseopenmojiblack{person with skullcap- medium-dark skin tone}{2798}
\showcaseopenmojiblack{person with skullcap- medium-light skin tone}{2799}
\showcaseopenmojiblack{person with skullcap}{2800}
\showcaseopenmojiblack{person with veil- dark skin tone}{2801}
\showcaseopenmojiblack{person with veil- light skin tone}{2802}
\showcaseopenmojiblack{person with veil- medium skin tone}{2803}
\showcaseopenmojiblack{person with veil- medium-dark skin tone}{2804}
\showcaseopenmojiblack{person with veil- medium-light skin tone}{2805}
\showcaseopenmojiblack{person with veil}{2806}
\showcaseopenmojiblack{person with white cane facing right}{2807}
\showcaseopenmojiblack{person with white cane- dark skin tone}{2808}
\showcaseopenmojiblack{person with white cane- light skin tone}{2809}
\showcaseopenmojiblack{person with white cane- medium skin tone}{2810}
\showcaseopenmojiblack{person with white cane- medium-dark skin tone}{2811}
\showcaseopenmojiblack{person with white cane- medium-light skin tone}{2812}
\showcaseopenmojiblack{person with white cane}{2813}
\showcaseopenmojiblack{person- bald}{2814}
\showcaseopenmojiblack{person- beard}{2815}
\showcaseopenmojiblack{person- blond hair}{2816}
\showcaseopenmojiblack{person- curly hair}{2817}
\showcaseopenmojiblack{person- dark skin tone, bald}{2818}
\showcaseopenmojiblack{person- dark skin tone, beard}{2819}
\showcaseopenmojiblack{person- dark skin tone, blond hair}{2820}
\showcaseopenmojiblack{person- dark skin tone, curly hair}{2821}
\showcaseopenmojiblack{person- dark skin tone, red hair}{2822}
\showcaseopenmojiblack{person- dark skin tone, white hair}{2823}
\showcaseopenmojiblack{person- dark skin tone}{2824}
\showcaseopenmojiblack{person- light skin tone, bald}{2825}
\showcaseopenmojiblack{person- light skin tone, beard}{2826}
\showcaseopenmojiblack{person- light skin tone, blond hair}{2827}
\showcaseopenmojiblack{person- light skin tone, curly hair}{2828}
\showcaseopenmojiblack{person- light skin tone, red hair}{2829}
\showcaseopenmojiblack{person- light skin tone, white hair}{2830}
\showcaseopenmojiblack{person- light skin tone}{2831}
\showcaseopenmojiblack{person- medium skin tone, bald}{2832}
\showcaseopenmojiblack{person- medium skin tone, beard}{2833}
\showcaseopenmojiblack{person- medium skin tone, blond hair}{2834}
\showcaseopenmojiblack{person- medium skin tone, curly hair}{2835}
\showcaseopenmojiblack{person- medium skin tone, red hair}{2836}
\showcaseopenmojiblack{person- medium skin tone, white hair}{2837}
\showcaseopenmojiblack{person- medium skin tone}{2838}
\showcaseopenmojiblack{person- medium-dark skin tone, bald}{2839}
\showcaseopenmojiblack{person- medium-dark skin tone, beard}{2840}
\showcaseopenmojiblack{person- medium-dark skin tone, blond hair}{2841}
\showcaseopenmojiblack{person- medium-dark skin tone, curly hair}{2842}
\showcaseopenmojiblack{person- medium-dark skin tone, red hair}{2843}
\showcaseopenmojiblack{person- medium-dark skin tone, white hair}{2844}
\showcaseopenmojiblack{person- medium-dark skin tone}{2845}
\showcaseopenmojiblack{person- medium-light skin tone, bald}{2846}
\showcaseopenmojiblack{person- medium-light skin tone, beard}{2847}
\showcaseopenmojiblack{person- medium-light skin tone, blond hair}{2848}
\showcaseopenmojiblack{person- medium-light skin tone, curly hair}{2849}
\showcaseopenmojiblack{person- medium-light skin tone, red hair}{2850}
\showcaseopenmojiblack{person- medium-light skin tone, white hair}{2851}
\showcaseopenmojiblack{person- medium-light skin tone}{2852}
\showcaseopenmojiblack{person- red hair}{2853}
\showcaseopenmojiblack{person- white hair}{2854}
\showcaseopenmojiblack{person}{2855}
\showcaseopenmojiblack{petri dish}{2856}
\showcaseopenmojiblack{phoenix}{2857}
\showcaseopenmojiblack{pick}{2858}
\showcaseopenmojiblack{pickup truck}{2859}
\showcaseopenmojiblack{picture}{2860}
\showcaseopenmojiblack{pie}{2861}
\showcaseopenmojiblack{pig face}{2862}
\showcaseopenmojiblack{pig nose}{2863}
\showcaseopenmojiblack{pig}{2864}
\showcaseopenmojiblack{pigeon}{2865}
\showcaseopenmojiblack{pile of poo}{2866}
\showcaseopenmojiblack{pill}{2867}
\showcaseopenmojiblack{pills}{2868}
\showcaseopenmojiblack{pilot- dark skin tone}{2869}
\showcaseopenmojiblack{pilot- light skin tone}{2870}
\showcaseopenmojiblack{pilot- medium skin tone}{2871}
\showcaseopenmojiblack{pilot- medium-dark skin tone}{2872}
\showcaseopenmojiblack{pilot- medium-light skin tone}{2873}
\showcaseopenmojiblack{pilot}{2874}
\showcaseopenmojiblack{pinata}{2875}
\showcaseopenmojiblack{pinched fingers- dark skin tone}{2876}
\showcaseopenmojiblack{pinched fingers- light skin tone}{2877}
\showcaseopenmojiblack{pinched fingers- medium skin tone}{2878}
\showcaseopenmojiblack{pinched fingers- medium-dark skin tone}{2879}
\showcaseopenmojiblack{pinched fingers- medium-light skin tone}{2880}
\showcaseopenmojiblack{pinched fingers}{2881}
\showcaseopenmojiblack{pinching hand- dark skin tone}{2882}
\showcaseopenmojiblack{pinching hand- light skin tone}{2883}
\showcaseopenmojiblack{pinching hand- medium skin tone}{2884}
\showcaseopenmojiblack{pinching hand- medium-dark skin tone}{2885}
\showcaseopenmojiblack{pinching hand- medium-light skin tone}{2886}
\showcaseopenmojiblack{pinching hand}{2887}
\showcaseopenmojiblack{pine decoration}{2888}
\showcaseopenmojiblack{pineapple}{2889}
\showcaseopenmojiblack{ping pong}{2890}
\showcaseopenmojiblack{pink heart}{2891}
\showcaseopenmojiblack{pinterest}{2892}
\showcaseopenmojiblack{pirate flag}{2893}
\showcaseopenmojiblack{pisces}{2894}
\showcaseopenmojiblack{pixelfed}{2895}
\showcaseopenmojiblack{pizza}{2896}
\showcaseopenmojiblack{placard}{2897}
\showcaseopenmojiblack{place of worship}{2898}
\showcaseopenmojiblack{plaster}{2899}
\showcaseopenmojiblack{plastic bottle}{2900}
\showcaseopenmojiblack{play button}{2901}
\showcaseopenmojiblack{play or pause button}{2902}
\showcaseopenmojiblack{playground slide}{2903}
\showcaseopenmojiblack{pleading face}{2904}
\showcaseopenmojiblack{plunger}{2905}
\showcaseopenmojiblack{plus}{2906}
\showcaseopenmojiblack{polar bear}{2907}
\showcaseopenmojiblack{polar explorer man}{2908}
\showcaseopenmojiblack{polar explorer woman}{2909}
\showcaseopenmojiblack{polar explorer}{2910}
\showcaseopenmojiblack{polar research station}{2911}
\showcaseopenmojiblack{police car light}{2912}
\showcaseopenmojiblack{police car}{2913}
\showcaseopenmojiblack{police officer- dark skin tone}{2914}
\showcaseopenmojiblack{police officer- light skin tone}{2915}
\showcaseopenmojiblack{police officer- medium skin tone}{2916}
\showcaseopenmojiblack{police officer- medium-dark skin tone}{2917}
\showcaseopenmojiblack{police officer- medium-light skin tone}{2918}
\showcaseopenmojiblack{police officer}{2919}
\showcaseopenmojiblack{pomegranate}{2920}
\showcaseopenmojiblack{poodle}{2921}
\showcaseopenmojiblack{pool 8 ball}{2922}
\showcaseopenmojiblack{popcorn}{2923}
\showcaseopenmojiblack{poppy}{2924}
\showcaseopenmojiblack{porpoise}{2925}
\showcaseopenmojiblack{post office}{2926}
\showcaseopenmojiblack{postal horn}{2927}
\showcaseopenmojiblack{postbox}{2928}
\showcaseopenmojiblack{pot of food}{2929}
\showcaseopenmojiblack{potable water}{2930}
\showcaseopenmojiblack{potato}{2931}
\showcaseopenmojiblack{potentiometer}{2932}
\showcaseopenmojiblack{potted plant}{2933}
\showcaseopenmojiblack{poultry leg}{2934}
\showcaseopenmojiblack{pound banknote}{2935}
\showcaseopenmojiblack{pouring liquid}{2936}
\showcaseopenmojiblack{pouting cat}{2937}
\showcaseopenmojiblack{power on symbol}{2938}
\showcaseopenmojiblack{power on-off symbol}{2939}
\showcaseopenmojiblack{power sleep symbol}{2940}
\showcaseopenmojiblack{power symbol}{2941}
\showcaseopenmojiblack{prayer beads}{2942}
\showcaseopenmojiblack{pregnant man- dark skin tone}{2943}
\showcaseopenmojiblack{pregnant man- light skin tone}{2944}
\showcaseopenmojiblack{pregnant man- medium skin tone}{2945}
\showcaseopenmojiblack{pregnant man- medium-dark skin tone}{2946}
\showcaseopenmojiblack{pregnant man- medium-light skin tone}{2947}
\showcaseopenmojiblack{pregnant man}{2948}
\showcaseopenmojiblack{pregnant person- dark skin tone}{2949}
\showcaseopenmojiblack{pregnant person- light skin tone}{2950}
\showcaseopenmojiblack{pregnant person- medium skin tone}{2951}
\showcaseopenmojiblack{pregnant person- medium-dark skin tone}{2952}
\showcaseopenmojiblack{pregnant person- medium-light skin tone}{2953}
\showcaseopenmojiblack{pregnant person}{2954}
\showcaseopenmojiblack{pregnant woman- dark skin tone}{2955}
\showcaseopenmojiblack{pregnant woman- light skin tone}{2956}
\showcaseopenmojiblack{pregnant woman- medium skin tone}{2957}
\showcaseopenmojiblack{pregnant woman- medium-dark skin tone}{2958}
\showcaseopenmojiblack{pregnant woman- medium-light skin tone}{2959}
\showcaseopenmojiblack{pregnant woman}{2960}
\showcaseopenmojiblack{pretzel}{2961}
\showcaseopenmojiblack{prince- dark skin tone}{2962}
\showcaseopenmojiblack{prince- light skin tone}{2963}
\showcaseopenmojiblack{prince- medium skin tone}{2964}
\showcaseopenmojiblack{prince- medium-dark skin tone}{2965}
\showcaseopenmojiblack{prince- medium-light skin tone}{2966}
\showcaseopenmojiblack{prince}{2967}
\showcaseopenmojiblack{princess- dark skin tone}{2968}
\showcaseopenmojiblack{princess- light skin tone}{2969}
\showcaseopenmojiblack{princess- medium skin tone}{2970}
\showcaseopenmojiblack{princess- medium-dark skin tone}{2971}
\showcaseopenmojiblack{princess- medium-light skin tone}{2972}
\showcaseopenmojiblack{princess}{2973}
\showcaseopenmojiblack{printer}{2974}
\showcaseopenmojiblack{prohibited}{2975}
\showcaseopenmojiblack{purple circle}{2976}
\showcaseopenmojiblack{purple flag}{2977}
\showcaseopenmojiblack{purple heart}{2978}
\showcaseopenmojiblack{purple hexagon}{2979}
\showcaseopenmojiblack{purple square}{2980}
\showcaseopenmojiblack{purse}{2981}
\showcaseopenmojiblack{pushpin}{2982}
\showcaseopenmojiblack{puzzle piece}{2983}
\showcaseopenmojiblack{qr code}{2984}
\showcaseopenmojiblack{quarantine}{2985}
\showcaseopenmojiblack{quebec flag}{2986}
\showcaseopenmojiblack{rabbit face}{2987}
\showcaseopenmojiblack{rabbit}{2988}
\showcaseopenmojiblack{raccoon}{2989}
\showcaseopenmojiblack{racing car}{2990}
\showcaseopenmojiblack{radio button}{2991}
\showcaseopenmojiblack{radio}{2992}
\showcaseopenmojiblack{radioactive waste}{2993}
\showcaseopenmojiblack{radioactive}{2994}
\showcaseopenmojiblack{railway car}{2995}
\showcaseopenmojiblack{railway track}{2996}
\showcaseopenmojiblack{rainbow flag}{2997}
\showcaseopenmojiblack{rainbow hexagon}{2998}
\showcaseopenmojiblack{rainbow}{2999}
\showcaseopenmojiblack{raised back of hand- dark skin tone}{3000}
\showcaseopenmojiblack{raised back of hand- light skin tone}{3001}
\showcaseopenmojiblack{raised back of hand- medium skin tone}{3002}
\showcaseopenmojiblack{raised back of hand- medium-dark skin tone}{3003}
\showcaseopenmojiblack{raised back of hand- medium-light skin tone}{3004}
\showcaseopenmojiblack{raised back of hand}{3005}
\showcaseopenmojiblack{raised fist- dark skin tone}{3006}
\showcaseopenmojiblack{raised fist- light skin tone}{3007}
\showcaseopenmojiblack{raised fist- medium skin tone}{3008}
\showcaseopenmojiblack{raised fist- medium-dark skin tone}{3009}
\showcaseopenmojiblack{raised fist- medium-light skin tone}{3010}
\showcaseopenmojiblack{raised fist}{3011}
\showcaseopenmojiblack{raised hand- dark skin tone}{3012}
\showcaseopenmojiblack{raised hand- light skin tone}{3013}
\showcaseopenmojiblack{raised hand- medium skin tone}{3014}
\showcaseopenmojiblack{raised hand- medium-dark skin tone}{3015}
\showcaseopenmojiblack{raised hand- medium-light skin tone}{3016}
\showcaseopenmojiblack{raised hand}{3017}
\showcaseopenmojiblack{raising hands- dark skin tone}{3018}
\showcaseopenmojiblack{raising hands- light skin tone}{3019}
\showcaseopenmojiblack{raising hands- medium skin tone}{3020}
\showcaseopenmojiblack{raising hands- medium-dark skin tone}{3021}
\showcaseopenmojiblack{raising hands- medium-light skin tone}{3022}
\showcaseopenmojiblack{raising hands}{3023}
\showcaseopenmojiblack{ram}{3024}
\showcaseopenmojiblack{raspberry pi}{3025}
\showcaseopenmojiblack{rat}{3026}
\showcaseopenmojiblack{razor}{3027}
\showcaseopenmojiblack{receipt}{3028}
\showcaseopenmojiblack{record button}{3029}
\showcaseopenmojiblack{recycling symbol}{3030}
\showcaseopenmojiblack{red and black flag}{3031}
\showcaseopenmojiblack{red apple}{3032}
\showcaseopenmojiblack{red circle}{3033}
\showcaseopenmojiblack{red envelope}{3034}
\showcaseopenmojiblack{red exclamation mark}{3035}
\showcaseopenmojiblack{red eye}{3036}
\showcaseopenmojiblack{red flag}{3037}
\showcaseopenmojiblack{red hair}{3038}
\showcaseopenmojiblack{red heart}{3039}
\showcaseopenmojiblack{red hexagon}{3040}
\showcaseopenmojiblack{red paper lantern}{3041}
\showcaseopenmojiblack{red question mark}{3042}
\showcaseopenmojiblack{red square}{3043}
\showcaseopenmojiblack{red triangle pointed down}{3044}
\showcaseopenmojiblack{red triangle pointed up}{3045}
\showcaseopenmojiblack{reddit}{3046}
\showcaseopenmojiblack{regional indicator a}{3047}
\showcaseopenmojiblack{regional indicator b}{3048}
\showcaseopenmojiblack{regional indicator c}{3049}
\showcaseopenmojiblack{regional indicator d}{3050}
\showcaseopenmojiblack{regional indicator e}{3051}
\showcaseopenmojiblack{regional indicator f}{3052}
\showcaseopenmojiblack{regional indicator g}{3053}
\showcaseopenmojiblack{regional indicator h}{3054}
\showcaseopenmojiblack{regional indicator i}{3055}
\showcaseopenmojiblack{regional indicator j}{3056}
\showcaseopenmojiblack{regional indicator k}{3057}
\showcaseopenmojiblack{regional indicator l}{3058}
\showcaseopenmojiblack{regional indicator m}{3059}
\showcaseopenmojiblack{regional indicator n}{3060}
\showcaseopenmojiblack{regional indicator o}{3061}
\showcaseopenmojiblack{regional indicator p}{3062}
\showcaseopenmojiblack{regional indicator q}{3063}
\showcaseopenmojiblack{regional indicator r}{3064}
\showcaseopenmojiblack{regional indicator s}{3065}
\showcaseopenmojiblack{regional indicator t}{3066}
\showcaseopenmojiblack{regional indicator u}{3067}
\showcaseopenmojiblack{regional indicator v}{3068}
\showcaseopenmojiblack{regional indicator w}{3069}
\showcaseopenmojiblack{regional indicator x}{3070}
\showcaseopenmojiblack{regional indicator y}{3071}
\showcaseopenmojiblack{regional indicator z}{3072}
\showcaseopenmojiblack{registered}{3073}
\showcaseopenmojiblack{relieved face}{3074}
\showcaseopenmojiblack{reminder ribbon}{3075}
\showcaseopenmojiblack{repeat button}{3076}
\showcaseopenmojiblack{repeat single button}{3077}
\showcaseopenmojiblack{rescue worker s helmet}{3078}
\showcaseopenmojiblack{restroom}{3079}
\showcaseopenmojiblack{return}{3080}
\showcaseopenmojiblack{reusable bag}{3081}
\showcaseopenmojiblack{reverse button}{3082}
\showcaseopenmojiblack{revolving hearts}{3083}
\showcaseopenmojiblack{rhinoceros}{3084}
\showcaseopenmojiblack{ribbon}{3085}
\showcaseopenmojiblack{rice ball}{3086}
\showcaseopenmojiblack{rice cracker}{3087}
\showcaseopenmojiblack{right anger bubble}{3088}
\showcaseopenmojiblack{right arrow curving down}{3089}
\showcaseopenmojiblack{right arrow curving left}{3090}
\showcaseopenmojiblack{right arrow curving up}{3091}
\showcaseopenmojiblack{right arrow}{3092}
\showcaseopenmojiblack{right-facing fist- dark skin tone}{3093}
\showcaseopenmojiblack{right-facing fist- light skin tone}{3094}
\showcaseopenmojiblack{right-facing fist- medium skin tone}{3095}
\showcaseopenmojiblack{right-facing fist- medium-dark skin tone}{3096}
\showcaseopenmojiblack{right-facing fist- medium-light skin tone}{3097}
\showcaseopenmojiblack{right-facing fist}{3098}
\showcaseopenmojiblack{rightwards hand- dark skin tone}{3099}
\showcaseopenmojiblack{rightwards hand- light skin tone}{3100}
\showcaseopenmojiblack{rightwards hand- medium skin tone}{3101}
\showcaseopenmojiblack{rightwards hand- medium-dark skin tone}{3102}
\showcaseopenmojiblack{rightwards hand- medium-light skin tone}{3103}
\showcaseopenmojiblack{rightwards hand}{3104}
\showcaseopenmojiblack{rightwards pushing hand- dark skin tone}{3105}
\showcaseopenmojiblack{rightwards pushing hand- light skin tone}{3106}
\showcaseopenmojiblack{rightwards pushing hand- medium skin tone}{3107}
\showcaseopenmojiblack{rightwards pushing hand- medium-dark skin tone}{3108}
\showcaseopenmojiblack{rightwards pushing hand- medium-light skin tone}{3109}
\showcaseopenmojiblack{rightwards pushing hand}{3110}
\showcaseopenmojiblack{ring buoy}{3111}
\showcaseopenmojiblack{ring}{3112}
\showcaseopenmojiblack{ringed planet}{3113}
\showcaseopenmojiblack{roasted coffee bean}{3114}
\showcaseopenmojiblack{roasted sweet potato}{3115}
\showcaseopenmojiblack{robot}{3116}
\showcaseopenmojiblack{rock}{3117}
\showcaseopenmojiblack{rocket}{3118}
\showcaseopenmojiblack{roll of paper}{3119}
\showcaseopenmojiblack{rolled-up newspaper}{3120}
\showcaseopenmojiblack{roller coaster}{3121}
\showcaseopenmojiblack{roller skate}{3122}
\showcaseopenmojiblack{rolling on the floor laughing}{3123}
\showcaseopenmojiblack{rooster}{3124}
\showcaseopenmojiblack{rose}{3125}
\showcaseopenmojiblack{rosette}{3126}
\showcaseopenmojiblack{round pushpin}{3127}
\showcaseopenmojiblack{rounded symbol for cai}{3128}
\showcaseopenmojiblack{rounded symbol for fu}{3129}
\showcaseopenmojiblack{rounded symbol for lu}{3130}
\showcaseopenmojiblack{rounded symbol for shou}{3131}
\showcaseopenmojiblack{rounded symbol for shuangxi}{3132}
\showcaseopenmojiblack{rounded symbol for xi}{3133}
\showcaseopenmojiblack{ruby}{3134}
\showcaseopenmojiblack{rugby football}{3135}
\showcaseopenmojiblack{running shirt}{3136}
\showcaseopenmojiblack{running shoe}{3137}
\showcaseopenmojiblack{sad but relieved face}{3138}
\showcaseopenmojiblack{safari}{3139}
\showcaseopenmojiblack{safety pin}{3140}
\showcaseopenmojiblack{safety vest}{3141}
\showcaseopenmojiblack{safety}{3142}
\showcaseopenmojiblack{sagittarius}{3143}
\showcaseopenmojiblack{sailboat}{3144}
\showcaseopenmojiblack{sake}{3145}
\showcaseopenmojiblack{saline drip}{3146}
\showcaseopenmojiblack{salt}{3147}
\showcaseopenmojiblack{saluting face}{3148}
\showcaseopenmojiblack{sandwich}{3149}
\showcaseopenmojiblack{sanitizer spray}{3150}
\showcaseopenmojiblack{santa claus- dark skin tone}{3151}
\showcaseopenmojiblack{santa claus- light skin tone}{3152}
\showcaseopenmojiblack{santa claus- medium skin tone}{3153}
\showcaseopenmojiblack{santa claus- medium-dark skin tone}{3154}
\showcaseopenmojiblack{santa claus- medium-light skin tone}{3155}
\showcaseopenmojiblack{santa claus}{3156}
\showcaseopenmojiblack{sari}{3157}
\showcaseopenmojiblack{satellite antenna}{3158}
\showcaseopenmojiblack{satellite}{3159}
\showcaseopenmojiblack{sauropod}{3160}
\showcaseopenmojiblack{save}{3161}
\showcaseopenmojiblack{saw}{3162}
\showcaseopenmojiblack{saxophone}{3163}
\showcaseopenmojiblack{scale}{3164}
\showcaseopenmojiblack{scales}{3165}
\showcaseopenmojiblack{scarf}{3166}
\showcaseopenmojiblack{school}{3167}
\showcaseopenmojiblack{schwabisch gmund forum gold und silber}{3168}
\showcaseopenmojiblack{schwabisch gmund funfknopfturm}{3169}
\showcaseopenmojiblack{schwabisch gmund ratshaus}{3170}
\showcaseopenmojiblack{scientist- dark skin tone}{3171}
\showcaseopenmojiblack{scientist- light skin tone}{3172}
\showcaseopenmojiblack{scientist- medium skin tone}{3173}
\showcaseopenmojiblack{scientist- medium-dark skin tone}{3174}
\showcaseopenmojiblack{scientist- medium-light skin tone}{3175}
\showcaseopenmojiblack{scientist}{3176}
\showcaseopenmojiblack{scissors}{3177}
\showcaseopenmojiblack{scorpio}{3178}
\showcaseopenmojiblack{scorpion}{3179}
\showcaseopenmojiblack{screwdriver}{3180}
\showcaseopenmojiblack{scroll horizontal}{3181}
\showcaseopenmojiblack{scroll}{3182}
\showcaseopenmojiblack{sea level rise}{3183}
\showcaseopenmojiblack{seal}{3184}
\showcaseopenmojiblack{seat}{3185}
\showcaseopenmojiblack{see-no-evil monkey}{3186}
\showcaseopenmojiblack{seedling}{3187}
\showcaseopenmojiblack{selfie- dark skin tone}{3188}
\showcaseopenmojiblack{selfie- light skin tone}{3189}
\showcaseopenmojiblack{selfie- medium skin tone}{3190}
\showcaseopenmojiblack{selfie- medium-dark skin tone}{3191}
\showcaseopenmojiblack{selfie- medium-light skin tone}{3192}
\showcaseopenmojiblack{selfie}{3193}
\showcaseopenmojiblack{service dog}{3194}
\showcaseopenmojiblack{service mark}{3195}
\showcaseopenmojiblack{seven o clock}{3196}
\showcaseopenmojiblack{seven-thirty}{3197}
\showcaseopenmojiblack{sewing needle}{3198}
\showcaseopenmojiblack{shaking face}{3199}
\showcaseopenmojiblack{shallow pan of food}{3200}
\showcaseopenmojiblack{shamrock}{3201}
\showcaseopenmojiblack{share}{3202}
\showcaseopenmojiblack{shark}{3203}
\showcaseopenmojiblack{shaved ice}{3204}
\showcaseopenmojiblack{sheaf of rice}{3205}
\showcaseopenmojiblack{shelter}{3206}
\showcaseopenmojiblack{shield}{3207}
\showcaseopenmojiblack{shinto shrine}{3208}
\showcaseopenmojiblack{ship}{3209}
\showcaseopenmojiblack{shooting star}{3210}
\showcaseopenmojiblack{shopping bags}{3211}
\showcaseopenmojiblack{shopping cart}{3212}
\showcaseopenmojiblack{shortcake}{3213}
\showcaseopenmojiblack{shorts}{3214}
\showcaseopenmojiblack{shower}{3215}
\showcaseopenmojiblack{shrimp}{3216}
\showcaseopenmojiblack{shuffle tracks button}{3217}
\showcaseopenmojiblack{shushing face}{3218}
\showcaseopenmojiblack{sign of the horns- dark skin tone}{3219}
\showcaseopenmojiblack{sign of the horns- light skin tone}{3220}
\showcaseopenmojiblack{sign of the horns- medium skin tone}{3221}
\showcaseopenmojiblack{sign of the horns- medium-dark skin tone}{3222}
\showcaseopenmojiblack{sign of the horns- medium-light skin tone}{3223}
\showcaseopenmojiblack{sign of the horns}{3224}
\showcaseopenmojiblack{signal}{3225}
\showcaseopenmojiblack{signpost}{3226}
\showcaseopenmojiblack{simple}{3227}
\showcaseopenmojiblack{singer- dark skin tone}{3228}
\showcaseopenmojiblack{singer- light skin tone}{3229}
\showcaseopenmojiblack{singer- medium skin tone}{3230}
\showcaseopenmojiblack{singer- medium-dark skin tone}{3231}
\showcaseopenmojiblack{singer- medium-light skin tone}{3232}
\showcaseopenmojiblack{singer}{3233}
\showcaseopenmojiblack{six o clock}{3234}
\showcaseopenmojiblack{six-thirty}{3235}
\showcaseopenmojiblack{skateboard}{3236}
\showcaseopenmojiblack{skier}{3237}
\showcaseopenmojiblack{skis}{3238}
\showcaseopenmojiblack{skull and crossbones}{3239}
\showcaseopenmojiblack{skull}{3240}
\showcaseopenmojiblack{skunk}{3241}
\showcaseopenmojiblack{sled}{3242}
\showcaseopenmojiblack{sleeping face}{3243}
\showcaseopenmojiblack{sleepy face}{3244}
\showcaseopenmojiblack{slightly frowning face}{3245}
\showcaseopenmojiblack{slightly smiling face}{3246}
\showcaseopenmojiblack{slot machine}{3247}
\showcaseopenmojiblack{sloth}{3248}
\showcaseopenmojiblack{small airplane}{3249}
\showcaseopenmojiblack{small blue diamond}{3250}
\showcaseopenmojiblack{small orange diamond}{3251}
\showcaseopenmojiblack{smartwatch}{3252}
\showcaseopenmojiblack{smiling cat with heart-eyes}{3253}
\showcaseopenmojiblack{smiling face with halo}{3254}
\showcaseopenmojiblack{smiling face with heart-eyes}{3255}
\showcaseopenmojiblack{smiling face with hearts}{3256}
\showcaseopenmojiblack{smiling face with horns}{3257}
\showcaseopenmojiblack{smiling face with open hands}{3258}
\showcaseopenmojiblack{smiling face with smiling eyes}{3259}
\showcaseopenmojiblack{smiling face with sunglasses}{3260}
\showcaseopenmojiblack{smiling face with tear}{3261}
\showcaseopenmojiblack{smiling face}{3262}
\showcaseopenmojiblack{smirking face}{3263}
\showcaseopenmojiblack{snail}{3264}
\showcaseopenmojiblack{snake}{3265}
\showcaseopenmojiblack{sneezing face}{3266}
\showcaseopenmojiblack{snow-capped mountain}{3267}
\showcaseopenmojiblack{snowboarder- dark skin tone}{3268}
\showcaseopenmojiblack{snowboarder- light skin tone}{3269}
\showcaseopenmojiblack{snowboarder- medium skin tone}{3270}
\showcaseopenmojiblack{snowboarder- medium-dark skin tone}{3271}
\showcaseopenmojiblack{snowboarder- medium-light skin tone}{3272}
\showcaseopenmojiblack{snowboarder}{3273}
\showcaseopenmojiblack{snowflake}{3274}
\showcaseopenmojiblack{snowman without snow}{3275}
\showcaseopenmojiblack{snowman}{3276}
\showcaseopenmojiblack{soap}{3277}
\showcaseopenmojiblack{soccer ball}{3278}
\showcaseopenmojiblack{social distancing}{3279}
\showcaseopenmojiblack{socks}{3280}
\showcaseopenmojiblack{soft ice cream}{3281}
\showcaseopenmojiblack{softball}{3282}
\showcaseopenmojiblack{solar cell}{3283}
\showcaseopenmojiblack{solar energy}{3284}
\showcaseopenmojiblack{soon arrow}{3285}
\showcaseopenmojiblack{sort}{3286}
\showcaseopenmojiblack{sos button}{3287}
\showcaseopenmojiblack{sos stencil}{3288}
\showcaseopenmojiblack{sound recording copyright}{3289}
\showcaseopenmojiblack{space shuttle}{3290}
\showcaseopenmojiblack{spade suit}{3291}
\showcaseopenmojiblack{spade}{3292}
\showcaseopenmojiblack{spaghetti}{3293}
\showcaseopenmojiblack{sparkle}{3294}
\showcaseopenmojiblack{sparkler}{3295}
\showcaseopenmojiblack{sparkles}{3296}
\showcaseopenmojiblack{sparkling heart}{3297}
\showcaseopenmojiblack{spatzlepresse}{3298}
\showcaseopenmojiblack{speak-no-evil monkey}{3299}
\showcaseopenmojiblack{speaker high volume}{3300}
\showcaseopenmojiblack{speaker low volume}{3301}
\showcaseopenmojiblack{speaker medium volume}{3302}
\showcaseopenmojiblack{speaking head}{3303}
\showcaseopenmojiblack{speech balloon}{3304}
\showcaseopenmojiblack{speedboat}{3305}
\showcaseopenmojiblack{spider web}{3306}
\showcaseopenmojiblack{spider}{3307}
\showcaseopenmojiblack{spiral calendar}{3308}
\showcaseopenmojiblack{spiral notepad}{3309}
\showcaseopenmojiblack{spiral shell}{3310}
\showcaseopenmojiblack{sponge}{3311}
\showcaseopenmojiblack{spoon}{3312}
\showcaseopenmojiblack{sport utility vehicle}{3313}
\showcaseopenmojiblack{sports medal}{3314}
\showcaseopenmojiblack{spouting whale}{3315}
\showcaseopenmojiblack{spouting-orca}{3316}
\showcaseopenmojiblack{square with left half black}{3317}
\showcaseopenmojiblack{square with lower right diagonal black}{3318}
\showcaseopenmojiblack{square with right half black}{3319}
\showcaseopenmojiblack{square with upper left diagonal black}{3320}
\showcaseopenmojiblack{squid}{3321}
\showcaseopenmojiblack{squinting face with tongue}{3322}
\showcaseopenmojiblack{stadium}{3323}
\showcaseopenmojiblack{stairway}{3324}
\showcaseopenmojiblack{star and crescent}{3325}
\showcaseopenmojiblack{star of david}{3326}
\showcaseopenmojiblack{star with left half black}{3327}
\showcaseopenmojiblack{star with right half black}{3328}
\showcaseopenmojiblack{star-struck}{3329}
\showcaseopenmojiblack{star}{3330}
\showcaseopenmojiblack{station}{3331}
\showcaseopenmojiblack{statue of liberty}{3332}
\showcaseopenmojiblack{statue of zeus at olympia}{3333}
\showcaseopenmojiblack{steaming bowl}{3334}
\showcaseopenmojiblack{stethoscope}{3335}
\showcaseopenmojiblack{stick figure leaning left}{3336}
\showcaseopenmojiblack{stick figure leaning right}{3337}
\showcaseopenmojiblack{stick figure with arms raised}{3338}
\showcaseopenmojiblack{stick figure with dress and arms raised}{3339}
\showcaseopenmojiblack{stick figure with dress leaning left}{3340}
\showcaseopenmojiblack{stick figure with dress leaning right}{3341}
\showcaseopenmojiblack{stick figure with dress}{3342}
\showcaseopenmojiblack{stick figure}{3343}
\showcaseopenmojiblack{stomach}{3344}
\showcaseopenmojiblack{stop button}{3345}
\showcaseopenmojiblack{stop sign}{3346}
\showcaseopenmojiblack{stopwatch}{3347}
\showcaseopenmojiblack{straight ruler}{3348}
\showcaseopenmojiblack{strawberry}{3349}
\showcaseopenmojiblack{stretcher}{3350}
\showcaseopenmojiblack{structural fire}{3351}
\showcaseopenmojiblack{student- dark skin tone}{3352}
\showcaseopenmojiblack{student- light skin tone}{3353}
\showcaseopenmojiblack{student- medium skin tone}{3354}
\showcaseopenmojiblack{student- medium-dark skin tone}{3355}
\showcaseopenmojiblack{student- medium-light skin tone}{3356}
\showcaseopenmojiblack{student}{3357}
\showcaseopenmojiblack{studio microphone}{3358}
\showcaseopenmojiblack{stuffed flatbread}{3359}
\showcaseopenmojiblack{stuttgart fernsehturm}{3360}
\showcaseopenmojiblack{sun behind cloud}{3361}
\showcaseopenmojiblack{sun behind large cloud}{3362}
\showcaseopenmojiblack{sun behind rain cloud}{3363}
\showcaseopenmojiblack{sun behind small cloud}{3364}
\showcaseopenmojiblack{sun with face}{3365}
\showcaseopenmojiblack{sun}{3366}
\showcaseopenmojiblack{sunflower}{3367}
\showcaseopenmojiblack{sunglasses}{3368}
\showcaseopenmojiblack{sunrise over mountains}{3369}
\showcaseopenmojiblack{sunrise}{3370}
\showcaseopenmojiblack{sunset}{3371}
\showcaseopenmojiblack{superhero- dark skin tone}{3372}
\showcaseopenmojiblack{superhero- light skin tone}{3373}
\showcaseopenmojiblack{superhero- medium skin tone}{3374}
\showcaseopenmojiblack{superhero- medium-dark skin tone}{3375}
\showcaseopenmojiblack{superhero- medium-light skin tone}{3376}
\showcaseopenmojiblack{superhero}{3377}
\showcaseopenmojiblack{supervillain- dark skin tone}{3378}
\showcaseopenmojiblack{supervillain- light skin tone}{3379}
\showcaseopenmojiblack{supervillain- medium skin tone}{3380}
\showcaseopenmojiblack{supervillain- medium-dark skin tone}{3381}
\showcaseopenmojiblack{supervillain- medium-light skin tone}{3382}
\showcaseopenmojiblack{supervillain}{3383}
\showcaseopenmojiblack{surveillance}{3384}
\showcaseopenmojiblack{sushi}{3385}
\showcaseopenmojiblack{suspension railway}{3386}
\showcaseopenmojiblack{svg}{3387}
\showcaseopenmojiblack{swab pliers}{3388}
\showcaseopenmojiblack{swan}{3389}
\showcaseopenmojiblack{sweat droplets}{3390}
\showcaseopenmojiblack{swipe down}{3391}
\showcaseopenmojiblack{swipe left}{3392}
\showcaseopenmojiblack{swipe right}{3393}
\showcaseopenmojiblack{swipe up}{3394}
\showcaseopenmojiblack{swipe}{3395}
\showcaseopenmojiblack{switch}{3396}
\showcaseopenmojiblack{synagogue}{3397}
\showcaseopenmojiblack{syringe}{3398}
\showcaseopenmojiblack{t-rex}{3399}
\showcaseopenmojiblack{t-shirt}{3400}
\showcaseopenmojiblack{tablet}{3401}
\showcaseopenmojiblack{taco}{3402}
\showcaseopenmojiblack{takeout box}{3403}
\showcaseopenmojiblack{tamale}{3404}
\showcaseopenmojiblack{tanabata tree}{3405}
\showcaseopenmojiblack{tangerine}{3406}
\showcaseopenmojiblack{tap}{3407}
\showcaseopenmojiblack{tardis}{3408}
\showcaseopenmojiblack{taurus}{3409}
\showcaseopenmojiblack{taxi}{3410}
\showcaseopenmojiblack{teacher- dark skin tone}{3411}
\showcaseopenmojiblack{teacher- light skin tone}{3412}
\showcaseopenmojiblack{teacher- medium skin tone}{3413}
\showcaseopenmojiblack{teacher- medium-dark skin tone}{3414}
\showcaseopenmojiblack{teacher- medium-light skin tone}{3415}
\showcaseopenmojiblack{teacher}{3416}
\showcaseopenmojiblack{teacup without handle}{3417}
\showcaseopenmojiblack{teapot}{3418}
\showcaseopenmojiblack{tear-off calendar}{3419}
\showcaseopenmojiblack{technologist- dark skin tone}{3420}
\showcaseopenmojiblack{technologist- light skin tone}{3421}
\showcaseopenmojiblack{technologist- medium skin tone}{3422}
\showcaseopenmojiblack{technologist- medium-dark skin tone}{3423}
\showcaseopenmojiblack{technologist- medium-light skin tone}{3424}
\showcaseopenmojiblack{technologist}{3425}
\showcaseopenmojiblack{teddy bear}{3426}
\showcaseopenmojiblack{telephone receiver}{3427}
\showcaseopenmojiblack{telephone}{3428}
\showcaseopenmojiblack{telescope}{3429}
\showcaseopenmojiblack{television}{3430}
\showcaseopenmojiblack{temperature taking}{3431}
\showcaseopenmojiblack{temple of artemis at ephesus}{3432}
\showcaseopenmojiblack{ten o clock}{3433}
\showcaseopenmojiblack{ten-thirty}{3434}
\showcaseopenmojiblack{tennis}{3435}
\showcaseopenmojiblack{tent}{3436}
\showcaseopenmojiblack{test tube}{3437}
\showcaseopenmojiblack{texas flag}{3438}
\showcaseopenmojiblack{thermometer}{3439}
\showcaseopenmojiblack{thinking face}{3440}
\showcaseopenmojiblack{thong sandal}{3441}
\showcaseopenmojiblack{thought balloon}{3442}
\showcaseopenmojiblack{thread}{3443}
\showcaseopenmojiblack{three finger operation}{3444}
\showcaseopenmojiblack{three o clock}{3445}
\showcaseopenmojiblack{three-thirty}{3446}
\showcaseopenmojiblack{thumbs down- dark skin tone}{3447}
\showcaseopenmojiblack{thumbs down- light skin tone}{3448}
\showcaseopenmojiblack{thumbs down- medium skin tone}{3449}
\showcaseopenmojiblack{thumbs down- medium-dark skin tone}{3450}
\showcaseopenmojiblack{thumbs down- medium-light skin tone}{3451}
\showcaseopenmojiblack{thumbs down}{3452}
\showcaseopenmojiblack{thumbs up- dark skin tone}{3453}
\showcaseopenmojiblack{thumbs up- light skin tone}{3454}
\showcaseopenmojiblack{thumbs up- medium skin tone}{3455}
\showcaseopenmojiblack{thumbs up- medium-dark skin tone}{3456}
\showcaseopenmojiblack{thumbs up- medium-light skin tone}{3457}
\showcaseopenmojiblack{thumbs up}{3458}
\showcaseopenmojiblack{ticket}{3459}
\showcaseopenmojiblack{tiger face}{3460}
\showcaseopenmojiblack{tiger}{3461}
\showcaseopenmojiblack{timer clock}{3462}
\showcaseopenmojiblack{timer}{3463}
\showcaseopenmojiblack{tired face}{3464}
\showcaseopenmojiblack{toggle button state b}{3465}
\showcaseopenmojiblack{toggle button}{3466}
\showcaseopenmojiblack{toilet}{3467}
\showcaseopenmojiblack{tokyo tower}{3468}
\showcaseopenmojiblack{tomato}{3469}
\showcaseopenmojiblack{tongue}{3470}
\showcaseopenmojiblack{toolbox}{3471}
\showcaseopenmojiblack{tooth}{3472}
\showcaseopenmojiblack{toothbrush}{3473}
\showcaseopenmojiblack{top arrow}{3474}
\showcaseopenmojiblack{top hat}{3475}
\showcaseopenmojiblack{tornado}{3476}
\showcaseopenmojiblack{town}{3477}
\showcaseopenmojiblack{trackball}{3478}
\showcaseopenmojiblack{tractor}{3479}
\showcaseopenmojiblack{trade mark}{3480}
\showcaseopenmojiblack{train}{3481}
\showcaseopenmojiblack{tram car}{3482}
\showcaseopenmojiblack{tram}{3483}
\showcaseopenmojiblack{transgender flag}{3484}
\showcaseopenmojiblack{transgender symbol}{3485}
\showcaseopenmojiblack{transmission}{3486}
\showcaseopenmojiblack{triangular flag}{3487}
\showcaseopenmojiblack{triangular ruler}{3488}
\showcaseopenmojiblack{trident emblem}{3489}
\showcaseopenmojiblack{troll}{3490}
\showcaseopenmojiblack{trolleybus}{3491}
\showcaseopenmojiblack{trophy}{3492}
\showcaseopenmojiblack{tropical drink}{3493}
\showcaseopenmojiblack{tropical fish}{3494}
\showcaseopenmojiblack{trowel}{3495}
\showcaseopenmojiblack{true south (antarctica) flag}{3496}
\showcaseopenmojiblack{trump}{3497}
\showcaseopenmojiblack{trumpet}{3498}
\showcaseopenmojiblack{tsunami}{3499}
\showcaseopenmojiblack{tulip}{3500}
\showcaseopenmojiblack{tumbler glass}{3501}
\showcaseopenmojiblack{turkey}{3502}
\showcaseopenmojiblack{turtle}{3503}
\showcaseopenmojiblack{twelve o clock}{3504}
\showcaseopenmojiblack{twelve-thirty}{3505}
\showcaseopenmojiblack{twitter}{3506}
\showcaseopenmojiblack{two hearts}{3507}
\showcaseopenmojiblack{two o clock}{3508}
\showcaseopenmojiblack{two-hump camel}{3509}
\showcaseopenmojiblack{two-thirty}{3510}
\showcaseopenmojiblack{typescript}{3511}
\showcaseopenmojiblack{ubuntu}{3512}
\showcaseopenmojiblack{umbrella on ground}{3513}
\showcaseopenmojiblack{umbrella with rain drops}{3514}
\showcaseopenmojiblack{umbrella}{3515}
\showcaseopenmojiblack{unamused face}{3516}
\showcaseopenmojiblack{unicorn}{3517}
\showcaseopenmojiblack{united federation of planets flag (star trek)}{3518}
\showcaseopenmojiblack{unlocked}{3519}
\showcaseopenmojiblack{up arrow}{3520}
\showcaseopenmojiblack{up down black arrow}{3521}
\showcaseopenmojiblack{up! button}{3522}
\showcaseopenmojiblack{up-down arrow}{3523}
\showcaseopenmojiblack{up-left arrow}{3524}
\showcaseopenmojiblack{up-pointing triangle with left half black}{3525}
\showcaseopenmojiblack{up-pointing triangle with right half black}{3526}
\showcaseopenmojiblack{up-right arrow}{3527}
\showcaseopenmojiblack{upload}{3528}
\showcaseopenmojiblack{upside-down face}{3529}
\showcaseopenmojiblack{upwards button}{3530}
\showcaseopenmojiblack{valencian community flag}{3531}
\showcaseopenmojiblack{vampire- dark skin tone}{3532}
\showcaseopenmojiblack{vampire- light skin tone}{3533}
\showcaseopenmojiblack{vampire- medium skin tone}{3534}
\showcaseopenmojiblack{vampire- medium-dark skin tone}{3535}
\showcaseopenmojiblack{vampire- medium-light skin tone}{3536}
\showcaseopenmojiblack{vampire}{3537}
\showcaseopenmojiblack{vertical traffic light}{3538}
\showcaseopenmojiblack{vibration mode}{3539}
\showcaseopenmojiblack{victory hand- dark skin tone}{3540}
\showcaseopenmojiblack{victory hand- light skin tone}{3541}
\showcaseopenmojiblack{victory hand- medium skin tone}{3542}
\showcaseopenmojiblack{victory hand- medium-dark skin tone}{3543}
\showcaseopenmojiblack{victory hand- medium-light skin tone}{3544}
\showcaseopenmojiblack{victory hand}{3545}
\showcaseopenmojiblack{video camera}{3546}
\showcaseopenmojiblack{video game}{3547}
\showcaseopenmojiblack{videocassette}{3548}
\showcaseopenmojiblack{viennese coffee house}{3549}
\showcaseopenmojiblack{violin}{3550}
\showcaseopenmojiblack{virgo}{3551}
\showcaseopenmojiblack{virtual reality}{3552}
\showcaseopenmojiblack{volcano ashes}{3553}
\showcaseopenmojiblack{volcano eruption}{3554}
\showcaseopenmojiblack{volcano}{3555}
\showcaseopenmojiblack{volleyball}{3556}
\showcaseopenmojiblack{vs button}{3557}
\showcaseopenmojiblack{vulcan salute- dark skin tone}{3558}
\showcaseopenmojiblack{vulcan salute- light skin tone}{3559}
\showcaseopenmojiblack{vulcan salute- medium skin tone}{3560}
\showcaseopenmojiblack{vulcan salute- medium-dark skin tone}{3561}
\showcaseopenmojiblack{vulcan salute- medium-light skin tone}{3562}
\showcaseopenmojiblack{vulcan salute}{3563}
\showcaseopenmojiblack{waffle}{3564}
\showcaseopenmojiblack{waning crescent moon}{3565}
\showcaseopenmojiblack{waning gibbous moon}{3566}
\showcaseopenmojiblack{warning fire}{3567}
\showcaseopenmojiblack{warning strip right}{3568}
\showcaseopenmojiblack{warning strip}{3569}
\showcaseopenmojiblack{warning tsunami}{3570}
\showcaseopenmojiblack{warning volcano}{3571}
\showcaseopenmojiblack{warning}{3572}
\showcaseopenmojiblack{wash hands}{3573}
\showcaseopenmojiblack{washing machine}{3574}
\showcaseopenmojiblack{washington d}{3575}
\showcaseopenmojiblack{wastebasket}{3576}
\showcaseopenmojiblack{watch}{3577}
\showcaseopenmojiblack{water buffalo}{3578}
\showcaseopenmojiblack{water closet}{3579}
\showcaseopenmojiblack{water pistol}{3580}
\showcaseopenmojiblack{water wave}{3581}
\showcaseopenmojiblack{watermelon}{3582}
\showcaseopenmojiblack{waving hand- dark skin tone}{3583}
\showcaseopenmojiblack{waving hand- light skin tone}{3584}
\showcaseopenmojiblack{waving hand- medium skin tone}{3585}
\showcaseopenmojiblack{waving hand- medium-dark skin tone}{3586}
\showcaseopenmojiblack{waving hand- medium-light skin tone}{3587}
\showcaseopenmojiblack{waving hand}{3588}
\showcaseopenmojiblack{wavy dash}{3589}
\showcaseopenmojiblack{waxing crescent moon}{3590}
\showcaseopenmojiblack{waxing gibbous moon}{3591}
\showcaseopenmojiblack{weary cat}{3592}
\showcaseopenmojiblack{weary face}{3593}
\showcaseopenmojiblack{web syndication}{3594}
\showcaseopenmojiblack{webassembly}{3595}
\showcaseopenmojiblack{wedding}{3596}
\showcaseopenmojiblack{whale}{3597}
\showcaseopenmojiblack{wheel chair}{3598}
\showcaseopenmojiblack{wheel of dharma}{3599}
\showcaseopenmojiblack{wheel}{3600}
\showcaseopenmojiblack{wheelbarrow}{3601}
\showcaseopenmojiblack{wheelchair symbol}{3602}
\showcaseopenmojiblack{white cane}{3603}
\showcaseopenmojiblack{white circle}{3604}
\showcaseopenmojiblack{white exclamation mark}{3605}
\showcaseopenmojiblack{white flag}{3606}
\showcaseopenmojiblack{white flower}{3607}
\showcaseopenmojiblack{white hair}{3608}
\showcaseopenmojiblack{white heart}{3609}
\showcaseopenmojiblack{white hexagon}{3610}
\showcaseopenmojiblack{white large square}{3611}
\showcaseopenmojiblack{white medium square}{3612}
\showcaseopenmojiblack{white medium-small square}{3613}
\showcaseopenmojiblack{white pentagon}{3614}
\showcaseopenmojiblack{white question mark}{3615}
\showcaseopenmojiblack{white rectangle}{3616}
\showcaseopenmojiblack{white small square}{3617}
\showcaseopenmojiblack{white square button}{3618}
\showcaseopenmojiblack{white square}{3619}
\showcaseopenmojiblack{white vertical ellipse}{3620}
\showcaseopenmojiblack{wifi}{3621}
\showcaseopenmojiblack{wikidata}{3622}
\showcaseopenmojiblack{wild fire}{3623}
\showcaseopenmojiblack{wilted flower}{3624}
\showcaseopenmojiblack{wind chime}{3625}
\showcaseopenmojiblack{wind energy}{3626}
\showcaseopenmojiblack{wind face}{3627}
\showcaseopenmojiblack{window}{3628}
\showcaseopenmojiblack{windows}{3629}
\showcaseopenmojiblack{windsurfing}{3630}
\showcaseopenmojiblack{wine glass}{3631}
\showcaseopenmojiblack{wing}{3632}
\showcaseopenmojiblack{winking face with tongue}{3633}
\showcaseopenmojiblack{winking face}{3634}
\showcaseopenmojiblack{winrar}{3635}
\showcaseopenmojiblack{wire}{3636}
\showcaseopenmojiblack{wireframes}{3637}
\showcaseopenmojiblack{wireless}{3638}
\showcaseopenmojiblack{wolf}{3639}
\showcaseopenmojiblack{woman and man holding hands- dark skin tone, light skin tone}{3640}
\showcaseopenmojiblack{woman and man holding hands- dark skin tone, medium skin tone}{3641}
\showcaseopenmojiblack{woman and man holding hands- dark skin tone, medium-dark skin tone}{3642}
\showcaseopenmojiblack{woman and man holding hands- dark skin tone, medium-light skin tone}{3643}
\showcaseopenmojiblack{woman and man holding hands- dark skin tone}{3644}
\showcaseopenmojiblack{woman and man holding hands- light skin tone, dark skin tone}{3645}
\showcaseopenmojiblack{woman and man holding hands- light skin tone, medium skin tone}{3646}
\showcaseopenmojiblack{woman and man holding hands- light skin tone, medium-dark skin tone}{3647}
\showcaseopenmojiblack{woman and man holding hands- light skin tone, medium-light skin tone}{3648}
\showcaseopenmojiblack{woman and man holding hands- light skin tone}{3649}
\showcaseopenmojiblack{woman and man holding hands- medium skin tone, dark skin tone}{3650}
\showcaseopenmojiblack{woman and man holding hands- medium skin tone, light skin tone}{3651}
\showcaseopenmojiblack{woman and man holding hands- medium skin tone, medium-dark skin tone}{3652}
\showcaseopenmojiblack{woman and man holding hands- medium skin tone, medium-light skin tone}{3653}
\showcaseopenmojiblack{woman and man holding hands- medium skin tone}{3654}
\showcaseopenmojiblack{woman and man holding hands- medium-dark skin tone, dark skin tone}{3655}
\showcaseopenmojiblack{woman and man holding hands- medium-dark skin tone, light skin tone}{3656}
\showcaseopenmojiblack{woman and man holding hands- medium-dark skin tone, medium skin tone}{3657}
\showcaseopenmojiblack{woman and man holding hands- medium-dark skin tone, medium-light skin tone}{3658}
\showcaseopenmojiblack{woman and man holding hands- medium-dark skin tone}{3659}
\showcaseopenmojiblack{woman and man holding hands- medium-light skin tone, dark skin tone}{3660}
\showcaseopenmojiblack{woman and man holding hands- medium-light skin tone, light skin tone}{3661}
\showcaseopenmojiblack{woman and man holding hands- medium-light skin tone, medium skin tone}{3662}
\showcaseopenmojiblack{woman and man holding hands- medium-light skin tone, medium-dark skin tone}{3663}
\showcaseopenmojiblack{woman and man holding hands- medium-light skin tone}{3664}
\showcaseopenmojiblack{woman and man holding hands}{3665}
\showcaseopenmojiblack{woman artist- dark skin tone}{3666}
\showcaseopenmojiblack{woman artist- light skin tone}{3667}
\showcaseopenmojiblack{woman artist- medium skin tone}{3668}
\showcaseopenmojiblack{woman artist- medium-dark skin tone}{3669}
\showcaseopenmojiblack{woman artist- medium-light skin tone}{3670}
\showcaseopenmojiblack{woman artist}{3671}
\showcaseopenmojiblack{woman astronaut- dark skin tone}{3672}
\showcaseopenmojiblack{woman astronaut- light skin tone}{3673}
\showcaseopenmojiblack{woman astronaut- medium skin tone}{3674}
\showcaseopenmojiblack{woman astronaut- medium-dark skin tone}{3675}
\showcaseopenmojiblack{woman astronaut- medium-light skin tone}{3676}
\showcaseopenmojiblack{woman astronaut}{3677}
\showcaseopenmojiblack{woman barista}{3678}
\showcaseopenmojiblack{woman biking- dark skin tone}{3679}
\showcaseopenmojiblack{woman biking- light skin tone}{3680}
\showcaseopenmojiblack{woman biking- medium skin tone}{3681}
\showcaseopenmojiblack{woman biking- medium-dark skin tone}{3682}
\showcaseopenmojiblack{woman biking- medium-light skin tone}{3683}
\showcaseopenmojiblack{woman biking}{3684}
\showcaseopenmojiblack{woman bouncing ball- dark skin tone}{3685}
\showcaseopenmojiblack{woman bouncing ball- light skin tone}{3686}
\showcaseopenmojiblack{woman bouncing ball- medium skin tone}{3687}
\showcaseopenmojiblack{woman bouncing ball- medium-dark skin tone}{3688}
\showcaseopenmojiblack{woman bouncing ball- medium-light skin tone}{3689}
\showcaseopenmojiblack{woman bouncing ball}{3690}
\showcaseopenmojiblack{woman bowing- dark skin tone}{3691}
\showcaseopenmojiblack{woman bowing- light skin tone}{3692}
\showcaseopenmojiblack{woman bowing- medium skin tone}{3693}
\showcaseopenmojiblack{woman bowing- medium-dark skin tone}{3694}
\showcaseopenmojiblack{woman bowing- medium-light skin tone}{3695}
\showcaseopenmojiblack{woman bowing}{3696}
\showcaseopenmojiblack{woman cartwheeling- dark skin tone}{3697}
\showcaseopenmojiblack{woman cartwheeling- light skin tone}{3698}
\showcaseopenmojiblack{woman cartwheeling- medium skin tone}{3699}
\showcaseopenmojiblack{woman cartwheeling- medium-dark skin tone}{3700}
\showcaseopenmojiblack{woman cartwheeling- medium-light skin tone}{3701}
\showcaseopenmojiblack{woman cartwheeling}{3702}
\showcaseopenmojiblack{woman climbing- dark skin tone}{3703}
\showcaseopenmojiblack{woman climbing- light skin tone}{3704}
\showcaseopenmojiblack{woman climbing- medium skin tone}{3705}
\showcaseopenmojiblack{woman climbing- medium-dark skin tone}{3706}
\showcaseopenmojiblack{woman climbing- medium-light skin tone}{3707}
\showcaseopenmojiblack{woman climbing}{3708}
\showcaseopenmojiblack{woman construction worker- dark skin tone}{3709}
\showcaseopenmojiblack{woman construction worker- light skin tone}{3710}
\showcaseopenmojiblack{woman construction worker- medium skin tone}{3711}
\showcaseopenmojiblack{woman construction worker- medium-dark skin tone}{3712}
\showcaseopenmojiblack{woman construction worker- medium-light skin tone}{3713}
\showcaseopenmojiblack{woman construction worker}{3714}
\showcaseopenmojiblack{woman cook- dark skin tone}{3715}
\showcaseopenmojiblack{woman cook- light skin tone}{3716}
\showcaseopenmojiblack{woman cook- medium skin tone}{3717}
\showcaseopenmojiblack{woman cook- medium-dark skin tone}{3718}
\showcaseopenmojiblack{woman cook- medium-light skin tone}{3719}
\showcaseopenmojiblack{woman cook}{3720}
\showcaseopenmojiblack{woman dancing- dark skin tone}{3721}
\showcaseopenmojiblack{woman dancing- light skin tone}{3722}
\showcaseopenmojiblack{woman dancing- medium skin tone}{3723}
\showcaseopenmojiblack{woman dancing- medium-dark skin tone}{3724}
\showcaseopenmojiblack{woman dancing- medium-light skin tone}{3725}
\showcaseopenmojiblack{woman dancing}{3726}
\showcaseopenmojiblack{woman detective- dark skin tone}{3727}
\showcaseopenmojiblack{woman detective- light skin tone}{3728}
\showcaseopenmojiblack{woman detective- medium skin tone}{3729}
\showcaseopenmojiblack{woman detective- medium-dark skin tone}{3730}
\showcaseopenmojiblack{woman detective- medium-light skin tone}{3731}
\showcaseopenmojiblack{woman detective}{3732}
\showcaseopenmojiblack{woman elf- dark skin tone}{3733}
\showcaseopenmojiblack{woman elf- light skin tone}{3734}
\showcaseopenmojiblack{woman elf- medium skin tone}{3735}
\showcaseopenmojiblack{woman elf- medium-dark skin tone}{3736}
\showcaseopenmojiblack{woman elf- medium-light skin tone}{3737}
\showcaseopenmojiblack{woman elf}{3738}
\showcaseopenmojiblack{woman facepalming- dark skin tone}{3739}
\showcaseopenmojiblack{woman facepalming- light skin tone}{3740}
\showcaseopenmojiblack{woman facepalming- medium skin tone}{3741}
\showcaseopenmojiblack{woman facepalming- medium-dark skin tone}{3742}
\showcaseopenmojiblack{woman facepalming- medium-light skin tone}{3743}
\showcaseopenmojiblack{woman facepalming}{3744}
\showcaseopenmojiblack{woman factory worker- dark skin tone}{3745}
\showcaseopenmojiblack{woman factory worker- light skin tone}{3746}
\showcaseopenmojiblack{woman factory worker- medium skin tone}{3747}
\showcaseopenmojiblack{woman factory worker- medium-dark skin tone}{3748}
\showcaseopenmojiblack{woman factory worker- medium-light skin tone}{3749}
\showcaseopenmojiblack{woman factory worker}{3750}
\showcaseopenmojiblack{woman fairy- dark skin tone}{3751}
\showcaseopenmojiblack{woman fairy- light skin tone}{3752}
\showcaseopenmojiblack{woman fairy- medium skin tone}{3753}
\showcaseopenmojiblack{woman fairy- medium-dark skin tone}{3754}
\showcaseopenmojiblack{woman fairy- medium-light skin tone}{3755}
\showcaseopenmojiblack{woman fairy}{3756}
\showcaseopenmojiblack{woman farmer- dark skin tone}{3757}
\showcaseopenmojiblack{woman farmer- light skin tone}{3758}
\showcaseopenmojiblack{woman farmer- medium skin tone}{3759}
\showcaseopenmojiblack{woman farmer- medium-dark skin tone}{3760}
\showcaseopenmojiblack{woman farmer- medium-light skin tone}{3761}
\showcaseopenmojiblack{woman farmer}{3762}
\showcaseopenmojiblack{woman feeding baby- dark skin tone}{3763}
\showcaseopenmojiblack{woman feeding baby- light skin tone}{3764}
\showcaseopenmojiblack{woman feeding baby- medium skin tone}{3765}
\showcaseopenmojiblack{woman feeding baby- medium-dark skin tone}{3766}
\showcaseopenmojiblack{woman feeding baby- medium-light skin tone}{3767}
\showcaseopenmojiblack{woman feeding baby}{3768}
\showcaseopenmojiblack{woman firefighter- dark skin tone}{3769}
\showcaseopenmojiblack{woman firefighter- light skin tone}{3770}
\showcaseopenmojiblack{woman firefighter- medium skin tone}{3771}
\showcaseopenmojiblack{woman firefighter- medium-dark skin tone}{3772}
\showcaseopenmojiblack{woman firefighter- medium-light skin tone}{3773}
\showcaseopenmojiblack{woman firefighter}{3774}
\showcaseopenmojiblack{woman frowning- dark skin tone}{3775}
\showcaseopenmojiblack{woman frowning- light skin tone}{3776}
\showcaseopenmojiblack{woman frowning- medium skin tone}{3777}
\showcaseopenmojiblack{woman frowning- medium-dark skin tone}{3778}
\showcaseopenmojiblack{woman frowning- medium-light skin tone}{3779}
\showcaseopenmojiblack{woman frowning}{3780}
\showcaseopenmojiblack{woman genie}{3781}
\showcaseopenmojiblack{woman gesturing no- dark skin tone}{3782}
\showcaseopenmojiblack{woman gesturing no- light skin tone}{3783}
\showcaseopenmojiblack{woman gesturing no- medium skin tone}{3784}
\showcaseopenmojiblack{woman gesturing no- medium-dark skin tone}{3785}
\showcaseopenmojiblack{woman gesturing no- medium-light skin tone}{3786}
\showcaseopenmojiblack{woman gesturing no}{3787}
\showcaseopenmojiblack{woman gesturing ok- dark skin tone}{3788}
\showcaseopenmojiblack{woman gesturing ok- light skin tone}{3789}
\showcaseopenmojiblack{woman gesturing ok- medium skin tone}{3790}
\showcaseopenmojiblack{woman gesturing ok- medium-dark skin tone}{3791}
\showcaseopenmojiblack{woman gesturing ok- medium-light skin tone}{3792}
\showcaseopenmojiblack{woman gesturing ok}{3793}
\showcaseopenmojiblack{woman getting haircut- dark skin tone}{3794}
\showcaseopenmojiblack{woman getting haircut- light skin tone}{3795}
\showcaseopenmojiblack{woman getting haircut- medium skin tone}{3796}
\showcaseopenmojiblack{woman getting haircut- medium-dark skin tone}{3797}
\showcaseopenmojiblack{woman getting haircut- medium-light skin tone}{3798}
\showcaseopenmojiblack{woman getting haircut}{3799}
\showcaseopenmojiblack{woman getting massage- dark skin tone}{3800}
\showcaseopenmojiblack{woman getting massage- light skin tone}{3801}
\showcaseopenmojiblack{woman getting massage- medium skin tone}{3802}
\showcaseopenmojiblack{woman getting massage- medium-dark skin tone}{3803}
\showcaseopenmojiblack{woman getting massage- medium-light skin tone}{3804}
\showcaseopenmojiblack{woman getting massage}{3805}
\showcaseopenmojiblack{woman golfing- dark skin tone}{3806}
\showcaseopenmojiblack{woman golfing- light skin tone}{3807}
\showcaseopenmojiblack{woman golfing- medium skin tone}{3808}
\showcaseopenmojiblack{woman golfing- medium-dark skin tone}{3809}
\showcaseopenmojiblack{woman golfing- medium-light skin tone}{3810}
\showcaseopenmojiblack{woman golfing}{3811}
\showcaseopenmojiblack{woman guard- dark skin tone}{3812}
\showcaseopenmojiblack{woman guard- light skin tone}{3813}
\showcaseopenmojiblack{woman guard- medium skin tone}{3814}
\showcaseopenmojiblack{woman guard- medium-dark skin tone}{3815}
\showcaseopenmojiblack{woman guard- medium-light skin tone}{3816}
\showcaseopenmojiblack{woman guard}{3817}
\showcaseopenmojiblack{woman health worker- dark skin tone}{3818}
\showcaseopenmojiblack{woman health worker- light skin tone}{3819}
\showcaseopenmojiblack{woman health worker- medium skin tone}{3820}
\showcaseopenmojiblack{woman health worker- medium-dark skin tone}{3821}
\showcaseopenmojiblack{woman health worker- medium-light skin tone}{3822}
\showcaseopenmojiblack{woman health worker}{3823}
\showcaseopenmojiblack{woman in lotus position- dark skin tone}{3824}
\showcaseopenmojiblack{woman in lotus position- light skin tone}{3825}
\showcaseopenmojiblack{woman in lotus position- medium skin tone}{3826}
\showcaseopenmojiblack{woman in lotus position- medium-dark skin tone}{3827}
\showcaseopenmojiblack{woman in lotus position- medium-light skin tone}{3828}
\showcaseopenmojiblack{woman in lotus position}{3829}
\showcaseopenmojiblack{woman in manual wheelchair facing right}{3830}
\showcaseopenmojiblack{woman in manual wheelchair- dark skin tone}{3831}
\showcaseopenmojiblack{woman in manual wheelchair- light skin tone}{3832}
\showcaseopenmojiblack{woman in manual wheelchair- medium skin tone}{3833}
\showcaseopenmojiblack{woman in manual wheelchair- medium-dark skin tone}{3834}
\showcaseopenmojiblack{woman in manual wheelchair- medium-light skin tone}{3835}
\showcaseopenmojiblack{woman in manual wheelchair}{3836}
\showcaseopenmojiblack{woman in motorized wheelchair facing right}{3837}
\showcaseopenmojiblack{woman in motorized wheelchair- dark skin tone}{3838}
\showcaseopenmojiblack{woman in motorized wheelchair- light skin tone}{3839}
\showcaseopenmojiblack{woman in motorized wheelchair- medium skin tone}{3840}
\showcaseopenmojiblack{woman in motorized wheelchair- medium-dark skin tone}{3841}
\showcaseopenmojiblack{woman in motorized wheelchair- medium-light skin tone}{3842}
\showcaseopenmojiblack{woman in motorized wheelchair}{3843}
\showcaseopenmojiblack{woman in steamy room- dark skin tone}{3844}
\showcaseopenmojiblack{woman in steamy room- light skin tone}{3845}
\showcaseopenmojiblack{woman in steamy room- medium skin tone}{3846}
\showcaseopenmojiblack{woman in steamy room- medium-dark skin tone}{3847}
\showcaseopenmojiblack{woman in steamy room- medium-light skin tone}{3848}
\showcaseopenmojiblack{woman in steamy room}{3849}
\showcaseopenmojiblack{woman in tuxedo- dark skin tone}{3850}
\showcaseopenmojiblack{woman in tuxedo- light skin tone}{3851}
\showcaseopenmojiblack{woman in tuxedo- medium skin tone}{3852}
\showcaseopenmojiblack{woman in tuxedo- medium-dark skin tone}{3853}
\showcaseopenmojiblack{woman in tuxedo- medium-light skin tone}{3854}
\showcaseopenmojiblack{woman in tuxedo}{3855}
\showcaseopenmojiblack{woman judge- dark skin tone}{3856}
\showcaseopenmojiblack{woman judge- light skin tone}{3857}
\showcaseopenmojiblack{woman judge- medium skin tone}{3858}
\showcaseopenmojiblack{woman judge- medium-dark skin tone}{3859}
\showcaseopenmojiblack{woman judge- medium-light skin tone}{3860}
\showcaseopenmojiblack{woman judge}{3861}
\showcaseopenmojiblack{woman juggling- dark skin tone}{3862}
\showcaseopenmojiblack{woman juggling- light skin tone}{3863}
\showcaseopenmojiblack{woman juggling- medium skin tone}{3864}
\showcaseopenmojiblack{woman juggling- medium-dark skin tone}{3865}
\showcaseopenmojiblack{woman juggling- medium-light skin tone}{3866}
\showcaseopenmojiblack{woman juggling}{3867}
\showcaseopenmojiblack{woman kneeling facing right}{3868}
\showcaseopenmojiblack{woman kneeling- dark skin tone}{3869}
\showcaseopenmojiblack{woman kneeling- light skin tone}{3870}
\showcaseopenmojiblack{woman kneeling- medium skin tone}{3871}
\showcaseopenmojiblack{woman kneeling- medium-dark skin tone}{3872}
\showcaseopenmojiblack{woman kneeling- medium-light skin tone}{3873}
\showcaseopenmojiblack{woman kneeling}{3874}
\showcaseopenmojiblack{woman lifting weights- dark skin tone}{3875}
\showcaseopenmojiblack{woman lifting weights- light skin tone}{3876}
\showcaseopenmojiblack{woman lifting weights- medium skin tone}{3877}
\showcaseopenmojiblack{woman lifting weights- medium-dark skin tone}{3878}
\showcaseopenmojiblack{woman lifting weights- medium-light skin tone}{3879}
\showcaseopenmojiblack{woman lifting weights}{3880}
\showcaseopenmojiblack{woman mage- dark skin tone}{3881}
\showcaseopenmojiblack{woman mage- light skin tone}{3882}
\showcaseopenmojiblack{woman mage- medium skin tone}{3883}
\showcaseopenmojiblack{woman mage- medium-dark skin tone}{3884}
\showcaseopenmojiblack{woman mage- medium-light skin tone}{3885}
\showcaseopenmojiblack{woman mage}{3886}
\showcaseopenmojiblack{woman mechanic- dark skin tone}{3887}
\showcaseopenmojiblack{woman mechanic- light skin tone}{3888}
\showcaseopenmojiblack{woman mechanic- medium skin tone}{3889}
\showcaseopenmojiblack{woman mechanic- medium-dark skin tone}{3890}
\showcaseopenmojiblack{woman mechanic- medium-light skin tone}{3891}
\showcaseopenmojiblack{woman mechanic}{3892}
\showcaseopenmojiblack{woman mountain biking- dark skin tone}{3893}
\showcaseopenmojiblack{woman mountain biking- light skin tone}{3894}
\showcaseopenmojiblack{woman mountain biking- medium skin tone}{3895}
\showcaseopenmojiblack{woman mountain biking- medium-dark skin tone}{3896}
\showcaseopenmojiblack{woman mountain biking- medium-light skin tone}{3897}
\showcaseopenmojiblack{woman mountain biking}{3898}
\showcaseopenmojiblack{woman office worker- dark skin tone}{3899}
\showcaseopenmojiblack{woman office worker- light skin tone}{3900}
\showcaseopenmojiblack{woman office worker- medium skin tone}{3901}
\showcaseopenmojiblack{woman office worker- medium-dark skin tone}{3902}
\showcaseopenmojiblack{woman office worker- medium-light skin tone}{3903}
\showcaseopenmojiblack{woman office worker}{3904}
\showcaseopenmojiblack{woman pilot- dark skin tone}{3905}
\showcaseopenmojiblack{woman pilot- light skin tone}{3906}
\showcaseopenmojiblack{woman pilot- medium skin tone}{3907}
\showcaseopenmojiblack{woman pilot- medium-dark skin tone}{3908}
\showcaseopenmojiblack{woman pilot- medium-light skin tone}{3909}
\showcaseopenmojiblack{woman pilot}{3910}
\showcaseopenmojiblack{woman playing handball- dark skin tone}{3911}
\showcaseopenmojiblack{woman playing handball- light skin tone}{3912}
\showcaseopenmojiblack{woman playing handball- medium skin tone}{3913}
\showcaseopenmojiblack{woman playing handball- medium-dark skin tone}{3914}
\showcaseopenmojiblack{woman playing handball- medium-light skin tone}{3915}
\showcaseopenmojiblack{woman playing handball}{3916}
\showcaseopenmojiblack{woman playing water polo- dark skin tone}{3917}
\showcaseopenmojiblack{woman playing water polo- light skin tone}{3918}
\showcaseopenmojiblack{woman playing water polo- medium skin tone}{3919}
\showcaseopenmojiblack{woman playing water polo- medium-dark skin tone}{3920}
\showcaseopenmojiblack{woman playing water polo- medium-light skin tone}{3921}
\showcaseopenmojiblack{woman playing water polo}{3922}
\showcaseopenmojiblack{woman police officer- dark skin tone}{3923}
\showcaseopenmojiblack{woman police officer- light skin tone}{3924}
\showcaseopenmojiblack{woman police officer- medium skin tone}{3925}
\showcaseopenmojiblack{woman police officer- medium-dark skin tone}{3926}
\showcaseopenmojiblack{woman police officer- medium-light skin tone}{3927}
\showcaseopenmojiblack{woman police officer}{3928}
\showcaseopenmojiblack{woman pouting- dark skin tone}{3929}
\showcaseopenmojiblack{woman pouting- light skin tone}{3930}
\showcaseopenmojiblack{woman pouting- medium skin tone}{3931}
\showcaseopenmojiblack{woman pouting- medium-dark skin tone}{3932}
\showcaseopenmojiblack{woman pouting- medium-light skin tone}{3933}
\showcaseopenmojiblack{woman pouting}{3934}
\showcaseopenmojiblack{woman raising hand- dark skin tone}{3935}
\showcaseopenmojiblack{woman raising hand- light skin tone}{3936}
\showcaseopenmojiblack{woman raising hand- medium skin tone}{3937}
\showcaseopenmojiblack{woman raising hand- medium-dark skin tone}{3938}
\showcaseopenmojiblack{woman raising hand- medium-light skin tone}{3939}
\showcaseopenmojiblack{woman raising hand}{3940}
\showcaseopenmojiblack{woman rowing boat- dark skin tone}{3941}
\showcaseopenmojiblack{woman rowing boat- light skin tone}{3942}
\showcaseopenmojiblack{woman rowing boat- medium skin tone}{3943}
\showcaseopenmojiblack{woman rowing boat- medium-dark skin tone}{3944}
\showcaseopenmojiblack{woman rowing boat- medium-light skin tone}{3945}
\showcaseopenmojiblack{woman rowing boat}{3946}
\showcaseopenmojiblack{woman running facing right}{3947}
\showcaseopenmojiblack{woman running- dark skin tone}{3948}
\showcaseopenmojiblack{woman running- light skin tone}{3949}
\showcaseopenmojiblack{woman running- medium skin tone}{3950}
\showcaseopenmojiblack{woman running- medium-dark skin tone}{3951}
\showcaseopenmojiblack{woman running- medium-light skin tone}{3952}
\showcaseopenmojiblack{woman running}{3953}
\showcaseopenmojiblack{woman s boot}{3954}
\showcaseopenmojiblack{woman s clothes}{3955}
\showcaseopenmojiblack{woman s hat}{3956}
\showcaseopenmojiblack{woman s sandal}{3957}
\showcaseopenmojiblack{woman scientist- dark skin tone}{3958}
\showcaseopenmojiblack{woman scientist- light skin tone}{3959}
\showcaseopenmojiblack{woman scientist- medium skin tone}{3960}
\showcaseopenmojiblack{woman scientist- medium-dark skin tone}{3961}
\showcaseopenmojiblack{woman scientist- medium-light skin tone}{3962}
\showcaseopenmojiblack{woman scientist}{3963}
\showcaseopenmojiblack{woman shrugging- dark skin tone}{3964}
\showcaseopenmojiblack{woman shrugging- light skin tone}{3965}
\showcaseopenmojiblack{woman shrugging- medium skin tone}{3966}
\showcaseopenmojiblack{woman shrugging- medium-dark skin tone}{3967}
\showcaseopenmojiblack{woman shrugging- medium-light skin tone}{3968}
\showcaseopenmojiblack{woman shrugging}{3969}
\showcaseopenmojiblack{woman singer- dark skin tone}{3970}
\showcaseopenmojiblack{woman singer- light skin tone}{3971}
\showcaseopenmojiblack{woman singer- medium skin tone}{3972}
\showcaseopenmojiblack{woman singer- medium-dark skin tone}{3973}
\showcaseopenmojiblack{woman singer- medium-light skin tone}{3974}
\showcaseopenmojiblack{woman singer}{3975}
\showcaseopenmojiblack{woman sneezing into elbow}{3976}
\showcaseopenmojiblack{woman standing- dark skin tone}{3977}
\showcaseopenmojiblack{woman standing- light skin tone}{3978}
\showcaseopenmojiblack{woman standing- medium skin tone}{3979}
\showcaseopenmojiblack{woman standing- medium-dark skin tone}{3980}
\showcaseopenmojiblack{woman standing- medium-light skin tone}{3981}
\showcaseopenmojiblack{woman standing}{3982}
\showcaseopenmojiblack{woman student- dark skin tone}{3983}
\showcaseopenmojiblack{woman student- light skin tone}{3984}
\showcaseopenmojiblack{woman student- medium skin tone}{3985}
\showcaseopenmojiblack{woman student- medium-dark skin tone}{3986}
\showcaseopenmojiblack{woman student- medium-light skin tone}{3987}
\showcaseopenmojiblack{woman student}{3988}
\showcaseopenmojiblack{woman superhero- dark skin tone}{3989}
\showcaseopenmojiblack{woman superhero- light skin tone}{3990}
\showcaseopenmojiblack{woman superhero- medium skin tone}{3991}
\showcaseopenmojiblack{woman superhero- medium-dark skin tone}{3992}
\showcaseopenmojiblack{woman superhero- medium-light skin tone}{3993}
\showcaseopenmojiblack{woman superhero}{3994}
\showcaseopenmojiblack{woman supervillain- dark skin tone}{3995}
\showcaseopenmojiblack{woman supervillain- light skin tone}{3996}
\showcaseopenmojiblack{woman supervillain- medium skin tone}{3997}
\showcaseopenmojiblack{woman supervillain- medium-dark skin tone}{3998}
\showcaseopenmojiblack{woman supervillain- medium-light skin tone}{3999}
\showcaseopenmojiblack{woman supervillain}{4000}
\showcaseopenmojiblack{woman surfing- dark skin tone}{4001}
\showcaseopenmojiblack{woman surfing- light skin tone}{4002}
\showcaseopenmojiblack{woman surfing- medium skin tone}{4003}
\showcaseopenmojiblack{woman surfing- medium-dark skin tone}{4004}
\showcaseopenmojiblack{woman surfing- medium-light skin tone}{4005}
\showcaseopenmojiblack{woman surfing}{4006}
\showcaseopenmojiblack{woman swimming- dark skin tone}{4007}
\showcaseopenmojiblack{woman swimming- light skin tone}{4008}
\showcaseopenmojiblack{woman swimming- medium skin tone}{4009}
\showcaseopenmojiblack{woman swimming- medium-dark skin tone}{4010}
\showcaseopenmojiblack{woman swimming- medium-light skin tone}{4011}
\showcaseopenmojiblack{woman swimming}{4012}
\showcaseopenmojiblack{woman teacher- dark skin tone}{4013}
\showcaseopenmojiblack{woman teacher- light skin tone}{4014}
\showcaseopenmojiblack{woman teacher- medium skin tone}{4015}
\showcaseopenmojiblack{woman teacher- medium-dark skin tone}{4016}
\showcaseopenmojiblack{woman teacher- medium-light skin tone}{4017}
\showcaseopenmojiblack{woman teacher}{4018}
\showcaseopenmojiblack{woman technologist- dark skin tone}{4019}
\showcaseopenmojiblack{woman technologist- light skin tone}{4020}
\showcaseopenmojiblack{woman technologist- medium skin tone}{4021}
\showcaseopenmojiblack{woman technologist- medium-dark skin tone}{4022}
\showcaseopenmojiblack{woman technologist- medium-light skin tone}{4023}
\showcaseopenmojiblack{woman technologist}{4024}
\showcaseopenmojiblack{woman tipping hand- dark skin tone}{4025}
\showcaseopenmojiblack{woman tipping hand- light skin tone}{4026}
\showcaseopenmojiblack{woman tipping hand- medium skin tone}{4027}
\showcaseopenmojiblack{woman tipping hand- medium-dark skin tone}{4028}
\showcaseopenmojiblack{woman tipping hand- medium-light skin tone}{4029}
\showcaseopenmojiblack{woman tipping hand}{4030}
\showcaseopenmojiblack{woman vampire- dark skin tone}{4031}
\showcaseopenmojiblack{woman vampire- light skin tone}{4032}
\showcaseopenmojiblack{woman vampire- medium skin tone}{4033}
\showcaseopenmojiblack{woman vampire- medium-dark skin tone}{4034}
\showcaseopenmojiblack{woman vampire- medium-light skin tone}{4035}
\showcaseopenmojiblack{woman vampire}{4036}
\showcaseopenmojiblack{woman walking facing right}{4037}
\showcaseopenmojiblack{woman walking- dark skin tone}{4038}
\showcaseopenmojiblack{woman walking- light skin tone}{4039}
\showcaseopenmojiblack{woman walking- medium skin tone}{4040}
\showcaseopenmojiblack{woman walking- medium-dark skin tone}{4041}
\showcaseopenmojiblack{woman walking- medium-light skin tone}{4042}
\showcaseopenmojiblack{woman walking}{4043}
\showcaseopenmojiblack{woman wearing turban- dark skin tone}{4044}
\showcaseopenmojiblack{woman wearing turban- light skin tone}{4045}
\showcaseopenmojiblack{woman wearing turban- medium skin tone}{4046}
\showcaseopenmojiblack{woman wearing turban- medium-dark skin tone}{4047}
\showcaseopenmojiblack{woman wearing turban- medium-light skin tone}{4048}
\showcaseopenmojiblack{woman wearing turban}{4049}
\showcaseopenmojiblack{woman with headscarf- dark skin tone}{4050}
\showcaseopenmojiblack{woman with headscarf- light skin tone}{4051}
\showcaseopenmojiblack{woman with headscarf- medium skin tone}{4052}
\showcaseopenmojiblack{woman with headscarf- medium-dark skin tone}{4053}
\showcaseopenmojiblack{woman with headscarf- medium-light skin tone}{4054}
\showcaseopenmojiblack{woman with headscarf}{4055}
\showcaseopenmojiblack{woman with medical mask}{4056}
\showcaseopenmojiblack{woman with veil- dark skin tone}{4057}
\showcaseopenmojiblack{woman with veil- light skin tone}{4058}
\showcaseopenmojiblack{woman with veil- medium skin tone}{4059}
\showcaseopenmojiblack{woman with veil- medium-dark skin tone}{4060}
\showcaseopenmojiblack{woman with veil- medium-light skin tone}{4061}
\showcaseopenmojiblack{woman with veil}{4062}
\showcaseopenmojiblack{woman with white cane facing right}{4063}
\showcaseopenmojiblack{woman with white cane- dark skin tone}{4064}
\showcaseopenmojiblack{woman with white cane- light skin tone}{4065}
\showcaseopenmojiblack{woman with white cane- medium skin tone}{4066}
\showcaseopenmojiblack{woman with white cane- medium-dark skin tone}{4067}
\showcaseopenmojiblack{woman with white cane- medium-light skin tone}{4068}
\showcaseopenmojiblack{woman with white cane}{4069}
\showcaseopenmojiblack{woman zombie}{4070}
\showcaseopenmojiblack{woman- bald}{4071}
\showcaseopenmojiblack{woman- beard}{4072}
\showcaseopenmojiblack{woman- blond hair}{4073}
\showcaseopenmojiblack{woman- curly hair}{4074}
\showcaseopenmojiblack{woman- dark skin tone, bald}{4075}
\showcaseopenmojiblack{woman- dark skin tone, beard}{4076}
\showcaseopenmojiblack{woman- dark skin tone, blond hair}{4077}
\showcaseopenmojiblack{woman- dark skin tone, curly hair}{4078}
\showcaseopenmojiblack{woman- dark skin tone, red hair}{4079}
\showcaseopenmojiblack{woman- dark skin tone, white hair}{4080}
\showcaseopenmojiblack{woman- dark skin tone}{4081}
\showcaseopenmojiblack{woman- light skin tone, bald}{4082}
\showcaseopenmojiblack{woman- light skin tone, beard}{4083}
\showcaseopenmojiblack{woman- light skin tone, blond hair}{4084}
\showcaseopenmojiblack{woman- light skin tone, curly hair}{4085}
\showcaseopenmojiblack{woman- light skin tone, red hair}{4086}
\showcaseopenmojiblack{woman- light skin tone, white hair}{4087}
\showcaseopenmojiblack{woman- light skin tone}{4088}
\showcaseopenmojiblack{woman- medium skin tone, bald}{4089}
\showcaseopenmojiblack{woman- medium skin tone, beard}{4090}
\showcaseopenmojiblack{woman- medium skin tone, blond hair}{4091}
\showcaseopenmojiblack{woman- medium skin tone, curly hair}{4092}
\showcaseopenmojiblack{woman- medium skin tone, red hair}{4093}
\showcaseopenmojiblack{woman- medium skin tone, white hair}{4094}
\showcaseopenmojiblack{woman- medium skin tone}{4095}
\showcaseopenmojiblack{woman- medium-dark skin tone, bald}{4096}
\showcaseopenmojiblack{woman- medium-dark skin tone, beard}{4097}
\showcaseopenmojiblack{woman- medium-dark skin tone, blond hair}{4098}
\showcaseopenmojiblack{woman- medium-dark skin tone, curly hair}{4099}
\showcaseopenmojiblack{woman- medium-dark skin tone, red hair}{4100}
\showcaseopenmojiblack{woman- medium-dark skin tone, white hair}{4101}
\showcaseopenmojiblack{woman- medium-dark skin tone}{4102}
\showcaseopenmojiblack{woman- medium-light skin tone, bald}{4103}
\showcaseopenmojiblack{woman- medium-light skin tone, beard}{4104}
\showcaseopenmojiblack{woman- medium-light skin tone, blond hair}{4105}
\showcaseopenmojiblack{woman- medium-light skin tone, curly hair}{4106}
\showcaseopenmojiblack{woman- medium-light skin tone, red hair}{4107}
\showcaseopenmojiblack{woman- medium-light skin tone, white hair}{4108}
\showcaseopenmojiblack{woman- medium-light skin tone}{4109}
\showcaseopenmojiblack{woman- red hair}{4110}
\showcaseopenmojiblack{woman- white hair}{4111}
\showcaseopenmojiblack{woman}{4112}
\showcaseopenmojiblack{women holding hands- dark skin tone, light skin tone}{4113}
\showcaseopenmojiblack{women holding hands- dark skin tone, medium skin tone}{4114}
\showcaseopenmojiblack{women holding hands- dark skin tone, medium-dark skin tone}{4115}
\showcaseopenmojiblack{women holding hands- dark skin tone, medium-light skin tone}{4116}
\showcaseopenmojiblack{women holding hands- dark skin tone}{4117}
\showcaseopenmojiblack{women holding hands- light skin tone, dark skin tone}{4118}
\showcaseopenmojiblack{women holding hands- light skin tone, medium skin tone}{4119}
\showcaseopenmojiblack{women holding hands- light skin tone, medium-dark skin tone}{4120}
\showcaseopenmojiblack{women holding hands- light skin tone, medium-light skin tone}{4121}
\showcaseopenmojiblack{women holding hands- light skin tone}{4122}
\showcaseopenmojiblack{women holding hands- medium skin tone, dark skin tone}{4123}
\showcaseopenmojiblack{women holding hands- medium skin tone, light skin tone}{4124}
\showcaseopenmojiblack{women holding hands- medium skin tone, medium-dark skin tone}{4125}
\showcaseopenmojiblack{women holding hands- medium skin tone, medium-light skin tone}{4126}
\showcaseopenmojiblack{women holding hands- medium skin tone}{4127}
\showcaseopenmojiblack{women holding hands- medium-dark skin tone, dark skin tone}{4128}
\showcaseopenmojiblack{women holding hands- medium-dark skin tone, light skin tone}{4129}
\showcaseopenmojiblack{women holding hands- medium-dark skin tone, medium skin tone}{4130}
\showcaseopenmojiblack{women holding hands- medium-dark skin tone, medium-light skin tone}{4131}
\showcaseopenmojiblack{women holding hands- medium-dark skin tone}{4132}
\showcaseopenmojiblack{women holding hands- medium-light skin tone, dark skin tone}{4133}
\showcaseopenmojiblack{women holding hands- medium-light skin tone, light skin tone}{4134}
\showcaseopenmojiblack{women holding hands- medium-light skin tone, medium skin tone}{4135}
\showcaseopenmojiblack{women holding hands- medium-light skin tone, medium-dark skin tone}{4136}
\showcaseopenmojiblack{women holding hands- medium-light skin tone}{4137}
\showcaseopenmojiblack{women holding hands}{4138}
\showcaseopenmojiblack{women s room}{4139}
\showcaseopenmojiblack{women with bunny ears}{4140}
\showcaseopenmojiblack{women wrestling}{4141}
\showcaseopenmojiblack{wood}{4142}
\showcaseopenmojiblack{woozy face}{4143}
\showcaseopenmojiblack{world map}{4144}
\showcaseopenmojiblack{worm}{4145}
\showcaseopenmojiblack{worried face}{4146}
\showcaseopenmojiblack{wrapped gift}{4147}
\showcaseopenmojiblack{wrench}{4148}
\showcaseopenmojiblack{writing hand- dark skin tone}{4149}
\showcaseopenmojiblack{writing hand- light skin tone}{4150}
\showcaseopenmojiblack{writing hand- medium skin tone}{4151}
\showcaseopenmojiblack{writing hand- medium-dark skin tone}{4152}
\showcaseopenmojiblack{writing hand- medium-light skin tone}{4153}
\showcaseopenmojiblack{writing hand}{4154}
\showcaseopenmojiblack{x-ray}{4155}
\showcaseopenmojiblack{yarn}{4156}
\showcaseopenmojiblack{yawning face}{4157}
\showcaseopenmojiblack{yellow circle}{4158}
\showcaseopenmojiblack{yellow flag}{4159}
\showcaseopenmojiblack{yellow heart}{4160}
\showcaseopenmojiblack{yellow hexagon}{4161}
\showcaseopenmojiblack{yellow square}{4162}
\showcaseopenmojiblack{yen banknote}{4163}
\showcaseopenmojiblack{yin yang}{4164}
\showcaseopenmojiblack{yo-yo}{4165}
\showcaseopenmojiblack{youtube}{4166}
\showcaseopenmojiblack{zany face}{4167}
\showcaseopenmojiblack{zebra}{4168}
\showcaseopenmojiblack{zipper-mouth face}{4169}
\showcaseopenmojiblack{zombie}{4170}
\showcaseopenmojiblack{zzz}{4171}
\end{openmojicase}

\end{document}