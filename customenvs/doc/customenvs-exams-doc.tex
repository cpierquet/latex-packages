% !TeX TXS-program:compile = txs:///arara
% arara: pdflatex: {shell: no, synctex: no, interaction: batchmode}
% arara: pdflatex: {shell: no, synctex: no, interaction: batchmode} if found('log', '(undefined references|Please rerun|Rerun to get)')

\documentclass[french,11pt,a4paper]{article}
\usepackage{amsmath,amssymb}
\usepackage{kpfonts}
\usepackage{inconsolata}
\usepackage[T1]{fontenc}
\usepackage[utf8]{inputenc}
\usepackage[scale=0.875]{cabin}
\usepackage{customenvs-exams}
\usepackage{soul}
\usepackage{lipsum}
\usepackage{codehigh}
\usepackage{fontawesome5}
\usepackage{twemojis}
\usepackage{fancyvrb}
\usepackage{fancyhdr}
\fancyhf{}
\renewcommand{\headrulewidth}{0pt}
%\rhead{\sffamily\small\affloetalab[Legende]}
\lfoot{\sffamily\small [customenvs-exams]}
\cfoot{\sffamily\small - \thepage{} -}
\rfoot{\hyperlink{matoc}{\small\faArrowAltCircleUp[regular]}}
\usepackage{hologo}
\usepackage{xspace}
\providecommand\tikzlogo{Ti\textit{k}Z}
\providecommand\TeXLive{\TeX{}Live\xspace}
\providecommand\PSTricks{\textsf{PSTricks}\xspace}
\let\pstricks\PSTricks
\let\TikZ\tikzlogo

\usepackage{hyperref}
\urlstyle{same}
\hypersetup{pdfborder=0 0 0}
\usepackage[margin=2cm]{geometry}
\setlength{\parindent}{0pt}

\def\TPversion{0.1.0}
\def\TPdate{06/07/2025}
\usepackage{tcolorbox}

\sethlcolor{lightgray!25}
\NewDocumentCommand\MontreCode{ m }{%
	\hl{\vphantom{\texttt{pf}}\texttt{#1}}%
}

\usepackage{scontents}
\begin{scontents}[overwrite,write-out={automat-specif-pourcent-1.tex}]
\def\AutomatEnonce{%
L'opération qui permet de calculer $25\,\%$ de 480 est :
}

\def\AutomatReponses{%
	{$\dfrac{480}{25 \times 100}$} ,
	{$25 \times 480 \times 0,1$} ,
	{$\dfrac{480 \times 100}{25}$} ,
	{$\dfrac{1}{4} \times 480$}
}
\end{scontents}

\usepackage{babel}

\begin{document}

\pagestyle{fancy}

\thispagestyle{empty}

\begin{center}
	\begin{minipage}{0.88\linewidth}
	\begin{tcolorbox}[colframe=yellow,colback=yellow!15]
		\begin{center}
			\begin{tabular}{c}
				{\Huge \texttt{customenvs-exams}} \\
				\\
				{\LARGE (sous-package de \texttt{customenvs})}\\
				\\
				{\LARGE Commandes spéciales, type 'examens'.} \\
				\\
				{\small \texttt{Version \TPversion{} -- \TPdate}}
		\end{tabular}
		\end{center}
	\end{tcolorbox}
\end{minipage}
\end{center}

\begin{center}
	\begin{tabular}{c}
		\texttt{Cédric Pierquet}\\
		{\ttfamily cpierquet -- at -- outlook . fr}
	\end{tabular}
\end{center}

\begin{center}
	\begin{minipage}{0.85\linewidth}
		\begin{tcolorbox}[colframe=teal,colback=teal!10,halign=center,fontupper=\footnotesize]
			\texttt{\url{https://github.com/cpierquet/latex-packages/tree/main/customenvs}}
			
			~
			
			\texttt{\url{https://forge.apps.education.fr/pierquetcedric/packages-latex}}
		\end{tcolorbox}
	\end{minipage}
\end{center}

\vspace*{1.25cm}

\hrule

\medskip

\hfill\MontreCode{customenvs-exams} est inclus avec \MontreCode{customenvs}, mais peut être chargé indépendamment !\hfill\null

\medskip

\hrule

\vfill

\begin{tcolorbox}[colback=lightgray!2,boxrule=0.1mm]
\begin{TitreSujet}{SUJET}
Sujet d'examen, \today
\end{TitreSujet}

\SousTitreSujetEpreuve{$\leftrightsquigarrow$ Épreuve E2 $\leftrightsquigarrow$}

\begin{ExamExercice}{1}{4}
\begin{ExamAutomatQuest}{1}
L'opération qui permet de calculer $25\,\%$ de 480 est :

\smallskip

\ExamReponsesQCM%
{%
	{$\dfrac{480}{25 \times 100}$} ,
	{$25 \times 480 \times 0,1$} ,
	{$\dfrac{480 \times 100}{25}$} ,
	{$\dfrac{1}{4} \times 480$}
}
\end{ExamAutomatQuest}

\begin{ExamAutomatQuest}{2}
L'opération qui permet de calculer $25\,\%$ de 480 est :

\smallskip

\ExamReponsesQCM[Filets,Largeur=0.5\linewidth,HtAuto,NbCols=2,Labels={1)}]%
{%
	{$\dfrac{480}{25 \times 100}$} ,
	{$25 \times 480 \times 0,1$} ,
	{$\dfrac{480 \times 100}{25}$} ,
	{$\dfrac{1}{4} \times 480$}
}
\end{ExamAutomatQuest}
\end{ExamExercice}
\end{tcolorbox}

\vfill~

\pagebreak

\phantomsection

\hypertarget{matoc}{}

\tableofcontents

\vspace*{5mm}

\hrule

\vspace*{5mm}

\section{Le sous-package customenvs-exams}

\subsection{Idées}

L'idée est de proposer, de manière indépendante à \MontreCode{customenvs}, des commandes/environnements pour travailler sur une présentation type \textit{examen}.

\subsection{Commande de titre(s)}

\begin{codehigh}[language=latex/latex2]
\begin{TitreSujet}[Couleur=...,AlignH=...]<options tcbox>{titre onglet}
...
\end{TitreSujet}
\end{codehigh}

\begin{demohigh}[language=latex/latex2]
\begin{TitreSujet}[Couleur=red!50!black]{SUJET}
Métropole, SIO, 16 Mai 2024
\end{TitreSujet}
\end{demohigh}

\begin{demohigh}[language=latex/latex2]
\begin{TitreSujet}[Couleur=teal,AlignH=center]{CORRIGÉ}
Baccalauréat Centres étrangers Groupe 1\\
14 mars 2023
\end{TitreSujet}
\end{demohigh}

\begin{codehigh}[language=latex/latex2]
\SujetTitreExo[couleur]{titre}
\end{codehigh}

\begin{demohigh}[language=latex/latex2]
\SujetTitreExo{Exercice 4 (5 points)}

\SujetTitreExo[olive]{Exercice 1 [Matrices]\dotfill(5 points)}
\end{demohigh}

\subsection{Environnement (boîte) d'exercice(s)}

\begin{codehigh}[language=latex/latex2]
\begin{ExamExercice}{numéro}{nb points}
...
\end{ExamExercice}
\end{codehigh}

\begin{codehigh}[language=latex/latex2]
%commande pour le formattage du titre (\renewcommand si besoin)
\newcommand\ExamTitreExo[2]{%1=numéro,%2=nb points
  \textbf{\large\examdecoexo\underline{Exercice #1}~(#2 points)}%
}
\end{codehigh}

\begin{demohigh}[language=latex/latex2]
\begin{ExamExercice}{4}{5{,}5}
\lipsum[1]
\end{ExamExercice}
\end{demohigh}

\subsection{Réponses à un QCM}

\begin{codehigh}[language=latex/latex2]
\ExamReponsesQCM%
  [%
    EspacesCL={6pt/2pt},%               espaces H/V des cellules
    NbCols=1/2/4,%                      nombre de colonnes
    Filets=TF,%                         bordures
    PoliceLabels={\bfseries},%
    Labels={a.},%                       formattage des labels (box possible)
    EspaceLabels=\kern5pt,%
    Swap=false,%                        échange b<->c
    Largeur=0.99\linewidth,%
    Melange=false,%
    HtAuto=true,%                       hauteur automatique
  ]%
  {réponseA,réponseB,réponseC,réponseD}%
  <options tblr>
\end{codehigh}

\begin{demohigh}[language=latex/latex2]
\ExamReponsesQCM%
{%
  {$\dfrac{480}{25 \times 100}$} ,
  {$25 \times 480 \times 0,1$} ,
  {$\dfrac{480 \times 100}{25}$} ,
  {$\dfrac{1}{4} \times 480$}
}
\end{demohigh}

\begin{demohigh}[language=latex/latex2]
\ExamReponsesQCM[Filets,Largeur=0.8\linewidth]%
{%
  {$\dfrac{480}{25 \times 100}$} ,
  {$25 \times 480 \times 0,1$} ,
  {$\dfrac{480 \times 100}{25}$} ,
  {$\dfrac{1}{4} \times 480$}
}
\end{demohigh}

\begin{demohigh}[language=latex/latex2]
\ExamReponsesQCM[Filets,NbCols=2]%
{%
  {$\dfrac{480}{25 \times 100}$} ,
  {$25 \times 480 \times 0,1$} ,
  {$\dfrac{480 \times 100}{25}$} ,
  {$\dfrac{1}{4} \times 480$}
}

\ExamReponsesQCM[Filets,NbCols=2,Swap]%
{%
  {$\dfrac{480}{25 \times 100}$} ,
  {$25 \times 480 \times 0,1$} ,
  {$\dfrac{480 \times 100}{25}$} ,
  {$\dfrac{1}{4} \times 480$}
}
\end{demohigh}

\begin{demohigh}[language=latex/latex2]
\ExamReponsesQCM[Filets,NbCols=2,HtAuto]%
{%
  {$\dfrac{480}{25 \times 100}$} ,
  {$25 \times 480 \times 0,1$} ,
  {$\dfrac{480 \times 100}{25}$} ,
  {$\frac{1}{4} \times 480$}
}
\end{demohigh}

\begin{demohigh}[language=latex/latex2]
\ExamReponsesQCM[Filets,Melange,Largeur=0.8\linewidth]%
{%
  {$\dfrac{480}{25 \times 100}$} ,
  {$25 \times 480 \times 0,1$} ,
  {$\dfrac{480 \times 100}{25}$} ,
  {$\dfrac{1}{4} \times 480$}
}
\end{demohigh}

\subsection{Environnement complet 'Automatismes'}

\begin{codehigh}[language=latex/latex2]
%environnement 'Automatismes'
\begin{ExamAutomatQuest}[options tcbox]{numéro}
...
\end{ExamAutomatQuest}
\end{codehigh}

\begin{codehigh}[language=latex/latex2]
%présentation complète 'Automatismes'
\QuestionAutomatismes%
  {numéro}%
  [options tcbox]%
  {Énoncé}%
  [clés QCM]%
  <options tblr QCM>%
  {liste réponses}
\end{codehigh}

\begin{demohigh}[language=latex/latex2]
\begin{ExamAutomatQuest}{1}
L'opération qui permet de calculer $25\,\%$ de 480 est :\\

\medskip

\ExamReponsesQCM%
{%
  {$\dfrac{480}{25 \times 100}$} ,
  {$25 \times 480 \times 0,1$} ,
  {$\dfrac{480 \times 100}{25}$} ,
  {$\dfrac{1}{4} \times 480$}
}
\end{ExamAutomatQuest}
\end{demohigh}

\begin{demohigh}[language=latex/latex2]
\QuestionAutomatismes{4}
{L'opération qui permet de calculer $25\,\%$ de 480 est :}
  [Melange,Largeur=15cm,Filets,NbCols=2,HtAuto,Labels={1)}]
  {%
    {$\dfrac{480}{25 \times 100}$} ,
    {$25 \times 480 \times 0,1$} ,
    {$\dfrac{480 \times 100}{25}$} ,
    {$\dfrac{1}{4} \times 480$}
}
\end{demohigh}

\pagebreak

\subsection{Environnement complet 'Automatismes' via fichier externe}

\begin{codehigh}[language=latex/latex2]
%====automat-specif-pourcent-1.tex
\def\AutomatEnonce{%macro pour l'énoncé (nom fixé)
  L'opération qui permet de calculer $25\,\%$ de 480 est :
}
\def\AutomatReponses{%macro pour les réponses (nom fixé)
  {$\dfrac{480}{25 \times 100}$} ,
  {$25 \times 480 \times 0,1$} ,
  {$\dfrac{480 \times 100}{25}$} ,
  {$\dfrac{1}{4} \times 480$}
}
\end{codehigh}

\begin{demohigh}[language=latex/latex2]
\QuestionAutomatismesFichier{11}
  []                              %param de l'env
  [Melange]                       %param du QCM
  {automat-specif-pourcent-1}     %fichier
\end{demohigh}

\pagebreak

\section{Historique}

\verb|v0.1.0|~:~Version initiale

\end{document}