% !TeX TXS-program:compile = txs:///arara
% arara: pdflatex: {shell: no, synctex: no, interaction: batchmode}

\documentclass[a4paper,french,11pt]{article}
\usepackage{iftex}
\ifluatex
	\usepackage{fontspec}
	\newfontfamily{\qtancienolive}{QTAncientOlive}
	%\renewcommand\sffamily{\qtancienolive}
\else
	\usepackage[T1]{fontenc}
	\usepackage{cabin}
\fi
\usepackage[margin=1cm]{geometry}
\usepackage{babel}
\usepackage{codehigh}
\setlength\parindent{0pt}

\usepackage{framamemes}
\ifluatex
	\setKVdefault[framameme]{Police=\bfseries\qtancienolive}
\fi

\begin{document}

\begin{demohigh}[language=latex/latex2]
\FramaMeme
[%
GrilleAide,
TexteA={FRAMAMÈMES},EchelleA=1.35,
TexteB={MOI},EchelleB=1.5,
TexteC={LES AUTRES \\ GÉNÉRATEURS},
][height=6cm]{petitami}
\end{demohigh}

\begin{demohigh}[language=latex/latex2]
\FramaMeme
[%
TexteA={PARTAGER \\ DES MÈMES},EchelleA=0.62,
TexteB={RESPECTER LE \\ DROIT D'AUTEUR},EchelleB=0.4,
TexteC={MOI AVANT \\ L'OUVERTURE DE \\ FRAMAMÈMES},EchelleC=0.8
][height=8cm]{deuxboutons}
\end{demohigh}

\begin{demohigh}[language=latex/latex2]
\FramaMeme
[%
TexteA={UTILISER \\ UN GÉNÉRATEUR \\ PROPRIÉTAIRE},
TexteB={UTILISER \\ FRAMAMÈMES},
][height=8cm]{aimeaimepas}
\end{demohigh}

\begin{demohigh}[language=latex/latex2]
\FramaMeme
[%
TexteA={MÈMES},EchelleA=2,
TexteB={FRAMAMÈMES},EchelleB=2,
TexteC={LICENCE \\ LIBRE},EchelleC=2,
][height=8cm]{poigneemains}
\end{demohigh}

\begin{demohigh}[language=latex/latex2]
\FramaMeme
[%
TexteA={AUTRES GÉNÉRATEURS \\ DE MÈMES},EchelleA=0.55,
TexteB={FRAMAMÈMES},EchelleB=0.85,
TexteC={MOI},EchelleC=1.5,
][height=8cm]{sortieroute}
\end{demohigh}

\begin{demohigh}[language=latex/latex2]
\FramaMeme
[%
TexteA={JE VAIS PARTAGER \\ UN MÈME !},EchelleA=0.85,
TexteB={EN UTILISANT \\ FRAMAMÈME !},EchelleB=0.85,
TexteC={EN UTILISANT \\ FRAMAMÈME, HEIN ?},EchelleC=0.85,
][height=8cm]{anakinpadme}
\end{demohigh}

\begin{demohigh}[language=latex/latex2]
\FramaMeme
[%
TexteA={UTILISE UN AUTRE \\ GÉNÉRATEUR QUE \\ FRAMAMÈMES},EchelleA=0.5,
TexteB={MOI},EchelleB=2,
][height=8cm]{unopioche}
\end{demohigh}

\pagebreak

\begin{demohigh}[language=latex/latex2]
\def\VignetteA{J'AI DÉMONTRÉ QUE LE \\ TRIANGLE ÉTAIT RECTANGLE \\ AVEC PYTHAGORE}
\def\VignetteB{EN FAISANT DES CALCULS \\ SÉPARÉS ET EN APPLIQUANT \\ LA RÉCIPROQUE DU \\ THÉORÈME DE PYTHAGORE...}
\def\VignetteC{TU AS AU-MOINS \\ FAIT DES CALCULS \\ SÉPARÉS, HEIN ?}
\FramaMeme
[%
TexteA={\VignetteA},EchelleA=1,
TexteB={\VignetteB},EchelleB=1,
TexteC={\VignetteC},EchelleC=1,
][width=10cm]{anakinpadme}
\end{demohigh}

\begin{demohigh}[language=latex/latex2]
\def\VignetteA{J'AI UNE CENTAINE\\DE COPIES À CORRIGER}
\def\VignetteB{OH !\\UNE NOUVELLE\\APPLI SUR LA FORGE !}
\FramaMeme
[%
TexteA={\VignetteA},EchelleA=1,
TexteB={\VignetteB},EchelleB=0.75
][width=10cm]{toutvabien}
\end{demohigh}

\begin{demohigh}[language=latex/latex2]
\def\VignetteA{1. PRENDRE\\UNE IMAGE\\AU HASARD}
\def\VignetteB{2. FAIRE UN\\MÈME TRÈS\\DRÔLE AVEC}
\def\VignetteC{3. SE POSER\\LA QUESTION\\DU DROIT\\D'AUTEUR}
\FramaMeme
[%
TexteA={\VignetteA},EchelleA=0.85,
TexteB={\VignetteB},EchelleB=0.85,
TexteC={\VignetteC},EchelleC=0.85
][width=10cm]{plangru}
\end{demohigh}

\begin{demohigh}[language=latex/latex2]
\def\VignetteA{FRAMAMÈMES}
\def\VignetteB{LES AUTRES\\GÉNÉRATEURS}
\def\VignetteC{LES GENS QUI SE FOUTENT\\DES LICENCES}
\FramaMeme
[%
TexteA={\VignetteA},EchelleA=1.5,
TexteB={\VignetteB},EchelleB=1.5,
TexteC={\VignetteC},EchelleC=1.25
][width=10cm]{memeimage}
\end{demohigh}

\begin{demohigh}[language=latex/latex2]
\def\VignetteA{JE PRÉFÈRE LES\\MÈMES ORIGINAUX…}
\def\VignetteB{LE DÉTOURNEMENT,\\C'EST LA BASE\\DES MÈMES !}
\FramaMeme
[%
TexteA={\VignetteA},EchelleA=1,
TexteB={\VignetteB},EchelleB=1
][width=10cm]{batmanrobin}
\end{demohigh}

\begin{demohigh}[language=latex/latex2]
\def\VignetteA{NOUS AVONS BESOIN D'UNE IMAGE\\VIRALE POUR ILLUSTRER NOTRE ARTICLE.}
\def\VignetteB{ON EN PREND UNE AU HASARD\\SUR GOOGLE IMAGES !}
\def\VignetteC{ON UTILISE UNE IA\\POUR LA GÉNÉRER !}
\def\VignetteD{SINON, ON PEUT UTILISER\\FRAMAMÈMES…}
\FramaMeme
[%
TexteA={\VignetteA},EchelleA=0.75,
TexteB={\VignetteB},EchelleB=0.5,
TexteC={\VignetteC},EchelleC=0.5,
TexteD={\VignetteD},EchelleD=0.5
][width=10cm]{boardroom}
\end{demohigh}
%
%%\pagebreak
%%
%%\begin{demohigh}[language=latex/latex2]
%%\FramaMeme
%%	[%
%%	TexteA={FRAMAMÈMES},EchelleA=1.25,
%%	TexteB={MOI},EchelleB=1.5,
%%	TexteC={LES AUTRES \\ GÉNÉRATEURS},
%%	]
%%	<height=6cm>{petitami}
%%\end{demohigh}
%%%\imagememe[height=6cm]{boyfriend}{distractedboyfriend}

\end{document}