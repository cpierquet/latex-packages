On considère la fonction $g$ définie sur l'intervalle $]0;+\infty[$ par : \[g(x) = \ln \left(x^2\right) + x - 2.\]

\begin{enumerate}
	\item Déterminer les limites de la fonction $g$ aux bornes de son ensemble de définition.
	\item On admet que la fonction $g$ est dérivable sur l'intervalle $]0;+\infty[$.
	
	Étudier les variations de la fonction $g$ sur l'intervalle $]0;+\infty[$.
	\item 
	\begin{enumerate}
		\item Démontrer qu'il existe un unique réel strictement positif $\alpha$ tel que $g(\alpha) = 0$.
		\item Déterminer un encadrement de $\alpha$ d'amplitude $10^{-2}$.
	\end{enumerate}
	\item En déduire le tableau de signe de la fonction $g$ sur l'intervalle $]0;+\infty[$.
\end{enumerate}

\smallskip

\textbf{Partie B}

\medskip

On considère la fonction $f$ définie sur l'intervalle $]0;+\infty[$ par : \[f(x) = \dfrac{(x-2)}{x}\ln(x).\]
%
On note $\mathcal{C}_f$ sa courbe représentative dans un repère orthonormé.

\begin{enumerate}
	\item
	\begin{enumerate}
		\item Déterminer la limite de la fonction $f$ en 0.
		\item Interpréter graphiquement le résultat.
	\end{enumerate}
	\item  Déterminer la limite de la fonction $f$ en $+\infty$.
	\item On admet que la fonction $f$ est dérivable sur l'intervalle $]0;+\infty[$,
	
	Montrer que pour tout réel $x$ strictement positif, on a $f'(x)=\dfrac{g(x)}{x^2}$.
	
	\item En déduire les variations de la fonction $f$ sur l'intervalle $]0;+\infty[$,
\end{enumerate}

\smallskip

\textbf{Partie C}

\medskip

Étudier la position relative de la courbe $\mathcal{C}_f$ et de la courbe représentative de la fonction $\ln$ sur l'intervalle $]0;+\infty[$.
