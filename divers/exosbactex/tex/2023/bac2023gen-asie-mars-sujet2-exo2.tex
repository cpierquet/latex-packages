On considère la fonction $f$ définie sur $\mathbb{R}$ par $f(x)=\ln\big(\text{e}^{2x}-\text{e}^x+1\big)$.  On note $\mathcal{C}_f$ sa courbe 
représentée ci-dessous.

\begin{center}
	\begin{tikzpicture}[x=1.7cm,y=1.7cm,xmin=-3.5,xmax=3.15,xgrille=0.5,xgrilles=0.5,ymin=-2.15,ymax=3.15,ygrille=0.5,ygrilles=0.5]
		\GrilleTikz[Affs=false] \AxesTikz[ElargirOx=0/0,ElargirOy=0/0]
		\AxexTikz[Police=\footnotesize,AffOrigine=false]{-3.5,-3,...,3} 
		\AxeyTikz[Police=\footnotesize,AffOrigine=false]{-2,-1.5,...,2.5}
		\OrigineTikz[Police=\footnotesize]
		\clip (\xmin,\ymin) rectangle (\xmax,\ymax) ;
		\CourbeTikz[very thick,red,samples=250]{ln(exp(2*\x)-exp(\x)+1)}{\xmin:\xmax}
		\draw[red] (1.25,1.75) node[font=\large] {$\mathcal{C}_f$} ;
	\end{tikzpicture}
\end{center}

Un élève formule les conjectures suivantes à partir de cette représentation graphique :

\medskip

\begin{tblr}{vlines,width=\linewidth,colspec={X[l,m]}}
	\hline
	\hspace{3mm}1.~~L'équation $f(x) = 2$ semble admettre au moins une solution. \\
	\hspace{3mm}2.~~Le plus grand intervalle sur lequel la fonction $f$ semble être croissante est $[-0,5;+\infty[$. \\
	\hspace{3mm}3.~~L'équation de la tangente au point d'abscisse $a = 0$ semble être : $y = 1,5x$. \\ \hline
\end{tblr}

\bigskip

Le but de cet exercice est de valider ou rejeter les conjectures concernant la fonction $f$.

\bigskip

\textbf{Partie A : Étude d'une fonction auxiliaire}

\medskip

On définit sur $\mathbb{R}$ la fonction $g$ définie par $g(x) = \text{e}^{2x}-\text{e}^x+1$.

\begin{enumerate}
	\item Déterminer $\displaystyle\lim_{x \to -\infty} g(x)$.
	\item Montrer que $\displaystyle\lim_{x \to +\infty} g(x) = +\infty$.
	\item Montrer que $g'(x)=\text{e}^x \big(2\text{e}^x-1\big)$ pour tout $x \in \mathbb{R}$.
	\item Étudier le sens de variation de la fonction $g$ sur $\mathbb{R}$.
	
	Dresser le tableau des variations de la fonction $g$ en y faisant figurer la valeur exacte des extrema s'il y en a, ainsi que les limites de $g$ en $-\infty$ et $+\infty$.
	\item En déduire le signe de $g$ sur $\mathbb{R}$.
	\item Sans en mener nécessairement les calculs, expliquer comment on pourrait établir le résultat de la question {5.} en posant $X = \text{e}^x$.
\end{enumerate}

\medskip

\textbf{Partie B}

\smallskip

\begin{enumerate}
	\item Justifier que la fonction $f$ est bien définie sur $\mathbb{R}$.
	\item La fonction dérivée de la fonction $f$ est notée $f'$. Justifier que $f'(x) = \frac{g'(x)}{g(x)}$ pour tout $x \in \mathbb{R}$.
	\item Déterminer une équation de la tangente à la courbe au point d'abscisse $0$.
	\item Montrer que la fonction $f$ est strictement croissante sur $[-\ln(2);+\infty[$.
	\item Montrer que l'équation $f(x) = 2$ admet une unique solution $\alpha$ sur $[-\ln(2);+\infty[$ et déterminer une valeur approchée de $\alpha$ à $10^{-2}$ près. 
\end{enumerate}

\medskip

\textbf{Partie C}

\medskip

À l'aide des résultats de la \textbf{partie B}, indiquer, pour chaque conjecture de l'élève, si elle est vraie ou fausse. Justifier. 
