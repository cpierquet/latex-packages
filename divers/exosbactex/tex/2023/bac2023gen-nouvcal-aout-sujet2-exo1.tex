On considère le cube $ABCDEFGH$ d'arête 1 représenté ci-contre.

\begin{wrapstuff}[r,abovesep=-1.5\baselineskip]
	\begin{tikzpicture}[x={(-140:1.35cm)},y={(0:2.43cm)},z={(90:2.25cm)}]
		%placement des points avec labels
		\PlacePointsEspace{A/0,0,0/hd B/0,1,0/bd C/1,1,0/b D/1,0,0/b E/0,0,1/hg F/0,1,1/hd G/1,1,1/hg H/1,0,1/hg K/1,0.5,1/h}
		%segments pointillés
		\TraceSegmentsEspace[thick,dashed]{A/D A/B A/E}
		%segments pleins
		\TraceSegmentsEspace[thick]{D/C C/B B/F F/E E/H H/D H/G F/G G/C}
		%Marques points
		\MarquePointsEspace{A,B,C,D,E,F,G,H,K}
		%Axes
		\draw[thick] (D)--++(0.75,0,0) node[above] {$x$} ;
		\draw[thick] (B)--++(0,0.6,0) node[above left] {$y$} ;
		\draw[thick] (E)--++(0,0,0.5) node[below right] {$z$} ;
	\end{tikzpicture}
\end{wrapstuff}

\smallskip

On note $K$ le milieu du segment $[HG]$.

\smallskip

On se place dans le repère orthonormé $\left(A;\vect{AD},\vect{AB},\vect{AE}\right)$.

\begin{enumerate}
	\item Justifier que les points $C$, $F$ et $K$ définissent un plan.
	\item 
	\begin{enumerate}
		\item Donner, sans justifier, les longueurs $KG$, $GF$ et $GC$.
		\item Calculer l'aire du triangle $FGC$.
		\item Calculer le volume du tétraèdre $FGCK$.
		
		\bigskip
		
		On rappelle que le volume $\mathcal{V}$ d'un tétraèdre est donné par :%
		\[ \mathcal{V}=\frac13 \mathcal{B} \times h, \]%
		où $\mathcal{B}$ est l'aire d'une base et $h$ la hauteur correspondante.\hfill~%
	\end{enumerate}
	\item 
	\begin{enumerate}
		\item On note $\vect{n}$ le vecteur de coordonnées $\begin{pmatrix}1\\2\\1\end{pmatrix}$.
		
		Démontrer que $\vect{n}$ est normal au plan $(CFK)$.
		\item En déduire qu'une équation cartésienne du plan $(CFK)$ est :%
		\[ x+2y+z-3=0. \]
	\end{enumerate}
	\item On note $\Delta$ la droite passant par le point $G$ et orthogonale au plan $(CFK)$.
	
	Démontrer qu'une représentation paramétrique de la droite $\Delta$ est :%
	\[ \begin{dcases} x=1+t\\y=1+2t\\z=1+t\end{dcases} \qquad (t \in \R). \]
	\item Soit $L$ le point d'intersection entre la droite $A$ et le plan $(CFK)$.
	\begin{enumerate}
		\item Déterminer les coordonnées du point $L$.
		\item En déduire que $LG = \dfrac{\sqrt{6}}{6}$.
	\end{enumerate}
	\item En utilisant la question \textbf{2.}, déterminer la valeur exacte de l'aire du triangle $CFK$.
\end{enumerate}
