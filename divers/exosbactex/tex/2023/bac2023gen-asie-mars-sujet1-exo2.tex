On considère le cube $ABCDEFGH$ qui est représenté en ANNEXE.

Dans le repère orthonormé $\left(A;\vect{AB},\vect{AD},\vect{AE}\right)$ on considère les points $M$, $N$ et $P$ de coordonnées :
\[ M\left(1;1;\frac34\right) \text{, } N\left(0;\frac12;1\right) \text{, } P\left(1;0;-\frac54\right). \]
%

Dans cet exercice, on se propose de calculer le volume du tétraèdre $FMNP$.

\begin{enumerate}
	\item Donner les coordonnées des vecteurs $\vect{MN}$ et $\vect{MP}$.
	\item Placer les points $M$, $N$ et $P$ sur la figure donnée en ANNEXE qui sera à rendre avec la copie.
	\item Justifier que les points $M$, $N$ et $P$ ne sont pas alignés.
	
	Des lors les trois points définissent le plan $(MNP)$.
	\item 
	\begin{enumerate}
		\item Calculer le produit scalaire $\vect{MN}\cdot\vect{MP}$, puis en déduire la nature du triangle $MNP$.
		\item Calculer l'aire du triangle $MNP$.
	\end{enumerate}
	\item 
	\begin{enumerate}
		\item Montrer que le vecteur $\vect{n}(5;-8;4)$ est un vecteur normal au plan $(MNP)$.
		\item En déduire qu'une équation cartésienne du plan $(MNP)$ est $5x-8y+4z=0$.
	\end{enumerate}
	\item On rappelle que le point $F$ a pour coordonnées $F(1;0;1)$. Déterminer une représentation paramétrique de la droite $d$ orthogonale au plan $(MNP)$ et passant par le point $F$.
	\item On note $L$ le projeté orthogonal du point $F$ sur le plan $(MNP)$.
	
	Montrer que les coordonnées du point $L$ sont $L\left(\dfrac{4}{7};\dfrac{24}{35};\dfrac{23}{35}\right)$.
	\item Montrer que $FL=\dfrac{3\sqrt{105}}{35}$ puis puis calculer le volume du tétraèdre $FMNP$.
	
	On rappelle que le volume $V$ d'un tétraèdre est donné par la formule : 
	
	\hspace{5mm}$V = \dfrac13  \times \text{aire d'une base} \times \text{hauteur associée à cette base}$.
\end{enumerate}