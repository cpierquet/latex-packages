\emph{Cet exercice est un questionnaire à choix multiple.\\ Pour chaque question, une seule des quatre réponses proposées est exacte. Le candidat indiquera sur sa copie le numéro de la question et la réponse choisie. Aucune justification n'est demandée.\\ Une réponse fausse, une réponse multiple ou l'absence de réponse à une question ne rapporte ni n'enlève de point.\\ Les cinq questions sont indépendantes.}

\bigskip

Une chaîne de fabrication produit des pièces mécaniques. On estime que 4\,\% des pièces produites par cette chaîne sont défectueuses.

\smallskip

On choisit au hasard $n$ pièces produites par la chaîne de fabrication. Le nombre de pièces produites est suffisamment grand pour que ce choix puisse être assimilé à un tirage avec remise. On note $X$ la variable aléatoire égale au nombre de pièces défectueuses tirées.

\medskip

Dans les trois questions suivantes, on prend $n = 50$.

\begin{enumerate}
	\item Quelle est la probabilité, arrondie au millième, de tirer au moins une pièce défectueuse ?
	\begin{enumerate}
		\item $1$
		\item $0,870$
		\item $0,600$
		\item $0,599$
	\end{enumerate}
	\pagebreak
	\item La probabilité $p(3 < X \leqslant 7)$ est égale à :
	\begin{enumerate}
		\item $p(X \leqslant 7) - p(X > 3)$
		\item $p(X \leqslant 7) - p(X \leqslant 3)$
		\item $p(X < 7) - p(X > 3)$
		\item $p(X < 7) - p(X \geqslant 3)$
	\end{enumerate}
	\item Quel est le plus petit entier naturel $k$ tel que la probabilité de tirer au plus $k$ pièces défectueuses soit supérieure ou égale à 95\,\% ?
	\begin{enumerate}
		\item 2
		\item 3
		\item 4
		\item 5
	\end{enumerate}
\end{enumerate}

Dans les questions suivantes, $n$ ne vaut plus nécessairement 50.

\begin{enumerate}[resume]
	\item Quelle est la probabilité de ne tirer que des pièces défectueuses ?
	\begin{enumerate}
		\item $0,04^n$
		\item $0,96^n$
		\item $1-0,04^n$
		\item $1-0,96^n$
	\end{enumerate}
	\item On considère la fonction \textsf{Python} ci-dessous. Que renvoie-t-elle ?
	
\begin{CodePythonLstAlt}*[Largeur=6cm]{center}
def seuil(x) :
	n = 1
	while 1-0.96**n < x :
		n = n+1
	return n
\end{CodePythonLstAlt}
	\begin{enumerate}
		\item Le plus petit nombre $n$ tel que la probabilité de tirer au moins une pièce défectueuse soit supérieure ou égale à $x$.
		\item Le plus petit nombre $n$ tel que la probabilité de ne tirer aucune pièce défectueuse soit supérieure ou égale à $x$.
		\item Le plus grand nombre $n$ tel que la probabilité de ne tirer que des pièces défectueuses soit supérieure ou égale à $x$.
		\item Le plus grand nombre $n$ tel que la probabilité de ne tirer aucune pièce défectueuse soit supérieure ou égale à $x$.
	\end{enumerate}
\end{enumerate}