L’espace est muni d’un repère orthonormé $(O;\vect{\imath},\vect{\jmath},\vect{k})$.

On considère :

\begin{itemize}
	\item le point $A(1;-1;-1)$ ;
	\item le plan $\mathcal{P}_1$ d’équation : $5x + 2y + 4z = 17$ ;
	\item le plan $\mathcal{P}_2$ d’équation : $10x + 14y + 3z = 19$ ;
	\item la droite $\mathcal{D}$ de représentation paramétrique : \[ \begin{dcases} x=1+2t \\ y=-t \\ z=3-2t \end{dcases} \text{ où } t \text{ décrit } \mathbb{R}.  \]
\end{itemize}

\begin{enumerate}
	\item Justifier que les plans $\mathcal{P}_1$ et $\mathcal{P}_2$ ne sont pas parallèles.
	\item Démontrer que $\mathcal{D}$ est la droite d’intersection de $\mathcal{P}_1$ et $\mathcal{P}_2$.
	\item 
	\begin{enumerate}
		\item Vérifier que $A$ n’appartient pas à $\mathcal{P}_1$.
		\item Justifier que $A$ n’appartient pas à $\mathcal{D}$.
	\end{enumerate}
	\item Pour tout réel $t$, on note $M$ le point de $\mathcal{D}$ de coordonnées $(1+2t;-t;3-2t)$.
	
	On considère alors $f$ la fonction qui à tout réel $t$ associe $AM^2$, soit $f(t)=AM^2$.
	\begin{enumerate}
		\item Démontrer que pour tout réel $t$, on a : $f(t)=9t^2-18t+17$.
		\item Démontrer que la distance $AM$ est minimale lorsque $M$ a pour coordonnées $(3;-1;1)$.
	\end{enumerate}
	\item On note $H$ le point de coordonnées $(3;-1;1)$. Démontrer que la droite $(AH)$ est perpendiculaire à $\mathcal{D}$.
\end{enumerate}
