\emph{Cet exercice est un questionnaire à choix multiple.\\ Pour chaque question, une seule des quatre réponses proposées est exacte. Le candidat 
indiquera sur sa copie le numéro de la question et la réponse choisie. Aucune justification n’est demandée. \\ Aucun point n’est enlevé en l’absence de réponse ou en cas de réponse inexacte.}

\medskip

Un jeu vidéo possède une vaste communauté de joueurs en ligne. Avant de débuter une partie, le joueur doit choisir entre deux « mondes » : soit le monde A, soit le monde B.

\smallskip

On choisit au hasard un individu dans la communauté des joueurs.

Lorsqu'il joue une partie, on admet que :

\begin{itemize}
	\item la probabilité que le joueur choisisse le monde A est égale à $\frac25$ ;
	\item si le joueur choisit le monde A, la probabilité qu’il gagne la partie est de $\frac{7}{10}$ ;
	\item la probabilité que le joueur gagne la partie est de $\frac{12}{25}$.
\end{itemize}

On considère les évènements suivants :

\begin{itemize}
	\item $A$ : « Le joueur choisit le monde A » ;
	\item $B$ : « Le joueur choisit le monde B » ;
	\item $G$ : « Le joueur gagne la partie ».
\end{itemize}

\begin{enumerate}
	\item La probabilité que le joueur choisisse le monde A et gagne la partie est égale à : 
	
	\smallskip
	
	\begin{tblr}{width=\linewidth,colspec={X[l,m]X[l,m]X[l,m]X[l,m]}}
		\textbf{a.}~~$\dfrac{7}{10}$ & \textbf{b.}~~$\dfrac{3}{25}$ & \textbf{c.}~~$\dfrac{7}{25}$ & \textbf{d.}~~$\dfrac{24}{125}$
	\end{tblr}
	\item La probabilité $P_B(G)$ de l’événement $G$ sachant que $B$ est réalisé est égale à :
	
	\smallskip
	
	\begin{tblr}{width=\linewidth,colspec={X[l,m]X[l,m]X[l,m]X[l,m]}}
		\textbf{a.}~~$\dfrac{1}{5}$ & \textbf{b.}~~$\dfrac{1}{3}$ & \textbf{c.}~~$\dfrac{7}{15}$ & \textbf{d.}~~$\dfrac{5}{12}$
	\end{tblr}
\end{enumerate}

Dans la suite de l’exercice, un joueur effectue 10 parties successives. On assimile cette situation à un tirage aléatoire avec remise. On rappelle que la probabilité de gagner une partie est de $\frac{12}{25}$.

\begin{enumerate}
	\item La probabilité, arrondie au millième, que le joueur gagne exactement 6 parties est égale à :
	
	\smallskip
	
	\begin{tblr}{width=\linewidth,colspec={X[l,m]X[l,m]X[l,m]X[l,m]}}
		\textbf{a.}~~$0,859$ & \textbf{b.}~~$0,671$ & \textbf{c.}~~$0,188$ & \textbf{d.}~~$0,187$
	\end{tblr}
	\item On considère un entier naturel $n$ pour lequel la probabilité, arrondie au millième, que le joueur gagne au plus $n$ parties est de $0,207$. Alors :
	
	\smallskip
	
	\begin{tblr}{width=\linewidth,colspec={X[l,m]X[l,m]X[l,m]X[l,m]}}
		\textbf{a.}~~$n=2$ & \textbf{b.}~~$n=3$ & \textbf{c.}~~$n=4$ & \textbf{d.}~~$n=5$
	\end{tblr}
	\item La probabilité que le joueur gagne au moins une partie est égale à :
	
	\smallskip
	
	\begin{tblr}{width=\linewidth,colspec={X[l,m]X[l,m]X[l,m]X[l,m]}}
		\textbf{a.}~~$1-\left(\dfrac{12}{25}\right)^{10}$ & \textbf{b.}~~$\left(\dfrac{13}{25}\right)^{10}$ & \textbf{c.}~~$\left(\dfrac{12}{25}\right)^{10}$ & \textbf{d.}~~$1-\left(\dfrac{13}{25}\right)^{10}$
	\end{tblr}
\end{enumerate}
