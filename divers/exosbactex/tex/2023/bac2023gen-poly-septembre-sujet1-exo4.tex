\emph{Cet exercice est un questionnaire à choix multiples.\\ Pour chacune des questions suivantes, une seule des quatre réponses proposées est exacte.\\ Pour répondre, indiquer sur la copie le numéro de la question et la lettre de la réponse choisie.\\ Aucune justification n'est demandée.\\ Une réponse fausse, une absence de réponse, ou une réponse multiple, ne rapporte ni n'enlève de point.}

\medskip

L'espace est muni d'un repère orthonormé $\Rijk$ dans lequel on considère :

\begin{itemize}
	\item les points $A(6;-6;6)$, $B(-6;0;6)$ et $C(-2;-2;11)$ ;
	\item la droite $(d)$ orthogonale aux deux droites sécantes $(AB)$ et $(BC)$ et passant par le point $A$ ;
	\item la droite $(d')$ de représentation paramétrique : \[ \begin{dcases}x=-6-8t\\y=4t\\z=6+5t\end{dcases}\text{, avec } t \in \R.\]
\end{itemize}

\medskip

\textbf{Question 1}

\medskip

Parmi les vecteurs suivants, lequel est un vecteur directeur de la droite $(d)$ ?

\smallskip

\ReponsesQCM[Labels=(a),EspaceLabels={~~~},PoliceLabels={}]{%
	$\Vecteur*{u}[1]  \CoordVecEsp{-6}{3}{0}$ §
	$\Vecteur*{u}[2]  \CoordVecEsp{1}{2}{6}$ §
	$\Vecteur*{u}[3]  \CoordVecEsp{1}{2}{0,2}$ §
	$\Vecteur*{u}[4]  \CoordVecEsp{1}{2}{0}$
}

\medskip

\textbf{Question 2}

\medskip

Parmi les équations suivantes, laquelle est une représentation paramétrique de la droite
$(AB)$ ?

\smallskip

\ReponsesQCM[NbCols=2,Labels=(a),EspaceLabels={~~~},PoliceLabels={},Swap]{%
	$\begin{dcases}x=2t-6\\y=-6\\z=t+6\end{dcases}\text{, avec } t \in \R.$ §
	$\begin{dcases}x=2t-6\\y=-6\\z=-t-6\end{dcases}\text{, avec } t \in \R.$ §
	$\begin{dcases}x=2t+6\\y=-t-6\\z=6\end{dcases}\text{, avec } t \in \R.$ §
	$\begin{dcases}x=2t+6\\y=t-6\\z=6\end{dcases}\text{, avec } t \in \R.$
}

\medskip

\textbf{Question 3}

\medskip

Un vecteur directeur de la droite $(d')$ est :

\smallskip

\ReponsesQCM[Labels=(a),EspaceLabels={~~~},PoliceLabels={}]{%
	$\Vecteur*{v}[1]  \CoordVecEsp{-6}{0}{6}$ §
	$\Vecteur*{v}[2]  \CoordVecEsp{-14}{4}{11}$ §
	$\Vecteur*{v}[3]  \CoordVecEsp{8}{-4}{-5}$ §
	$\Vecteur*{v}[4]  \CoordVecEsp{8}{-4}{5}$
}

\medskip

\textbf{Question 4}

\medskip

Lequel des quatre points suivants appartient à la droite $(d')$ ?

\smallskip

\ReponsesQCM[Labels=(a),EspaceLabels={~~~},PoliceLabels={},NbCols=2,Swap]{%
	$M_1(50;-28;-29)$ §
	$M_2(-14;-4;11)$ §
	$M_3(2;-4;-1)$ §
	$M_4(-3;0;3)$
}

\medskip

\textbf{Question 5}

\medskip

Le plan d'équation $x=1$ a pour vecteur normal :

\smallskip

\ReponsesQCM[Labels=(a),EspaceLabels={~~~},PoliceLabels={}]{%
	$\Vecteur*{n}[1] \CoordVecEsp{1}{0}{0}$ §
	$\Vecteur*{n}[2] \CoordVecEsp{0}{1}{1}$ §
	$\Vecteur*{n}[3] \CoordVecEsp{0}{1}{0}$ §
	$\Vecteur*{n}[4] \CoordVecEsp{1}{0}{1}$
}