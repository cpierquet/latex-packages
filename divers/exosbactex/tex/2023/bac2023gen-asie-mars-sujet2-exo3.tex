\emph{Marie Sklodowska-Curie (1867-1934) est une physicienne (mais aussi chimiste et mathématicienne), polonaise naturalisée française. Deux Prix Nobel lui ont été décernés : un en Physique (partagé avec son mari et Henri Becquerel) en 1903 et un en Chimie en 1911 pour la découverte de deux nouveaux éléments, le polonium (nom donné en hommage à ses origines) et le radium.}

\medskip

On décide d'étudier le rayonnement radioactif du polonium lors de la désintégration des noyaux atomiques au cours du temps.

\smallskip

Au début de l'expérience, on dispose d'un morceau de 2~g de polonium.

On sait que 1~g de polonium contient $3 \times 10^{21}$ noyaux atomiques.

On admet que, au bout de 24 heures, 0,5\,\% des noyaux se sont désintégrés et que, pour compenser cette disparition, on ajoute alors 0,005~g de polonium. 

\smallskip

On modélise la situation à l'aide d'une suite  $\left(v_n\right)_{n \in \mathbb{N}}$ ; on note $v_0$ le nombre de noyaux contenus dans le polonium au début de l'expérience. Pour $n \geqslant 1$, $v_n$ désigne le nombre de noyaux contenus dans le polonium au bout de $n$ jours écoulés.

\begin{enumerate}
	\item 
	\begin{enumerate}
		\item Vérifier que $v_0 = 6 \times 10^{21}$.
		\item Expliquer que, pour tout nombre entier naturel $n$, on a $v_{n+1} = 0,995v_n + 1,5 \times 10^{19}$.
	\end{enumerate}
	\item 
	\begin{enumerate}
		\item Démontrer, par récurrence sur $n$, que $0 \leqslant v_{n+1} \leqslant v_n$.
		\item En déduire que la suite $\left(v_n\right)_{n \in \mathbb{N}}$ est convergente.
	\end{enumerate}
	\item On considère la suite $\left(u_n\right)_{n \in \mathbb{N}}$ définie, pour tout entier naturel $n$, par : $u_n = v_n - 3 \times 10^{21}$.
	\begin{enumerate}
		\item Montrer que la suite $\left(u_n\right)_{n \in \mathbb{N}}$ est géométrique de raison $0,995$.
		\item En déduire que, pour tout entier naturel $n$, $v_n = 3 \times 10^{21} \big(0,0995^n+1\big)$.
		\item En déduire la limite de la suite $\left(v_n\right)_{n \in \mathbb{N}}$ et interpréter le résultat dans le contexte de l'exercice.
	\end{enumerate}
	\item Déterminer, par le calcul, au bout de combien de jours le nombre de noyaux de polonium sera
	inférieur à $4,5 \times 10^{21}$. Justifier la réponse.
	\item On souhaite disposer de la liste des termes de la suite $\left(v_n\right)_{n \in \mathbb{N}}$.
	
	Pour cela, on utilise une fonction appelée \texttt{noyaux} programmée en langage \textsf{Python} et retranscrite partiellement ci-après.

\begin{CodePythonLstAlt}[Largeur=10cm]{center}
def noyaux(n) :
	V = 6*10**21
	L = [V]
	for k in range(n) :
		V = ......
		L.append(V)
	return L
\end{CodePythonLstAlt}
	\begin{enumerate}
		\item À la lecture des questions précédentes, proposer deux solutions différentes pour compléter
		la \texttt{ligne 5} de la fonction noyaux afin qu'elle réponde au problème.
		\item Pour quelle valeur de l'entier \texttt{n} la commande \texttt{noyaux(n)} renverra-t-elle les relevés quotidiens du nombre de noyaux contenus dans l'échantillon de polonium pendant 52 semaines d'étude? 
	\end{enumerate}
\end{enumerate}
