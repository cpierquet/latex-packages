Pour chacune des affirmations suivantes, indiquer si elle est vraie ou fausse. 

Chaque réponse doit être justifiée. 

Une réponse non justifiée ne rapporte aucun point.

\begin{enumerate}
	\item \textbf{Affirmation :} La fonction $f$ définie sur $\mathbb{R}$ par $f(x) = \text{e}^x - x$ est convexe.
	\item \textbf{Affirmation :} L'équation $\left(2\text{e}^x - 6\right)\left(\text{e}^x + 2\right) = 0$ admet $\ln (3)$ comme unique solution dans $\mathbb{R}$.
	\item \textbf{Affirmation :} \[\displaystyle\lim_{x \to + \infty} \dfrac{\text{e}^{2x} - 1}{\text{e}^x - x} = 0.\]
	\item Soit $f$ la fonction définie sur $\mathbb{R}$ par $f(x) = (6x + 5)\text{e}^{3x}$ et soit $F$ la fonction définie sur $\mathbb{R}$ par :
	
	$F(x) = (2x + 1)\text{e}^{3x} + 4$.
	
	\textbf{Affirmation :} $F$ est la primitive de $f$ sur $\mathbb{R}$ qui prend la valeur 5 quand $x = 0$.
	\item On considère la fonction \texttt{mystere} définie ci-dessous qui prend une liste \texttt{L} de nombres en paramètre.
	
	On rappelle que \texttt{len(L)} représente la longueur de la liste \texttt{L}.
	
\begin{CodePythonLst}*[10cm]{center}
def mystere(L) :
	S = 0
	for i in range(len(L))
		S = S + L[i]
	return S / len(L)
\end{CodePythonLst}

	\textbf{Affirmation :} L'exécution de \texttt{mystere([1,9,9,5,0,3,6,12,0,5])} renvoie \texttt{50}.
\end{enumerate}
