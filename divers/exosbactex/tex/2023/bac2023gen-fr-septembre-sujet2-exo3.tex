On considère la suite $\suiten$ définie par : \[ \begin{dcases} u_1 = \frac{1}{\e} \\ u_{n+1} = \frac{1}{\e} \left(1+\frac1n\right)u_n \text{ pour tout entier } n \geqslant 1. \end{dcases} \]

\begin{enumerate}
	\item Calculer les valeurs exactes de $u_2$ et $u_3$. On détaillera les calculs.
	\item On considère une fonction écrite en langage \textsf{Python} qui, pour un entier naturel $n$ donné, affiche le terme $u_n$. Compléter les lignes \textsf{L2} et \textsf{L4} de ce programme.

\begin{CodePythonLstAlt}[Largeur=8cm]{center}
def suite(n):
	..................
	for i in range(1, n) :
		u = ..................
	return u
\end{CodePythonLstAlt}
	\item On admet que tous les termes de la suite $\suiten$ sont strictement positifs.
	\begin{enumerate}
		\item Montrer que pour tout entier naturel $n$ non nul, on a : $1+\dfrac1n \leqslant \e$.
		\item En déduire que la suite $\suiten$ est décroissante.
		\item La suite $\suiten$ est-elle convergente ? Justifier votre réponse.
	\end{enumerate}
	\item 
	\begin{enumerate}
		\item Montrer par récurrence que pour tout entier naturel non nul, on a : $u_n = \dfrac{n}{\e^n}$.
		\item En déduire, si elle existe, la limite de la suite $\suiten$.
	\end{enumerate}
\end{enumerate}
