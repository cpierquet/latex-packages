\emph{Cet exercice est un questionnaire à choix multiples. Pour chaque question, une seule des quatre propositions est exacte. Indiquer sur la copie le numéro de la question et la lettre de la proposition choisie. Aucune justification n'est demandée.\\
Pour chaque question, une réponse exacte rapporte un point. Une réponse fausse, une réponse multiple ou l'absence de réponse ne rapporte ni n'enlève de point.}

\bigskip

\textbf{Question 1 :}

\smallskip

Soit $f$ la fonction définie sur $\mathbb{R}$ par $f(x)=x\,\text{e}^x$. Une primitive $F$ sur $\mathbb{R}$ de la fonction $f$ est définie par :

\medskip

\begin{tblr}{width=\linewidth,colspec={X[l,m]X[l,m]}}%X[l,m]X[l,m]}}
	(a)~~$F(x)=\frac{x^2}{2}\text{e}^x$ & (b)~~$F(x)=(x-1)\text{e}^x$ \\
	(c)~~$F(x)=(x+1)\text{e}^x$ & (d)~~$F(x)=\frac12x\,\text{e}^{x^2}$
\end{tblr}

\bigskip

\textbf{Question 2 :}

\medskip

La courbe $\mathcal{C}$ ci-dessous représente une fonction $f$ définie et deux fois dérivable sur $]0;+\infty[$.

On sait que :

\begin{itemize}
	\item le maximum de la fonction $f$ est atteint au point d'abscisse 3 ;
	\item le point $P$ d'abscisse 5 est l'unique point d'inflexion de la courbe $\mathcal{C}$.
\end{itemize}

\begin{center}
	\begin{tikzpicture}[x=0.5cm,y=0.5cm,xmin=-1,xmax=14,xgrille=1,xgrilles=0.2,ymin=-3,ymax=6,ygrille=1,ygrilles=0.2]
		\GrilleTikz[][very thin,gray][very thin,lightgray]
		\AxesTikz[Epaisseur=0.75pt,ElargirOx=0/0,ElargirOy=0/0]
		\OrigineTikz[Police=\tiny,Decal=0pt]
		\AxexTikz[Epaisseur=0.75pt,AffOrigine=false,Police=\tiny,HautGrad=3pt]{-1,0,...,13}
		\AxeyTikz[Epaisseur=0.75pt,AffOrigine=false,Police=\tiny,HautGrad=3pt]{-3,-2,...,5}
		\FenetreTikz
		\CourbeTikz[thick,red,samples=500]{10*(\x-1)*exp(-0.5*\x)}{0:14}
		\draw[<->,>=latex,CouleurVertForet,thick] ({2.25},{(10*3-10)*exp(-0.5*3)}) -- ({3.75},{(10*3-10)*exp(-0.5*3)}) ;
		\draw[red] (10,1.25) node[below right=0pt,font=\scriptsize] {$y=f(x)$} ;
		\CourbeTikz[thick,densely dashed,blue,samples=2]{-10*exp(-2.5)*(\x-5)+40*exp(-2.5)}{0:14}
		\filldraw (5,{(10*5-10)*exp(-0.5*5)}) circle[radius=1.5pt] node[above right,font=\scriptsize] {$P$} ;
	\end{tikzpicture}
\end{center}

On a :

\begin{tblr}{width=\linewidth,colspec={X[l,m]}}
	(a)~~pour tout réel $x \in ]0;5[$, $f(x)$ et $f'(x)$ sont de même signe \\
	(b)~~pour tout réel $x \in ]5;+\infty[$, $f(x)$ et $f'(x)$ sont de même signe \\
	(c)~~pour tout réel $x \in ]0;5[$, $f'(x)$ et $f''(x)$ sont de même signe \\
	(d)~~pour tout réel $x \in ]5;+\infty[$, $f(x)$ et $f''(x)$ sont de même signe
\end{tblr}

\bigskip

\textbf{Question 3 :}

\medskip

On considère la fonction $g$ définie sur $[0;+\infty[$ par $g(t)=\frac{a}{b+\text{e}^{-t}}$ où $a$ et $b$ sont deux nombres réels.

On sait que $g(0)=2$ et $\displaystyle\lim_{t \to +\infty} g(t)=3$. Les valeurs de $a$ et $b$ sont :

\medskip

\begin{tblr}{width=\linewidth,colspec={X[l,m]X[l,m]}}
	(a)~~$a=2$ et $b=3$ & (b)~~$a=4$ et $b=\frac43$ \\
	(c)~~$a=4$ et $b=1$ & (d)~~$a=6$ et $b=2$
\end{tblr}

\bigskip

\textbf{Question 4 :}

\medskip

Alice dispose de deux urnes A et B contenant chacune quatre boules indiscernables au toucher. L'urne A contient deux boules vertes et deux boules rouges. L'urne B contient trois boules vertes et une boule rouge.

\smallskip

Alice choisit au hasard une urne puis une boule dans cette urne. Elle obtient une boule verte. La probabilité qu'elle ait choisi l'urne B est :

\medskip

\begin{tblr}{width=\linewidth,colspec={X[l,m]X[l,m]}}
	(a)~~$\frac38$ & (b)~~$\frac12$ \\
	(c)~~$\frac35$ & (d)~~$\frac58$ \\
\end{tblr}

\bigskip

\textbf{Question 5 :}

\medskip

On pose $S=1+\frac12+\frac13+\frac14+\ldots+\frac{1}{100}$.

\smallskip

Parmi les scripts \textsf{Python} ci-dessous, celui est permet de calculer la somme $S$ est :

\begin{tblr}{width=\linewidth,colspec={X[l,m]X[l,m]}}
	(a)~~ & (b)~~
\end{tblr}

\begin{minipage}{0.5\linewidth}
\begin{CodePythonLstAlt}[Largeur=8cm]{}
def somme_a() :
	S = 0
	for k in range(100) :
		S = 1/(k+1)
	return S
\end{CodePythonLstAlt}
\end{minipage}
\begin{minipage}{0.5\linewidth}
\begin{CodePythonLstAlt}[Largeur=8cm]{}
def somme_b() :
	S = 0
	for k in range(100) :
		S = S + 1/(k+1)
	return S
\end{CodePythonLstAlt}
\end{minipage}

\begin{tblr}{width=\linewidth,colspec={X[l,m]X[l,m]}}
	(c)~~ & (d)~~
\end{tblr}

\begin{minipage}{0.5\linewidth}
\begin{CodePythonLstAlt}[Largeur=8cm]{}
def somme_c() :
	k = 0
	while S < 100 :
		S = S + 1/(k+1)
	return S
\end{CodePythonLstAlt}
\end{minipage}
\begin{minipage}{0.5\linewidth}
\begin{CodePythonLstAlt}[Largeur=8cm]{}
def somme_d() :
	k = 0
	while k < 100 :
		S = S + 1/(k+1)
	return S
\end{CodePythonLstAlt}
\end{minipage}
