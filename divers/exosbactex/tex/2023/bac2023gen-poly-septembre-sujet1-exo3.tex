On considère la fonction $f$ définie sur $\R$ par \[ f(x)=\dfrac34x^2-2x+3. \]

\begin{enumerate}
	\item Dresser le tableau de variations de $f$ sur $\R$.
	\item En déduire, que pour tout $x$ appartenant à l'intervalle $\IntervalleFF{\frac43}{2}$, $f(x)$ appartient à l'intervalle $\IntervalleFF{\frac43}{2}$.
	\item Démontrer que pour tout $x$ réel, $x \leqslant f(x)$.
	
	Pour cela, on pourra démontrer que pour tout réel $x$ : \[ f(x)-x = \dfrac34(x-2)^2. \]
\end{enumerate}

On considère la suite $\Suite{u}$ définie par un réel $u_0$ et pour tout entier naturel $n$ : \[ u_{n+1} = f\big(u_n\big).\]

On a donc, pour tout entier naturel $n$, $u_{n+1} = \dfrac34u_n^2-2u_n+3.$

\begin{enumerate}[resume]
	\item Étude du cas : $\frac34 \leqslant u_0 \leqslant 2$.
	\begin{enumerate}
		\item Démontrer par récurrence que, pour tout entier naturel $n$, \[ u_{n+1} \leqslant u_n \leqslant 2. \]
		\item En déduire que la suite $\Suite{u}$ est convergente.
		\item Prouver que la limite de la suite est égale à 2.
	\end{enumerate}
	\item Étude du cas particulier $u_0=3$.
	
	On admet que dans ce cas la suite $\Suite{u}$ tend vers $+\infty$.
	
	Recopier et compléter la fonction « \texttt{seuil} » suivante écrite en \textsf{Python}, afin qu'elle renvoie la plus petite valeur de $n$ telle que $u_n$ soit supérieur ou égal à 100.
	
\begin{CodePythonLstAlt}*[Largeur=7cm]{center}
def seuil() :
	u = 3
	n = 0
	while ......... :
		u = .........
		n = .........
	return n
\end{CodePythonLstAlt}
	\item Étude du cas : $u_0 > 2$.
	
	À l'aide d'un raisonnement par l'absurde, montrer que $\Suite{u}$ n'est pas convergente.
\end{enumerate}
