\emph{Cet exercice est un questionnaire à choix multiple.\\
	Pour chaque question, une seule des quatre réponses proposées est exacte. Le candidat indiquera sur sa copie le numéro de la question et la réponse choisie. Aucune justification n'est demandée.\\
	Une réponse fausse, une réponse multiple ou l'absence de réponse à une question ne rapporte ni n'enlève de point.\\
	Les cinq questions sont indépendantes.}

\medskip

L'espace est muni d'un repère orthonormé $\Rijk$.

\smallskip

On considère les points $A(-1;2;5)$, $B(3;6;3)$, $C(3;0;9)$ et $D(8;-3;-8)$.

On admet que les points $A$, $B$ et $C$ ne sont pas alignés.

\begin{enumerate}
	\item $ABC$ est un triangle :
	
	\medskip
	
	\begin{tblr}{width=\linewidth,colspec={*{2}{X[m,l]}}}
		(a)~~isocèle rectangle en A & (b)~~isocèle rectangle en B\\
		(c)~~isocèle rectangle en C & (d)~~équilatéral\\
	\end{tblr}

	\item Une équation cartésienne du plan $(BCD)$ est :
	
	\medskip
	
	\begin{tblr}{width=\linewidth,colspec={*{2}{X[m,l]}}}
		(a)~~$2 x+y+z-15=0$ & (b)~~$9 x-5 y+3=0$\\
		(c)~~$4 x+y+z-21=0$ & (d)~~$11 x+5 z-73=0$\\
	\end{tblr}
	
	\item On admet que le plan $(ABC)$ a pour équation cartésienne $x- 2y - 2z + 15 = 0$.
	
	On appelle $H$ le projeté orthogonal du point $D$ sur le plan $(ABC)$.
	
	On peut affirmer que :
	
	\medskip
	
	\begin{tblr}{width=\linewidth,colspec={*{2}{X[m,l]}}}
		(a)~~$H(-2;17;12)$ & (b)~~$H(3;7;2)$\\
		(c)~~$H(3;2;7)$& (d)~~$H(-15;1;-1)$\\
	\end{tblr}
	\item Soit la droite $\Delta$ de représentation paramétrique $\begin{dcases}x=\phantom{-}5+t \\ y=\phantom{-}3-t \\ z =-1 + 3t \end{dcases}$, avec $t$ réel.
	
	Les droites $(BC)$ et $\Delta$ sont :
	
	\medskip
	
	\begin{tblr}{width=\linewidth,colspec={*{2}{X[m,l]}}}
		(a)~~confondues & (b)~~strictement parallèles\\
		(c)~~sécantes & (d)~~non coplanaires\\
	\end{tblr}
	\item On considère le plan $\mathcal{P}$ d'équation cartésienne $2x - y + 2z - 6 = 0$.
	
	On admet que le plan $(ABC)$ a pour équation cartésienne $x-2 y-2 z+15=0$.
	
	On peut affirmer que :
	
	\medskip
	
	\begin{tblr}{width=\linewidth,colspec={*{1}{X[m,l]}}}
		(a)~~les plans $\mathcal{P}$ et $(ABC)$ sont strictement parallèles \\
		(b)~~les plans $\mathcal{P}$ et $(ABC)$ sont sécants et leur intersection est la droite (AB)\\
		(c)~~les plans $\mathcal{P}$ et $(ABC)$ sont sécants et leur intersection est la droite (AC)\\
		(d)~~les plans $\mathcal{P}$ et $(ABC)$ sont sécants et leur intersection est la droite (BC)\\
	\end{tblr}
\end{enumerate}