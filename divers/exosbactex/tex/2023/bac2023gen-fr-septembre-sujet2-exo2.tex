On considère la fonction $f$ définie sur l’intervalle $\intervOO{0}{+\infty}$ par : \[ f(x)=(2-\ln(x)) \times \ln(x), \]%
où $\ln$ désigne la fonction logarithme népérien.

On admet que la fonction $f$ est deux fois dérivable sur $\intervOO{0}{+\infty}$.

On note $\mathcal{C}$ la courbe représentative de la fonction $f$ dans un repère orthogonal et $\mathcal{C}'$ la courbe représentative de la fonction $f'$, fonction dérivée de la fonction $f$.

La \textbf{courbe} $\bm{\mathcal{C}_f}$ est donnée ci-dessous ainsi que son unique tangente horizontale $(T)$.

\begin{center}
	\begin{tikzpicture}[xmin=0,xmax=17.5,x=0.75cm,xgrille=1,ymin=-0.38,ymax=2.2,y=3.75cm,ygrille=0.2]
		\GrilleTikz[Affs=false] \AxesTikz[ElargirOx=0,ElargirOy=0]
		\AxexTikz[AffOrigine=false]{0,1,...,17}
		\AxeyTikz{-0.2,0,0.2,0.4,0.6,0.8,1,1.2,1.4,1.6,1.8,2}
		\FenetreTikz
		\CourbeTikz[red,line width=1.25pt,samples=500]{(2-2*ln(\x))/(\x)}{0.75:17.5}
		\draw[teal,line width=1.25pt,densely dashed] ({exp(2)},0)--++ (0,{2*(1-ln(exp(2)))/exp(2)}) ;
		\draw[teal,line width=1.25pt,densely dashed] (0,{2*(1-ln(exp(2)))/exp(2)}) --++(\xmax,0) ;
		\draw[teal] (1.15,{2*(1-ln(exp(2)))/exp(2)}) node[below left] {$(T)$} ;
		\draw[fill=white] (3,1.4) rectangle (16,2) node[midway,text width=\fpeval{13*0.75}cm,align=center] {On rappelle que \textbf{cette courbe $\bm{\mathcal{C}'}$ est la courbe représentative de la fonction dérivée $\bm{f'}$}} ;
	\end{tikzpicture}
\end{center}

\begin{enumerate}
	\item Par lecture graphique, avec la précision que permet le tracé ci-dessus, donner :
	\begin{enumerate}
		\item le coefficient directeur de la tangente à $\mathcal{C}$ au point d’abscisse 1 ;
		\item le plus grand intervalle sur lequel la fonction f est convexe.
	\end{enumerate}
	\item 
	\begin{enumerate}
		\item Calculer la limite de la fonction $f$ en $+\infty$.
		\item Calculer $\lim\limits_{x \to 0} f(x)$. Interpréter graphiquement ce résultat.
	\end{enumerate}
	\item Montrer que la courbe $\mathcal{C}$ coupe l’axe des abscisses en deux points exactement dont on précisera les coordonnées.
	\item 
	\begin{enumerate}
		\item Montrer que pour tout réel $x$ appartenant à $\intervOO{0}{+\infty}$, $f'(x)=\dfrac{2(1-\ln(x))}{x}$.
		\item En déduire, en justifiant, le tableau de variations de la fonction $f$ sur  $\intervOO{0}{+\infty}$.
	\end{enumerate}
	\item On note $f''$ la dérivée seconde de $f$ et on admet que pour tout réel $x$ appartenant à $\intervOO{0}{+\infty}$, $f''(x)=\dfrac{2(\ln(x)-2)}{x^2}$.
	
	Déterminer par le calcul le plus grand intervalle sur lequel la fonction $f$ est convexe et préciser les coordonnées du point d’inflexion de la courbe $\mathcal{C}$.
\end{enumerate}
