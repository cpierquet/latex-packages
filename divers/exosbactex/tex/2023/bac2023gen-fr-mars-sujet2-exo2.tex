Des biologistes étudient l’évolution d’une population d’insectes dans un jardin botanique.

Au début de l’étude la population est de \num{100000} insectes.

Pour préserver l’équilibre du milieu naturel le nombre d’insectes ne doit pas dépasser \num{400000}.

\bigskip

\textbf{Partie A : Étude d’un premier modèle en laboratoire}

\medskip

L’observation de l’évolution de ces populations d’insectes en laboratoire, en l’absence de tout prédateur, montre que le nombre d’insectes augmente de $60$\,\% chaque mois.

En tenant compte de cette observation, les biologistes modélisent l’évolution de la population d’insectes à l’aide d’une suite $\left(u_n\right)$ où, pour tout entier naturel $n$, $u_n$ modélise le nombre d’insectes, exprimé en millions, au bout de $n$ mois. On a donc $u_0 = 0,1$.

\begin{enumerate}
	\item Justifier que pour tout entier naturel $n$ : $u_n = 0,1 \times 1,6^n$.
	\item Déterminer la limite de la suite $\left(u_n\right)$.
	\item En résolvant une inéquation, déterminer le plus petit entier naturel $n$ à partir duquel $u_n > 0,4$.
	\item Selon ce modèle, l’équilibre du milieu naturel serait-il préservé ? Justifier la réponse.
\end{enumerate}

\smallskip

\textbf{Partie B : Étude d’un second modèle}

\medskip

En tenant compte des contraintes du milieu naturel dans lequel évoluent les insectes, les biologistes choisissent une nouvelle modélisation.

Ils modélisent le nombre d’insectes à l’aide de la suite $\left(v_n\right)$, définie par : $v_0=0,1$ et, pour tout 
entier naturel $n$, $v_{n+1}=1,6v_n-1,6v_n^2$, où, pour tout entier naturel $n$, $v_n$ est le nombre d’insectes, exprimé en millions, au bout de $n$ mois.

\begin{enumerate}
	\item Déterminer le nombre d’insectes au bout d’un mois.
	\item On considère la fonction $f$ définie sur l’intervalle $\left[0;\frac12\right]$ par $f(x)=1,6x-1,6x^2$.
	\begin{enumerate}
		\item Résoudre l’équation $f(x)=x$.
		\item Montrer que la fonction $f$ est croissante sur l'intervalle $\left[0;\frac12\right]$.
	\end{enumerate}
	\item 
	\begin{enumerate}
		\item Montrer par récurrence que, pour tout entier naturel $n$, $0 \leqslant v_n \leqslant v_{n+1} \leqslant \frac12$.
		\item Montrer que la suite $\left(v_n\right)$ est convergente.
		
		\smallskip
		
		On note $\ell$ la valeur de sa limite. On admet que $\ell$ est solution de l’équation $f(x)=x$.
		\item Déterminer la valeur de $\ell$. Selon ce modèle, l’équilibre du milieu naturel sera-t-il préservé ? Justifier la réponse.
	\end{enumerate}
	\item On donne ci-dessous la fonction \texttt{seuil}, écrite en langage \textsf{Python}.

\begin{CodePythonLstAlt}*[Largeur=5.5cm]{center}
def seuil(a) :
	v = 0.1
	n = 0
	while v < a :
		v = 1.6*v - 1.6*v*v
		n = n+1
	return n
\end{CodePythonLstAlt}
	\begin{enumerate}
		\item Qu’observe-t-on si on saisit \texttt{seuil(0.4)} ?
		\item Déterminer la valeur renvoyée par la saisie de \texttt{seuil(0.4)}.
		
		Interpréter cette valeur dans le contexte de l’exercice. 
	\end{enumerate}
\end{enumerate}
