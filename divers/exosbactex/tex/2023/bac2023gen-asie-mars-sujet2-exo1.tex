On considère deux cubes $ABCDEFGH$ et $BKLCFJMG$ positionnés comme sur la figure suivante :

\begin{center}
	\begin{tikzpicture}[x={(-19:3cm)},y={(14:4cm)},z={(90:3.8cm)},line join=bevel]
		\coordinate (A) at (0,0,0) ; \coordinate (B) at (1,0,0) ;
		\coordinate (C) at (1,1,0) ; \coordinate (D) at (0,1,0) ;
		\coordinate (E) at (0,0,1) ; \coordinate (F) at (1,0,1) ;
		\coordinate (G) at (1,1,1) ; \coordinate (H) at (0,1,1) ;
		\coordinate (K) at (2,0,0) ; \coordinate (L) at (2,1,0) ;
		\coordinate (M) at (2,1,1) ; \coordinate (J) at (2,0,1) ;
		\coordinate (I) at ($(E)!0.5!(F)$) ;
		\draw[semithick,dashed] (D)--(H) (D)--(L) (B)--(C)--(G)--cycle ;
		\draw[thick,darkgray,dashed,->,>-latex] (A)--(D) ;
		\draw[thick,darkgray,->,>-latex] (A)--(B) (A)--(E) ;
		\draw[semithick] (A)--(B)--(F)--(E)--cycle  (B)--(K)--(J)--(F)--cycle (K)--(L)--(M)--(J)--cycle (E)--(F)--(G)--(H)--cycle (F)--(J)--(M)--(G)--cycle (B)--(I)--(G);
		\foreach \point/\pos in {A/below left,B/below,C/above right,D/above left,E/left,F/below left,G/above right,H/above,I/above left,J/below right,K/below,L/right,M/above right}
			{\draw (\point) node[\pos,font=\large] {$\point$} ;}
	\end{tikzpicture}
\end{center}

Le point $I$ est le milieu de $[EF]$.

Dans toute la suite de l'exercice, on se place dans le repère orthonormé $\left(A;\vect{AB},\vect{AD},\vect{AE}\right)$.

Ainsi, par exemple, les points $F$, $G$ et $J$ ont pour coordonnées $F(1;0;1)$, $G(1;1;1)$ et $J(2;0;1)$.

\begin{enumerate}
	\item Montrer que le volume du tétraèdre $FIGB$ est égal à $\frac{1}{12}$ d'unité de volume.
	
	\smallskip
	
	On On rappelle que le volume $\mathcal{V}$ d'un tétraèdre est donné par la formule : \[ \mathcal{V}=\frac13 \times \text{aire d'une base} \times \text{hauteur correspondante}. \]
	\item Déterminer les coordonnées du point $I$.
	\item Montrer que le vecteur $\vect{DJ}$ un vecteur normal au plan $(BIG)$.
	\item Montrer qu'une équation cartésienne du plan $(BIG)$ est $2x-y+z-2=0$.
	\item Déterminer une représentation paramétrique de la droite $d$, orthogonale à $(BIG)$ et passant par $F$.
	\item 
	\begin{enumerate}
		\item La droite $d$ coupe le plan $(BIG)$ au point $P$.
		
		Montrer que les coordonnées du point $P$ sont $\left(\dfrac23;\dfrac16;\dfrac56\right)$.
		\item Calculer la longueur $FP$.
		\item Déduire des questions précédentes l'aire du triangle $IGB$.
	\end{enumerate}
\end{enumerate}
