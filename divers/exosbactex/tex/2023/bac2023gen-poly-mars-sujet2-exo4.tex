\textit{Pour chacune des affirmations suivantes, indiquer si elle est vraie ou fausse. Chaque réponse doit être justifiée. Une réponse non justifiée ne rapporte aucun point.}

\medskip

\begin{enumerate}
	\item \textbf{Affirmation} : La suite $u$ définie pour tout entier naturel $n$ par$u_n = \frac{(-1)^n}{n+1}$ est bornée.
	\item \textbf{Affirmation} : Toute suite bornée est convergente.
	\item \textbf{Affirmation} : Toute suite croissante tend vers $+\infty$.
	\item Soit la fonction $f$ définie sur $\mathbb{R}$ par $f(x)=\ln(x^2+2x+2)$.
	
	\smallskip
	
	\textbf{Affirmation} : La fonction $f$ est convexe sur l’intervalle $[-3;1]$.
	\item On considère la fonction \texttt{mystere} définie ci-dessous qui prend une liste \texttt{L} de nombres en paramètre. On rappelle que \texttt{len(L)} renvoie la longueur, c’est-à-dire le nombre d’éléments de la liste \texttt{L}.
	
\begin{CodePythonLst}*[Largeur=13cm]{center}
def mystere(L) :
	M = L[0]
	#On initialise M avec le premier élément de la liste L
	for i in range(len(L)) :
		if L[i] > M :
			M = L[i]
	return M
\end{CodePythonLst}

	\textbf{Affirmation} : L’exécution de \texttt{mystere([2,3,7,0,6,3,2,0,5])} renvoie \texttt{7}.
\end{enumerate}