Un commerçant vend deux types de matelas: matelas RESSORTS et matelas MOUSSE. 

On suppose que chaque client achète un seul matelas.

On dispose des informations suivantes :

\begin{itemize}
	\item 20\,\% des clients achètent un matelas RESSORTS. 
	
	Parmi eux, 90\,\% sont satisfaits de leur achat.
	\item 82\,\% des clients sont satisfaits de leur achat.
\end{itemize}

\emph{Les deux parties peuvent être traitées de manière indépendante.}

\medskip

\textbf{Partie A}

\medskip

On choisit au hasard un client et on note les évènements :

\begin{itemize}
	\item $R$ : \og le client achète un matelas RESSORTS \fg,
	\item $S$ : \og le client est satisfait de son achat \fg.
\end{itemize}

On note $x = P_{\overline{R}}(S)$, où $P_{\overline{R}}(S)$ désigne la probabilité de $S$ sachant que $R$ n'est pas réalisé.

\begin{wrapstuff}[r]
\begin{forest} for tree = {grow'=0,math content,l=2.5cm,s sep=0.5cm},
	[,name=Omega
		[R , fleche , aproba=\ldots , name=A11
			[S , fleche , aproba=\ldots , name=A21]
			[\overline{S} , fleche , bproba=\ldots , name=A22]
		]
		[\overline{R} , fleche , bproba=\ldots , name=A12
			[S , fleche , aproba=x , name=A23]
			[\overline{S} , fleche , bproba=\ldots , name=A24]
		]
	]
\end{forest}
\end{wrapstuff}

\begin{enumerate}
	\item Recopier et compléter l'arbre pondéré ci-contre décrivant la situation.
	\item Démontrer que $x = 0,8$.
	\item On choisit un client satisfait de son achat.
	
	Quelle est la probabilité qu'il ait acheté un matelas RESSORTS ?
	
	On arrondira le résultat à $10^{-2}$.
\end{enumerate}

\medskip

\textbf{Partie B}

\smallskip

\begin{enumerate}
	\item On choisit 5 clients au hasard. 
	
	On considère la variable aléatoire $X$ qui donne le nombre de clients satisfaits de leur achat parmi ces 5 clients.
	\begin{enumerate}
		\item On admet que $X$ suit une loi binomiale. Donner ses paramètres.
		\item Déterminer la probabilité qu'au plus trois clients soient satisfaits de leur achat.
		
		On arrondira le résultat à $10^{-3}$.
	\end{enumerate}
	\item Soit $n$ un entier naturel non nul.
	
	On choisit à présent $n$ clients au hasard. Ce choix peut être assimilé à un tirage au sort avec remise.
	\begin{enumerate}
		\item On note $p_n$ la probabilité que les $n$ clients soient tous satisfaits de leur achat.
		
		Démontrer que $p_n = 0,82^n$.
		\item Déterminer les entiers naturels $n$ tels que $p_n < 0,01$. 
		
		Interpréter dans le contexte de l'exercice.
	\end{enumerate}
\end{enumerate}
