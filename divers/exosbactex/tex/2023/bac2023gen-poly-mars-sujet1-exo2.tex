L'espace est muni d'un repère orthonormée $\left(\text{O};\vect{\imath},\vect{\jmath},\vect{k}\right)$.

On considère :

\begin{itemize}
	\item $d_1$ la droite passant par le point $H(2;3;0)$ et de vecteur directeur $\vect{u}\begin{pmatrix}1\\-1\\1\end{pmatrix}$ ;
	\item $d_2$  la droite de représentation paramétrique : \[ \begin{dcases}x=2k-3\\y=k\\z=5\end{dcases} \: \text{ où } k \text{ décrit } \mathbb{R}.\]
\end{itemize}

Le but de cet exercice est de déterminer une représentation paramétrique d'une droite $\Delta$ qui soit perpendiculaire aux droites $d_1$ et $d_2$.

\begin{enumerate}
	\item
	\begin{enumerate}
		\item Déterminer un vecteur directeur $\vect{v}$ de la droite $d_2$.
		\item Démontrer que les droites $d_1$ et $d_2$ ne sont pas parallèles.
		\item Démontrer que les droites $d_1$ et $d_2$  ne sont pas sécantes.
		\item Quelle est la position relative des droites $d_1$ et $d_2$ ?
	\end{enumerate}
	\item 
	\begin{enumerate}
		\item Vérifier que le vecteur $\vect{w}\begin{pmatrix}-1\\2\\3\end{pmatrix}$ est orthogonal à $\vect{u}$ et à $\vect{v}$.
		\item On considère le plan $P$ passant par le point $H$ et dirigé par les vecteurs $\vect{u}$ et $\vect{w}$.
		
		On admet qu'une équation cartésienne de ce plan est : \[5x + 4y - z - 22 = 0.\]
		%
		Démontrer que l'intersection du plan $P$ et de la droite $d_2$ est le point $M(3;3;5)$.
	\end{enumerate}
	\item Soit $\Delta$ la droite de vecteur directeur $\vect{w}$ passant par le point $M$.
	
	Une représentation paramétrique de $\Delta$ est donc donnée par : \[ \begin{dcases}x=-r+3\\y=2r+3\\z=3r+5\end{dcases} \: \text{ où } r \text{ décrit } \mathbb{R}.\]
	\begin{enumerate}
		\item Justifier que les droites $\Delta$ et $d_1$ sont perpendiculaires en un point L dont on déterminera les coordonnées.
		\item Expliquer pourquoi la droite $\Delta$ est solution du problème posé.
	\end{enumerate}
\end{enumerate}
