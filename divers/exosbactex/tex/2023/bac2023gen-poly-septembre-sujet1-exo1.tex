Une concession automobile vend des véhicules à moteur électrique et des véhicules à moteur thermique.

Certains clients, avant de se rendre sur le site de la concession, ont consulté la plate-forme numérique de la concession. On a ainsi observé que :

\begin{itemize}
	\item 20\,\% des clients sont intéressés par les véhicules à moteur électrique et 80\,\% préfèrent s'orienter vers l'achat d'un véhicule à moteur thermique ;
	\item lorsqu'un client souhaite acheter un véhicule à moteur électrique, la probabilité pour que le client ait consulté la plate-forme numérique est de $0,5$ ;
	\item lorsqu'un client souhaite acheter un véhicule à moteur thermique, la probabilité pour que le client ait consulté la plate-forme numérique est de $0,375$.
\end{itemize}

On considère les événements suivants :

\begin{itemize}
	\item $C$ : « un client a consulté la plate-forme numérique » ;
	\item $E$ : « un client souhaite acquérir un véhicule à moteur électrique » ;
	\item $T$ : « un client souhaite acquérir un véhicule à moteur thermique ».
\end{itemize}

Les clients font des choix indépendants les uns des autres.

\begin{enumerate}
	\item 
	\begin{enumerate}
		\item Calculer la probabilité qu'un client choisi au hasard souhaite acquérir un véhicule à moteur électrique et ait consulté la plate-forme numérique.
		
		On pourra utiliser un arbre pondéré.
		\item Démontrer que $P(C)=0,4$.
		\item On suppose qu'un client a consulté la plate-forme numérique.
		
		Calculer la probabilité que le client souhaite acheter un véhicule à moteur électrique.
	\end{enumerate}
	\item La concession accueille quotidiennement 17 clients en moyenne.
	
	On note $X$ la variable aléatoire donnant le nombre de clients souhaitant acquérir
	un véhicule à moteur électrique.
	\begin{enumerate}
		\item Préciser la nature et les paramètres de la loi de probabilité suivie par $X$.
		\item Calculer la probabilité qu'au moins trois des clients souhaitent acheter un
		véhicule à moteur électrique lors d'une journée. Donner le résultat arrondi à $10^{-2}$ près.
	\end{enumerate}
\end{enumerate}
