\emph{Cet exercice est un questionnaire à choix multiple.\\ Pour chaque question, une seule des quatre réponses proposées est exacte.\\ Le candidat indiquera sur sa copie le numéro de la question et la réponse choisie.\\ Aucune justification n'est demandée.\\ Une réponse fausse, une réponse multiple ou l'absence de réponse à une question ne rapporte ni n'enlève de point.\\ Les cinq questions sont indépendantes.}

\bigskip

Dans tout l'exercice, $\mathbb{R}$ désigne l'ensemble des nombres réels.

\begin{enumerate}
	\item Une primitive de la fonction $f$, définie sur $\mathbb{R}$ par $f(x) =x\,\text{e}^x$, est la fonction $F$, définie sur $\mathbb{R}$, par :
	
	\smallskip
	
	\begin{tblr}{width=\linewidth,colspec={X[l,m]X[l,m]}}
		\textbf{a.}~~$F(x) =\dfrac{x^2}{2}\,\text{e}^x$&\textbf{c.}~~$F(x) = (x + 1)\,\text{e}^x$\\
		\textbf{b.}~~$F(x) = (x - 1)\,\text{e}^x$ &\textbf{d.}~~$F(x) =x^2 \, \text{e}^{x^2}$ 
	\end{tblr}

	\item On considère la fonction $g$ définie par $g(x) = \ln \left(\dfrac{x - 1}{2x+ 4}\right).$
	
	La fonction $g$ est définie sur :
	
	\smallskip
	
	\begin{tblr}{width=\linewidth,colspec={X[l,m]X[l,m]}}
			\textbf{a.}~~$\mathbb{R}$		& \textbf{c.}~~$]-\infty;-2[ \cup ]1;+\infty[$\\
			\textbf{c.}~~$]-2;+\infty[$ 	& \textbf{d.}~~$]-2;1[$
	\end{tblr}
	\item La fonction $h$ définie sur $\mathbb{R}$ par $h(x)= (x + 1)\,\text{e}^{x}$ est :
	
	\smallskip
	
	\begin{tblr}{width=\linewidth,colspec={X[l,m]X[l,m]}}
		\textbf{a.}~~concave sur $\mathbb{R}$	& \textbf{c.}~~convexe sur $]-\infty;-3]$ et concave sur $[-3;+\infty[$\\
		\textbf{b.}~~convexe sur $\mathbb{R}$	& \textbf{d.}~~concave sur $]-\infty;-3]$ et convexe sur $[-3;+\infty[$
	\end{tblr}
	\item Une suite $\left(u_n\right)$ est minorée par 3 et converge vers un réel $\ell$.
	
	On peut affirmer que :
	
	\smallskip
	
	\begin{tblr}{width=\linewidth,colspec={X[l,m]X[l,m]}}
		\textbf{a.}~~$\ell = 3$ 		&\textbf{c.}~~La suite $\left(u_n\right)$ est décroissante.\\
		\textbf{b.}~~$\ell \geqslant 3$ &\textbf{d.}~~La suite $\left(u_n\right)$ est constante à partir d'un certain rang.
	\end{tblr}
	\item La suite $\left(w_n\right)$ est définie par $w_1 = 2$ et pour tout entier naturel $n$ strictement positif, $w_{n+1} = \dfrac1n w_n$.
	
	\smallskip
	
	\begin{tblr}{width=\linewidth,colspec={X[l,m]X[l,m]}}
		\textbf{a.}~~La suite $\left(w_n\right)$ est géométrique&\textbf{c.}~~$w_5 = \dfrac{1}{15}$\\
		\textbf{b.}~~La suite $\left(w_n\right)$ n'admet pas de limite&\textbf{d.}~~La suite $\left(w_n\right)$ converge vers 0.
	\end{tblr}
\end{enumerate}
