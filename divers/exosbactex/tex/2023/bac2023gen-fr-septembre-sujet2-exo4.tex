\emph{Cet exercice est un questionnaire à choix multiples.\\Pour chacune des questions suivantes, une seule des quatre réponses proposées est exacte.\\	Une réponse exacte rapporte un point. Une réponse fausse, une réponse multiple ou l’absence de réponse à une question ne rapporte ni n’enlève de point.\\Pour répondre, indiquer sur la copie le numéro de la question et la lettre de la réponse choisie.\\Aucune justification n’est demandée.}

\bigskip

L’espace est rapporté à un repère orthonormé $\Rijk$.

On considère :

\begin{itemize}
	\item les points $A(-1;-2;3)$, $B(1;-2;7)$ et $C(1;0;2)$ ;
	\item la droite $\Delta$ de représentation paramétrique $\begin{dcases} x = 1- t \\ y = 2 \\ z = -4+3t \end{dcases}$, où $t \in \R$ ;
	\item le plan $\mathcal{P}$ d’équation cartésienne : $3x +2y + z -4 = 0$ ;
	\item le plan $\mathcal{Q}$ d’équation cartésienne : $-6x -4y -2z +7 = 0$.
\end{itemize}

\begin{enumerate}
	\item Lequel des points suivants appartient au plan $\mathcal{P}$ ?
	
	\medskip
	
	\qcm{$R(1;-3;1)$}{$S(1;2;-1)$}{$T(1;0;1)$}{$U(2;-1;1)$}
	\item Le triangle $ABC$ est :
	
	\medskip
	
	\qcmdeux{équilatéral}{rectangle isocèle}{isocèle non rectangle}{rectangle non isocèle}
	\item La droite $\Delta$ est :
	
	\medskip
	
	\qcmdeux{orthogonale au plan $\mathcal{P}$}{sécante au plan $\mathcal{P}$}{incluse dans le plan $\mathcal{P}$}{strictement parallèle au plan $\mathcal{P}$}
	\item On donne le produit scalaire $\vect{BA}\cdot\vect{BC}=20$.
	
	Une mesure au degré près de l’angle ABC est
	
	\medskip
	
	\qcm{\ang{34}}{\ang{120}}{\ang{90}}{\ang{0}}
	\item L’intersection des plans $\mathcal{P}$ et $\mathcal{Q}$ est :
	
	\medskip
	
	\qcmdeux{un plan}{l’ensemble vide}{une droite}{réduite à un point}
\end{enumerate}