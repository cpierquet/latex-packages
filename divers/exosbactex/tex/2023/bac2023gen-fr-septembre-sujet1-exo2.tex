L’espace est rapporté à un repère orthonormé $\Rijk$.

On considère les points : 

\hfill$A(1;0;-1)$, $B(3;-1;2)$, $C(2;-2;-1)$ et $D(4;-1;-2)$.\hfill~

\smallskip

On note $\Delta$ la droite de représentation paramétrique : \[ \begin{dcases} x = 0 \\ y = 2+t \\ z = -1+t \end{dcases} \text{, avec } t \in \R. \]
%
\begin{enumerate}
	\item 
	\begin{enumerate}
		\item Montrer que les points $A$, $B$ et $C$ définissent un plan que l’on notera $\mathcal{P}$.
		\item Montrer que la droite $(CD)$ est orthogonale au plan $\mathcal{P}$.
		
		Sur le plan $\mathcal{P}$, que représente le point $C$ par rapport à $D$ ?
		\item Montrer qu’une équation cartésienne du plan $\mathcal{P}$ est : $2x+y-z-3 = 0$.
	\end{enumerate}
	\item 
	\begin{enumerate}
		\item Calculer la distance $CD$.
		\item Existe-t-il un point $M$ du plan $\mathcal{P}$ différent de $C$ vérifiant $MD = \sqrt{6}$ ? Justifier la réponse.
	\end{enumerate}
	\item 
	\begin{enumerate}
		\item Montrer que la droite $\Delta$ est incluse dans le plan $\mathcal{P}$.
		\item Soit $H$ le projeté orthogonal du point $D$ sur la droite $\Delta$.
		
		Montrer que $H$ est le point de $\Delta$ associé à la valeur $t = -2$ dans la représentation paramétrique de $\Delta$ donnée ci-dessus.
		\item En déduire la distance du point $D$ à la droite $\Delta$.
	\end{enumerate}
\end{enumerate}
