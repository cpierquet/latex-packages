\emph{Cet exercice est un questionnaire à choix multiples.\\Pour chacune des questions suivantes, une seule des quatre réponses proposées est exacte.\\	Une réponse exacte rapporte un point. Une réponse fausse, une réponse multiple ou l’absence de réponse à une question ne rapporte ni n’enlève de point.\\Pour répondre, indiquer sur la copie le numéro de la question et la lettre de la réponse choisie.\\Aucune justification n’est demandée.}

\bigskip

\begin{enumerate}
	\item On considère la fonction $f$ définie sur $\R$ par : \[ f(x)=x\e^{x^2-3}. \]
	Une des primitives $F$ de la fonction $f$ sur $\R$ est définie par :
	
	\medskip
	
	\begin{tblr}{width=\linewidth,colspec={*{2}{X[m,l]}}}
		\textbf{a.}~~$F(x)=2x\e^{x^2-3}$ & \textbf{b.}~~$F(x)=\big(2x^2+1\big)\e^{x^2-3}$ \\
		\textbf{c.}~~$F(x)=\frac12 x\e^{x^2-3}$ & \textbf{d.}~~$F(x)=\frac12 \e^{x^2-3}$
	\end{tblr}
	\item On considère la suite $\suiten$ définie pour tout entier naturel $n$ par : \[ u_n = \e^{2n+1}. \]
	La suite $\suiten$ est :
	
	\medskip
	
	\begin{tblr}{width=\linewidth,colspec={*{2}{X[m,l]}}}
		\textbf{a.}~~arithmétique de raison 2 & \textbf{b.}~~géométrique de raison $\e$ \\
		\textbf{c.}~~géométrique de raison $\e^2$ & \textbf{d.}~~convergente vers $\e$
	\end{tblr}
\end{enumerate}

Pour les questions \textbf{3.} et \textbf{4.}, on considère la suite $\suiten$ définie sur $\N$ par : \[ u_0=15 \text{ et, pour tout entier naturel }n \text{ : } u_{n+1}=1,2u_n+12. \]

\begin{enumerate}[resume]
	\item La fonction \textsf{Python} suivante, dont la ligne \textsf{4} est incomplète, doit renvoyer la plus petite valeur de l’entier $n$ telle que$u_n > \num{10000}$.

\begin{CodePythonLstAlt}[Largeur=7cm]{center}
def seuil() :
	n = 0
	u = 15
	while ........:
		n = n+1
		u = 1,2∗u + 12
	return(n)
\end{CodePythonLstAlt}
	
	À la ligne \textsf{4}, on complète par :
	
	\medskip
	
	\begin{tblr}{width=\linewidth,colspec={*{4}{X[m,l]}}}
		\textbf{a.}~~\texttt{u <= 10000} & \textbf{b.}~~\texttt{u = 10000} & \textbf{c.}~~\texttt{u > 10000} & \textbf{d.}~~\texttt{n <= 10000}
	\end{tblr}
	\item On considère la suite $\suiten[v]$ définie sur $\N$ par : $v_n = u_n +60$.
	
	La suite $\suiten[v]$ est :
	
	\medskip
	
	\begin{tblr}{width=\linewidth,colspec={*{2}{X[m,l]}}}
		\textbf{a.}~~une suite décroissante & \textbf{b.}~~une suite géométrique de raison $1,2$ \\ \textbf{c.}~~une suite arithmétique de raison 60 & \textbf{d.}~~une suite ni géométrique ni arithmétique
	\end{tblr}
\end{enumerate}
