On se place dans l'espace rapporté à un repère orthonormé $\Rijk$.

On considère le point $A(1;1;0)$ et le vecteur $\vect{u}\begin{pmatrix}0\\2\\- 1\end{pmatrix}$.

On considère le plan $\mathcal{P}$ d'équation : $x + 4y + 2z + 1 = 0$.

\begin{enumerate}
	\item On note $(d)$ la droite passant par A et dirigée par le vecteur $\vect{u}$.
	
	Déterminer une représentation paramétrique de $(d)$.
	\item Justifier que la droite $(d)$ et le plan $\mathcal{P}$ sont sécants en un point B dont les coordonnées sont $(1;-1;1)$.
	\item On considère le point $(1;-1;-1)$.
	\begin{enumerate}
		\item Vérifier que les points $A$, $B$ et $C$ définissent bien un plan.
		\item Montrer que le vecteur $\vect{n}\begin{pmatrix}1\\0\\0\end{pmatrix}$ est un vecteur normal au plan $(ABC)$.
		\item Déterminer une équation cartésienne du plan (ABC).
	\end{enumerate}
	\item 
	\begin{enumerate}
		\item Justifier que le triangle $ABC$ est isocèle en $A$.
		\item Soit $H$ le milieu du segment $[BC]$.
		
		Calculer la longueur $AH$ puis l'aire du triangle $ABC$.
	\end{enumerate}
	\item Soit $D$ le point de coordonnées $(0;-1;1)$.
	\begin{enumerate}
		\item Montrer que la droite $(BD)$ est une hauteur de la pyramide $ABCD$.
		\item Déduire des questions précédentes le volume de la pyramide $ABCD$.
	\end{enumerate}
	\hfill On rappelle que le volume $V$ d'une pyramide est donné par: \[V = \dfrac13 \mathcal{B} \times h,\] où $\mathcal{B}$ est l'aire d'une base et $h$ la hauteur correspondante.\hfill~
\end{enumerate}
