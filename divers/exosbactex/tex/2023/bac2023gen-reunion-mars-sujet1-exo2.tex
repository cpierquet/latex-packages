On considère la fonction $f$ définie sur $]0;+\infty[$ par : \[ \bm{f(x) = 3x + 1 - 2x \ln (x)}.\]
%
On admet que la fonction $f$ est deux fois dérivable sur $]0;+\infty[$. 

On note $f'$ sa dérivée et $f''$ sa dérivée seconde.

On note $\mathcal{C}_f$ sa courbe représentative dans un repère du plan.

\begin{enumerate}
	\item Déterminer la limite de la fonction $f$ en 0 et en $+\infty$.
	\item 
	\begin{enumerate}
		\item Démontrer que pour tout réel $x$ strictement positif : $f'(x) = 1- 2\ln (x)$.
		\item Étudier le signe de $f'$ et dresser le tableau de variation de la fonction $f$ sur l'intervalle $]0;+\infty[$.
		
		On fera figurer dans ce tableau les limites ainsi que la valeur exacte de l'extremum.
	\end{enumerate}
	\item
	\begin{enumerate}
		\item Démontrer que l'équation $f(x) = 0$ admet une unique solution sur $]0;+\infty[$. On notera $\alpha$ cette solution.
		\item En déduire le signe de la fonction $f$ sur $]0;+\infty[$.
	\end{enumerate}	
	\item On considère une primitive quelconque de la fonction $f$ sur l'intervalle $]0;+\infty[$. On la note $F$.
	
	Peut-on affirmer que la fonction $F$ est strictement décroissante sur l'intervalle $\left[\e^{\frac12};+ \infty\right[$ ? Justifier.
	\item
	\begin{enumerate}
		\item Étudier la convexité de la fonction $f$ sur $]0;+\infty[$.
		
		Quelle est la position de la courbe $\mathcal{C}_f$ par rapport à ses tangentes ?
		\item Déterminer une équation de la tangente $T$ à la courbe $\mathcal{C}_f$ au point d'abscisse 1.
		\item Déduire des questions 5.(a) et 5.(b) que pour tout réel $x$ strictement positif : \[\ln (x) \geqslant  1 - \dfrac 1x.\].
	\end{enumerate}
\end{enumerate}
