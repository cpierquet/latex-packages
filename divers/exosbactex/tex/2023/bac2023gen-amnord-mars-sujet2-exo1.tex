\textbf{Partie A}

\medskip

Le plan est muni d'un repère orthogonal.

On considère une fonction $f$ définie et dérivable sur $\R$. On note $f'$ sa fonction dérivée.

On donne ci-dessous la courbe représentative de \textbf{la fonction dérivée} $\bm{f'}$.

\begin{center}
	\begin{tikzpicture}[x=0.6cm,y=0.6cm,xmin=-6.75,xmax=3.5,xgrille=1,xgrilles=0.5,ymin=-7.75,ymax=2.75,ygrille=1,ygrilles=0.5]
		\GrilleTikz[Affs=false]
		\AxesTikz[Epaisseur=0.75pt,ElargirOx=0/0,ElargirOy=0/0]
		\AxexTikz[Epaisseur=0.75pt,AffOrigine=false,Police=\scriptsize,HautGrad=2pt]{-6,-5,...,3}
		\AxeyTikz[Epaisseur=0.75pt,AffOrigine=false,Police=\scriptsize,HautGrad=2pt]{-7,-6,...,2}
		\OrigineTikz[Decal=0pt,Police=\scriptsize]
		\clip (\xmin,\ymin) rectangle (\xmax,\ymax) ;
		\CourbeTikz[thick,red,samples=250]{(\x*\x-3*\x+1)*exp(\x)}{\xmin:\xmax}
	\end{tikzpicture}
\end{center}

Dans cette partie, les résultats seront obtenus par lecture graphique de la courbe représentative de la fonction dérivée $f'$. Aucune justification n'est demandée.

\begin{enumerate}
	\item Donner le sens de variation de la fonction $f$ sur $\R$. On utilisera des valeurs approchées si besoin.
	\item Donner les intervalles sur lesquels la fonction $f$ semble être convexe.
\end{enumerate}

\medskip

\textbf{Partie B}

\medskip

On admet que la fonction $f$ de la \textbf{partie A} est définie sur $\R$ par $f(x) = (x^2-5x+6)\,\e^x$.

On note $\mathcal{C}$ la courbe représentative de la fonction $f$ dans un repère.

\begin{enumerate}
	\item 
	\begin{enumerate}
		\item Déterminer la limite de la fonction $f$ en $+\infty$.
		\item Déterminer la limite de la fonction $f$ en $-\infty$.
	\end{enumerate}
	\item Montrer que, pour tout réel $x$, on a $f'(x) = (x^2 - 3x + 1)\,e^x$.
	\item En déduire le sens de variation de la fonction $f$.
	\item D~terminer l'équation réduite de la tangente $(\mathcal{T})$ à la courbe $\mathcal{C}$ au point d'abscisse $0$.
\end{enumerate}

On admet que la fonction $f$ est deux fois dérivable sur $\R$.

On note $f''$ la fonction dérivée seconde de la fonction $f$.

On admet que, pour tout réel $x$, on a \mbox{$f''(x) = (x + 1)(x- 2)\,\e^x$}.

\begin{enumerate}[resume]
	\item 
	\begin{enumerate}
		\item Étudier la convexité de la fonction $f$ sur $\R$.
		\item Montrer que, pour tout $x$ appartenant à l'intervalle $[-1;2]$, on a $f(x) \leqslant x + 6$.
	\end{enumerate}
\end{enumerate}