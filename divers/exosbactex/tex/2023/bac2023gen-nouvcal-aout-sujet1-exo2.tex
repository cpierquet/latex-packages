On considère la suite $\left(u_n\right)$ définie par $u_0=3$ et, pour tout entier naturel $n$, par :%
\[ u_{n+1}=5u_n-4n-3. \]
%
\begin{enumerate}
	\item 
	\begin{enumerate}
		\item Démontrer que $u_1=12$.
		\item Déterminer $u_2$ en détaillant le calcul.
		\item À l'aide de la calculatrice, conjecturer le sens de variation ainsi que la limite de la suite $\left(u_n\right)$.
	\end{enumerate}
	\item 
	\begin{enumerate}
		\item Démontrer par récurrence que, pour tout entier naturel $n$, on a :%
		\[ u_n \geqslant n+1. \]
		\item En déduire la limite de la suite $\left(u_n\right)$.
	\end{enumerate}
	\item On considère la suite $\left(v_n\right)$ définie pour tout entier naturel $n$ par :%
	\[ v_n = u_n - n - 1. \]
	\begin{enumerate}
		\item Démontrer que la suite $\left(v_n\right)$ est géométrique.
		
		Donner sa raison et son premier terme $v_0$.
		\item En déduire, pour tout entier naturel $n$, l'expression de $v_n$ en fonction de $n$.
		\item En déduire que pour tout entier naturel $n$ :%
		\[ u_n = 2 \times 5^n + n + 1. \]
		\item En déduire le sens de variation de la suite $\left(u_n\right)$.
	\end{enumerate}
	\item On considère la fonction ci-dessous, écrite de manière incomplète en langage Python et destinée à renvoyer le plus petit entier naturel $n$ tel que $u_n \geqslant 10^7$.
	
\begin{CodePythonLstAlt}*[Largeur=7cm]{center}
def suite() :
	u = 3
	n = 0
	while ......... :
		u = .........
		n = n+1
	return n
\end{CodePythonLstAlt}
	\begin{enumerate}
		\item Recopier le programme et compléter les deux instructions manquantes.
		\item Quelle est la valeur renvoyée par cette fonction ?
	\end{enumerate}
\end{enumerate}
