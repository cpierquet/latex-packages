Dans l'espace muni d'un repère orthonormé d'unité 1~cm, on considère les points :

\smallskip

\hfill$D(3;1;5)$ \quad $E(3;-2;-1)$ \quad $F(-1;2;1)$ \quad $G(3;2;-3)$.\hfill~

\begin{enumerate}
	\item 
	\begin{enumerate}
		\item Déterminer les coordonnées des vecteurs $\vect{EF}$ et $\vect{FG}$.
		\item Justifier que les points $E$, $F$ et $G$ ne sont pas alignés.
	\end{enumerate}
	\item 
	\begin{enumerate}
		\item Déterminer une représentation paramétrique de la droite $(FG)$.
		\item On appelle $H$ le point de coordonnées $(2;2;-2)$.
		
		Vérifier que $H$ est le projeté orthogonal de $E$ sur la droite $(FG)$.
		\item Montrer que l'aire du triangle $EFG$ est égale à 12~cm\up{2}.
	\end{enumerate}
	\item 
	\begin{enumerate}
		\item Démontrer que le vecteur $\vect{n}{\begin{pmatrix}2\\1\\2\end{pmatrix}}$ est un vecteur normal au plan $(EFG)$.
		\item Déterminer une équation cartésienne du plan $(EFG)$.
		\item Déterminer une représentation paramétrique de la droite $(d)$ passant par le point $D$ et orthogonale au plan $(EFG)$.
		\item On note $K$ le projeté orthogonal du point $D$ sur le plan $(EFG)$.
		
		À l'aide des questions précédentes, calculer les coordonnées du point $K$.
	\end{enumerate}
	\item 
	\begin{enumerate}
		\item Vérifier que la distance $DK$ est égale à 5~cm.
		\item En déduire le volume du tétraèdre $DEFG$.
	\end{enumerate}
\end{enumerate}