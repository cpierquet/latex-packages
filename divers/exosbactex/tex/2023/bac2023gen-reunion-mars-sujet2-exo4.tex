\emph{Cet exercice est un questionnaire à choix multiples.\\ Pour chacune des questions suivantes, une seule des quatre réponses proposées est exacte.\\ Pour répondre, indiquer sur la copie le numéro de la question et la lettre de la réponse choisie.\\ Aucune justification n'est demandée.\\ Une réponse fausse, une absence de réponse, ou une réponse multiple, ne rapporte ni n'enlève de point.}

\medskip

\begin{enumerate}
	\item On considère la fonction $f$ définie sur $\R$ par $f(x) = 2x\,\e^x$.
	
	Le nombre de solutions sur $\R$ de l'équation $f(x) = - \dfrac{73}{100}$ est égal à : 
	
	\medskip
	
	\begin{tblr}{width=\linewidth,colspec={*{4}{X[m,l]}}}
		(a)~~0 &(b)~~1 &(c)~~2&(d)~~une infinité.
	\end{tblr}
	\item On considère la fonction $g$ définie sur $\R$ par : \[g(x) = \dfrac{x+ 1}{\e^x}.\]
	%
	La limite de la fonction $g$ en $- \infty$ est égale à : 
	
	\medskip
	
	\begin{tblr}{width=\linewidth,colspec={*{4}{X[m,l]}}}
		(a)~~ $-\infty$&(b)~~ $+\infty$ &(c)~~ 0&(d)~~elle n'existe pas.
	\end{tblr}
	
	\item On considère la fonction $h$ définie sur $\R$ par : \[h(x) = (4x - 16)\e^{2x}.\]
	%
	On note $\mathcal{C}_h$ la courbe représentative de $h$ dans un repère orthogonal.
	
	On peut affirmer que:
	
	\medskip
	
	\begin{tblr}{width=\linewidth,colspec={*{2}{X[m,l]}}}
		(a)~~ $h$ est convexe sur $\R$.&(b)~~ $\mathcal{C}_h$ possède un point d'inflexion en $x = 3$.\\
		(c)~~ $h$ est concave sur $\R$.&(d)~~ $\mathcal{C}_h$ possède un point d'inflexion en $x = 3,5$.
	\end{tblr}
	\item On considère la fonction $k$ définie sur l'intervalle $]0; +\infty[$ par : \[k(x) = 3 \ln (x) - x.\]
	%
	On note $\mathcal{C}$ la courbe représentative de la fonction $k$ dans un repère orthonormé. 
	
	On note $T$ la tangente à la courbe $\mathcal{C}$ au point d'abscisse $x = \e$.
	
	Une équation de $T$ est:
	
	\medskip
	
	\begin{tblr}{width=\linewidth,colspec={*{2}{X[m,l]}}}
		(a)~~$y = (3 - \e)x$&(b)~~$y = \left(\dfrac{3 - \e}{\e}\right)x$\\
		(c)~~$y = \left(\dfrac{3}{\e}- 1\right)x + 1$&(d)~~$y = (\e - 1)x + 1$
	\end{tblr}
	\item On considère l'équation $[\ln (x)]^2 + 10 \ln (x) + 21 = 0$, avec $x \in ]0;+\infty[$.
	
	Le nombre de solutions de cette équation est égal à :
	
	\medskip
	
	\begin{tblr}{width=\linewidth,colspec={*{4}{X[m,l]}}}
		(a)~~0&(b)~~1&(c)~~2&(d)~~une infinité.
	\end{tblr}
\end{enumerate}