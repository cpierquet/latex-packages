Une entreprise de location de bateaux de tourisme propose à ses clients deux types de bateaux : bateau à voile et bateau à moteur.

\smallskip

Par ailleurs, un client peut prendre l'option PILOTE. Dans ce cas, le bateau, qu'il soit à voile ou à moteur, est loué avec un pilote.

\smallskip

On sait que :

\begin{itemize}
	\item 60\,\% des clients choisissent un bateau à voile ; parmi eux, 20\,\% prennent l'option PILOTE.
	\item 42\,\% des clients prennent l'option PILOTE.
\end{itemize}

On choisit un hasard un client et on considère les événements :

\begin{itemize}
	\item $V$ : \og le client un bateau à voile \fg{} ;
	\item $L$ : \og le client prend l'option PILOTE \fg.
\end{itemize}

\medskip

\hfill~\textit{Les trois parties peuvent être traitées de manière indépendante.}\hfill~

\medskip

\textbf{Partie A}

\smallskip

\begin{enumerate}
	\item Traduire la situation par un arbre pondéré que l'on complètera au fur et à mesure.
	\item Calculer la probabilité que le client choisisse un bateau à voile et qu'il ne prenne pas l'option PILOTE.
	\item Démontrer que la probabilité que le client choisisse un bateau à moteur et qu'il prenne l'option PILOTE est égale à $0,30$.
	\item En déduire $P_{\overline{V}}(L)$, probabilité de $L$ sachant que $V$ n'est pas réalisé.
	\item Un client a pris l'option PILOTE.
	
	Quelle est la probabilité qu'il ait choisi un bateau à voile ? Arrondir à $0,01$ près.
\end{enumerate}

\smallskip

\textbf{Partie B}

\medskip

Lorsqu'un client ne prend pas l'option PILOTE, la probabilité que son bateau subisse une avarie est égale à $0,12$. Cette probabilité n'est que de $\num{0,005}$ si le client prend l'option PILOTE.

On considère un client. On note $A$ l'événement : \og son bateau subit une avarie \fg.

\begin{enumerate}
	\item Déterminer $P\big(L \cap A\big)$ et $P\big(\overline{L} \cap A\big)$.
	\item L'entreprise loue \num{1000} bateaux. À combien d'avaries peut-on s'attendre ?
\end{enumerate}

\smallskip

\textbf{Partie C}

\medskip

On rappelle que la probabilité qu'un client donné prenne l'option PILOTE est égale à $0,42$.

On considère un échantillon aléatoire de 40 clients. On note $X$ la variable aléatoire comptant le nombre de clients de l'échantillon prenant l'option PILOTE.

\begin{enumerate}
	\item On admet que la variable aléatoire $X$ suit une loi binomiale.
	
	Donner sans justification ses paramètres.
	\item Calculer la probabilité, arrondie à $10^{-3}$, qu'au moins 15 clients prennent l'option PILOTE.
\end{enumerate}
