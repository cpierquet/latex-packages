On considère la fonction $f$ définie sur $]0;+\infty[$ par $f(x)=x^2-8\,\ln(x)$, où $\ln$ désigne la fonction logarithme népérien.

\smallskip

On admet que $f$ est dérivable sur $]0;+\infty[$, on note $f'$ sa fonction dérivée.

\begin{enumerate}
	\item Déterminer $\displaystyle\lim_{x \to 0} f(x)$.
	\item On admet que, pour tout réel $x$ de $]0;+\infty[$, $f(x)=x^2 \left(1-8\frac{\ln(x)}{x}\right)$.
	
	Déterminer $\displaystyle\lim_{x \to +\infty} f(x)$.
	\item Montrer que, pour tout réel $x$ de $]0;+\infty[$, $f'(x)=\frac{2\big(x^2-4\big)}{x}$.
	\item Étudier les variations de $f$ sur $]0;+\infty[$ et dresser son tableau de variations complet.
	
	On précisera la valeur exacte du minimum de $f$ sur $]0;+\infty[$.
	\item Démontrer que, sur l'intervalle $]0;2]$, l’équation $f(x)=0$ admet une solution unique $\alpha$ (on ne cherchera pas à déterminer la valeur de $\alpha$).
	\item On admet que, sur l’intervalle $[2;+\infty[$, l’équation $f(x)=0$ admet une solution unique $\beta$ (on ne cherchera pas à déterminer la valeur de $\beta$).
	
	En déduire le signe de $f$ sur l’intervalle $]0;+\infty[$.
	\item Pour tout nombre réel $k$, on considère la fonction $g_k$ définie sur $]0;+\infty[$ par : \[ g_k(x)=x^2-8\,\ln(x)+k. \]
	%
	En s’aidant du tableau de variations de $f$, déterminer la plus petite valeur de $k$ pour laquelle la fonction $g_k$ est positive sur l’intervalle $]0;+\infty[$.
\end{enumerate}
