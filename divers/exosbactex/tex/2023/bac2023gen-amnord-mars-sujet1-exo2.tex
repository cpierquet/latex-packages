On considère la fonction $f$ définie sur $\R$ par : \[f(x)=\e^{3 x}-(2 x+1) \e^{x}.\]
%
Le but de cet exercice est d'étudier la fonction $f$ sur $\R$.

\medskip

\textbf{Partie A - Étude d'une fonction auxiliaire}

\medskip

On définit la fonction $g$ sur $\R$ par : \[g(x)=3 \e^{2 x}- 2 x - 3.\]
%
\begin{enumerate}
	\item 
	\begin{enumerate}
		\item Déterminer la limite de la fonction $g$ en $-\infty$.
		\item Déterminer la limite de la fonction $g$ en $+\infty$.
	\end{enumerate}
	\item 
	\begin{enumerate}
		\item On admet que la fonction $g$ est dérivable sur $\R$, et on note $g'$ sa fonction dérivée. Démontrer que pour tout nombre réel $x$, on a $g'(x)=6 \e^{2 x}-2$.
		\item Étudier le signe de la fonction dérivée $g'$ sur $\R$.
		\item En déduire le tableau de variations de la fonction $g$ sur $\R$. Vérifier que la fonction $g$ admet un minimum égal à $\ln (3) - 2$.
	\end{enumerate}
	\item 
	\begin{enumerate}
		\item Montrer que $x=0$ est solution de l'équation $g(x) = 0$.
		\item Montrer que l'équation $g(x)=0$ admet une deuxième solution, non nulle, notée $\alpha$, dont on donnera un encadrement d'amplitude $10^{-1}$.
	\end{enumerate}
	\item Déduire des questions précédentes le signe de la fonction $g$ sur $\R$.
\end{enumerate}

\medskip

\textbf{Partie B - Étude de la fonction} $\bm{f}$

\medskip

\begin{enumerate}
	\item La fonction $f$ est dérivable sur $\R$, et on note $f'$ sa fonction dérivée.
	
	Démontrer que pour tout nombre réel $x$, on a $f'(x)=\e^{x} g(x)$, où $g$ est la fonction définie dans la \textbf{partie A}.
	\item En déduire alors le signe de la fonction dérivée $f'$ puis les variations de la fonction $f$ sur $\R$.
	\item Pourquoi la fonction $f$ n'est-elle pas convexe sur $\R$ ? Expliquer.
\end{enumerate}