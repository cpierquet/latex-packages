Dans l'espace muni d'un repère orthonormé $\left(O;\vect{\imath},\,\vect{\jmath},\,\vect{k}\right)$, on considère les
points : \[ A(-1;-3;2) \text{, } B(3;-2;6) \text{ et } C(1;2;-4). \]

\begin{enumerate}
	\item Démontrer que les points $A$, $B$ et $C$ définissent un plan que l'on notera $\mathcal{P}$.
	\item 
	\begin{enumerate}
		\item Montrer que le vecteur $\vect{n}\begin{pmatrix}13\\-16\\-9\end{pmatrix}$  est normal au plan $\mathcal{P}$.
		\item Démontrer qu'une équation cartésienne du plan $P$ est $13x - 16y - 9z- 17 = 0$.
	\end{enumerate}
\end{enumerate}

On note $\mathcal{D}$ la droite passant par le point $F(15;-16;-8)$ et orthogonale au plan $\mathcal{P}$.

\begin{enumerate}[resume]
	\item Donner une représentation paramétrique de la droite $\mathcal{D}$.
	\item On appelle $E$ le point d'intersection de la droite $\mathcal{D}$ et du plan $\mathcal{P}$.
	
	Démontrer que le point $E$ a pour coordonnées $(2;0;1)$.
	\item Déterminer la valeur exacte de la distance du point $F$ au plan $\mathcal{P}$.
	\item Déterminer les coordonnées du ou des point(s) de la droite $\mathcal{D}$ dont la distance au plan $\mathcal{P}$ est égale à la moitié de la distance du point $F$ au plan $\mathcal{P}$.
\end{enumerate}