\emph{Les parties A et B peuvent être traitées indépendamment}

\medskip

\textbf{Partie A}

\medskip

Chaque jour, un athlète doit sauter une haie en fin d’entraînement. Son entraîneur estime, au vu de la saison précédente, que :

\begin{itemize}
	\item si l’athlète franchit la haie un jour, alors il la franchira dans 90\,\% des cas le jour suivant ;
	\item si l’athlète ne franchit pas la haie un jour, alors dans 70\,\% des cas il ne la franchira pas non plus le lendemain.
\end{itemize}

On note pour tout entier naturel $n$ :

\begin{itemize}
	\item $R_n$ l'événement : « L’athlète réussit à franchir la haie lors de la $n$-ième séance »,
	\item $p_n$ la probabilité de l'événement $R_n$. On considère que $p_0=0,6$.
\end{itemize}

\begin{enumerate}
	\item Soit $n$ un entier naturel, recopier l’arbre pondéré ci-dessous et compléter les pointillés.
	
	\begin{center}
		\def\ArbreDeuxDeux{
			$R_n$/$p_n$/above,$R_{n+1}$/$\ldots$/above,$\overline{R_{n+1}}$/$\ldots$/below,
			$\overline{R_{n}}$/$1-p_n$/below,$R_{n+1}$/$\ldots$/above,$\overline{R_{n+1}}$/$\ldots$/below
		}
		\ArbreProbasTikz{\ArbreDeuxDeux}
	\end{center}
	\item Justifier en vous aidant de l’arbre que, pour tout entier naturel $n$, on a :\[ p_{n+1}=0,6p_n+0,3. \]
	\item On considère la suite $\left(u_n\right)$ définie, pour tout entier naturel $n$, par $u_n = p_n - 0,75$.
	\begin{enumerate}
		\item Démontrer que la suite $\left(u_n\right)$ est une suite géométrique dont on précisera la raison et le premier terme.
		\item Démontrer que, pour tout entier naturel $n$ : $p_n = 0,75 - 0,15 \times 0,6^n$.
		\item En déduire que la suite $\left(p_n\right)$ est convergente et déterminer sa limite $\ell$.
		\item Interpréter la valeur de $\ell$ dans le cadre de l'exercice.
	\end{enumerate}
\end{enumerate}

\smallskip

\textbf{Partie B}

\medskip

Après de nombreuses séances d'entraînement, l’entraineur estime maintenant que l’athlète franchit chaque haie avec une probabilité de $0,75$ et ce indépendamment d’avoir franchi ou non les haies précédentes.

\smallskip

On note $X$ la variable aléatoire qui donne le nombre de haies franchies par l’athlète à l’issue d’un 400 mètres haies qui comporte 10 haies.

\begin{enumerate}
	\item Préciser la nature et les paramètres de la loi de probabilité suivie par $X$.
	\item Déterminer, à $10^{-3}$ près, la probabilité que l’athlète franchisse les 10 haies.
	\item Calculer $P(X \geqslant 9)$, à $10^{-3}$ près.
\end{enumerate}
