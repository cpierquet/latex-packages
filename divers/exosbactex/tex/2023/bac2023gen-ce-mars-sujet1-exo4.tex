Un biologiste a modélisé l'évolution d'une population de bactéries (en milliers d'entités) par la fonction $f$ définie sur $[0;+\infty[$ par $f(t) = \text{e}^3 -  \text{e}^{-0,5t^2+t+2}$ où $t$ désigne le temps en heures depuis le début de l'expérience.

\smallskip

À partir de cette modélisation, il propose les trois affirmations ci-dessous. Pour chacune d'elles, indiquer, en justifiant, si elle est vraie ou fausse.

\begin{itemize}
	\item Affirmation 1 : « La population augmente en permanence ».
	\item Affirmation 2 : « À très long terme, la population dépassera \num{21000} bactéries ».
	\item Affirmation 3 : « La population de bactéries aura un effectif de \num{10000} à deux reprises au cours du temps ».
\end{itemize}