On considère la suite $\left(u_n\right)$ telle que $u_0=0$ et pour tout entier naturel $n$ :%
\[ u_{n+1} = \frac{-u_n-4}{u_n+3}. \]%
On admet que $u_n$ est défini pour tout entier naturel $n$.

\begin{enumerate}
	\item Calculer les valeurs exactes de $u_1$ et $u_2$
	\item On considère la fonction \texttt{terme} ci-dessous écrite de manière incomplète en langage Python :
	
	\smallskip

\begin{minipage}{0.5\linewidth}
\begin{CodePythonLstAlt}*[Largeur=7cm]{}
def terme(n) :
	u = ......
	for i in range(n) :
		u = ......
	return(u)
\end{CodePythonLstAlt}
\end{minipage}
\hfill
\begin{minipage}{0.4\linewidth}
On rappelle qu'en langage Python,\\
\og \texttt{i \textcolor{CouleurVertForet}{in} \textcolor{magenta}{range}(n)} \fg{} signifie que \texttt{i} varie de \texttt{0} à \texttt{n-1}.
\end{minipage}
\hfill~
	\smallskip
	
	Recopier et compléter le cadre ci-dessus de sorte que, pour tout entier naturel $n$, l'instruction \texttt{terme(n)} renvoie la valeur de $u_n$.
	\item Soit la fonction $f$ définie sur $\intervOO{-3}{+\infty}$ par :%
	\[ f(x)=\frac{-x-4}{x+3}. \]%
	Ainsi, pour tout entier naturel $n$, on a $u_{n+1}=f\big(u_n\big)$.
	
	Démontrer que la fonction $f$ est strictement croissante sur $\intervOO{-3}{+\infty}$.
	\item Démontrer par récurrence que pour tout entier naturel $n$ :%
	\[ -2 < u_{n+1} \leqslant u_n. \]
	\item En déduire que la suite $\left(u_n\right)$ est convergente.
	\item Soit la suite $\left(v_n\right)$ définie pour tout entier naturel $n$ par :%
	\[ v_n = \frac{1}{u_n+2}. \]
	\begin{enumerate}
		\item Donner $v_0$.
		\item Démontrer que la suite $\left(v_n\right)$ est arithmétique de raison 1.
		\item En déduire que pour tout entier naturel $n$ :%
		\[ u_n = \frac{1}{n+0,5}-2. \]
		\item Déterminer la limite de la suite $\left(u_n\right)$.
	\end{enumerate}
\end{enumerate}
