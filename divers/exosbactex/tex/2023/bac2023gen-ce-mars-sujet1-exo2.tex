Les deux parties de cet exercice sont indépendantes.

\smallskip

Dans une grande ville française, des trottinettes électriques sont mises à disposition des usagers. Une entreprise, chargée de l'entretien du parc de trottinettes, contrôle leur état chaque lundi.

\bigskip

\textbf{Partie A}

\medskip

On estime que :

\begin{itemize}
	\item lorsqu'une trottinette est en bon état un lundi, la probabilité qu'elle soit encore en bon état le lundi suivant est $0,9$ ;
	\item lorsqu'une trottinette est en mauvais état un lundi, la probabilité qu'elle soit en bon
	état le lundi suivant est $0,4$ 
\end{itemize}

On s'intéresse à l'état d'une trottinette lors des phases de contrôle.

Soit $n$ un entier naturel. On note $B_n$ l'évènement « la trottinette est en bon état $n$ semaines
après sa mise en service » et $p_n$ la probabilité de $B_n$.

Lors de sa mise en service, la trottinette est en bon état. On a donc $p_0 = 1$.

\begin{enumerate}
	\item Donner $p_1$, et montrer que $p_2 = 0,85$. On pourra s'appuyer sur un arbre pondéré.
	\item Recopier et compléter l'arbre pondéré ci-dessous :
	
	\begin{center}
		\def\ArbreDeuxDeux{
			$B_n$/$p_n$/above,$B_{n+1}$//above,$\overline{B_{n+1}}$//above,
			$\overline{B_n}$//below,$B_{n+1}$//below,$\overline{B_{n+1}}$//below
		}
		\ArbreProbasTikz{\ArbreDeuxDeux}
	\end{center}
	\item En déduire que, pour tout entier naturel $n$, $p_{n+1} = 0,5p_n + 0,4$.
	\item 
	\begin{enumerate}
		\item Démontrer par récurrence que pour tout entier naturel $n$, $p_n \geqslant 0,8$.
		\item À partir de ce résultat, quelle communication l'entreprise peut-elle envisager pour valoriser la fiabilité du parc ?
	\end{enumerate}
	\item 
	\begin{enumerate}
		\item On considère la suite $\left(u_n\right)$ définie pour tout entier naturel $n$ par $u_n = p_n - 8$.
		
		Montrer que $\left(u_n\right)$ est une suite géométrique dont on donnera le premier terme et la raison.
		\item En déduire l’expression de $u_n$, puis de $p_n$ en fonction de $n$.
		\item En déduire la limite de la suite $\left(p_n\right)$.
	\end{enumerate}
\end{enumerate}

\smallskip

\textbf{Partie B}

\medskip

Dans cette partie, on modélise la situation de la façon suivante :

\begin{itemize}
	\item l'état d'une trottinette est indépendant de celui des autres ;
	\item la probabilité qu'une trottinette soit en bon état est égale à $0,8$.
\end{itemize}

On note $X$ la variable aléatoire qui, à un lot de 15 trottinettes, associe le nombre de trottinettes en bon état. Le nombre de trottinettes du parc étant très important, le prélèvement de 15 trottinettes peut être assimilé à un tirage avec remise.

\begin{enumerate}
	\item Justifier que $X$ suit une loi binomiale et préciser les paramètres de cette loi.
	\item Calculer la probabilité que les 15 trottinettes soient en bon état.
	\item Calculer la probabilité qu'au moins 10 trottinettes soient en bon état dans un lot de 15.
	\item On admet que $\mathbb{E}(X) = 12$. Interpréter le résultat 
\end{enumerate}
