\emph{Les parties \text{A} et \text{B} peuvent être traitées indépendamment}

\medskip

Les utilisateurs de vélo d'une ville sont classés en deux catégories disjointes:

\begin{itemize}
	\item ceux qui utilisent le vélo dans leurs déplacements professionnels ;
	\item ceux qui utilisent le vélo uniquement pour leurs loisirs.
\end{itemize}

Un sondage donne les résultats suivants:

\begin{itemize}
	\item 21\,\% des utilisateurs ont moins de 35 ans.
	
	Parmi eux, 68\,\% utilisent leur vélo uniquement pour leurs loisirs alors que les autres l'utilisent dans leurs déplacements professionnels ;
	\item parmi les 35 ans ou plus, seuls 20\,\% utilisent leur vélo dans leurs déplacements professionnels, les autres l'utilisent uniquement pour leurs loisirs.
\end{itemize}

On interroge au hasard un utilisateur de vélo de cette ville.

Dans tout l'exercice on considère les évènements suivants:

\begin{itemize}
	\item $J$ : \og la personne interrogée a moins de 35 ans \fg{} ;
	\item $T$ : \og la personne interrogée utilise le vélo dans ses déplacements professionnels \fg{} ;
	\item $\overline{J}$ et $\overline{T}$ sont les évènements contraires de $J$ et $T$.
\end{itemize}

\textbf{Partie A}

\smallskip

\begin{enumerate}
	\item Calculer la probabilité que la personne interrogée ait moins de $35$~ans et utilise son vélo dans ses déplacements professionnels.
	
	\emph{On pourra s'appuyer sur un arbre pondéré.}
	\item Calculer la valeur exacte de la probabilité de $T$.
	\item On considère à présent un habitant qui utilise son vélo dans ses déplacements professionnels.
	
	Démontrer que la probabilité qu'il ait moins de 35 ans est $0,30$ à $10^{-2}$ près.
\end{enumerate}

\textbf{Partie B}

\medskip

Dans cette partie, on s'intéresse uniquement aux personnes utilisant leur vélo dans leurs déplacements professionnels.

\medskip

On admet que 30\,\% d'entre elles ont moins de $35$~ans.

On sélectionne au hasard parmi elles un échantillon de $120$~personnes auxquelles on va soumettre un questionnaire supplémentaire. 

On assimile la sélection de cet échantillon à un tirage aléatoire avec remise.

\medskip

On demande à chaque individu de cet échantillon son âge.

\medskip

X représente le nombre de personnes de l'échantillon ayant moins de $35$~ans.

\medskip

\emph{Dans cette partie, les résultats seront arrondis à $10^{-3}$ près.}

\begin{enumerate}
	\item Déterminer la nature et les paramètres de la loi de probabilité suivie par $X$.
	\item Calculer la probabilité qu'au moins $50$ utilisateurs de vélo parmi les $120$ aient moins de $35$~ans.
\end{enumerate}
