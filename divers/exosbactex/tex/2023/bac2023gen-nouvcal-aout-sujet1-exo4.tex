On considère la fonction $f$ définie pour tout réel $x$ de l'intervalle $\intervOO{0}{+\infty}$ par :%
\[ f(x)=5x^2 + 2x - 2x^2\,\ln(x). \]
%
On note $\mathcal{C}_f$ la courbe représentative de $f$ dans un repère orthogonal du plan.

On admet que $f$ est deux fois dérivable sut l'intervalle $\intervOO{0}{+\infty}$.

On note $f'$ sa dérivée et $f''$ sa dérivée seconde.

\begin{enumerate}
	\item 
	\begin{enumerate}
		\item Démontre: que la limite de la fonction $f$ en 0 est égale à 0.
		\item Déterminer la limite de la fonction $f$ en
		$+\infty$.
	\end{enumerate}
	\item Déterminer $f'(x)$ pour tout réel $x$ de l'intervalle $\intervOO{0}{+\infty}$.
	\item 
	\begin{enumerate}
		\item Démontrer que pour tout réel $x$ de l'intervalle $\intervOO{0}{+\infty}$ :%
		\[ f''(x) = 4(1-\ln(x)). \]
		\item En déduire le plus grand intervalle sur lequel la courbe $\mathcal{C}_f$, est au-dessus de ses tangentes.
		\item Dresser le tableau des variations de la fonction $f'$ sur l'intervalle $\intervOO{0}{+\infty}$.
		
		(On admettra que $\lim\limits_{\substack{x \rightarrow 0 \\ x > 0 }} f'(x) = 2$ et que $\lim\limits_{x \rightarrow +\infty} f'(x) = -\infty$.)
	\end{enumerate}
	\item 
	\begin{enumerate}
		\item Montrer que l'équation $f'(x) = 0$ admet dans l'intervalle $\intervOO{0}{+\infty}$ une unique solution $\alpha$ dont on donnera un encadrement d'amplitude $10^{-2}$.
		\item En déduire le signe de $f'(x)$ sur l'intervalle $\intervOO{0}{+\infty}$ ainsi que le tableau des variations de la fonction $f$ sur l'intervalle $\intervOO{0}{+\infty}$.
	\end{enumerate}
	\item 
	\begin{enumerate}
		\item En utilisant l'égalité $f'(\alpha) = 0$, démontrer que :%
		\[ \ln(\alpha) = \frac{4\alpha+1}{2\alpha}. \]%
		En déduire que $f(\alpha)=\alpha^2+\alpha$.
		\item En déduire un encadrement d'amplitude $10^{-1}$ du maximum de la fonction $f$.
	\end{enumerate}
\end{enumerate}