Dans un souci de préservation de l'environnement, Monsieur Durand décide de se rendre chaque matin au travail en utilisant son vélo ou les transports en commun.

S'il choisit de prendre les transports en commun un matin, il reprend les transports en commun le lendemain avec une probabilité égale à $0,8$.

S'il utilise son vélo un matin, il reprend son vélo le lendemain avec une probabilité égale à $0,4$.

\medskip

Pour tout entier naturel $n$ non nul, on note:

\begin{itemize}
	\item $T_n$ l'évènement \og Monsieur Durand utilise les transports en commun le $n$-ième jour \fg{} ;
	\item $V_n$ l'évènement \og  Monsieur Durand utilise son vélo le $n$-ième jour \fg{} ;
	\item On note $p_n$ la probabilité de l'évènement $T_n$.
\end{itemize}

Le premier matin, il décide d'utiliser les transports en commun. Ainsi, la probabilité de l'évènement $T_1$ est $p_1 = 1$.

\begin{enumerate}
	\item Recopier et compléter l'arbre pondéré ci-dessous représentant la situation pour les
	2\ieme{} et 3\ieme{} jours.
	
	\begin{center}
		\def\ArbreDeuxDeux{
			$T_2$/$\ldots$/above,$T_3$/$\ldots$/above,$V_3$/$\ldots$/below,
			$V_2$/$\ldots$/below,$T_3$/$\ldots$/above,$V_3$/$\ldots$/below
		}
		\begin{EnvArbreProbasTikz}{\ArbreDeuxDeux}
			\draw (R) node[fill=white] {$T_1$} ;
		\end{EnvArbreProbasTikz}
	\end{center}
	\item Calculer $p_3$
	\item Le 3\ieme{} jour, M. Durand utilise son vélo.
	
	Calculer la probabilité qu'il ait pris les transports en commun la veille.
	
	\item Recopier et compléter l'arbre pondéré ci-dessous représentant la situation pour les
	$n$-ième et $(n + 1)$-ième jours.
	
	\begin{center}
		\def\ArbreDeuxDeux{
			$T_n$/$p_n$/above,$T_{n+1}$/$\ldots$/above,$V_{n+1}$/$\ldots$/below,
			$V_n$/$\ldots$/below,$T_{n+1}$/$\ldots$/above,$V_{n+1}$/$\ldots$/below
		}
		\ArbreProbasTikz{\ArbreDeuxDeux}
	\end{center}
	\item Montrer que, pour tout entier naturel $n$ non nul, $p_{n+1} = 0,2p_n + 0,6$.
	\item Montrer par récurrence que, pour tout entier naturel $n$ non nul, on a \[p_n = 0,75 + 0,25 \times 0,2^{n-1}.\]
	\item Déterminer la limite de la suite $(p_n)$ et interpréter le résultat dans le contexte de l'exercice.
\end{enumerate}
