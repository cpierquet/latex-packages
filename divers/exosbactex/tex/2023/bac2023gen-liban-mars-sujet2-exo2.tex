\textbf{Partie A}

\medskip

On considère la fonction $f$ définie par : \[ f(x)=x-\ln(1+x). \]
%
\begin{enumerate}
	\item Justifier que la fonction $f$ est définie sur l'intervalle $]-1;+\infty[$.
	\item On admet que la fonction $f$ est dérivable sur $]-1;+\infty[$.
	
	Déterminer l'expression de sa fonction dérivée $f'$.
	\item 
	\begin{enumerate}
		\item En déduire le sens de variation de la fonction $f$ sur l'intervalle $]-1;+\infty[$.
		\item En déduire le signe de la fonction $f$ sur l'intervalle $]-1;+\infty[$.
	\end{enumerate}
	\item 
	\begin{enumerate}
		\item Montrer que, pour tout $x$ appartenant à l'intervalle $]-1;+\infty[$[, on a : \[ f(x) = \ln {\left(\frac{\text{e}^x}{1+x}\right)}. \]
		\item En déduire la limite en $+\infty$ de la fonction $f$.
	\end{enumerate}
\end{enumerate}

\medskip

\textbf{Partie B}

\medskip

On considère la suite $\big(u_n\big)$ définie par $u_0=10$ et, pour tout entier naturel $n$, \[ u_{n+1}=u_n-\ln\big(1+u_n\big). \]
%
On admet que la suite $\big(u_n\big)$ est bien définie.

\begin{enumerate}
	\item Donner la valeur arrondie au millième de $u_1$.
	\item En utilisant la question \textbf{3.a.} de la \textbf{partie A}, démontrer par récurrence que, pour tout entier naturel $n$, on a $u_n \geqslant 0$.
	\item Démontrer que la suite $\big(u_n\big)$ est décroissante.
	\item Déduire des questions précédentes que la suite $\big(u_n\big)$ converge.
	\item Déterminer la limite de la suite $\big(u_n\big)$.
\end{enumerate}
