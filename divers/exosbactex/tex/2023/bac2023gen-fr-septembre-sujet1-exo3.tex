\textit{Les parties \textbf{A} et \textbf{B} sont indépendantes.}

\textit{Les probabilités demandées seront données à $10^{-3}$ près.}

\smallskip

Pour aider à la détection de certaines allergies, on peut procéder à un test sanguin dont le résultat est soit positif, soit négatif.

Dans une population, ce test donne les résultats suivants :

\begin{itemize}
	\item si un individu est allergique, le test est positif dans 97\,\% des cas ;
	\item Si un individu n’est pas allergique, le test est négatif dans $95,7$\,\% des cas.
\end{itemize}

Par ailleurs, 20\,\% des individus de la population concernée présentent un test positif.

\smallskip

On choisit au hasard un individu dans la population, et on note :

\begin{itemize}
	\item $A$ l’évènement « l’individu est allergique » ;
	\item $T$ l’évènement « l’individu présente un test positif ».
\end{itemize}

On notera $\overline{A}$ et $\overline{T}$ les évènements contraires de $A$ et $T$.

On appelle par ailleurs $x$ la probabilité de l’évènement $A$ : $x = p(A)$.

\medskip

\textbf{Partie A}

\begin{wrapstuff}[r]
\ArbreProbasTikz[EspaceNiveau=2]{$A$/$x$/above,$T$/$\ldots$/above,$\overline{T}$/$\ldots$/below,
	$\overline{A}$/$\ldots$/below,$T$/$\ldots$/above,$\overline{T}$/$\ldots$/below}
\end{wrapstuff}

\begin{enumerate}
	\item Reproduire et compléter l’arbre ci-contre décrivant la situation, en indiquant sur chaque branche la probabilité correspondante.
	\item 
	\begin{enumerate}
		\item Démontrer l’égalité : $p(T) = 0,927x +0,043$.
		\item En déduire la probabilité que l’individu choisi soit allergique.
	\end{enumerate}
	\item Justifier par un calcul l’affirmation suivante :
	
	\og Si le test d’un individu choisi au hasard est positif, il y a plus de 80\,\% de chances que cet individu soit allergique \fg.
\end{enumerate}

\medskip

\textbf{Partie B}

\medskip

On réalise une enquête sur les allergies dans une ville en interrogeant 150 habitants choisis au hasard, et on admet que ce choix se ramène à des tirages successifs indépendants avec remise.

On sait que la probabilité qu’un habitant choisi au hasard dans cette ville soit allergique est égale à $0,08$.

On note $X$ la variable aléatoire qui à un échantillon de 150 habitants choisis au hasard associe le nombre de personnes allergiques dans cet échantillon.

\begin{enumerate}
	\item Quelle est la loi de probabilité suivie par la variable aléatoire $X$ ?
	
	Préciser ses paramètres.
	\item Déterminer la probabilité que 20 personnes exactement parmi les 150 interrogées soient allergiques.
	\item Déterminer la probabilité qu’au moins 10\,\% des personnes parmi les 150 interrogées soient allergiques.
\end{enumerate}
