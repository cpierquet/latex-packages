\emph{Cet exercice est un questionnaire à choix multiple. Pour chaque question, une seule des quatre réponses proposées est exacte. Le candidat indiquera sur sa copie le numéro de la question et la réponse choisie. Aucune justification n'est demandée.\\ Une réponse fausse, une réponse multiple ou l'absence de réponse à une question ne rapporte ni n'enlève de point. Les cinq questions sont indépendantes.}

\medskip

\begin{enumerate}
	\item On considère la fonction $f$ définie sur l'intervalle $]1;+\infty[$ par $f(x)=0,05 - \frac{\ln(x)}{x-1}$.
	
	La limite de la fonction $f$ en $+\infty$ est égale à :
	\begin{enumerate}
		\item $+\infty$
		\item $0,05$
		\item $-\infty$
		\item 0
	\end{enumerate}
	\item On considère une fonction $h$ continue sur l'intervalle $[-2;4]$ telle que :
	
	$h(-1) = 0$ \quad  $h(1) = 4$ \quad  $h(3) = -1$
	
	\smallskip
	
	On peut affirmer que :
	\begin{enumerate}
		\item la fonction $h$ est croissante sur l'intervalle $[-1;1]$.
		\item la fonction $h$ est positive sur l'intervalle $[-1;1]$.
		\item il existe au moins un nombre réel $a$ dans l'intervalle $[1;3]$ tel que $h(a) = 1$.
		\item l'équation $h(x) = 1$ admet exactement deux solutions dans l'intervalle $[-2;4]$.
	\end{enumerate}
	\item On considère deux suites $\suiten$ et $\suiten[v]$ à termes strictement positifs telles que $\displaystyle\lim_{n \to +\infty} u_n = +\infty$ et $\suiten[v]$ converge vers 0. On peut affirmer que :
	\begin{enumerate}
		\item la suite $\left(\frac{1}{v_n}\right)$ converge.
		\item la suite $\left(\frac{u_n}{v_n}\right)$ converge.
		\item la suite $\suiten$ est croissante.
		\item $\displaystyle\lim_{n \to +\infty} (-u_n)^n = -\infty$
	\end{enumerate}
	\item Pour participer à un jeu, un joueur doit payer 4\,€. Il lance ensuite un dé équilibré à six faces :
	
	\begin{itemize}
		\item s'il obtient 1, il remporte 12\,€ ;
		\item s'il obtient un nombre pair, il remporte 3\,€ ;
		\item sinon, il ne remporte rien.
	\end{itemize}
	
	En moyenne, le joueur :
	\begin{enumerate}
		\item gagne $3,5$\,€.
		\item perd 3\,€.
		\item perd $1,5$\,€.
		\item perd $0,5$\,€.
	\end{enumerate}
	\item On considère la variable aléatoire $X$ suivant la loi binomiale $\mathcal{B}(3;p)$.
	
	On sait que $P(X = 0) = \frac{1}{125}$. On peut affirmer que :
	\begin{enumerate}
		\item $p=\frac15$.
		\item $P(X=1)=\frac{124}{125}$.
		\item $p=\frac45$.
		\item $P(X=1)=\frac45$.
	\end{enumerate}
\end{enumerate}