Une société de production s'interroge sur l'opportunité de programmer un jeu télévisé. Ce jeu réunit quatre candidats et se déroule en deux phases.

\smallskip

La première phase est une phase de qualification. Cette phase ne dépend que du hasard'

Pour chaque candidat, la probabilité de se qualifier est $0,6$.

\smallskip

La deuxième phase est une compétition entre les candidats qualifiés. Elle n'a lieu que si deux candidats au moins sont qualifiés. Sa durée dépend du nombre de candidats qualifiés comme l'indique le tableau ci-dessous (lorsqu'il n'y a pas de deuxième phase, on considère que sa durée est nulle).

\begin{center}
	\begin{tblr}{hlines,vlines,width=12cm,colspec={Q[5.5cm,l,m]*{5}{X[m,c]}}}
		Nombre de candidats qualifiés pour la deuxième phase	& 0 & 1 & 2 & 3 & 4 \\
		Durée de la deuxième phase en minutes					& 0 & 0 & 5 & 9 & 11 \\
	\end{tblr}
\end{center}

Pour que la société décide de retenir ce jeu, il faut que les deux conditions suivantes soient vérifiées :

\smallskip

\underline{Condition n°1} : La deuxième phase doit avoir lieu dans au moins 80\,\% des cas

\underline{Condition n°2} : La durée moyenne de la deuxième phase ne doit pas excéder 6 minutes.

\medskip

\textbf{Le jeu peut-il être retenu ?}