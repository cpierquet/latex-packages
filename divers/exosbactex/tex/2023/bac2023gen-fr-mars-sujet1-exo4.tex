On considère le cube $ABCDEFGH$ d’arête 1.

On appelle $I$ le point d’intersection du plan $(GBD)$ avec la droite $(EC)$.

\begin{center}
	\begin{tikzpicture}[x={(0:4cm)},y={(27:2cm)},z={(90:4cm)},line join=bevel]
		\coordinate (A) at (0,0,0) ; \coordinate (B) at (1,0,0) ;
		\coordinate (C) at (1,1,0) ; \coordinate (D) at (0,1,0) ;
		\coordinate (E) at (0,0,1) ; \coordinate (F) at (1,0,1) ;
		\coordinate (G) at (1,1,1) ; \coordinate (H) at (0,1,1) ;
		\coordinate (I) at ({2/3},{2/3},{1/3}) ;
		\coordinate (J) at ($(D)!0.5!(B)$) ;
		\coordinate (U) at ($(E)!-0.16!(C)$) ;%prolongement de (EC) à gauche
		\coordinate (V) at ($(E)!1.19!(C)$) ;%prolongement de (EC) à droite
		\draw[semithick,dashed] (A)--(D)--(C) (D)--(H) ;
		\draw[semithick,densely dashed] (D)--(B) (E)--(C) (D)--(G) ;
		\draw[semithick] (A)--(B)--(F)--(E)--cycle  (B)--(C)--(G)--(F)--cycle (E)--(F)--(G)--(H)--cycle (B)--(G) (U)--(E) (C)--(V) ;
		\foreach \point/\pos in {A/below,B/below,C/below,D/below,E/above,F/above,G/above,H/above,I/below left,J/below left}
			{\filldraw (\point) circle[radius=1.25pt] node[\pos] {$\point$} ;}
	\end{tikzpicture}
\end{center}

L’espace est rapporté au repère $\left(A;\vect{AB},\,\vect{AD},\,\vect{AE}\right)$.

\begin{enumerate}
	\item Donner dans ce repère les coordonnées des points $E$, $C$, $G$.
	\item Déterminer une représentation paramétrique de la droite $(EC)$.
	\item Démontrer que la droite $(EC)$ est orthogonale au plan $(GBD)$.
	\item 
	\begin{enumerate}
		\item Justifier qu’une équation cartésienne du plan $(GBD)$ est : $x + y - z - 1 = 0$.
		\item Montrer que le point $I$ a pour coordonnées $\left(\frac23;\frac23;\frac13\right)$.
		\item En déduire que la distance du point $E$ au plan $(GBD)$ est égale à $\frac{2\sqrt{3}}{3}$.
	\end{enumerate}
	\item 
	\begin{enumerate}
		\item Démontrer que le triangle $BDG$ est équilatéral.
		\item Calculer l’aire du triangle $BDG$. On pourra utiliser le point $J$, milieu du segment $[BD]$.
	\end{enumerate}
	\item Justifier que le volume du tétraèdre $EGBD$ est égal à $\frac13$.
	
	\smallskip
	
	\textit{On rappelle que le volume d’un tétraèdre est donné par : $\mathcal{V}=\frac13 \mathcal{B}h$ où $\mathcal{B}$ est l’aire d’une base du tétraèdre et $h$ est la hauteur relative à cette base.}
\end{enumerate}