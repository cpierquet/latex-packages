On considère la fonction $f$ définie sur $\mathbb{R}$ par : \[ f(x)=\frac{1}{1+\text{e}^{-3x}}. \]
%
On note $\mathcal{C}_f$ sa courbe représentative dans un repère orthogonal du plan.

On nomme $A$ le point de coordonnées $\left(0;\frac12\right)$ et $B$ le point de coordonnées $\left(1;\frac54\right)$.

On a tracé ci-dessous la courbe $\mathcal{C}_f$ et $\mathcal{T}$ la tangente à la courbe $\mathcal{C}_f$, au point d'abscisse $0$.

\begin{center}
	\begin{tikzpicture}[x=1.5cm,y=3cm,xmin=-3,xmax=4,xgrille=0.5,xgrilles=0.5,ymin=-0.75,ymax=1.5,ygrille=0.25,ygrilles=0.25]
		\GrilleTikz \AxesTikz[ElargirOx=0/0,ElargirOy=0/0]
		\draw (0,0) node[below left=1pt] {$0$} ;
		\draw[thick] (1,1.75pt)--++(0,-3.5pt) node[below] {$1$} ;
		\draw[thick] (1.75pt,1)--++(-3.5pt,0) node[left] {$1$} ;
		\clip (\xmin,\ymin) rectangle (\xmax,\ymax) ;
		\draw[thick,red,samples=250,domain=\xmin:\xmax] plot (\x,{1/(1+exp(-3*\x))}) ;
		\draw[thick,blue,samples=2] plot (\x,{0.75*\x+0.5}) ;
		\filldraw (0,0.5) circle[radius=1.5pt] node[below right] {$A$} ;
		\filldraw (1,1.25) circle[radius=1.5pt] node[below right] {$B$} ;
		\draw (3.75,1.125) node[red] {$\mathcal{C}_f$} ;
		\draw (-1.25,-0.625) node[blue] {$\mathcal{T}$} ;
	\end{tikzpicture}
\end{center}

\medskip

\textbf{Partie A : lectures graphiques}

\medskip

Dans cette partie, les résultats seront obtenus par lecture graphique. Aucune justification n'est demandée.

\begin{enumerate}
	\item Déterminer l'équation réduite de la tangente $\mathcal{T}$.
	\item Donner les intervalles sur lesquels la fonction $f$ semble convexe ou concave.
\end{enumerate}

\smallskip

\textbf{Partie B : Étude de la fonction}

\smallskip

\begin{enumerate}
	\item On admet que la fonction $f$ est dérivable sur $\mathbb{R}$.
	
	Déterminer l'expression de sa fonction dérivée $f'$.
	\item Justifier que la fonction $f$ est strictement croissante sur $\mathbb{R}$.
	\item 
	\begin{enumerate}
		\item Déterminer la limite en $+\infty$ de la fonction $f$.
		\item Déterminer la limite en $-\infty$ de la fonction $f$.
	\end{enumerate}
	\item D~terminer la valeur exacte de la solution $\alpha$ de l'équation $f(x) = 0,99$.
\end{enumerate}

\smallskip

\textbf{Partie C : Tangente et convexité}

\smallskip

\begin{enumerate}
	\item Déterminer par le calcul une équation de la tangente $\mathcal{T}$ à la courbe $\mathcal{C}_f$ au point d'abscisse $0$.
\end{enumerate}

On admet que la fonction $f$ est deux fois dérivable sur $\mathbb{R}$. On note $f''$ la fonction dérivée seconde de la fonction $f$.

On admet que $f''$ est définie sur $\mathbb{R}$ par : \[ f''(x)=\frac{9\text{e}^{-3x}\big(\text{e}^{-3x}-1\big)}{\big(1+\text{e}^{-3x}\big)^3}. \]
%
\begin{enumerate}[resume]
	\item Étudier le signe de la fonction $f''$ sur $\mathbb{R}$.
	\item 
	\begin{enumerate}
		\item Indiquer, en justifiant, sur quel(s) intervalle(s) la fonction $f$ est convexe.
		\item Que représente le point $A$ pour la courbe $\mathcal{C}_f$ ?
	\end{enumerate}
	\item En déduire la position relative de la tangente $\mathcal{T}$ et de la courbe $\mathcal{C}_f$. Justifier la réponse. 
\end{enumerate}
