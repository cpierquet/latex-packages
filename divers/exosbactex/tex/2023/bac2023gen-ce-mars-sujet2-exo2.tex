On considère la fonction $f$ définie sur $]-1,5 ; +\infty[$ par $f(x)=\ln(2x+3)-1$.

\smallskip

Le but de cet exercice est d'étudier la convergence de la suite $\left(u_n\right)$ définie par :%
\[ u_0=0 \text{ et } u_{n+1} = f{\big(u_n\big)} \text{ pour tout entier naturel } n. \]
%
\textbf{Partie A : Étude d'une fonction auxiliaire}

\medskip

On considère la fonction $g$ définie sur $]-1,5 ; +\infty[$ par $g(x) = f (x) - x$.

\begin{enumerate}
	\item Déterminer la limite de la fonction $g$ en $-1,5$.
\end{enumerate}

On admet que la limite de la fonction $g$ en $+\infty$ est $-\infty$.

\begin{enumerate}[resume]
	\item Étudier les variations de la fonction $g$ sur $]-1,5 ; +\infty[$.
	\item 
	\begin{enumerate}
		\item Démontrer que, dans l'intervalle $]-0,5 ; +\infty[$, l'équation $g(x)=0$ admet une unique solution $\alpha$.
		\item Déterminer un encadrement de $\alpha$ d'amplitude $10^{-2}$.
	\end{enumerate}
\end{enumerate}

\textbf{Partie B : Étude de la suite \boldmath$\left(u_n\right)$\unboldmath}

\medskip

On admet que la fonction $f$ est strictement croissante sur $]-1,5 ; +\infty[$.

\begin{enumerate}
	\item Soit $x$ un nombre réel. Montrer que si $x \in [-1;\alpha]$ alors $f(x) \in [-1; \alpha]$.
	\item 
	\begin{enumerate}
		\item Démontrer par récurrence que pour tout entier naturel $n$ : \[ -1 \leqslant u_n \leqslant u_{n+1} \leqslant \alpha. \]
		\item En déduire que la suite $\left(u_n\right)$ converge.
	\end{enumerate}
\end{enumerate}
