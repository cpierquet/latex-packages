\textbf{Partie A}

\medskip

On considère la suite $\left(u_n\right)_{n \in \mathbb{N}}$ définie par $u_0 = 400$ et pour tout entier naturel $n$ : $u_{n+1} = 0,9u_n + 60$.

\begin{enumerate}
	\item 
	\begin{enumerate}
		\item Calculer $u_1$ et $u_2$.
		\item Conjecturer le sens de variation de la suite $\left(u_n\right)_{n \in \mathbb{N}}$.
	\end{enumerate}
	\item Montrer, par récurrence, que pour tout entier naturel $n$, on a l'inégalité $0 \leqslant u_n \leqslant u_{n+1} \leqslant 600$.
	\item 
	\begin{enumerate}
		\item Montrer que la suite $\left(u_n\right)_{n \in \mathbb{N}}$ est convergente.
		\item Déterminer la limite de la suite $\left(u_n\right)_{n \in \mathbb{N}}$. Justifier.
	\end{enumerate}
	\item On donne une fonction écrite en langage \textsf{Python} :

\begin{CodePythonLstAlt}*[Largeur=8cm]{center}
def mystere(seuil) :
	n = 0
	u = 400
	while u <= seuil :
		n = n + l
		u = 0.9*u + 60
	return n
\end{CodePythonLstAlt}
	Quelle valeur obtient-on en tapant dans la console de \textsf{Python} : \texttt{mystere(500)} ?
\end{enumerate}

\medskip

\textbf{Partie B}

\medskip

Un arboriculteur possède un verger dans lequel il a la place de cultiver au maximum 500 arbres.

Chaque année il vend 10\,\% des arbres de son verger et puis il replante 60 nouveaux arbres. Le verger compte 400 arbres en 2023.

\smallskip

L'arboriculteur pense qu'il pourra continuer à vendre et à planter les arbres au même rythme pendant les années à venir.

\medskip

Va-t-il être confronté à un problème de place dans son verger ? Expliquer votre réponse.