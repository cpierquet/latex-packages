\emph{Cet exercice est un questionnaire à choix multiples.\\ Pour chacune des questions suivantes, une seule des quatre réponses proposées est exacte.\\ Pour répondre, indiquer sur la copie le numéro de la question et la lettre de la réponse choisie.\\ Aucune justification n'est demandée.\\ Une réponse fausse, une absence de réponse, ou une réponse multiple, ne rapporte ni n'enlève de point.}

\smallskip

\begin{enumerate}
	\item On considère la fonction $f$ définie sur $\R$ par : $f(x) = (x+1)\,\e^x$.
	
	Une primitive $F$ de $f$ sur $\R$ est définie par :
	
	\medskip
	
	\begin{tblr}{width=\linewidth,colspec={*{2}{X[m,l]}}}
		\textbf{a.}~~$F(x)=1+x\,\e^{x}$ &\textbf{b.}~~$F(x)=(1+x)\,\e^{x}$ \\
		\textbf{c.}~~$F(x)=(2+x)\,\e^{x}$&\textbf{d.}~~$F(x)=\left(\frac{x^2}{2}+x\right)\,\e^x$
	\end{tblr}
\end{enumerate}

\begin{center}
	$\ast\ast$
	
	\vspace*{-0.5\baselineskip}
	
	$\ast$
\end{center}

Dans toute la suite de l'exercice, on se place dans l'espace muni d'un repère orthonormé $\Rijk$.

\begin{enumerate}[resume]
	\item On considère les droites $(d_1)$ et $(d_2)$ dont des représentations paramétriques sont respectivement :%
	\[ (d_1) \quad \begin{dcases}x=2+r\\y=1+r\\z=-r\end{dcases} \quad (r \in \R) \text{ ; } \qquad (d_2) \quad \begin{dcases}x=1-s\\y=-1+s\\z=2-s\end{dcases} \quad (s \in \R). \]
	Les droites $(d_1)$ et $(d_2)$ sont :
	
	\medskip
	
	\begin{tblr}{width=\linewidth,colspec={*{2}{X[m,l]}}}
		\textbf{a.}~~sécantes. &\textbf{b.}~~strictement parallèles. \\
		\textbf{c.}~~confondues. &\textbf{d.}~~non coplanaires.
	\end{tblr}
	\item On considère le plan $(P)$ dont une équation cartésienne est : $2x-y+z-1=0$.
	
	On considère la droite $(\Delta)$ dont une représentation paramétrique est :%
	\[ \begin{dcases}x=2+u\\y=4+u\\z=1-u\end{dcases} \quad (u \in \R). \]
	La droite $(\Delta)$ est :
	
	\medskip
	
	\begin{tblr}{width=\linewidth,colspec={*{2}{X[m,l]}}}
		\textbf{a.}~~sécante et non orthogonale au plan $(P)$. &
		\textbf{b.}~~incluse dans le plan $(P)$. \\
		\textbf{c.}~~strictement parallèle au plan $(P)$. &
		\textbf{d.}~~orthogonale au plan $(P)$.
	\end{tblr}
	\item On considère le plan $(P_1)$ dont une équation cartésienne est $x - 2y + z +1= 0$, ainsi que le plan $(P_2)$ dont une équation cartésienne est $2x + y +z-6= 0$.
	
	Les plans $(P_1)$ et $(P_2)$ sont :
	
	\medskip
	
	\begin{tblr}{width=\linewidth,colspec={*{2}{X[m,l]}}}
		\textbf{a.}~~sécants et perpendiculaires. &
		\textbf{b.}~~confondus. \\
		\textbf{c.}~~sécants et non perpendiculaires. &
		\textbf{d.}~~strictement parallèles.
	\end{tblr}
	\item On considère les points $E(1;2;1)$, $F(2;4;3)$ et $G(-2;2;5)$.
	
	On peut affirmer que la mesure $\alpha$ de l'angle $\widehat{FEG}$ vérifie :
	
	\medskip
	
	\begin{tblr}{width=\linewidth,colspec={*{4}{X[m,l]}}}
		\textbf{a.}~~$\alpha = \ang{90}$ &
		\textbf{b.}~~$\alpha > \ang{90}$ &
		\textbf{c.}~~$\alpha = \ang{0}$ &
		\textbf{d.}~~$\alpha \approx \ang{71}$
	\end{tblr}
\end{enumerate}
