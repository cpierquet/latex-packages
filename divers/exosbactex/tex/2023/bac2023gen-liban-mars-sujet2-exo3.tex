L'espace est muni d'un repère orthonormé $\left(O;\vect{\imath},\vect{\jmath},\vect{k}\right)$

On considère les points $A(3;0;1)$, $B(2;1;2)$ et $C(-2;-5;1)$.

\begin{enumerate}
	\item Démontrer que les points $A$, $B$ et $C$ ne sont pas alignés.
	\item Démontrer que le triangle $ABC$ est rectangle en $A$.
	\item Vérifier que le plan $(ABC)$ a pour équation cartésienne : \[ -x + y - 2 z + 5 = 0. \]
	\item On considère le point $S(1;-2;4)$.
	
	Déterminer la représentation paramétrique de la droite $(\Delta)$, passant par $S$ et orthogonale au plan $(ABC)$.
	\item On appelle $H$ le point d'intersection de la droite $(\Delta)$ et du plan $(ABC)$.
	
	Montrer que les coordonnées de $H$ sont $(0;-1;2)$.
	\item Calculer la valeur exacte de la distance $SH$.
	\item On considère le cercle $\mathcal{C}$, inclus dans le plan $(ABC)$, de centre $H$, passant par le point $B$. On appelle $\mathcal{D}$ le disque délimité par le cercle $\mathcal{C}$.
	
	Déterminer la valeur exacte de l'aire du disque $\mathcal{D}$.
	\item En déduire la valeur exacte du volume du cône de sommet $S$ et de base le disque $\mathcal{D}$.
\end{enumerate}
