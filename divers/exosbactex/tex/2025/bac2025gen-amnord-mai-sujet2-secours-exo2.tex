\textit{Pour chacune des affirmations suivantes, préciser si elle est vraie ou fausse puis justifier la réponse donnée. Toute réponse non argumentée ne sera pas prise en compte.}

\medskip

$\Rijk$ est un repère de l'espace.

On considère la droite $D$ qui a pour représentation paramétrique $\begin{dcases}x=\phantom{-}3-\phantom{3}t\\y=-2 + 3t\\z=\phantom{-}1 + 4t\end{dcases}$, $t \in \R$ et le plan $P$ qui a pour équation cartésienne : $2x -3y + z - 6 = 0$.

\begin{enumerate}
	\item \textbf{Affirmation :} 
	La droite $D'$, qui a pour représentation paramétrique $\begin{dcases}x=2+2 t\\y=4-6 t \\z=9-8 t\end{dcases}$, $t \in \R$, est parallèle à la droite $D$.
	\item On admet que les points $A(-2;3;1)$, $B(1;3;-4)$ et $C(6;3;9)$ ne sont pas alignés.
	
	\textbf{Affirmation :} La droite $D$ est orthogonale au plan défini par les trois points $A$, $B$ et $C$.	
	\item \textbf{Affirmation :} La droite $D$ est sécante avec la droite $\Delta$ qui a pour représentation paramétrique :\[\begin{dcases}x=-4+2 t' \\ y=\phantom{-}1-3 t' \\ z=\phantom{-}2+\phantom{2}t'\end{dcases},~t' \in \R.\]
	\item \textbf{Affirmation :} Le point $F(-3;-3;3)$ est le projeté orthogonal du point $E(-5;0;2)$ sur le plan $P$.
	\item \textbf{Affirmation :} Il existe exactement une valeur du paramètre réel $a$ telle que le plan $P'$ d'équation ${-3x + y - a^{2} z+3=0}$ soit parallèle à la droite $D$.
\end{enumerate}