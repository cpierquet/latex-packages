Au basket-ball, il est possible de marquer des paniers rapportant un point, deux points ou trois points.

\textit{Les \textbf{PARTIES A} et \textbf{B} sont indépendantes.}

\medskip

\textbf{PARTIE A}

\medskip

L'entraineur d'une équipe de basket décide d'étudier les statistiques de réussite des lancers de ses joueurs. Il constate qu'à l'entrainement, lorsque Victor tente un panier à trois points, il le réussit avec une probabilité de $0,32$.
Lors d'un entrainement, Victor effectue une série de 15 lancers à trois points. On suppose que ces lancers sont indépendants.

\smallskip

On note $N$ la variable aléatoire qui donne le nombre de paniers marqués.

\smallskip

\textit{Les résultats des probabilités demandées seront, si nécessaire, arrondis au millième.}

\begin{enumerate}
	\item On admet que la variable aléatoire $N$ suit une loi binomiale. Préciser ses paramètres.
	\item Calculer la probabilité que Victor réussisse exactement 4 paniers lors de cette série.
	\item Déterminer la probabilité que Victor réussisse au plus 6 paniers lors de cette série.
	\item Déterminer l'espérance de la variable aléatoire $N$.
	\item On note $T$ la variable aléatoire qui donne le nombre de points marqués après cette série de lancers.
	\begin{enumerate}
		\item Exprimer $T$ en fonction de $N$.
		\item En déduire l'espérance de la variable aléatoire $T$. Donner une interprétation de cette valeur dans le contexte de l'exercice.
		\item Calculer $P(12 \leqslant T \leqslant 18)$.
	\end{enumerate}
\end{enumerate}

\textbf{PARTIE B}

\medskip

On note $X$ la variable aléatoire donnant le nombre de points marqués par Victor lors d'un match.

\smallskip

On admet que l'espérance $\Esper{X}=22$ et la variance $\Varianc{X}=65$.

\smallskip

Victor joue $n$ matchs, où $n$ est un nombre entier strictement positif.
On note $X_{1}$, $X_{2}$, \ldots, $X_{n}$ les variables aléatoires donnant le nombre de points marqués au cours des 1\up{er}, 2\up{e}, \ldots, $n$-ième matchs. On admet que les variables aléatoires $X_{1}$, $X_{2}$, \ldots, $X_{n}$ sont indépendantes et suivent la même loi que celle de $X$.

\smallskip

On pose $M_{n}=\frac{X_{1}+X_{2}+\ldots+X_{n}}{n}$.

\begin{enumerate}
	\item Dans cette question, on prend $n=50$.
	\begin{enumerate}
		\item Que représente la variable aléatoire $M_{50}$ ?
		\item Déterminer l'espérance et la variance de $M_{50}$.
		\item Démontrer que $P\big(\left|M_{50}-22\right| \geqslant 3\big) \leqslant \frac{13}{90}$.
		\item En déduire que la probabilité de l'événement « $19 < M_{50} < 25$ » est strictement supérieure à $0,85$.
	\end{enumerate}
	\item Indiquer, en justifiant, si l'affirmation suivante est vraie ou fausse :
	
	« Il n'existe aucun entier naturel $n$ tel que $P\big(\left|M_{n}-22\right| \geqslant 3\big)< 0,01$ ».
\end{enumerate}