\textit{Pour chacune des affirmations suivantes, indiquer si elle est vraie ou fausse. Chaque réponse doit être justifiée. Une réponse non justifiée ne rapporte aucun point.}

\medskip

L'espace est rapporté à un repère orthonormé $\Rijk$.

\smallskip

On considère la droite $(d)$ dont une représentation paramétrique est : \[ \begin{cases} x = 3 - 2t \\ y = -1 \\ z = 2 - 6t\end{cases}, \text{ où } t \in \R.\]
%
On considère également les points suivants :

\begin{itemize}[leftmargin=1.5cm]
	\item $A(3; -3; -2)$
	\item $B(5; -4; -1)$
	\item $C$ le point de la droite $(d)$ d'abscisse 2
	\item $H$ le projeté orthogonal du point B sur le plan $\mathcal{P}$ d'équation $x + 3z - 7 = 0$
\end{itemize}

\textbf{Affirmation 1}

La droite $(d)$ et l'axe des ordonnées sont deux droites non coplanaires.

\medskip

\textbf{Affirmation 2}

Le plan passant par $A$ et orthogonal à la droite $(d)$ a pour équation cartésienne : \[ x + 3z + 3 = 0 \]

\textbf{Affirmation 3}

Une mesure, exprimée en radian, de l'angle géométrique $\widehat{BAC}$ est $\dfrac{\pi}{6}$.

\medskip

\textbf{Affirmation 4}

La distance $BH$ est égale à $\dfrac{\sqrt{10}}{2}$.