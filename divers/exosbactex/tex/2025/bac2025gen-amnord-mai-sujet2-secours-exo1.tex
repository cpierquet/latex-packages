On considère la fonction $f$ définie sur $\R$ par \[f(x)=x \e^{-x}+2 x-1.\]

On admet que la fonction $f$ est deux fois dérivable sur $\R$.

On appelle $\mathcal{C}_{f}$ sa courbe représentative dans un repère orthogonal du plan.

On note $f'$ la fonction dérivée de la fonction $f$ et $f''$ la fonction dérivée seconde de $f$, c'est-à-dire la fonction dérivée de la fonction $f'$.

\medskip

\textbf{Partie A: Étude de la fonction $\bm{f}$}

\begin{enumerate}
	\item Déterminer les limites de la fonction $f$ en $-\infty$ et en $+\infty$.
	\item Pour tout réel $x$, calculer $f'(x)$.
	\item Montrer que pour tout réel $x$ : \[f''(x)= (x-2) \e^{-x}\]
	\item Étudier la convexité de la fonction $f$.
	\item Étudier les variations de la fonction $f'$ sur $\R$, puis dresser son tableau de variations en y faisant apparaître la valeur exacte de l'extremum.
	
	Les limites de la fonction $f'$ aux bornes de l'intervalle de définition ne sont pas attendues.
	
	\item En déduire le signe de la fonction $f'$ sur $\R$, puis justifier que la fonction $f$ est strictement croissante sur $\R$.
	\item Justifier qu'il existe un unique réel $\alpha$ tel que $f(\alpha)=0$.
	
	Donner un encadrement de $\alpha$, au centième près.
	\item On considère la droite $\Delta$ d'équation $y=2 x-1$.
	
	Étudier la position relative de la courbe $\mathcal{C}_{f}$ par rapport à la droite $\Delta$.
\end{enumerate}

\textbf{Partie B : Calcul d'aire}

\medskip

Soit $n$ un entier naturel non nul. On considère l'aire du domaine $D_{n}$ délimité par la courbe $\mathcal{C}_{f}$, la droite $\Delta$ et les droites d'équations respectives $x=1$ et $x=n$. On note \[ I_{n}=\int_{1}^{n} x \e^{-x} \dx.\]
%
\begin{enumerate}
	\item À l'aide d'une intégration par parties, exprimer $I_{n}$ en fonction de $n$.
	\item 
	\begin{enumerate}
		\item Justifier que l'aire du domaine $D_{n}$ est $I_{n}$.
		
		\item Calculer la limite de l'aire du domaine $D_{n}$ quand $n$ tend vers $+\infty$.
	\end{enumerate}
\end{enumerate}