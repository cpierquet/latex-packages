%----tkz-grapheur + ProfLycee (ecritures)----

\textit{La partie C est indépendante des parties A et B.}

\begin{center}
	\textbf{Partie A}
\end{center}

On donne ci-dessous, dans un repère orthogonal, les courbes $\mathcal{C}_{1}$ et $\mathcal{C}_{2}$, représentations graphiques de deux fonctions définies et dérivables sur $\R$. L'une des deux fonctions représentées est la fonction dérivée de l'autre. On les notera $g$ et $g^{\prime}$.

On précise également que :

\begin{itemize}
	\item La courbe $\mathcal{C}_{1}$ coupe l'axe des ordonnées au point de coordonnées $(0; 1)$.
	\item La courbe $\mathcal{C}_{2}$ coupe l'axe des ordonnées au point de coordonnées $(0; 2)$ et l'axe des abscisses aux points de coordonnées $(-2; 0)$ et $(1; 0)$.
\end{itemize}

\begin{Centrage}
	\begin{GraphiqueTikz}[x=1.9cm,y=0.6cm,Xmin=-3,Xmax=5,Xgrille=1,Xgrilles=1,Ymin=-8,Ymax=7,Ygrille=1,Ygrilles=1]
		\tikzset{pflgrillep/.style={thin,gray,densely dotted}}
		\tikzset{pflgrilles/.style={thin,gray,densely dotted}}
		\TracerAxesGrilles[Police=\small,Elargir=2.5mm]{auto}{auto}
		\TracerCourbe[Couleur=red]{(x^2+3*x+1)*exp(-x)}
		\begin{scope}
			\tikzset{pflcourbe/.append style={densely dashed}}
			\TracerCourbe[Couleur=blue]{(-x^2-x+2)*exp(-x)}
		\end{scope}
		\PlacerTexte[Police=\large,Couleur=red]{(2.25,2)}{$\mathcal{C}_{1}$}
		\PlacerTexte[Police=\large,Couleur=blue]{(2.25,-1.5)}{$\mathcal{C}_{2}$}
	\end{GraphiqueTikz}
\end{Centrage}

\begin{enumerate}
	\item En justifiant, associer à chacune des fonctions $g$ et $g^{\prime}$ sa représentation graphique.
	\item Justifier que l'équation réduite de la tangente à la courbe représentative de la fonction $g$ au point d'abscisse 0 est $y = 2x + 1$.
\end{enumerate}

\begin{center}
	\textbf{Partie B}
\end{center}

On considère $(E)$ l'équation différentielle $y + y^{\prime} = (2x + 3)\e^{-x}$, où $y$ est une fonction de la variable réelle $x$.

\begin{enumerate}
	\item Montrer que la fonction $f_{0}$ définie pour tout nombre réel $x$ par $f_{0}(x) = (x^{2} + 3x)e^{-x}$ est une solution particulière de l'équation différentielle $(E)$.
	\item Résoudre l'équation différentielle $(E_{0})$ : $y + y^{\prime} = 0$.
	\item Déterminer les solutions de l'équation différentielle $(E)$.
	\item On admet que la fonction $g$ décrite dans la \textbf{partie A} est une solution de l'équation différentielle $(E)$. Déterminer alors l'expression de la fonction $g$.
	\item Déterminer les solutions de l'équation différentielle $(E)$ dont la courbe admet exactement deux points d'inflexion.
\end{enumerate}

\begin{center}
	\textbf{Partie C}
\end{center}

On considère la fonction f définie pour tout nombre réel $x$ par : \[ f(x) = (x^{2} + 3x + 2)e^{-x}. \]
%
\begin{enumerate}
	\item Démontrer que la limite de la fonction $f$ en $+\infty$ est égale à 0.
	
	On admet par ailleurs que la limite de la fonction $f$ en $-\infty$ est égale à $+\infty$.
	\item On admet que la fonction $f$ est dérivable sur $\R$. On note $f^{\prime}$ la fonction dérivée de $f$ sur $\R$.
	\begin{enumerate}
		\item Vérifier que, pour tout nombre réel $x$, $f^{\prime}(x) = (-x^{2} - x + 1)\e^{-x}$.
		\item Déterminer le signe de la fonction dérivée $f^{\prime}$ sur $\R$ puis en déduire les variations de la fonction $f$ sur $\R$.
	\end{enumerate}
	\item Expliquer pourquoi la fonction $f$ est positive sur l'intervalle $\IntervalleFO{0}{+\infty}$.
	\item On notera $\mathcal{C}_{f}$ la courbe représentative de la fonction $f$ dans un repère orthogonal $\Rij$. On admet que la fonction $F$ définie pour tout nombre réel $x$ par $F(x) = (-x^{2} - 5x - 7)\e^{-x}$ est une primitive de la fonction $f$.
	
	Soit $\alpha$ un nombre réel positif.
	
	Déterminer l'aire $\mathcal{A}(\alpha)$, exprimée en unité d'aire, du domaine du plan délimité par l'axe des abscisses, la courbe $\mathcal{C}_{f}$ et les droites d'équation $x = 0$ et $x = \alpha$.
\end{enumerate}