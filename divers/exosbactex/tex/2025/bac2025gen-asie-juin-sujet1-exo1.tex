L'espace est rapporté à un repère orthonormé $\Rijk$. On considère :

\begin{itemize}
	\item $\alpha$ un réel quelconque;
	\item les points $A(1;1;0)$, $B(2;1;0)$ et $C(\alpha;3;\alpha)$;
	\item $(d)$ la droite dont une représentation paramétrique est : $\begin{dcases}x=1+t \\ y=2 t \\ z=-t\end{dcases}$, $t \in \R$.
\end{itemize}

Pour chacune des affirmations suivantes, préciser si elle est vraie ou fausse, puis justifier la réponse donnée. Une réponse non argumentée ne sera pas prise en compte.

\bigskip

\textbf{Affirmation 1 :}

\begin{itemize}
	\item[] Pour toutes les valeurs de $\alpha$, les points $A$, $B$ et $C$ définissent un plan et un vecteur normal à ce plan est $\vect{\jmath}\begin{pmatrix}0\\1\\0\end{pmatrix}$.
\end{itemize}

\medskip

\textbf{Affirmation 2 :}

\begin{itemize}
	\item[] Il existe exactement une valeur du réel $\alpha$ telle que les droites $(A C)$ et $(d)$ sont parallèles.
\end{itemize}

\medskip

\textbf{Affirmation 3 :}

\begin{itemize}
	\item[] Une mesure de l'angle $\widehat{OAB}$ est $135^{\circ}$.
\end{itemize}

\medskip

\textbf{Affirmation 4 :}

\begin{itemize}
	\item[] Le projeté orthogonal du point $A$ sur la droite (d) est le point $H$ de coordonnées : $H(1;2;2)$.
\end{itemize}

\medskip

\textbf{Affirmation 5 :}

\begin{itemize}
	\item[] La sphère de centre $O$ et de rayon 1 rencontre la droite (d) en deux points distincts. On rappelle que la sphère de centre $\Omega$ et de rayon $r$ est l'ensemble des points de l'espace situés à une distance $r$ de $\Omega$.
\end{itemize}