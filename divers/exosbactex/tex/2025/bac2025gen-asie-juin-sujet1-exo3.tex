%\usepackage{scontents}
%script python
\begin{scontents}[overwrite,write-out=asie2025j1exo3.py]
def mystere(k) :
	n = 1
	s = 2
	while s < k :
		n = n + 1
		s = 10 - 40/n + (40*0.8**n)/n
\end{scontents}

%===

Un patient doit prendre toutes les heures une dose de 2 mL d'un médicament. On introduit la suite $\left(u_{n}\right)$ telle que le terme $u_{n}$ représente la quantité de médicament, exprimée en mL, présente dans l'organisme immédiatement après $n$ prises de médicament.

On a $u_{1}=2$ et pour tout entier naturel $n$ strictement positif : $u_{n+1}=2+0,8 u_{n}$.

\medskip

\textbf{Partie A}

\medskip

En utilisant ce modèle, un médecin cherche à savoir à partir de combien de prises du médicament la quantité présente dans l'organisme du patient est strictement supérieure à 9~mL.

\begin{enumerate}
	\item Calculer la valeur $u_{2}$.
	\item Montrer, par récurrence sur $n$, que $u_{n}=10-8 \times 0,8^{n-1}$ pour tout entier naturel $n$ strictement positif.
	\item Déterminer $\lim\limits_{n \to +\infty} u_{n}$ et donner une interprétation de ce résultat dans le contexte de l'exercice.
	\item Soit $N$ un entier naturel strictement positif, l'inéquation $u_{N} \geqslant 10$ admet-elle des solutions ?
	
	Interpréter le résultat de cette question dans le contexte de l'exercice.
	\item Déterminer à partir de combien de prises de médicament la quantité de médicament présente dans l'organisme du patient est strictement supérieure à 9~mL. Justifier votre démarche.
\end{enumerate}

\textbf{Partie B}

\medskip

En utilisant la même modélisation, le médecin s'intéresse à la quantité moyenne de médicament présente dans l'organisme du malade au cours du temps. On définit pour cela la suite $\left(S_{n}\right)$ définie pour tout entier naturel $n$ strictement positif par \[ S_{n}=\frac{u_{1}+u_{2}+\cdots+u_{n}}{n}. \]
%
On admet que la suite $\left(S_{n}\right)$ est croissante.

\begin{enumerate}
	\item Calculer $S_{2}$.
	\item Montrer que, pour tout entier naturel $n$ strictement positif, \[ u_{1}+u_{2}+\cdots+u_{n}=10 n-40+40 \times 0,8^{n}. \]
	\item Calculer $\lim\limits_{n \to +\infty} S_{n}$.
	\item On donne la fonction \AffVignette[Type=py]{mystere} suivante, écrite en langage \textsf{Python}.
	
	\CodePythonLstFichierAlt[9cm]{center}{asie2025j1exo3.py}
	
	Dans le contexte de l'énoncé, que représente la valeur renvoyée par la saisie \AffVignette[Type=py]{mystere(9)} ?
	\item Justifier que cette valeur est strictement supérieure à 10.
\end{enumerate}