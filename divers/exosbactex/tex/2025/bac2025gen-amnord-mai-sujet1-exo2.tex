%----scontents + ProfLycee (ecritures) + customenvs

\begin{scontents}[overwrite,write-out=an2025j1exo2.py]
def algo(p) :
	u = 2
	n = 0
	while u-1 > p :
		u = (2*u-1) / (u+2)
	return (n, u)
\end{scontents}

On considère la suite numérique $\Suite{u}$ définie par son premier terme $u_{0}=2$ et pour tout entier naturel $n$, par : \[ u_{n+1}=\frac{2 u_{n}+1}{u_{n}+2}. \]
%
On admet que la suite $\Suite{u}$ est bien définie.

\begin{enumerate}
	\item Calculer le terme $u_{1}$.
	\item On définit la suite $\Suite{a}$ pour tout entier naturel $n$, par : \[ a_{n}=\frac{u_{n}}{u_{n}-1}. \]
	%
	On admet que la suite $\Suite{a}$ est bien définie.
	\begin{enumerate}
		\item Calculer $a_{0}$ et $a_{1}$.
		\item Démontrer que, pour tout entier naturel n, $a_{n+1}=3 a_{n}-1$.
		\item Démontrer par récurrence que, pour tout entier naturel $n$ supérieur ou égal à 1, $a_{n} \geqslant 3 n-1$
		\item En déduire la limite de la suite $\Suite{a}$.
	\end{enumerate}
	\item On souhaite étudier la limite de la suite $\Suite{u}$.
	\begin{enumerate}
		\item Démontrer que pour tout entier naturel $n$, $u_{n}=\frac{a_{n}}{a_{n}-1}$.
		\item En déduire la limite de la suite $\Suite{u}$.
	\end{enumerate}
	\item On admet que la suite $\Suite{u}$ est décroissante.
	
	On considère le programme suivant écrit en langage \AffVignette[Type=py]{Python} :
	
	\CodePythonLstFichierAlt[0.75\linewidth]{center}{an2025j1exo2.py}
	\begin{enumerate}
		\item Interpréter les valeurs \AffVignette[Type=py]{n} et \AffVignette[Type=py]{u} renvoyées par l'appel de la fonction \AffVignette[Type=py]{algo(p)} dans le contexte de l'exercice.
		\item Donner, sans justifier, la valeur de \AffVignette[Type=py]{n} pour \AffVignette[Type=py]{p = 0.001}.
	\end{enumerate}
\end{enumerate}