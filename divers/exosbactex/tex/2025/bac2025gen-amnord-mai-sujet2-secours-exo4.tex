L'objectif de cet exercice est d'étudier la suite $\left(u_{n}\right)$ définie pour tout entier naturel $n$ par : \[\begin{dcases}u_{0}=0\\u_{1}=\dfrac{1}{2}\\u_{n+2}=u_{n+1}-\dfrac{1}{4}u_{n}\end{dcases}.\]

\textbf{Partie A : Conjecture}

\begin{enumerate}
	\item Recopier et compléter le tableau ci-dessous.
	
	\textit{Aucune justification n'est demandée.}
	
	\begin{Centrage}
		\begin{tblr}{width=0.8\linewidth,hlines,vlines,colspec={*{7}{X[m,c]}}}
			$n$ & 0 & 1 & 2 & 3 & 4 & 5 \\
			$u_{n}$ & 0 & $\frac{1}{2}$ & $\frac{1}{2}$ & & & \\
		\end{tblr}
	\end{Centrage}
	\item Conjecturer la limite de la suite $\left(u_{n}\right)$.
\end{enumerate}

\textbf{Partie B : Étude d'une suite auxiliaire}

\medskip

Soit $\left(w_{n}\right)$ la suite définie pour tout entier naturel $n$ par : \[w_{n}=u_{n+1}-\dfrac{1}{2} u_{n}.\]

\begin{enumerate}
	\item Calculer $w_{0}$.
	\item Démontrer que la suite $\left(w_{n}\right)$ est géométrique de raison $\dfrac{1}{2}$.
	\item Pour tout entier naturel $n$, exprimer $w_{n}$ en fonction de $n$.
	\item Montrer que pour tout entier naturel $n$, on a : \[u_{n+1}=\left(\dfrac{1}{2}\right)^{n+1}+\dfrac{1}{2} u_{n}.\]
	\item Démontrer par récurrence que, pour tout $n \in \N$, $ u_{n}=n\left(\dfrac{1}{2}\right)^{n}$.
\end{enumerate}

\textbf{Partie C : Étude de la suite $\bm{\left(u_{n}\right)}$}

\begin{enumerate}
	\item Montrer que la suite $\left(u_{n}\right)$ est décroissante à partir du rang $n = 1$.
	\item En déduire que la suite $\left(u_{n}\right)$ est convergente sans chercher à calculer la valeur de la limite.
	\item On admet que la limite de la suite $\left(u_{n}\right)$ est solution de l'équation : $\ell=\ell-\dfrac{1}{4} \ell$.
	
	Déterminer la limite de la suite $\left(u_{n}\right)$.
\end{enumerate}