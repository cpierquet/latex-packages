On considère $f$ la fonction définie sur l'intervalle $\IntervalleOO{0}{+\infty}$ par $f(x)=\dfrac{\e^{\sqrt{x}}}{2 \sqrt{x}}$ et on appelle $C_{f}$ sa courbe représentative dans un repère orthonormé.

\begin{enumerate}
	\item On définit la fonction $g$ sur l'intervalle $\IntervalleOO{0}{+\infty}$ par $g(x)=\e^{\sqrt{x}}$.
	\begin{enumerate}
		\item Montrer que $g^{\prime}(x)=f(x)$ pour tout $x$ de l'intervalle $\IntervalleOO{0}{+\infty}$.
		\item Pour tout réel $x$ de l'intervalle $\IntervalleOO{0}{+\infty}$, calculer $f^{\prime}(x)$ et montrer que :\[ f^{\prime}(x)=\frac{\mathrm{e}^{\sqrt{x}}(\sqrt{x}-1)}{4 x \sqrt{x}}. \]
	\end{enumerate}
	\item
	\begin{enumerate}
		\item Déterminer la limite de la fonction $f$ en 0.
		\item Interpréter graphiquement ce résultat.
	\end{enumerate}
	\item
	\begin{enumerate}
		\item Déterminer la limite de la fonction $f$ en $+\infty$.
		\item Étudier le sens de variation de la fonction $f$ sur $\IntervalleOO{0}{+\infty}$.
		
		Dresser le tableau de variations de la fonction $f$ en y faisant figurer les limites aux bornes de l'intervalle de définition.
		\item Démontrer que l'équation $f(x)=2$ admet une unique solution sur l'intervalle $\IntervalleFO{1}{+\infty}$ et donner une valeur approchée à $10^{-1}$ près de cette solution.
	\end{enumerate}
	\item On pose $I=\displaystyle\int_{1}^{2} f(x) \dx$.
	\begin{enumerate}
		\item Calculer $I$.
		\item Interpréter graphiquement le résultat obtenu.
	\end{enumerate}
	\item On admet que la fonction $f$ est deux fois dérivable sur l'intervalle $\IntervalleOO{0}{+\infty}$ et que : \[ f^{\prime \prime}(x)=\frac{\mathrm{e}^{\sqrt{x}}(x-3 \sqrt{x}+3)}{8 x^{2} \sqrt{x}} \]
	\begin{enumerate}
		\item En posant $X=\sqrt{x}$, montrer que $x-3 \sqrt{x}+3>0$ pour tout réel $x$ de l'intervalle $\IntervalleOO{0}{+\infty}$.
		\item Étudier la convexité de la fonction $f$ sur l'intervalle $\IntervalleOO{0}{+\infty}$.
	\end{enumerate}
\end{enumerate}