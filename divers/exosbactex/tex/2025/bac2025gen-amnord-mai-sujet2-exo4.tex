%----tkz-grapheur----
On désigne par $f$ la fonction définie sur l'intervalle $\IntervalleFF{0}{\pi}$ par $f(x)=\e^{x}\,\sin (x)$.

On note $\mathcal{C}_{f}$ la courbe représentative de $f$ dans un repère.

\medskip

\textbf{PARTIE A}

\begin{enumerate}
	\item
	\begin{enumerate}
		\item Démontrer que pour tout réel $x$ de l'intervalle $\IntervalleFF{0}{\pi}$, $f^{\prime}(x)=\e^{x}\,(\sin (x)+\cos (x))$.
		\item Justifier que la fonction $f$ est strictement croissante sur l'intervalle $\IntervalleFF{0}{\frac{\pi}{2}}$.
	\end{enumerate}
	\item
	\begin{enumerate}
		\item Déterminer une équation de la tangente $T$ à la courbe $\mathcal{C}_{f}$ au point d'abscisse 0 .
		\item Démontrer que la fonction $f$ est convexe sur l'intervalle $\IntervalleFF{0}{\frac{\pi}{2}}$.
		\item En déduire que pour tout réel $x$ de l'intervalle $\IntervalleFF{0}{\frac{\pi}{2}}$, $\e^{x}\,\sin (x) \geqslant x$.
	\end{enumerate}
	\item Justifier que le point d'abscisse $\frac{\pi}{2}$ de la courbe représentative de la fonction $f$ est un point d'inflexion.
\end{enumerate}

\textbf{PARTIE B}

\medskip

On note $I=\int_{0}^{\frac{\pi}{2}} \e^{x}\,\sin (x) \dx$ et $J=\int_{0}^{\frac{\pi}{2}} \e^{x}\,\cos(x) \dx$.

\begin{enumerate}
	\item En intégrant par parties l'intégrale I de deux manières différentes, établir les deux relations suivantes : $\quad I=1+J \quad$ et $\quad I=\e^{\frac{\pi}{2}}-J$.
	\item En déduire que $I=\frac{1+\e^{\frac{\pi}{2}}}{2}$.
	\item On note $g$ la fonction définie sur $\R$ par $g(x)=x$.
	
	Les courbes représentatives des fonctions $f$ et $g$ sont tracées dans le repère orthogonal ci-dessous sur l'intervalle $\IntervalleFF{0}{\pi}$.
	
	Calculer la valeur exacte de l'aire du domaine hachuré situé entre les courbes $\mathcal{C}_{f}$ et $\mathcal{C}_{g}$ et les droites d'équation $x=0$ et $x=\frac{\pi}{2}$.
	
	\begin{Centrage}
		\begin{GraphiqueTikz}[x=1cm,y=0.5cm,Xmin=-1.25,Xmax=4,Xgrille=1,Xgrilles=1,Ymin=-1.5,Ymax=8.75,Ygrille=1,Ygrilles=1]
			\TracerAxesGrilles[Grille=false,Police=\small,Elargir=2.5mm]{-1,1,2,3,4}{2,4,6,8}
			\draw (-2pt,-2pt) node[below left] {$0$} ;
			\DefinirCourbe[Nom=cf]< f >{exp(x)*sin(x)}
			\DefinirCourbe[Nom=cg]< g >{x}
			\TracerIntegrale[Type=fct/fct,Style=hachures,Couleur=darkgray]{f(x)}[g(x)]{0}{pi/2}
			\draw[darkgray,pfltraitantec] ({pi/2},{pi/2})--({pi/2},0) node[below,font=\small] {$\frac{\pi}{2}$} ;
			\TracerCourbe[Couleur=red,Debut=0,Fin=pi]{f(x)}
			\TracerCourbe[Couleur=blue,Debut=0,Fin=pi]{g(x)}
			\PlacerTexte[Police=\large,Couleur=red]{(3,7)}{$\mathcal{C}_{f}$}
			\PlacerTexte[Police=\large,Couleur=blue]{(2.25,3)}{$\mathcal{C}_{g}$}
		\end{GraphiqueTikz}
	\end{Centrage}
\end{enumerate}