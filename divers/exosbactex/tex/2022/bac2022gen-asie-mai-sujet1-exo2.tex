Un médicament est administré à un patient par voie intraveineuse. 

\bigskip

\textbf{Partie A : modèle discret de la quantité médicamenteuse}

\medskip

Après une première injection de 1 mg de médicament, le patient est placé sous perfusion.

On estime que, toutes les $30$~minutes, l'organisme du patient élimine 10\,\% de la quantité de médicament présente dans le sang et qu'il reçoit une dose supplémentaire de $0,25$ mg de la substance médicamenteuse.

On étudie l'évolution de la quantité de médicament dans le sang avec le modèle suivant :

pour tout entier naturel $n$, on note $u_n$ la quantité, en mg, de médicament dans le sang du patient au bout de $n$ périodes de trente minutes. On a donc $u_0 = 1$.

\begin{enumerate}
	\item Calculer la quantité de médicament dans le sang au bout d'une demi-heure.
	\item Justifier que, pour tout entier naturel $n$, $u_{n+1} = 0,9u_n + 0,25$.
	\item 
	\begin{enumerate}
		\item Montrer par récurrence sur $n$ que, pour tout entier naturel $n$, $u_n \leqslant  u_{n+1} < 5$.
		\item En déduire que la suite $\left(u_n\right)$ est convergente.
	\end{enumerate}
	\item On estime que le médicament est réellement efficace lorsque sa quantité dans le sang du patient est supérieure ou égale à $1,8$ mg.
	\begin{enumerate}
		\item Recopier et compléter le script écrit en langage \textsf{Python} suivant de manière à déterminer au bout de combien de périodes de trente minutes le médicament commence à être réellement efficace.
		
\begin{CodePythonLstAlt}*[Largeur=8cm]{center}
def efficace() :
	u = 1
	n = 0
	while ...... :
		u = ......
		n = n+1
	return n
\end{CodePythonLstAlt}
		\item Quelle est la valeur renvoyée par ce script ? Interpréter ce résultat dans le contexte de l'exercice.
	\end{enumerate}
	\item Soit $\left(v_n\right)$ la suite définie, pour tout entier naturel $n$, par $v_n = 2,5 - u_n$.
	\begin{enumerate}
		\item Montrer que $\left(v_n\right)$ est une suite géométrique dont on précisera la raison et le premier terme $v_0$.
		\item Montrer que, pour tout entier naturel $n$, $u_n = 2,5 - 1,5 \times 0,9^n$.
		\item Le médicament devient toxique lorsque sa quantité présente dans le sang du patient dépasse $3$ mg. 
		
		D'après le modèle choisi, le traitement présente-t-il un risque pour le patient? Justifier.
	\end{enumerate}
\end{enumerate}

\textbf{Partie B : modèle continu de la quantité médicamenteuse}

\medskip

Après une injection initiale de $1$~mg de médicament, le patient est placé sous perfusion.

Le débit de la substance médicamenteuse administrée au patient est de $0,5$ mg par heure.

La quantité de médicament dans le sang du patient, en fonction du temps, est modélisée
par la fonction $f$, définie sur $[0;+\infty[$, par \[f(t) = 2,5 - 1,5\text{e}^{-0,2t},\]%
%
où $t$ désigne la durée de la perfusion exprimée en heure.

On rappelle que ce médicament est réellement efficace lorsque sa quantité dans le sang du patient est supérieure ou égale à $1,8$~mg.

\begin{enumerate}
	\item Le médicament est-il réellement efficace au bout de 3~h 45~min ?
	\item Selon ce modèle, déterminer au bout de combien de temps le médicament devient réellement efficace.
	\item Comparer le résultat obtenu avec celui obtenu à la question 4.(b) du modèle discret de la \textbf{Partie A}.
\end{enumerate}

