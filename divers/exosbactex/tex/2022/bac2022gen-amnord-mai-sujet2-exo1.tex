Dans une région touristique, une société propose un service de location de vélo pour la journée.

\smallskip

La société dispose de deux points de location distinctes, le point A et le point B. Les vélos peuvent être empruntés est restitués indifféremment dans l'un où l'autre des deux poins de location.

\smallskip

On admettra que le nombre total de vélos est constant et que tous les matins, à l'ouverture du service, chaque vélo se trouve au point A ou au point B.

\smallskip

D'après une étude statistique :

\begin{itemize}[leftmargin=*]
	\item si un vélo se trouve au point A un matin, la probabilité qu'il se trouve au point A le matin suivant est égale à $0,84$ ;
	\item si un vélo se trouve au point B un matin la probabilité qu'il se trouve au point B le matin suivant est égale à $0,76$.
\end{itemize}

À l'ouverture du service le premier matin, la société a disposé la moitié de ses vélos au point A, l'autre moitié au point B.

\smallskip

On considère un vélo de la société pris au hasard.

\smallskip

Pour tout entier naturel non nul $n$, on définit les évènements suivants :

\begin{itemize}
	\item $A_n$ : \og le vélo se trouve au point A le $n$-ième matin \fg{}
	\item $B_n$ : \og Le vélo se trouve au point B le $n$-ième matin \fg.
\end{itemize}

Pour tout entier naturel non nul $n$, on note $a_n$ la probabilité de l'évènement $A_n$ et $b_n$ celle de l'évènement $B_n$.

Ainsi $a_1 = 0,5$ et $b_1 =  0,5$.

\begin{enumerate}
	\item Recopier et compléter l'arbre pondéré ci-dessous qui modélise la situation pour les deux premiers matins :
	
	\begin{center}
		\begin{forest} for tree = {grow'=0,math content,l=3cm,s sep=0.5cm},
			[,name=Omega
			[A_1 , fleche , aproba=\ldots , name=A11
			[A_2 , fleche , aproba=\ldots , name=A21]
			[\overline{A_2} , fleche , bproba=\ldots , name=A22]
			]
			[\overline{A_1} , fleche , bproba=\ldots , name=A12
			[A_2 , fleche , aproba=\ldots , name=A23]
			[\overline{A_2} , fleche , bproba=\ldots , name=A24]
			]
			]
		\end{forest}
	\end{center}
	\item
	\begin{enumerate}
		\item Calculer $a_2$.
		\item le vélo se trouve au point à le deuxième matin. Calculer la probabilité qu'il se trouve au point B le premier matin. La probabilité sera arrondie au millième.
	\end{enumerate}
	\item
	\begin{enumerate}
		\item Recopier et compléter l'arbre pondéré ci-dessous qui modélise la situation pour les $n$-ième et $n + 1$-ième  matins.
		
		\begin{center}
			\begin{forest} for tree = {grow'=0,math content,l=3cm,s sep=0.5cm},
				[,name=Omega
				[A_n , fleche , aproba=a_n , name=A11
				[A_{n+1} , fleche , aproba=\ldots , name=A21]
				[\overline{A_{n+1}} , fleche , bproba=\ldots , name=A22]
				]
				[\overline{A_n} , fleche , bproba=\ldots , name=A12
				[A_{n+1} , fleche , aproba=\ldots , name=A23]
				[\overline{A_{n+1}} , fleche , bproba=\ldots , name=A24]
				]
				]
			\end{forest}
		\end{center}
		\item Justifier que pour tout entier naturel non nul $n$, $a_{n+1} = 0,6a_n + 0,24$.
	\end{enumerate}
	\item Montrer par récurrence que, pour tout entier naturel non nul $n$, $a_n = 0,6 - 0,1 \times 0,6^{n - 1}$.
	\item Déterminer la limite de la suite $\left(a_n\right)$ et interpréter cette limite dans le contexte de l'exercice. 
	\item Déterminer le plus petit  entier naturel $n$ tel que $a_n \geqslant  0,599$ et interpréter le résultat obtenu dans le contexte de l'exercice.
\end{enumerate}

