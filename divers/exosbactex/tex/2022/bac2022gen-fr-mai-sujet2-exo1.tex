Le coyote est un animal sauvage proche du loup, qui vit en Amérique du Nord.

Dans l’état d’Oklahoma, aux États-Unis, 70\,\% des coyotes sont touchés par une maladie appelée ehrlichiose.

Il existe un test aidant à la détection de cette maladie. Lorsque ce test est appliqué à un coyote, son résultat est soit positif, soit négatif, et on sait que :

\begin{itemize}
	\item si le coyote est malade, le test est positif dans 97\,\% des cas ;
	\item si le coyote n’est pas malade, le test est négatif dans 95\,\% des cas.
\end{itemize}
%
\begin{center}
	\textbf{Partie A}
\end{center}
%
Des vétérinaires capturent un coyote d’Oklahoma au hasard et lui font subir un test pour l’ehrlichiose.

On considère les événements suivants :

\begin{itemize}
	\item M : « le coyote est malade » ;
	\item T : « le test du coyote est positif ».
\end{itemize}

On note $\overline{M}$ et $\overline{T}$ respectivement les événements contraires de $M$ et $T$.

\begin{enumerate}
	\item Recopier et compléter l’arbre pondéré ci-dessous qui modélise la situation.
	
	\begin{center}
		\begin{forest} for tree = {grow'=0,math content,l=3cm,s sep=0.75cm},
			[,name=Omega
				[M , fleche , aproba=\ldots , name=A11
					[T , fleche , aproba=\ldots , name=A21]
					[\overline{T} , fleche , bproba=\ldots , name=A22]
				]
				[\overline{M} , fleche , bproba=\ldots , name=A12
					[T , fleche , aproba=\ldots , name=A23]
					[\overline{T} , fleche , bproba=\ldots , name=A24]
				]
			]
		\end{forest}
	\end{center}
	\item Déterminer la probabilité que le coyote soit malade et que son test soit positif.
	\item Démontrer que la probabilité de T est égale à 0,694.
	\item On appelle « valeur prédictive positive du test » la probabilité que le coyote soit effectivement  malade sachant que son test est positif.
	
	Calculer la valeur prédictive positive du test. On arrondira le résultat au millième.
	\item 
	\begin{enumerate}
		\item Par analogie avec la question précédente, proposer une définition de la « valeur prédictive négative du test », et calculer cette valeur en arrondissant au millième.
		\item Comparer les valeurs prédictives positive et négative du test, et interpréter.
	\end{enumerate}
\end{enumerate}
%
\begin{center}
	\textbf{Partie B}
\end{center}
%
On rappelle que la probabilité qu’un coyote capturé au hasard présente un test positif est de $0,694$.

\begin{enumerate}
	\item Lorsqu'on capture au hasard cinq coyotes, on assimile ce choix à un tirage avec remise.
	
	On note X la variable aléatoire qui à un échantillon de cinq coyotes capturés au hasard associe le nombre de coyotes dans cet échantillon ayant un test positif.
	\begin{enumerate}
		\item Quelle est la loi de probabilité suivie par X ? Justifier et préciser ses paramètres.
		\item Calculer la probabilité que dans un échantillon de cinq coyotes capturés au hasard, un seul ait un test positif. On arrondira le résultat au centième.
		\item Un vétérinaire affirme qu’il y a plus d’une chance sur deux qu’au moins quatre coyotes sur  cinq aient un test positif : cette affirmation est-elle vraie ? Justifier la réponse.
	\end{enumerate}
	\item Pour tester des médicaments, les vétérinaires ont besoin de disposer d’un coyote présentant un test positif. Combien doivent-ils capturer de coyotes pour que la probabilité qu’au moins l’un d’entre eux présente un test positif soit supérieure à 0,99 ?
\end{enumerate}

