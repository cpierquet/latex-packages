Dans l’espace, rapporté à un repère orthonormé $\Rijk$, on considère les points :

\smallskip

\hfill{}$A(2;0;3)$, $B(0;2;1)$, $C(-1;-1;2)$ et $D(3;-3;-1)$.\hfill~

\begin{enumerate}
	\item \textbf{Calcul d’un angle}
	\begin{enumerate}
		\item Calculer les coordonnées des vecteurs $\vect{AB}$ et $\vect{AC}$ et en déduire que les points $A$, $B$ et $C$ ne sont pas alignés.
		\item Calculer les longueurs AB et AC.
		\item À l’aide du produit scalaire  $\vect{AB} \cdot \vect{AC}$, déterminer la valeur du cosinus de l’angle $\widehat{BAC}$ puis donner une valeur approchée de la mesure de l'angle $\widehat{BAC}$ au dixième de degré.
	\end{enumerate}
	\item \textbf{Calcul d’une aire}
	\begin{enumerate}
		\item Déterminer une équation du plan $\mathcal{P}$ passant par le point C et perpendiculaire à la droite $(AB)$.
		\item Donner une représentation paramétrique de la droite $(AB)$.
		\item En déduire les coordonnées du projeté orthogonal E du point C sur la droite $(AB)$, c'est-à-dire du point d’intersection de la droite $(AB)$ et du plan $\mathcal{P}$.
		\item Calculer l’aire du triangle $ABC$.
	\end{enumerate}
	\item \textbf{Calcul d’un volume}
	\begin{enumerate}
		\item Soit le point $F(1;-1;3)$. Montrer que les points A, B, C et F sont coplanaires.
		\item Vérifier que la droite $(FD)$ est orthogonale au plan $(ABC)$.
		\item Sachant que le volume d’un tétraèdre est égal au tiers de l’aire de sa base multiplié par sa hauteur, calculer le volume du tétraèdre ABCD.
	\end{enumerate}
\end{enumerate}

