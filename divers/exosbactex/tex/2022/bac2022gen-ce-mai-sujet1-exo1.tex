\textit{Cet exercice est un questionnaire à choix multiples. Pour chacune des questions suivantes, une seule des quatre réponses proposées est exacte. Les six questions sont indépendantes.\\
	Une réponse incorrecte, une réponse multiple ou l’absence de réponse à une question ne rapporte ni n’enlève de point. Pour répondre, indiquer sur la copie le numéro de la question et la lettre de la réponse choisie.\\
	Aucune justification n’est demandée.}

\begin{enumerate}
	\item On considère la fonction $f$ définie pour tout réel $x$ par $f(x)=\ln\big(1+x^2\big)$.
	
	Sur $\R$, l'équation $f(x)=2022$ :
	
	\begin{tblr}{width=\linewidth,colspec={X[l]X[l]}}
		(a)~~n'admet aucune solution&(b)~~admet exactement une solution \\
		(c)~~admet exactement deux solutions&(d)~~admet une infinité de solutions
	\end{tblr}
	%
	\item Soit la fonction $g$ définie pour tout réel $x$ strictement positif par : $g(x) =x\,\ln(x)-x$.
	
	On note $\mathcal{C}_g$ sa courbe représentative dans un repère du plan.
	
	\begin{tblr}{width=\linewidth,colspec={X[l]X[l]}}
		(a)~~La fonction $g$ est convexe sur $]0;+\infty[$&(b)~~La fonction $g$ est concave sur $]0;+\infty[$\\
		(c)~~La courbe $\mathcal{C}_g$ admet exactement un point d'inflexion sur $]0;+\infty[$&(d)~~La courbe $\mathcal{C}_g$ admet exactement deux points d'inflexion sur $]0;+\infty[$
	\end{tblr}
	%
	\item On considère la fonction $f$ définie sur $]-1;1[$ par $f(x)=\dfrac{x}{1-x^2}$.
	
	Une primitive de la fonction $f$ est la fonction $g$ définie sur l’intervalle $]-1;1[$ par :
	
	\begin{tblr}{width=\linewidth,colspec={X[l]X[l]}}
		(a)~~$g(x)=-\frac12 \ln\big(1-x^2\big)$&(b)~~$g(x)=\frac{1+x^2}{(1-x^2)^2}$\\
		(c)~~$g(x)=\frac{x^2}{2\big(x-\tfrac{x^3}{3}\big)^2}$&(d)~~$g(x)=\frac{x^2}{2} \ln\big(1-x^2\big)$
	\end{tblr}
	%
	\item La fonction $x \mapsto \ln(-x^2-x+6)$ est définie sur :
	
	\begin{tblr}{width=\linewidth,colspec={X[l]X[l]}}
		(a)~~$]-3;2[$&(b)~~$]-\infty;6[$\\
		(c)~~$]0;+\infty[$&(d)~~$]2;+\infty[$
	\end{tblr}
	%
	\item On considère la fonction $f$ définie sur $]0,5;+\infty[$ par $f(x)=x^2-4x+3\,\ln(2x-1)$.
	
	Une équation de la tangente à la courbe représentative de $f$ au point d’abscisse 1 est :
	
	\begin{tblr}{width=\linewidth,colspec={X[l]X[l]}}
		(a)~~$y=4x-7$&(b)~~$y=2x-4$\\
		(c)~~$y=-3(x-1)+4$&(d)~~$y=2x-1$
	\end{tblr}
	%
	\item L’ensemble $\mathcal{S}$ des solutions dans $\R$ de l’inéquation $\ln(x +3) < 2\,\ln(x +1)$ est :
	
	\begin{tblr}{width=\linewidth,colspec={X[l]X[l]}}
		(a)$\mathcal{S}= ]-\infty;-2[ \cup ]1;+\infty[$&(b)~~$\mathcal{S}=]1;+\infty[$\\
		(c)~~$\mathcal{S}=\varnothing$&(d)~~$\mathcal{S}=]-1;1[$
	\end{tblr}
\end{enumerate}

