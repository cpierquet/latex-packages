\textit{Les résultats seront arrondis si besoin à $10^{-4}$ près.}

\medskip

Une étude statistique réalisée dans une entreprise fournit les informations suivantes :

\begin{itemize}
	\item 48\,\% des salariés sont des femmes. Parmi elles, $16,5$\,\% exercent une profession de cadre ; 
	\item 52\,\% des salariés sont des hommes. Parmi eux, $21,5$\,\% exercent une profession de cadre.
\end{itemize}

On choisit une personne au hasard parmi les salariés.

On considère les événements suivants :

\begin{itemize}
	\item F : « la personne choisie est une femme »;
	\item C : « la personne choisie exerce une profession de cadre ». 
\end{itemize}

\begin{enumerate}
	\item Représenter la situation par un arbre pondéré.
	\item Calculer la probabilité que la personne choisie soit une femme qui exerce une profession de cadre.
	\item 
	\begin{enumerate}
		\item Démontrer que la probabilité que la personne choisie exerce une profession de cadre est égale à $0,191$.
		\item Les événements F et C sont-ils indépendants ? Justifier.
	\end{enumerate}
	\item Calculer la probabilité de $F$ sachant $C$, notée $P_C(F)$. Interpréter le résultat dans le contexte de l'exercice.
	\item On choisit au hasard un échantillon de 15 salariés. Le grand nombre de salariés dans l'entreprise permet d'assimiler ce choix à un tirage avec remise.
	
	On note $X$ la variable aléatoire donnant le nombre de cadres au sein de l'échantillon de 15 salariés.
	
	On rappelle que la probabilité qu'un salarié choisi au hasard soit un cadre est égale à $0,191$.
	\begin{enumerate}
		\item Justifier que $X$ suit une loi binomiale dont on précisera les paramètres. 
		\item Calculer la probabilité que l'échantillon contienne au plus 1 cadre.
		\item Déterminer l'espérance de la variable aléatoire $X$.
	\end{enumerate}
	\item Soit $n$ un entier naturel.
	
	On considère dans cette question un échantillon de $n$ salariés.
	
	Quelle doit être la valeur minimale de $n$ pour que la probabilité qu'il y ait au moins un cadre au sein de l'échantillon soit supérieure ou égale à $0,99$ ?
\end{enumerate}

