\textit{Cet exercice est un questionnaire à choix multiples. Pour chacune des questions suivantes, une seule des quatre réponses proposées est exacte. Les six questions sont indépendantes.\\
	Une réponse incorrecte, une réponse multiple ou l’absence de réponse à une question ne rapporte ni n’enlève de point. Pour répondre, indiquer sur la copie le numéro de la question et la lettre de la réponse choisie. Aucune justification n’est demandée.}

\begin{enumerate}
	\item Un récipient contenant initialement 1 litre d'eau est laissé au soleil.
	
	Toutes les heures, le volume d'eau diminue de 15\,\%.
	
	Au bout de quel nombre entier d'heures le volume d'eau devient-il inférieur à un
	quart de litre ?
	
	\begin{tblr}{width=\linewidth,colspec={X[l]X[l]X[l]X[l]}}
		(a)~~2 heures&(b)~~8 heures&(c)~~9 heures&(d)~~13 heures
	\end{tblr}
	%
	\item On considère la suite $\suiten$, définie pour tout entier naturel $n$ par $u_{n+1} =\frac12 u_n + 3$ et $u_0=6$. On peut affirmer que :
	
	\begin{tblr}{width=\linewidth,colspec={X[l]X[l]}}
		(a)~~la suite $\suiten$ est strictement croissante&(b)~~la suite $\suiten$ est strictement décroissante\\
		(c)~~la suite $\suiten$ n'est pas monotone&(d)~~la suite $\suiten$ est constante
	\end{tblr}
	%
	\item On considère la fonction $f$ définie sur l'intervalle $]0;+\infty[$ par $f(x)=4\,\ln(3x)$.
	
	Pour tout réel $x$ de l'intervalle $]0;+\infty[$,on a:
	
	\begin{tblr}{width=\linewidth,colspec={X[l]X[l]}}
		(a)~~$f(2x)=f(x)+\ln(24)$&(b)~~$f(2x)=f(x)+\ln(16)$\\
		(c)~~$f(2x)=\ln(2)+f(x)$&(d)~~$f(2x)=2f(x)$
	\end{tblr}
	%
	\item On considère la fonction $g$ définie sur l'intervalle $]1;+\infty[$ par $g(x)=\dfrac{\ln(x)}{x-1}$.
	
	On note $\mathcal{C}_g$ la courbe représentative de la fonction $g$ dans un repère orthogonal. La courbe $\mathcal{C}_g$ admet : 
	
	\begin{tblr}{width=\linewidth,colspec={X[l]X[l]}}
		(a)~~une asymptote verticale et une asymptote horizontale.&(b)~~une asymptote verticale
		et aucune asymptote horizontale.\\
		(c)~~aucune asymptote verticale et une asymptote horizontale.&(d)~~aucune asymptote verticale et aucune asymptote horizontale.
	\end{tblr}
\end{enumerate}

Dans la suite de l'exercice, on considère la fonction $h$ définie sur l'intervalle $]0;2]$ par $h(x)=x^2\big(1 + 2\,\ln(x)\big)$.

On note $\mathcal{C}_h$ la courbe représentative de $h$ dans un repère du plan.

On admet que $h$ est deux fois dérivable sur l'intervalle $]0;2]$.

On note $h'$ sa dérivée et $h''$ sa dérivée seconde.

\smallskip

On admet que, pour tout réel $x$ de l'intervalle $]0;2]$, on a $h'(x) = 4x\big(1 + \ln(x)\big)$.

\begin{enumerate}[resume]
	\item Sur l'intervalle $\left[\dfrac{1}{\e};2\right]$, la fonction $h$ s'annule :
	
	\begin{tblr}{width=\linewidth,colspec={X[l]X[l]}}
		(a)~~exactement 0 fois&(b)~~exactement 1 fois\\
		(c)~~exactement 2 fois&(d)~~exactement 3 fois
	\end{tblr}
	%
	\item Une équation de la tangente à $\mathcal{C}_h$ au point d'abscisse $\sqrt{\e}$ est : 
	
	\begin{tblr}{width=\linewidth,colspec={X[l]X[l]}}
		(a)~~$y=\left(6\e^{\frac12}\right)\cdot x$&(b)~~$y=\left(6\sqrt{\e}\right)\cdot  x+2\e$\\
		(c)~~$y=6\e^{\frac{x}{2}}$&2(d)~~$y=\left(6\e^{\frac12}\right)\cdot x-4\e$
	\end{tblr}
	%
	\item Sur l'intervalle $]0;2]$, le nombre de points d'inflexion de la courbe $\mathcal{C}_h$ est égal à :
	
	\begin{tblr}{width=\linewidth,colspec={X[l]X[l]X[l]X[l]}}
		(a)~~0&(b)~~1&(c)~~2&(d)~~3
	\end{tblr}
\end{enumerate}

