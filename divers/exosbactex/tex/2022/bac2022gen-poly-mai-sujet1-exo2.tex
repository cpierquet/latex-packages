Selon les autorités sanitaires d'un pays, 7\,\% des habitants sont affectés par une certaine maladie.

Dans ce pays, un test est mis au point pour détecter cette maladie. Ce test a les caractéristiques suivantes :
%
\begin{itemize}
	\item pour les individus malades, le test donne un résultat négatif dans $20 \,\%$ des cas ;
	\item pour les individus sains, le test donne un résultat positif dans $1\,\%$ des cas.
\end{itemize}
%
Une personne est choisie au hasard dans la population et testée.

On considère les évènements suivants :
%
\begin{itemize}
	\item $M$ \og la personne est malade \fg{} ;
	\item $T$ \og le test est positif \fg{}.
\end{itemize}

\begin{enumerate}
	\item Calculer la probabilité de l'évènement $M \cap T$. On pourra s'appuyer sur un arbre pondéré.
	\item Démontrer que la probabilité que le test  de la personne choisie au hasard soit positif, et de \num{0,0653}.
	\item Dans un contexte de dépistage de la maladie, est-il plus pertinent de connaître $P_M(T)$ ou $P_T(M)$ ?
	\item On considère dans cette question que la personne choisie au hasard a eu un test positif.
	
	Quelle est la probabilité qu'elle soit malade ? On arrondira le résultat à $10^{-2}$ près.
	\item On choisit des personnes au hasard dans la population. La taille de la population de ce pays permet d'assimiler ce prélèvement à un tirage avec remise.
	
	On note $X$ la variable aléatoire qui donne le nombre d'individus ayant un test positif parmi les 10 personnes. 
	\begin{enumerate}
		\item Préciser la nature et les paramètres de la loi de probabilité suivie par $X$.
		\item Déterminer la probabilité pour qu'exactement deux personnes aient un test positif. On arrondira le résultat à $10^{-2}$ près.
	\end{enumerate}	
	\item Déterminer le nombre minimum de personnes à tester dans ce pays pour que la probabilité qu'au moins l'une d'entre elle ait un test positif, soit supérieur à $99\,\%$.
\end{enumerate}

