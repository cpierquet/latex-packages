L'espace est muni d'un repère orthonormé $\Rijk$.

On considère les points \[A(3;-2;2) \text{, } B(6;1;5) \text{, } C(6;-2;-1) \text{ et } D(0;4;-1).\]%
\emph{On rappelle que le volume d'un tétraèdre est donné par la formule :}\[V = \dfrac13 \mathcal{A} \times h\]%
\emph{où $\mathcal{A}$ est l'aire de la base et $h$ la hauteur correspondante.}

\begin{enumerate}
	\item Démontrer que les points A, B, C et D ne sont pas coplanaires.
	\item 
	\begin{enumerate}
		\item Montrer que le triangle ABC est rectangle.
		\item Montrer que la droite (AD) est perpendiculaire au plan (ABC).
		\item En déduire le volume du tétraèdre ABCD.
	\end{enumerate}	
	\item On considère le point $H(5;0;1)$.
	\begin{enumerate}
		\item Montrer qu'il existe des réels $\alpha$ et $\beta$ tels que $\vect{\text{BH}} = \alpha \vect{\text{BC}} + \beta \vect{\text{BD}}$.
		\item Démontrer que H est le projeté orthogonal du point A sur le plan (BCD). 
		\item En déduire la distance du point A au plan (BCD).
	\end{enumerate}	
	\item Déduire des questions précédentes l'aire du triangle BCD.
\end{enumerate}

