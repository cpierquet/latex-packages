Une entreprise fabrique des composants pour l'industrie automobile. Ces composants sont conçus sur trois chaînes de montage numérotées de 1 à 3.

\begin{itemize}
	\item La moitié des composants est conçue sur la chaîne \no 1 ;
	\item 30\,\% des composants sont conçus sur la chaîne \no 2;
	\item les composants restant sont conçus sur la chaîne \no 3.
\end{itemize}

À l'issue du processus de fabrication, il apparaît que 1\,\% des pièces issues de la chaîne \no 1 présentent un défaut, de même que 0,5\,\% des pièces issues de la chaîne \no 2 et 4\,\% des pièces issues de la chaîne \no 3.

On prélève au hasard un de ces composants. On note :

\begin{itemize}
	\item $C_1$ l'évènement \og le composant provient de la chaîne \no 1 \fg{} ;
	\item $C_2$ l'évènement \og le composant provient de la chaîne \no 2 \fg{} ;
	\item $C_3$ l'évènement \og le composant provient de la chaîne \no 3 \fg{} ;
	\item $D$ l'évènement \og le composant est défectueux\fg et $\overline{D}$ son évènement contraire.
\end{itemize}

\emph{Dans tout l'exercice, les calculs de probabilité seront donnés en valeur décimale exacte ou arrondie à $10^{-4}$ si nécessaire}.

\medskip

\textbf{Partie A}

\begin{enumerate}
	\item Représenter cette situation par un arbre pondéré.
	\item Calculer la probabilité que le composant prélevé provienne de la chaîne \no 3 et soit défectueux.
	\item Montrer que la probabilité de l'évènement $D$ est $P(D) = \num{0,0145}$.
	\item Calculer la probabilité qu'un composant défectueux provienne de la chaîne \no 3.
\end{enumerate}

\textbf{Partie B}

\medskip

L'entreprise décide de conditionner les composants produits en constituant des lots de $n$ unités. On note $X$ la variable aléatoire qui, à chaque lot de $n$ unités, associe le nombre de composants défectueux de ce lot.

Compte tenu des modes de production et de conditionnement de l'entreprise, on peut considérer que $X$ suit la loi binomiale de paramètres $n$ et $p = \num{0,0145}$.

\begin{enumerate}
	\item Dans cette question, les lots possèdent $20$ unités. On pose $n =20$.
	\begin{enumerate}
		\item Calculer la probabilité pour qu'un lot possède exactement trois composants défectueux.
		\item Calculer la probabilité pour qu'un lot ne possède aucun composant défectueux.
		
		En déduire la probabilité qu'un lot possède au moins un composant défectueux.
	\end{enumerate}
	\item  Le directeur de l'entreprise souhaite que la probabilité de n'avoir aucun composant
	défectueux dans un lot de $n$ composants soit supérieure à $0,85$. 
	
	Il propose de former des lots de 11 composants au maximum. A-t-il raison ? Justifier la réponse.
\end{enumerate}

\textbf{Partie C}

\medskip

Les coûts de fabrication des composants de cette entreprise sont de $15$ euros s'ils proviennent de la chaîne de montage \no 1, 12 euros s'ils proviennent de la chaîne de montage \no 2 et 9 euros s'ils proviennent de la chaîne de montage \no 3.

Calculer le coût moyen de fabrication d'un composant pour cette entreprise.

