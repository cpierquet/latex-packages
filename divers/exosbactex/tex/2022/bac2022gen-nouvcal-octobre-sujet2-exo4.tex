\emph{Cet exercice est un questionnaire à choix multiples.\\
	Pour chacune des questions suivantes, une seule des quatre réponses proposées est exacte.\\
	Une réponse fausse, une réponse multiple ou l'absence de réponse à une question ne rapporte ni n'enlève de point.\\
	Pour répondre, indiquer sur la copie le numéro de la question et la lettre de la réponse choisie.\\
	Aucune justification n'est demandée.}

\medskip

\begin{enumerate}
	\item On considère la suite $\left(u_n\right)$ définie pour tout entier naturel $n$ par :\[u_n  = \dfrac{(- 1)^n}{n + 1}.\]
	%
	On peut affirmer que :
	
	\smallskip
	
	\begin{tblr}{width=\linewidth,colspec={X[m]X[m]}}
		(a)~~la suite $\left(u_n\right)$ diverge vers $+\infty$. &(b)~~la suite $\left(u_n\right)$ diverge vers $-\infty$.\\
		(c)~~la suite $\left(u_n\right)$ n'a pas de limite. &(d)~~la suite $\left(u_n\right)$ converge.
	\end{tblr}
	
	\begin{center} \decosix \decosix \decosix  \end{center}
\end{enumerate}

Dans les questions 2. et 3., on considère deux suites $\left(v_n\right)$  et $\left(w_n\right)$  vérifiant la relation : \[w_n = \e^{- 2v_n} + 2.\]
%
\begin{enumerate}[resume]
	\item  Soit $a$ un nombre réel strictement positif. On a $v_0 = \ln (a)$.
	
	\smallskip
	
	\begin{tblr}{width=\linewidth,colspec={X[m]X[m]}}
		(a)~~$w_0 = \dfrac{1}{a^2}  +2$ &(b)~~$w_0 = \dfrac{1}{a^2  +2}$\\
		(c)~~$w_0 = -2a +2$ &(d)~~$w_0 = \dfrac{1}{- 2a} + 2$
	\end{tblr}
	\item On sait que la suite $\left(v_n\right)$ est croissante. On peut affirmer que la suite $\left(w_n\right)$ est :
	
	\smallskip
	
	\begin{tblr}{width=\linewidth,colspec={X[m]X[m]}}
		(a)~~décroissante et majorée par 3. &(b)~~décroissante et minorée par 2.\\
		(c)~~croissante et majorée par 3. &(d)~~croissante et minorée par 2.
	\end{tblr}
	\item On considère la suite $\left(a_n\right)$ ainsi définie : \[a_0 = 2 \text{ et, pour tout entier naturel }n, \:a_{n+1} = \dfrac13a_n + \dfrac83.\]
	%
	Pour tout entier naturel $n$,on a :
	
	\smallskip
	
	\begin{tblr}{width=\linewidth,colspec={X[m]X[m]}}
		(a)~~$a_n = 4 \times \left(\dfrac13\right)^n - 2$ &(b)~~$a_n = - \dfrac{2}{3^n} + 4$\\
		(c)~~$a_n = 4 - \left(\dfrac13\right)^n$ &(d)~~$a_n = 2 \times \left(\dfrac13\right)^n + \dfrac{8n}{3}$ \\
	\end{tblr}
	\item On considère une suite $\left(b_n\right)$ telle que, pour tout entier naturel $n$, on a :  \[b_{n+1} = b_n + \ln \left(\dfrac{2}{\left(b_n \right)^2 + 3}\right).\]
	%
	On peut affirmer que :
	
	\smallskip
	
	\begin{tblr}{width=\linewidth,colspec={X[m]X[m]}}
		(a)~~la suite $\left(b_n\right)$ est croissante. &(b)~~la suite $\left(b_n\right)$ est décroissante.\\
		(c)~~la suite $\left(b_n\right)$ n'est pas monotone. &(d)~~Le sens de variation de la suite $\left(b_n\right)$ dépend de $b_0$.
	\end{tblr}
	\item On considère la fonction $g$ définie sur l'intervalle $\intervOO{0}{+\infty}$ par : \[g(x) = \dfrac{\e^x}{x}.\]
	%
	On note $\mathcal{C}_g$ la courbe représentative de la fonction $g$ dans un repère orthogonal.
	
	La courbe $\mathcal{C}_g$ admet :
	
	\smallskip
	
	\begin{tblr}{width=\linewidth,colspec={X[m]X[m]}}
		(a)~~une asymptote verticale et une asymptote horizontale. &(b)~~une asymptote verticale et aucune asymptote horizontale.\\
		(c)~~aucune asymptote verticale et une asymptote horizontale. &(d)~~aucune asymptote verticale et aucune asymptote horizontale.
	\end{tblr}
	\item Soit $f$ la fonction définie sur $\R$ par \[f(x) = x\e^{x^2+1}.\]
	%
	Soit $F$ une primitive sur $\R$ de la fonction $f$. Pour tout réel $x$, on a :
	
	\smallskip
	
	\begin{tblr}{width=\linewidth,colspec={X[m]X[m]}}
		(a)~~$F(x) = \dfrac12x^2\e^{x^2+1}$ &(b)~~$F(x) = \left(1 + 2x^2 \right)\e^{x^2+1}$ \\
		(c)~~$F(x) = \e^{x^2+1}$ &(d)~~$F(x) = \dfrac12\e^{x^2+1}$
	\end{tblr}
\end{enumerate}