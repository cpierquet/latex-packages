Dans l'espace rapporté à un repère orthonormé $\Rijk$, on considère :
%
\begin{itemize}
	\item la droite $\mathcal{D}$ passant par le point $A(2;4;0)$ et dont un vecteur directeur est $\vect{u}\begin{pmatrix}1\\2\\0\end{pmatrix}$ ;
	\item la droite $\mathcal{D}'$ dont une représentation paramétrique est : $\begin{dcases}x=3\\y=3 + t\\z=3 + t \end{dcases}$ \:, $t \in \R$.
\end{itemize}

\begin{enumerate}
	\item 
	\begin{enumerate}
		\item Donner les coordonnées d'un vecteur directeur $\vect{u'}$ de la droite $\mathcal{D}'$.
		\item Montrer que les droites $\mathcal{D}$ et $\mathcal{D}'$ ne sont pas parallèles.
		\item Déterminer une représentation paramétrique de la droite $\mathcal{D}$.
	\end{enumerate}
\end{enumerate}

On admet dans la suite de cet exercice qu'il existe une unique droite $\Delta$ perpendiculaire aux droites $\mathcal{D}$ et $\mathcal{D}'$. Cette droite $\Delta$ coupe chacune des droites $\mathcal{D}$ et $\mathcal{D}'$. On appellera M le point d'intersection de $\Delta$ et $\mathcal{D}$, et M$'$ le point d'intersection de $\Delta$ et $\mathcal{D}'$.

On se propose de déterminer la distance MM$'$ appelée \og distance entre les droites $\mathcal{D}$ et $\mathcal{D}'$ \fg.
%
\begin{enumerate}[resume]
	\item Montrer que le vecteur $\vect{v}\begin{pmatrix}2\\- 1\\1\end{pmatrix}$ est un vecteur directeur de la droite $\Delta$.
	\item On note $\mathcal{P}$ le plan contenant les droites $\mathcal{D}$ et $\Delta$, c'est-à-dire le plan passant par le point A et de vecteurs directeurs $\vect{u}$ et $\vect{v}$.
	\begin{enumerate}
		\item Montrer que le vecteur $\vect{n}\begin{pmatrix}2\\-1\\-5 \end{pmatrix}$ est un vecteur normal au plan $\mathcal{P}$.
		\item En déduire qu'une équation du plan $\mathcal{P}$ est : $2x - y - 5z = 0$.
		\item On rappelle que M$'$ est le point d'intersection des droites $\Delta$ et $\mathcal{D}'$.
		
		Justifier que M$'$ est également le point d'intersection de $\mathcal{D}'$ et du plan $\mathcal{P}$.
		
		En déduire que les coordonnées du point M$'$ sont $(3;1;1)$.
	\end{enumerate}
	\item 
	\begin{enumerate}
		\item Déterminer une représentation paramétrique de la droite $\Delta$.
		\item Justifier que le point M a pour coordonnées $(1;2;0)$.
		\item Calculer la distance $MM'$.
	\end{enumerate}
	\item On considère la droite $d$ de représentation paramétrique $\begin{dcases}x = 5t\\y = 2 + 5t \\z = 1 + t \end{dcases}$ avec $t \in \R$.
	\begin{enumerate}
		\item Montrer que la droite $d$ est parallèle au plan $\mathcal{P}$.
		\item On note $\ell$ la distance d'un point N de la droite $d$ au plan $\mathcal{P}$ . 
		
		Exprimer le volume du  tétraèdre $ANMM'$ en fonction de $\ell$.
		
		On rappelle que le volume d'un tétraèdre est donné par : $V = \dfrac13 \times B \times h$ où $B$ désigne 
		l'aire d'une base et $h$ la hauteur relative à cette base.
		\item Justifier que, si $N_1$ et $N_2$ sont deux points quelconques de la droite $d$, les tétraèdres $AN_1MM'$ et $AN_2MM'$ ont le même volume.
	\end{enumerate}
\end{enumerate}