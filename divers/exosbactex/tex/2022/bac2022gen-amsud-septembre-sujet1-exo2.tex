Soit $\left(u_n\right)$ la suite définie par $u_0 = 4$ et, pour tout entier naturel $n$, $u_{n+1} = \dfrac15 u_n^2$.

\begin{enumerate}
	\item 
	\begin{enumerate}
		\item Calculer $u_1$ et $u_2$.
		\item Recopier et compléter la fonction ci-dessous écrite en langage \textsf{Python}.
		Cette fonction est nommée \texttt{suite\_u} et prend pour paramètre l'entier naturel \texttt{p}.
		
		Elle renvoie la valeur du terme de rang $p$ de la suite $\left(u_n\right)$.

\begin{CodePythonLstAlt}[Largeur=8cm]{center}
def suite_u(p) :
	u = ...
	for i in range(1,...) :
		u = ...
	return u
\end{CodePythonLstAlt}
	\end{enumerate}
	\item 
	\begin{enumerate}
		\item Démontrer par récurrence que pour tout entier naturel $n$, $0 < u_n \leqslant 4$.
		\item Démontrer que la suite $\left(u_n\right)$ est décroissante.
		\item En déduire que la suite $\left(u_n\right)$ est convergente.
	\end{enumerate}
	\item 
	\begin{enumerate}
		\item Justifier que la limite $\ell$ de la suite $\left(u_n\right)$ vérifie l'égalité $\ell = \dfrac15 \ell^2$.
		\item En déduire la valeur de $\ell$.
	\end{enumerate}
	\item Pour tout entier naturel $n$, on pose $v_n = \ln \left(u_n\right)$ et $w_n = v_n - \ln (5)$.
	\begin{enumerate}
		\item Montrer que, pour tout entier naturel $n$, $v_{n+1} = 2v_n - \ln (5)$.
		\item Montrer que la suite $\left(w_n\right)$ est géométrique de raison 2.
		\item Pour tout entier naturel $n$, donner l'expression de $w_n$ en fonction de $n$ et montrer que :
		
		\hfill$v_n = \ln \left(\dfrac45 \right) \times 2^n + \ln (5)$.\hfill~
	\end{enumerate}
	\item Calculer $\displaystyle\lim_{n \to + \infty} v_n$ et retrouver $\displaystyle\lim_{n \to + \infty} u_n$.
\end{enumerate}

