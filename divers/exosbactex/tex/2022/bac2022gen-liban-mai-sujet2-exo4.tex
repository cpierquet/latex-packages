On considère la fonction $f$ définie pour tout réel $x$ de $\intervOF{0}{1}$ par : \[ f(x)=\e^{-x} + \ln(x). \]
%
\begin{enumerate}
	\item Calculer la limite de $f$ en $0$.
	\item On admet que $f$ est dérivable sur $\intervOF{0}{1}$. On note $f'$ sa fonction dérivée.
	
	Démontrer que, pour tout réel $x$ appartenant à $\intervOF{0}{1}$, on a : \[ f'(x)=\dfrac{1-x\,\e^{-x}}{x}. \]
	\item Justifier que, pour tout réel $x$ appartenant à $\intervOF{0}{1}$, on a $x\,\e^{-x}<1$.
	
	En déduire le tableau de variation de $f$ sur $\intervOF{0}{1}$.
	\item Démontrer qu'il existe un unique réel $\ell$ appartenant à $\intervOF{0}{1}$ tel que $f\big(\ell\big) = \ell$.
\end{enumerate}

\textbf{\large Partie B}
%
\begin{enumerate}
	\item On définit deux suites $\suiten[a]$ et $\suiten[b]$ par : \[ \begin{dcases} a_0=\tfrac{1}{10} \\ b_0 = 1 \end{dcases} \text{ et, pour tout entier naturel }n \text{, } \begin{dcases} a_{n+1}=\e^{-b_n} \\ b_{n+1}=\e^{-a_n} \end{dcases}. \]
	\begin{enumerate}
		\item Calculer $a_1$ et $b_1$. On donnera des valeurs approchées à $10^{-2}$ près.
		\item On considère ci-dessous la fonction \texttt{termes}, écrite en langage \textsf{Python}.
		
\begin{CodePythonLstAlt}*[Largeur=8cm]{center}
def termes(n) :
	a = 1/10
	b = 1
	for k in range(0,n) :
		c = ...
		b = ...
		a = c
	return(a,b)
\end{CodePythonLstAlt}
	\end{enumerate}
	\item Recopier et compléter sans justifier le cadre ci-dessus de telle sorte que la fonction \texttt{termes} calcule les termes des suites $\suiten[a]$ et $\suiten[b]$.
	\item On rappelle que la fonction $x \mapsto e^{-x}$ est décroissante sur $\R$.
	\begin{enumerate}
		\item Démontrer par récurrence que, pour tout entier naturel $n$, on a : \[ 0 < a_n \leqslant a_{n+1} \leqslant b_{n+1} \leqslant b_n \leqslant 1. \]%
		\item En déduire que les suites $\suiten[a]$ et $\suiten[b]$ sont convergentes.
	\end{enumerate}
	\item On note $A$ la limite de $\suiten[a]$ et $B$ la limite de $\suiten[b]$.
	
	On admet que A et B appartiennent à l'intervalle $\intervOF{0}{1}$,et que $A=\e^{-B}$ et $B=\e^{-A}$.
	\begin{enumerate}
		\item Démontrer que $f(A) =0$.
		\item Déterminer $A - B$.
	\end{enumerate}
\end{enumerate}