La population d'une espèce en voie de disparition est surveillée de près dans une réserve naturelle.

Les conditions climatiques ainsi que le braconnage font que cette population diminue de 10\,\% chaque année.

Afin de compenser ces pertes, on réintroduit dans la réserve $100$ individus à la fin de chaque année.

On souhaite étudier l'évolution de l'effectif de cette population au cours du temps. Pour cela, on modélise l'effectif de la population de l'espèce par la suite $\left(u_n\right)$ où $u_n$ représente l'effectif de la population au début de l'année $2020 + n$. 

On admet que pour tout entier naturel $n$, $u_n \geqslant 0$.

Au début de l'année 2020, la population étudiée compte \num{2000} individus, ainsi $u_0 = \num{2000}$.

\begin{enumerate}
	\item Justifier que la suite $\left(u_n\right)$ vérifie la relation de récurrence :\[u_{n+1} = 0,9u_n + 100.\]
	\item Calculer $u_1$ puis $u_2$.
	\item Démontrer par récurrence que pour tout entier naturel $n$, $\num{1000} < u_{n+1}  \leqslant u_n$.
	\item La suite $\left(u_n\right)$ est-elle convergente ? Justifier la réponse.
	\item On considère la suite $\left(v_n\right)$ définie pour tout entier naturel $n$ par $v_n = u_n - \num{1000}$.
	\begin{enumerate}
		\item Montrer que la suite $\left(v_n\right)$ est géométrique de raison $0,9$.
		\item En déduire que, pour tout entier naturel $n$, $u_n = \num{1000} \left(1 + 0,9^n\right)$.
		\item Déterminer la limite de la suite $\left(u_n\right)$.
		
		En donner une interprétation dans le contexte de cet exercice.
	\end{enumerate}
	\item On souhaite déterminer le nombre d'années nécessaires pour que l'effectif de la population passe en dessous d'un certain seuil $S$ (avec $S > \num{1000}$).
	\begin{enumerate}
		\item Déterminer le plus petit entier $n$ tel que $u_n \leqslant \num{1020}$.
		
		Justifier la réponse par un calcul.
		\item Dans le programme \textsf{Python} ci-dessous, la variable \texttt{n} désigne le nombre d'années écoulées depuis 2020, la variable \texttt{u} désigne l'effectif de la population. 
		
		Recopier et compléter ce programme afin qu'il retourne le nombre d'années nécessaires pour que l'effectif de la population passe en dessous du seuil $S$.
		
\begin{CodePythonLstAlt}*[Largeur=6cm]{center}
def population(S) :
	n = 0
	u = 2000
	while ...... :
		u = ......
		n = ......
	return ......
\end{CodePythonLstAlt}
	\end{enumerate}
\end{enumerate}

