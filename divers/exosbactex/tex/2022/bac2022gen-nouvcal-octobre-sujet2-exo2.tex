Soit $f$ la fonction définie sur l'intervalle $\intervOO{0}{+\infty}$ par : \[f(x) = x \ln (x) - x - 2.\]
%
On admet que la fonction $f$ est deux fois dérivable sur $\intervOO{0}{+\infty}$.

On note $f'$ sa dérivée, $f''$ sa dérivée seconde et $\mathcal{C}_f$ sa courbe représentative dans un repère.

\begin{enumerate}
	\item 
	\begin{enumerate}
		\item Démontrer que, pour tout $x$ appartenant à $\intervOO{0}{+\infty}$, on a $f'(x) = \ln(x)$.
		\item Déterminer une équation de la tangente $T$ à la courbe $\mathcal{C}_f$ au point
		d'abscisse $x = \e$.
		\item Justifier que la fonction $f$ est convexe sur l'intervalle $\intervOO{0}{+\infty}$.
		\item En déduire la position relative de la courbe $\mathcal{C}_f$ et de la tangente $T$.
	\end{enumerate}
	\item 
	\begin{enumerate}
		\item Calculer la limite de la fonction $f$ en $0$.
		\item Démontrer que la limite de la fonction $f$ en $+\infty$ est égale à $+\infty$.
	\end{enumerate}
	\item Dresser le tableau de variations de la fonction $f$ sur l'intervalle $\intervOO{0}{+\infty}$.
	\item 
	\begin{enumerate}
		\item Démontrer que l'équation $f(x) = 0$ admet une unique solution dans l'intervalle $\intervOO{0}{+\infty}$. On note $\alpha$ cette solution.
		\item Justifier que le réel $\alpha$ appartient à l'intervalle $\intervOO{4,3}{4,4}$.
		\item En déduire le signe de la fonction $f$ sur l'intervalle $\intervOO{0}{+\infty}$.
	\end{enumerate}
	\item On considère la fonction \texttt{seuil} suivante écrite dans le langage \textsf{Python} :
	
	On rappelle que la fonction \texttt{log} du module \texttt{math} (que l'on suppose importé)
	désigne la fonction logarithme népérien $\ln$.
	
\begin{CodePythonLstAlt}*[Largeur=8cm]{center}
def seuil(pas) :
	x = 4.3
	while x*log(x) - x - 2 < 0 :
		x = x + pas
	return x
\end{CodePythonLstAlt}
	
	Quelle est la valeur renvoyée à l'appel de la fonction \texttt{seuil(0.01)} ?
	
	Interpréter ce résultat dans le contexte de l'exercice.
\end{enumerate}

