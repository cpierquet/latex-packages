Dans un repère orthonormé $\Rijk$ de l'espace, on considère les points \[A(-3;1;3)\text{, } B(2;2;3)\text{, } C(1;7;-1)\text{, } D(-4;6;-1) \text{ et } K(-3;14;14).\]
%
\begin{enumerate}
	\item 
	\begin{enumerate}
		\item Calculer les coordonnées des vecteurs $\vect{AB}$, $\vect{OC}$ et $\vect{AD}$.
		\item Montrer que le quadrilatère $ABCD$ est un rectangle.
		\item Calculer l'aire du rectangle $ABCD$.
	\end{enumerate}
	\item 
	\begin{enumerate}
		\item Justifier que les points $A$, $B$ et $D$ définissent un plan.
		\item Montrer que le vecteur $\vect{n}(-2;10;13)$ est un vecteur normal au plan $(ABD)$.
		\item  En déduire une équation cartésienne du plan $(ABD)$.
	\end{enumerate}
	\item
	\begin{enumerate}
		\item Donner une représentation paramétrique de la droite $\Delta$ orthogonale au plan $(ABD)$ et qui passe par le point $K$.
		\item Déterminer les coordonnées du point $I$, projeté orthogonal du point $K$ sur le plan $(ABD)$.
		\item Montrer que la hauteur de la pyramide $KABCD$ de base $ABCD$ et de sommet $K$ vaut $\sqrt{273}$.
	\end{enumerate}
	\item Calculer le volume $V$ de la pyramide $KABCD$.
	
	On rappelle que le volume V d'une pyramide est donné par la formule: \[V= \dfrac13 \times \text{aire de la base} \times \text{hauteur}.\]
\end{enumerate}

