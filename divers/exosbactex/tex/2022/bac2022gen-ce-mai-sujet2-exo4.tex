Une urne contient des jetons blancs et noirs tous indiscernables au toucher.

Une partie consiste à prélever au hasard successivement et avec remise deux jetons de cette urne.

On établit la règle de jeu suivante:

\begin{itemize}
	\item un joueur perd 9 euros si les deux jetons tirés sont de couleur blanche ;
	\item un joueur perd 1 euro si les deux jetons tirés sont de couleur noire ;
	\item un joueur gagne 5 euros si les deux jetons tirés sont de couleurs différentes.
\end{itemize}

\begin{enumerate}
	\item On considère que l'urne contient 2 jetons noirs et 3 jetons blancs.
	\begin{enumerate}
		\item Modéliser la situation à l'aide d'un arbre pondéré.
		\item Calculer la probabilité de perdre 9\,€ sur une partie.
	\end{enumerate}	
	\item On considère maintenant que l'urne contient 3 jetons blancs et au moins deux jetons noirs mais on ne connait pas le nombre exact de jetons noirs. On appellera $N$ le nombre de jetons noirs.
	\begin{enumerate}
		\item Soit $X$ la variable aléatoire donnant le gain du jeu pour une partie.
		
		Déterminer la loi de probabilité de cette variable aléatoire.
		\item Résoudre l'inéquation pour $x$ réel : \[-x^2  + 30x - 81 > 0.\]
		\item En utilisant le résultat de la question précédente, déterminer le nombre de jetons noirs que l'urne doit contenir afin que ce jeu soit favorable au joueur.
		\item Combien de jetons noirs le joueur doit-il demander afin d'obtenir un gain moyen maximal ?
	\end{enumerate}
	\item On observe $10$ joueurs qui tentent leur chance en effectuant une partie de ce jeu, indépendamment les uns des autres. On suppose que 7 jetons noirs ont été placés dans l'urne (avec 3 jetons blancs).
	
	Quelle est la probabilité d'avoir au moins $1$ joueur gagnant $5$ euros?
\end{enumerate}