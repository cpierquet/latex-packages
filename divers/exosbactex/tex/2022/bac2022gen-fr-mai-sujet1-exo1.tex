Dans le cadre d’un essai clinique, on envisage deux protocoles de traitement d’une maladie.

L’objectif de cet exercice est d’étudier, pour ces deux protocoles, l’évolution de la quantité de médicament présente dans le sang d’un patient en fonction du temps.

\smallskip

\textit{Les parties A et B sont indépendantes.}

\medskip

\textbf{Partie A : Étude du premier protocole}

\medskip

Le premier protocole consiste à faire absorber un médicament, sous forme de comprimé, au patient.

On modélise la quantité de médicament présente dans le sang du patient, exprimée en mg, par la fonction $f$ définie sur l’intervalle $[0;10]$ par $f(t)=3t\,\e^{-0,5t+1}$, où $t$ désigne le temps, exprimé en heure, écoulé depuis la prise du comprimé. 

\begin{enumerate}
	\item 
	\begin{enumerate}
		\item On admet que la fonction $f$ est dérivable sur l’intervalle $[0;10]$ et on note $f'$ sa fonction dérivée.
		
		Montrer que, pour tout nombre réel $t$ de $[0;10]$, on a : $f'(t)=3(-0,5t+1)\,\e^{-0,5t+1}$.
		\item En déduire le tableau de variations de la fonction $f$ sur l’intervalle $[0;10]$.
		\item Selon cette modélisation, au bout de combien de temps la quantité de médicament présente dans le sang du patient sera-t-elle maximale ? Quelle est alors cette quantité maximale ?
	\end{enumerate}
	\item 
	\begin{enumerate}[series=mylist]
		\item Montrer que l’équation $f(t)=5$ admet une unique solution sur l’intervalle $[0;2]$, notée $\alpha$, dont on donnera une valeur approchée à $10^{-2}$ près.
	\end{enumerate}
\end{enumerate}

On admet que l’équation $f(t)=5$ admet une unique solution sur l’intervalle $[2;10]$, notée $\beta$, et qu’une valeur approchée de $\beta$ à $10^{-2}$ près est $3,46$.

\begin{enumerate}
	\item[]
	\begin{enumerate}[resume=mylist]
		\item On considère que ce traitement est efficace lorsque la quantité de médicament présente dans le sang du patient est supérieure ou égale à 5 mg.
		
		Déterminer, à la minute près, la durée d’efficacité du médicament dans le cas de ce protocole. 
	\end{enumerate}
\end{enumerate}

\textbf{Partie B : Étude du deuxième protocole}

\medskip

Le deuxième protocole consiste à injecter initialement au patient, par piqûre intraveineuse, une dose de 2 mg de médicament puis à réinjecter toutes les heures une dose de 1,8 mg. 

On suppose que le médicament se diffuse instantanément dans le sang et qu’il est ensuite progressivement éliminé.

On estime que lorsqu'une heure s’est écoulée après une injection, la quantité de médicament dans le sang a diminué de 30\,\% par rapport à la quantité présente immédiatement après cette injection.

On modélise cette situation à l’aide de la suite $\suiten$ où, pour tout entier naturel $n$, $u_n$ désigne la quantité de médicament, exprimée en mg, présente dans le sang du patient immédiatement après l’injection de la $n$-ème heure. On a donc $u_0 = 2$.

\begin{enumerate}
	\item Calculer, selon cette modélisation, la quantité $u_1$ de médicament (en mg) présente dans le sang du patient immédiatement après l’injection de la première heure.
	\item Justifier que, pour tout entier naturel $n$, on a : $u_{n+1} = 0,7u_n + 1,8$.
	\item 
	\begin{enumerate}
		\item Montrer par récurrence que, pour tout entier naturel $n$, on a : $u_n \leqslant u_{n+1} < 6$.
		\item En déduire que la suite $\suiten$ est convergente. On note $\ell$ sa limite. 
		\item Déterminer la valeur de $\ell$. Interpréter cette valeur dans le contexte de l’exercice. 
	\end{enumerate}
	\item On considère la suite $\suiten[v]$ définie, pour tout entier naturel $n$, par $v_n=6-u_n$.
	\begin{enumerate}
		\item Montrer que la suite $\suiten[v]$ est une suite géométrique de raison $0,7$ dont on précisera le premier
		terme. 
		\item Déterminer l’expression de $\suiten[v]$ en fonction de $n$, puis de $u_n$ en fonction de $n$.
		\item Avec ce protocole, on arrête les injections lorsque la quantité de médicament présente dans le sang du patient est supérieure ou égale à $5,5$ mg.
		
		Déterminer, en détaillant les calculs, le nombre d’injections réalisées en appliquant ce protocole.
	\end{enumerate}
\end{enumerate}

