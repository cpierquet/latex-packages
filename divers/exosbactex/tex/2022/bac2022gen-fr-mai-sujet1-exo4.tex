\textit{Cet exercice est un questionnaire à choix multiple.\\
	Pour chaque question, une seule des quatre réponses proposées est exacte. Le candidat indiquera sur sa copie le numéro de la question et la réponse choisie. Aucune justification n’est demandée.\\
	Une réponse fausse, une réponse multiple ou l’absence de réponse à une question ne rapporte ni n’enlève de point.\\
	Les six questions sont indépendantes.}

\begin{enumerate}
	\item La courbe représentative de la fonction $f$ définie sur $\R$ par $f(x)=\dfrac{-2x^2+3x-1}{x^2+1}$ admet pour asymptote la droite d’équation :
	
	\smallskip
	
	\begin{tblr}{width=\linewidth,colspec={X[l,m]X[l,m]}}
		(a)~~$x=-2$&(b)~~$y=-1$ \\
		(c)~~$y=-2$&(d)~~$y=0$
	\end{tblr}
	%
	\item Soit $f$ la fonction définie sur $\R$ par $f(x)=x\,\e^{x^2}$.
	
	La primitive $F$ de $f$ sur $\R$ qui vérifie $F(0)=1$ est définie par :
	
	\begin{tblr}{width=\linewidth,colspec={X[l,m]X[l,m]}}
		(a)~~$F(x)=\dfrac{x^2}{2}\e^{x^2}$&(b)~~$F(x)=\dfrac{1}{2}\e^{x^2}$ \\
		(c)~~$F(x)=\big(1+2x^2\big)\e^{x^2}$&(d)~~$F(x)=\dfrac{1}{2}\e^{x^2}+\dfrac12$
	\end{tblr}
	%
	\item On donne ci-dessous la représentation graphique $\mathcal{C}_{f'}$ de la fonction dérivée $f'$ d’une fonction $f$ définie sur $\R$.
	
	\begin{center}
		\begin{tikzpicture}[x=0.75cm,y=0.75cm,xmin=-1,xmax=10,xgrille=1,xgrilles=0.25,ymin=-5,ymax=1.25,ygrille=1,ygrilles=0.25]
			\GrilleTikz \AxesTikz[ElargirOx=0/0,ElargirOy=0/0] \AxexTikz[Epaisseur=1pt,HautGrad=2pt]{1,2} \AxeyTikz[Epaisseur=1pt,HautGrad=2pt]{1}
			\draw (0,0) node[below left=2pt] {$0$} ;
			\clip (\xmin,\ymin) rectangle (\xmax,\ymax) ;
			\draw[line width=1pt,red,domain=\xmin:\xmax,samples=500] plot (\x,{(2*\x-4)*exp(-0.5*\x)}) ;
			\draw[red] (0.5,-3) node[right,font=\large] {$\mathcal{C}_{f'}$} ;
		\end{tikzpicture}
	\end{center}
	On peut affirmer que la fonction $f$ est :
	
	\begin{tblr}{width=\linewidth,colspec={X[l,m]X[l,m]}}
		(a)~~concave sur $]0;+\infty[$&(b)~~convexe sur $]0;+\infty[$\\
		(c)~~concave sur $[0;2]$&(d)~~convexe sur $[2;+\infty[$
	\end{tblr}
	%
	\item Parmi les primitives de la fonction $f$ définie sur $\R$ par $f(x)=3\,\e{-x^2}+2$ :
	
	\begin{tblr}{width=\linewidth,colspec={X[l,m]X[l,m]}}
		(a)~~toutes sont croissantes sur $\R$&(b)~~toutes sont décroissantes sur $\R$\\
		(c)~~certaines sont croissantes sur $\R$ et d’autres
		décroissantes sur $\R$&(d)~~toutes sont croissantes sur $]0;+\infty[$ et décroissantes sur $]0;+\infty[$
	\end{tblr}
	%
	\item La limite en $+\infty$ de la fonction $f$ définie sur l’intervalle $]0;+\infty[$ par $f(x)=\dfrac{2\,\ln(x)}{3x^2+1}$ est égale à :
	
	\begin{tblr}{width=\linewidth,colspec={X[l,m]X[l,m]X[l,m]X[l,m]}}
		(a)~~$\frac23$&(b)~~$+\infty$&
		(c)~~$-\infty$&(d)~~$0$
	\end{tblr}
	%
	\item L’équation $\e^{2x}+\e^{x}-12=0$ admet dans $\R$ :
	
	\begin{tblr}{width=\linewidth,colspec={X[l,m]X[l,m]X[l,m]X[l,m]}}
		(a)~~trois solutions&(b)~~deux solutions&(c)~~une seule solution&(d)~~aucune solution
	\end{tblr}
\end{enumerate}