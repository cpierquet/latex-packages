\textit{Cet exercice est un questionnaire à choix multiples. Pour chacune des questions suivantes, une seule des quatre réponses proposées est exacte. Les six questions sont indépendantes.\\
	Une réponse incorrecte, une réponse multiple ou l’absence de réponse à une question ne rapporte ni n’enlève de point. Pour répondre, indiquer sur la copie le numéro de la question et la lettre de la réponse choisie.\\
	Aucune justification n’est demandée.}

\begin{enumerate}
	\item Soit $f$ la fonction définie sur $\R$ par $f(x)=\dfrac{x}{\e^x}$.
	
	On suppose que $f$ est dérivable sur $\R$ et on note $f'$ sa fonction dérivée.
	
	\begin{tblr}{width=\linewidth,colspec={X[l]X[l]}}
		(a)~~$f'(x)=\e^{-x}$&(b)~~$f'(x)=x\,\e^{-x}$\\
		(c)~~$f'(x)=(1-x)\e^{-x}$&(d)~~$f'(x)=(1+x)\e^{-x}$
	\end{tblr}
	%
	\item Soit $f$ une fonction deux fois dérivable sur l'intervalle $[-3;1]$. On donne ci-dessous la représentation graphique de sa fonction dérivée seconde $f''$.
	
	\begin{center}
		\begin{tikzpicture}[x=2cm,y=1cm,xmin=-3.5,xmax=1.5,xgrille=1,xgrilles=0.5,ymin=-3.5,ymax=3.5,ygrille=1,ygrilles=0.5]
			\GrilleTikz[Affs=false] \AxesTikz[ElargirOx=0/0,ElargirOy=0/0]
			\AxexTikz{-3,-2,-1,1} \AxeyTikz{-3,-2,-1,1,2,3}
			\draw (0,0) node[below left=2pt] {0} ;
			\clip (\xmin,\ymin) rectangle (\xmax,\ymax) ;
			\draw[very thick,red,domain=-3:1,samples=250] plot (\x,{0.9743*\x*\x*\x + 2.9229*\x*\x - 0.9743*\x - 2.9229}) ;
		\end{tikzpicture}
	\end{center}
	
	On peut alors affirmer que :
	
	\begin{tblr}{width=\linewidth,colspec={X[l]X[l]}}
		(a)~~La fonction $f$ est convexe sur l'intervalle $[-1;1]$&(b)~~La fonction $f$ est concave sur l'intervalle $[-2;0]$\\
		(c)~~La fonction $f'$ est décroissante sur $[-2;0]$&(d)~~La fonction $f'$ admet un maximum en $x=-1$
	\end{tblr}
	%
	\item On considère la fonction $f$ définie sur $\R$ par $f(x)=x^3\,\e^{-x^2}$.
	
	Une primitive $F$ de la fonction $f$ est définie sur $\R$ par :
	
	\begin{tblr}{width=\linewidth,colspec={X[l]X[l]}}
		(a)~~$F(x)=-\frac16 (x^3+1)\e^{-x^2}$&(b)~~$F(x)=-\frac14 x^4\,\e^{-x^2}$\\
		(c)~~$F(x)=-\frac12 (x^2+1)\e^{-x^2}$&(d)~~$F(x)=x^2(3-2x^2)\e^{-x^2}$
	\end{tblr}
	%
	\item Que vaut $\displaystyle\lim_{x \to +\infty} \dfrac{\e^x+1}{\e^x-1}$ ?
	
	\begin{tblr}{width=\linewidth,colspec={X[l]X[l]}}
		(a)~~$-1$&(b)~~$1$\\
		(c)~~$+\infty$&(d)~~Elle n'existe pas
	\end{tblr}
	%
	\item On considère la fonction $f$ définie sur $\R$ par $f(x)=\e^{2x+1}$.
	
	La seule primitive $F$ sur $\R$ de la fonction $f$ telle que $F(0)=1$ est la fonction :
	
	\begin{tblr}{width=\linewidth,colspec={X[l]X[l]}}
		(a)~~$x \mapsto 2\,\e^{2x+1}-2\e+1$&(b)~~$x \mapsto \e^{2x+1}-\e$\\
		(c)~~$x \mapsto \dfrac12 \e^{2x+1} - \dfrac12 \e+1$&(d)~~$x \mapsto \e^{x^2+x}$
	\end{tblr}
	%
	\item Dans un repère, on a tracé ci-dessous la courbe représentative d'une fonction $f$ définie et deux fois dérivable sur $[-2;4]$.
	
	\begin{center}
		\begin{tikzpicture}[x=0.75cm,y=0.75cm,xmin=-2.5,xmax=5.5,xgrille=1,xgrilles=0.5,ymin=-2,ymax=3,ygrille=1,ygrilles=0.5]
			\tikzset{courbe/.style={line width=1pt,red,samples=250}}
			%grilles & axes
			\GrilleTikz \AxesTikz[ElargirOx=0/0,ElargirOy=0/0]
			\AxexTikz[AffGrad=false]{-2,-1,...,4}
			\AxeyTikz[AffGrad=false]{-2,-1,...,2}
			\draw (0,0) node[below left=2pt,font=\scriptsize] {0} ;
			\draw (1,0) node[below=4pt,font=\scriptsize] {1} ;
			\draw (0,1) node[left=4pt,font=\scriptsize] {1} ;
			\clip (\xmin,\ymin) rectangle (\xmax,\ymax) ; %on restreint les fonctions à la fenêtre
			\draw[very thick,blue,domain=\xmin:\xmax,samples=250] plot (\x,{1/6*\x*\x*\x-1/2*\x*\x+1}) ;
		\end{tikzpicture}
	\end{center}
	
	Parmi les courbes suivantes, laquelle représente la fonction $f''$, dérivée seconde de $f$ ?
	
	\begin{tblr}{width=\linewidth,colspec={X[l]X[l]}}
		(a)~~%
		{\begin{tikzpicture}[x=0.75cm,y=0.75cm,xmin=-2,xmax=4,xgrille=1,xgrilles=0.5,ymin=-3,ymax=3,ygrille=1,ygrilles=0.5,baseline=(current bounding box.north)]
				\tikzset{courbe/.style={line width=1pt,red,samples=250}}
				%grilles & axes
				\GrilleTikz \AxesTikz[ElargirOx=0/0,ElargirOy=0/0]
				\AxexTikz[AffGrad=false]{-2,-1,...,4}
				\AxeyTikz[AffGrad=false]{-2,-1,...,2}
				\draw (0,0) node[below left=2pt,font=\scriptsize] {0} ;
				\draw (1,0) node[below=4pt,font=\scriptsize] {1} ;
				\draw (0,1) node[left=4pt,font=\scriptsize] {1} ;
				\clip (\xmin,\ymin) rectangle (\xmax,\ymax) ; %on restreint les fonctions à la fenêtre
				\draw[very thick,red,domain=\xmin:\xmax,samples=2] plot (\x,{\x-1}) ;
		\end{tikzpicture}}%
		&(b)~~%
		{\begin{tikzpicture}[x=0.75cm,y=0.75cm,xmin=-2,xmax=4,xgrille=1,xgrilles=0.5,ymin=-3,ymax=3,ygrille=1,ygrilles=0.5,baseline=(current bounding box.north)]
				\tikzset{courbe/.style={line width=1pt,red,samples=250}}
				%grilles & axes
				\GrilleTikz \AxesTikz[ElargirOx=0/0,ElargirOy=0/0]
				\AxexTikz[AffGrad=false]{-2,-1,...,4}
				\AxeyTikz[AffGrad=false]{-2,-1,...,2}
				\draw (0,0) node[below left=2pt,font=\scriptsize] {0} ;
				\draw (1,0) node[below=4pt,font=\scriptsize] {1} ;
				\draw (0,1) node[left=4pt,font=\scriptsize] {1} ;
				\clip (\xmin,\ymin) rectangle (\xmax,\ymax) ; %on restreint les fonctions à la fenêtre
				\draw[very thick,red,domain=\xmin:\xmax,samples=2] plot (\x,{1-\x}) ;
		\end{tikzpicture}}%
		\\
		(c)~~%
		{\begin{tikzpicture}[x=0.75cm,y=0.75cm,xmin=-2,xmax=4,xgrille=1,xgrilles=0.5,ymin=-3,ymax=3,ygrille=1,ygrilles=0.5,baseline=(current bounding box.north)]
				\tikzset{courbe/.style={line width=1pt,red,samples=250}}
				%grilles & axes
				\GrilleTikz \AxesTikz[ElargirOx=0/0,ElargirOy=0/0]
				\AxexTikz[AffGrad=false]{-2,-1,...,4}
				\AxeyTikz[AffGrad=false]{-2,-1,...,2}
				\draw (0,0) node[below left=2pt,font=\scriptsize] {0} ;
				\draw (1,0) node[below=4pt,font=\scriptsize] {1} ;
				\draw (0,1) node[left=4pt,font=\scriptsize] {1} ;
				\clip (\xmin,\ymin) rectangle (\xmax,\ymax) ; %on restreint les fonctions à la fenêtre
				\draw[very thick,red,domain=\xmin:\xmax,samples=250] plot (\x,{1/3*\x*(\x-2)}) ;
		\end{tikzpicture}}%
		&(d)~~%
		{\begin{tikzpicture}[x=0.75cm,y=0.75cm,xmin=-2,xmax=4,xgrille=1,xgrilles=0.5,ymin=-3,ymax=3,ygrille=1,ygrilles=0.5,baseline=(current bounding box.north)]
				\tikzset{courbe/.style={line width=1pt,red,samples=250}}
				%grilles & axes
				\GrilleTikz \AxesTikz[ElargirOx=0/0,ElargirOy=0/0]
				\AxexTikz[AffGrad=false]{-2,-1,...,4}
				\AxeyTikz[AffGrad=false]{-2,-1,...,2}
				\draw (0,0) node[below left=2pt,font=\scriptsize] {0} ;
				\draw (1,0) node[below=4pt,font=\scriptsize] {1} ;
				\draw (0,1) node[left=4pt,font=\scriptsize] {1} ;
				\clip (\xmin,\ymin) rectangle (\xmax,\ymax) ; %on restreint les fonctions à la fenêtre
				\draw[very thick,red,domain=\xmin:\xmax,samples=250] plot (\x,{0.5*(\x-1)*(\x-1)}) ;
		\end{tikzpicture}}%
	\end{tblr}
\end{enumerate}

