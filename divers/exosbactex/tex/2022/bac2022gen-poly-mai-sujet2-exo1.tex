\emph{Cet exercice est un questionnaire à choix multiples. Pour chacune des six questions suivantes, une seule des quatre réponses proposées est exacte.\\
	Une réponse fausse, une réponse multiple ou l'absence de réponse à une question ne rapporte ni n'enlève de point.\\
	Pour répondre, indiquer sur la copie le numéro de la question et la lettre de la réponse
	choisie. Aucune justification n'est demandée.}

\begin{enumerate}
	\item On considère la fonction $f$ définie et dérivable sur $\intervOO{0}{+\infty}$ par $f(x)=x\,\ln(x)-x+1$.
	
	Parmi les quatre expressions suivantes, laquelle est celle de la fonction dérivée de $f$ ?
	
	\begin{tblr}{width=\linewidth,colspec={X[l]X[l]X[l]X[l]}}
		(a)~~$\ln(x)$&(b)~~$\dfrac{1}{x}-1$&(c)~~$\ln(x)-2$&(d)~~$\ln(x)-1$
	\end{tblr}
	\item On considère la fonction $g$ définie sur $\intervOO{0}{+\infty}$ par $g(x) = x^2 \left[1-\ln(x)\right]$.
	
	Parmi les quatre affirmations suivantes, laquelle est correcte ?
	
	\begin{tblr}{width=\linewidth,colspec={X[l]X[l]}}
		(a)~~$\lim_{x \to 0} g(x)=+\infty$&(b)~~$\lim_{x \to 0} g(x)=-\infty$\\
		(c)~~$\lim_{x \to 0} g(x)=0$&(d)~~{La fonction $g$ n'admet pas de limite en 0}
	\end{tblr}
	\item On considère la fonction $f$ définie sur $\R$ par $f(x) = x^3-0,9x^2-0,1x$.
	
	Le nombre de solutions de l'équation $f(x) = 0$ sur $\R$ est :
	
	\begin{tblr}{width=\linewidth,colspec={X[l]X[l]X[l]X[l]}}
		(a)~~$0$&(b)~~$1$&(c)~~$2$&(d)~~$3$
	\end{tblr}
	\item Si $H$ est une primitive d'une fonction $h$ définie et continue sur $\R$, et si $k$ est la fonction définie sur $\R$ par $k(x) = h(2x)$, alors, une primitive $K$
	de $k$ est définie sur $\R$ par :
	
	\begin{tblr}{width=\linewidth,colspec={X[l]X[l]X[l]X[l]}}
		(a)~~$K(x)=H(2x)$&(b)~~$K(x)=2H(2x)$&(c)~~$K(x)=\frac12H(2x)$&(d)~~$K(x)=2H(x)$
	\end{tblr}
	\item L'équation réduite de la tangente au point d'abscisse 1 de la courbe de la fonction $f$ définie sur $\R$ par \mbox{$f(x) = x\,\e^{x}$} est :
	
	\begin{tblr}{width=\linewidth,colspec={X[l]X[l]X[l]X[l]}}
		(a)~~$y=\e\,x+\e$&(b)~~$y=2\e\,x-\e$&(c)~~$y=2\e\,x+\e$&(d)~~$y=\e\,x$
	\end{tblr}
	\item Les nombres entiers $n$ solutions de l'inéquation $(0,2)^n < 0,001$ sont tous les
	nombres entiers $n$ tels que 
	
	\begin{tblr}{width=\linewidth,colspec={X[l]X[l]X[l]X[l]}}
		(a)~~$n \leqslant 4$&(b)~~$n \leqslant 5$&(c)~~$n \geqslant 4$&(d)~~$n \leqslant 5$
	\end{tblr}
\end{enumerate}

