Au basket-bail, il existe deux sortes de tir:

\begin{itemize}
	\item les tirs à deux points.
	
	\hspace{4mm}Ils sont réalisés près du panier et rapportent deux points s'ils sont réussis.
	\item les tirs à trois points.
	
	\hspace{4mm}Ils sont réalisés loin du panier et rapportent trois points s'ils sont réussis.
\end{itemize}

Stéphanie s'entraîne au tir. On dispose des données suivantes :

\begin{itemize}
	\item Un quart de ses tirs sont des tirs à deux points. Parmi eux, 60\,\% sont réussis.
	\item Trois quarts de ses tirs sont des tirs à trois points. Parmi eux, 35\,\% sont réussis.
\end{itemize}

\begin{enumerate}
	\item Stéphanie réalise un tir. 
	
	On considère les évènements suivants :
	%
	\begin{itemize}
		\item[] $D$ : \og Il s'agit d'un tir à deux points \fg.
		\item[] $R$ : \og le tir est réussi \fg.
	\end{itemize}
	\begin{enumerate}
		\item Représenter la situation à l'aide d'un arbre de probabilités.
		\item Calculer la probabilité $p\left(\overline{D} \cap R\right)$.
		\item Démontrer que la probabilité que Stéphanie réussisse un tir est égale à $0,4125$.
		\item Stéphanie réussit un tir. Calculer la probabilité qu'il s'agisse d'un tir à trois points. Arrondir le résultat au centième.
	\end{enumerate}
	\item Stéphanie réalise à présent une série de $10$ tirs à trois points.
	
	On note $X$ la variable aléatoire qui compte le nombre de tirs réussis.
	
	On considère que les tirs sont indépendants. On rappelle que la probabilité que Stéphanie réussisse un tir à trois points est égale à $0,35$.
	\begin{enumerate}
		\item Justifier que $X$ suit une loi binomiale. Préciser ses paramètres.
		\item Calculer l'espérance de $X$. Interpréter le résultat dans le contexte de l'exercice.
		\item Déterminer la probabilité que Stéphanie rate $4$ tirs ou plus. Arrondir le résultat au centième.
		\item Déterminer la probabilité que Stéphanie rate au plus $4$ tirs. Arrondir le résultat au centième.
	\end{enumerate}	
	\item Soit $n$ un entier naturel non nul.
	
	Stéphanie souhaite réaliser une série de $n$ tirs à trois points. 
	
	On considère que les tirs sont indépendants. On rappelle que la probabilité qu'elle réussisse un tir à trois points est égale à $0,35$.
	
	Déterminer la valeur minimale de $n$ pour que la probabilité que Stéphanie réussisse au moins un tir parmi les $n$ tirs soit supérieure ou égale à $0,99$.
\end{enumerate}

