L'espace est muni d'un repère orthonormé $\Rijk$.

\smallskip

On considère les points $A(5;0;-1)$, $B(1;4;-1)$, $C(1;0;3)$, $D(5;4;3)$ et $E(10;9;8)$.

\begin{enumerate}
	\item 
	\begin{enumerate}
		\item Soit $R$ le milieu du segment $[AB]$.
		
		Calculer les coordonnées du point $R$ ainsi que les coordonnées du vecteur $\vect{AB}$.
		\item Soit $\mathcal{P}_1$ le plan passant par le point $R$ et dont $\vect{AB}$ est un vecteur normal.
		
		Démontrer qu'une équation cartésienne du plan $\mathcal{P}_1$ est : \[ x-y-1=0.\]
		\item Démontrer que le point $E$ appartient au plan $\mathcal{P}_1$ et que $EA = EB$.
	\end{enumerate}
	\item On considère le plan $\mathcal{P}_2$ d'équation cartésienne $x-z-2=0$.
	\begin{enumerate}
		\item Justifier que les plans $\mathcal{P}_1$ et $\mathcal{P}_2$ sont sécants.
		\item On note la droite d'intersection de $\mathcal{P}_1$ et $\mathcal{P}_2$.
		
		Démontrer qu'une représentation paramétrique de la droite est : \[ \begin{dcases} x=2+t \\y=1+t \qquad (t \in \R). \\ z=t \end{dcases} \]
	\end{enumerate}
	\item On considère le plan $\mathcal{P}_3$ d'équation cartésienne $y+z-3=0$.
	
	Justifier que la droite $\Delta$ est sécante au plan $\mathcal{P}_3$ en un point $\Omega$ dont on déterminera les coordonnées. 
\end{enumerate}

Si $S$ et $T$ sont deux points distincts de l'espace, on rappelle que l'ensemble des points $M$ de l'espace tels que $MS= MT$ est un plan, appelé plan médiateur du segment $[ST]$.

On admet que les plans $\mathcal{P}_1$, $\mathcal{P}_2$ et $\mathcal{P}_3$ sont les plans médiateurs respectifs des segments $[AB]$, $[AC]$ et $[AD]$.

\begin{enumerate}[resume]
	\item 
	\begin{enumerate}
		\item Justifier que $\Omega A = \Omega B = \Omega C = \Omega D$.
		\item En déduire que les points $A$, $B$, $C$ et $D$ appartiennent à une même sphère dont on précisera le centre et le rayon. 
	\end{enumerate}
\end{enumerate}