\textit{Cet exercice est un questionnaire à choix multiple.\\
	Pour chaque question, une seule des quatre réponses proposées est exacte. Le candidat indiquera sur sa copie le numéro de la question et la réponse choisie. Aucune justification n’est demandée.\\
	Une réponse fausse, une réponse multiple ou l’absence de réponse à une question ne rapporte ni n’enlève de point.\\
	Les six questions sont indépendantes.}

\smallskip

Pour les questions \textbf{1} à \textbf{3} ci-dessous, on considère une fonction $f$ définie et deux fois dérivable sur $\R$.

La \textbf{courbe de sa fonction dérivée} $f'$ est donnée ci-dessous. On admet que $f'$ admet un maximum en $-\dfrac32$ et que sa courbe coupe l'axe des abscisses au point de coordonnées $\left( -\dfrac12;0 \right)$

\begin{center}
	\begin{tikzpicture}[x=1.1cm,y=1.1cm,xmin=-5.25,xmax=1.25,xgrille=1,xgrilles=0.25,ymin=-2.75,ymax=1.5,ygrille=1,ygrilles=0.25]
		\GrilleTikz \AxesTikz[ElargirOx=0/0,ElargirOy=0/0] \AxexTikz{-5,-4,-3,-2,-1,1} \AxeyTikz{-2,-1,1}
		\draw (0,0) node[below left=2pt] {0} ;
		\clip (\xmin,\ymin) rectangle (\xmax,\ymax) ;
		\draw[red,very thick,samples=250,domain=\xmin:\xmax] plot (\x,{(-2*\x-1)*exp(\x)}) ;
	\end{tikzpicture}
\end{center}

\textbf{Question 1 :}
%
\begin{enumerate}[label=(\alph*)]
	\item La fonction $f$ admet un maximum en $-\dfrac32$.
	\item La fonction $f$ admet un maximum en $-\dfrac12$.
	\item La fonction $f$ admet un minimum en $-\dfrac12$.
	\item Au point d’abscisse $-1$, la courbe de la fonction $f$ admet une tangente horizontale.
\end{enumerate}

\textbf{Question 2 :}
%
\begin{enumerate}[label=(\alph*)]
	\item La fonction $f$ est convexe sur $\left] -\infty ; -\dfrac32 \right[$.
	\item La fonction $f$ est convexe sur $\left] -\infty ; -\dfrac12 \right[$.
	\item La courbe $\mathcal{C}_f$ représentant la fonction $f$ n’admet pas de point d’inflexion.
	\item La fonction $f$ est concave sur $\left] -\infty ; -\dfrac12 \right[$.
\end{enumerate}

\textbf{Question 3 :}

\smallskip

La dérivée seconde $f''$ de la fonction $f$ vérifie :
%
\begin{enumerate}[label=(\alph*)]
	\item $f''(x) \geqslant 0$ pour $x \in \left] -\infty ; -\dfrac12 \right[$.
	\item $f''(x) \geqslant 0$ pour $x \in [-2;-1]$.
	\item $f''\left(\frac{-3}{2}\right)=0$.
	\item $f''(-3)=0$.
\end{enumerate}

\textbf{Question 4 :} :

\smallskip

On considère trois suites $\suiten$, $\suiten[v]$ et $\suiten[w]$.

On sait que, pour tout entier naturel $n$, on a : $u_n \leqslant v_n \leqslant w_n$ et de plus : $\lim_{n \to +\infty} u_n=1$ et $\lim_{n \to +\infty} w_n=3$.

On peut alors affirmer que :
%
\begin{enumerate}[label=(\alph*)]
	\item La suite $\suiten[v]$ converge.
	\item Si la suite $\suiten$ est croissante alors la suite $\suiten[v]$ est minorée par $u_0$.
	\item $1 \leqslant v_0 \leqslant 3$.
	\item La suite $\suiten[v]$ diverge.
\end{enumerate}

\textbf{Question 5 :}

\smallskip

On considère une suite $\suiten$ telle que, pour tout entier naturel $n$ non nul : $u_n \leqslant u_{n+1} \leqslant \dfrac{1}{n}$.

On peut alors affirmer que :
%
\begin{enumerate}[label=(\alph*)]
	\item La suite $\suiten$ diverge.
	\item La suite $\suiten$ converge.
	\item $\lim_{n \to +\infty} u_n=0$.
	\item $\lim_{n \to +\infty} u_n=1$.
\end{enumerate}

\textbf{Question 6 :}

\smallskip

On considère $\suiten$ une suite réelle telle que pour tout entier naturel $n$, on a : $n < u_n < n + 1$.

On peut affirmer que :
%
\begin{enumerate}[label=(\alph*)]
	\item Il existe un entier naturel $N$ tel que $u_N$ est un entier.
	\item La suite $\suiten$ est croissante.
	\item La suite $\suiten$ est convergente.
	\item La suite $\suiten$ n'a pas de limite.
\end{enumerate}

