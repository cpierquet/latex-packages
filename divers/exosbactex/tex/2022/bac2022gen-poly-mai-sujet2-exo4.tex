On considère le cube ABCDEFGH d'arête de longueur 1.

L'espace est muni du repère orthonormé $\left(A;\vect{AB}\vect{AD},\vect{AE} \right)$. Le point $I$ est le milieu du segment $[EF]$, $K$ le centre du carré $ADHE$ et $O$ le milieu du segment $[AG]$.

\begin{center}
	\begin{tikzpicture}[x=5cm,y=5cm,line join=bevel]
		\PaveTikz[Aff,Cube,Largeur=1,Angle=20,Fuite=0.33,Epaisseur={very thick}]
		\draw[thick] (0,0)--(0.5,1)--(G) ;
		\draw[thick,dotted] (A)--(G) ;
		\foreach \i in {A,B,C,D,E,F,G,H} \filldraw (\i) circle[radius=2pt] ;
		\filldraw ($(A)!0.5!(H)$) circle[radius=2pt] node[left] {K} ;
		\filldraw ($(E)!0.5!(F)$) circle[radius=2pt] node[above left] {I} ;
		\filldraw ($(A)!0.5!(G)$) circle[radius=2pt] node[below] {O} ;
	\end{tikzpicture}
\end{center}

\emph{Le but de l'exercice est de calculer de deux manières différentes, la distance du point $B$ au plan $(AIG)$.}

\medskip

\textbf{Partie 1. Première méthode}

\begin{enumerate}
	\item Donner, sans justification, les coordonnées des points $A$, $B$, et $G$.
	
	On admet que les points $I$ et $K$ ont pour coordonnées $I\left(\frac12;0;1\right)$ et $K\left(0;\frac12;\frac12\right)$.
	\item Démontrer que la droite $(BK)$ est orthogonale au plan $(AIG)$.
	\item Vérifier qu'une équation cartésienne du plan $(AIG)$ est : $2x-y-z=0$.
	\item Donner une représentation paramétrique de la droite $(BK)$.
	\item En déduire que le projeté orthogonal $L$ du point $B$ sur le plan $(AIG)$ a pour
	coordonnées $K\left(\frac13;\frac13;\frac13\right)$.
	\item Déterminer la distance du point $B$ au plan $(AIG)$.
\end{enumerate}

\textbf{Partie 2. Deuxième méthode}

\medskip

\emph{On rappelle que le volume $\mathcal{V}$ d'une pyramide est donné par la formule $\mathcal{V}=\dfrac13 \times b \times h$ où $b$ est l'aire d'une base et $h$ la hauteur associée à cette base.}

\begin{enumerate}
	\item 
	\begin{enumerate}
		\item Justifier que dans le tétraèdre $ABIG$, $[GF]$ est la hauteur relative à la base $AIB$.
		\item En déduire le volume du tétraèdre $ABIG$.
	\end{enumerate}
	\item On admet que $AI=IG=\dfrac{\sqrt{5}}{2}$ et que $AG=\sqrt{3}$.
	
	Démontrer que l'aire du triangle isocèle $AIG$ est égale à $\dfrac{\sqrt{6}}{4}$ unité d'aire.
	\item En déduire la distance du point $B$ au plan $(AIG)$.
\end{enumerate}