\textbf{Partie A}

\medskip

Le système d'alarme d'une entreprise fonctionne de telle sorte que, si un danger se présente, l'alarme s'active avec une probabilité de $0,97$. 

La probabilité qu'un danger se présente est de $0,01$ et la probabilité que l'alarme s'active est de \num{0,01465}.

On note $A$ l'évènement \og  l'alarme s'active \fg{}  et $D$ l'événement \og  un danger se présente \fg . 

On note $\overline{M}$ l'évènement contraire d'un évènement $M$ et $P(M)$ la probabilité de l'évènement $M$.

\begin{enumerate}
	\item Représenter la situation par un arbre pondéré qui sera complété au fur et à mesure de l'exercice.
	\item  
	\begin{enumerate}
		\item Calculer la probabilité qu'un danger se présente et que l'alarme s'active.
		\item En déduire la probabilité qu'un danger se présente sachant que l'alarme s'active.
		
		Arrondir le résultat à $10^{-3}$.
	\end{enumerate}
	\item  Montrer que la probabilité que l'alarme s'active sachant qu'aucun danger ne s'est présenté est $0,005$.
	\item On considère qu'une alarme ne fonctionne pas normalement lorsqu'un danger se présente et qu'elle ne s'active pas ou bien lorsqu'aucun danger ne se présente et qu'elle s'active.
	
	Montrer que la probabilité que l'alarme ne fonctionne pas normalement est inférieure à $0,01$.
\end{enumerate}

\textbf{Partie B}

\medskip

Une usine fabrique en grande quantité des systèmes d'alarme. On prélève successivement et au hasard $5$ systèmes d'alarme dans la production de l'usine. Ce prélèvement est assimilé à un tirage avec remise.

On note $S$ l'évènement \og l'alarme ne fonctionne pas normalement \fg{}  et on admet que $P(S) = \num{0,00525}$.

On considère $X$ la variable aléatoire qui donne le nombre de systèmes d'alarme ne fonctionnant pas normalement parmi les $5$ systèmes d'alarme prélevés.

Les résultats seront arrondis à $10^{-4}$.

\begin{enumerate}
	\item Donner la loi de probabilité suivie par la variable aléatoire X et préciser ses paramètres.
	\item Calculer la probabilité que, dans le lot prélevé, un seul système d'alarme ne fonctionne pas normalement.
	\item Calculer la probabilité que, dans le lot prélevé, au moins un système d'alarme ne fonctionne pas normalement.
\end{enumerate}

\textbf{Partie C}

\medskip

Soit $n$ un entier naturel non nul. On prélève successivement et au hasard $n$ systèmes d'alarme. Ce prélèvement est assimilé à un tirage avec remise.

Déterminer le plus petit entier $n$ tel que la probabilité d'avoir, dans le lot prélevé, au moins un système d'alarme qui ne fonctionne pas normalement soit supérieure à $0,07$.

