Soit $g$ la fonction définie sur l'intervalle $]0;+\infty[$ par \[g(x) = 1+ x^2[1 - 2 \ln (x)].\]

La fonction $g$ est dérivable sur l'intervalle $]0;+\infty[$ et on note $g'$ sa fonction dérivée.

On appelle $\mathcal{C}$ la courbe représentative de la fonction $g$ dans un repère orthonormé du plan.

\medskip

\textbf{Partie A}

\medskip

\begin{enumerate}
	\item Justifier que $g(\e)$ est strictement négatif.
	\item Justifier que $\displaystyle\lim_{x \to + \infty} g(x) = - \infty$.
	\item 
	\begin{enumerate}
		\item Montrer que, pour tout $x$ appartenant à l'intervalle $]0;+\infty[$, $g'(x) = -4x \ln (x)$.
		\item Étudier le sens de variation de la fonction $g$ sur l'intervalle $]0;+\infty[$ 
		\item Montrer que l'équation $g(x) = 0$ admet une unique solution, notée $\alpha$, sur l'intervalle $[1;+\infty[$.
		\item Donner un encadrement de $\alpha$ d'amplitude $10^{-2}$
	\end{enumerate}
	\item Déduire de ce qui précède le signe de la fonction $g$ sur l'intervalle $[1;+\infty[$.
\end{enumerate}

\textbf{Partie B}

\begin{enumerate}
	\item On admet que, pour tout $x$ appartenant à l'intervalle $[1;\alpha]$, $g''(x) = - 4[\ln (x) + 1]$.
	
	Justifier que la fonction $g$ est concave sur l'intervalle $[1;\alpha]$.
\end{enumerate}

\begin{wrapstuff}[r,leftsep=1.5em,rightsep=1em]
	\begin{tikzpicture}[x=1.3cm,y=1.3cm,xmin=0,xmax=3,ymin=0,ymax=3]
		\GrilleTikz \AxesTikz[ElargirOx=0/0,ElargirOy=0/0] \AxexTikz{0,1,2} \AxeyTikz{0,1,2}
		\draw[samples=250,very thick,red,domain=1:1.9] plot (\x,{1+\x*\x-2*\x*\x*ln(\x)}) ;
		\draw (1,2) node[above left,red,font=\large] {$A$} ;
		\draw (1.9,0) node[above right,red,font=\large] {$B$} ;
		\draw (1.9,0.05) node[below,font=\large] {$\alpha$} ;
		\draw (1.5,1.5) node[above right,red,font=\large] {$\mathcal{C}$} ;
	\end{tikzpicture}
\end{wrapstuff}
%
Sur la figure ci-contre, $A$ et $B$ sont les points de la courbe $\mathcal{C}$ d'abscisses respectives 1 et $\alpha$.

\begin{enumerate}[resume]
	\item 
	\begin{enumerate}
		\item Déterminer l'équation réduite de la droite $(AB)$.
		\item En déduire que pour tout réel $x$ appartenant à l'intervalle $[1;\alpha]$, on a :
		
		\smallskip
		
		\hfill$g(x) \geqslant \dfrac{- 2}{\alpha - 1} x + \dfrac{2\alpha}{\alpha - 1}$.\hfill~
		
		\vspace{3cm}
	\end{enumerate}
\end{enumerate}

