Dans l’espace rapporté à un repère orthonormé $\Rijk$, on considère :

\begin{itemize}
	\item le point $A$ de coordonnées $(-1;1;3)$ ;
	\item la droite $\mathcal{D}$ dont une représentation paramétrique est : $\begin{dcases} x=1+2t\\y=2-t\\z=2+2t\end{dcases}$, $t \in \mathbb{R}$.
\end{itemize}

On admet que le point $A$ n’appartient pas à la droite $\mathcal{D}$.

\begin{enumerate}
	\item 
	\begin{enumerate}
		\item Donner les coordonnées d’un vecteur directeur $\vect{u}$ de la droite $\mathcal{D}$.
		\item Montrer que le point $B(-1\,;\,3\,;\,0)$ appartient à la droite $\mathcal{D}$.
		\item Calculer le produit scalaire $\vect{AB}\cdot\vect{u}$.
	\end{enumerate}
	\item On note $\mathcal{P}$ le plan passant par le point $A$ et orthogonal à la droite $\mathcal{D}$, et on appelle $H$ le point d’intersection du plan $\mathcal{P}$ et de la droite $\mathcal{D}$. Ainsi, $H$ est le projeté orthogonal de $A$ sur la droite $\mathcal{D}$.
	\begin{enumerate}
		\item Montrer que le plan $\mathcal{P}$ admet pour équation cartésienne : $2x-y+2z-3=0$.
		\item En déduire que le point $H$ a pour coordonnées $\left( \dfrac{7}{9};\dfrac{19}{9};\dfrac{16}{9}\right)$.
		\item Calculer la longueur $AH$. On donnera une valeur exacte.
	\end{enumerate}
	\item Dans cette question, on se propose de retrouver les coordonnées du point $H$, projeté orthogonal du point $A$ sur la droite $\mathcal{D}$, par une autre méthode.
	
	\smallskip
	
	On rappelle que le point $B(-1;3;0)$ appartient à la droite $\mathcal{D}$ et que le vecteur $\vect{u}$ est un vecteur directeur de la droite $\mathcal{P}$.
	\begin{enumerate}
		\item Justifier qu’il existe un nombre réel $k$ tel que $\vect{HB}=k\vect{u}$.
		\item Montrer que $k=\dfrac{\vect{AB}\cdot\vect{u}}{||\vect{u}^2||}$.
		\item Calculer la valeur du nombre réel $k$ et retrouver les coordonnées du point $H$. 
	\end{enumerate}
	\item On considère un point $C$ appartenant au plan $\mathcal{P}$ tel que le volume du tétraèdre $ABCH$ soit égal à $\frac89$.
	
	Calculer l’aire du triangle $ACH$.
	
	On rappelle que le volume d’un tétraèdre est donné par : $\mathcal{V}=\dfrac13 \times \mathcal{B} \times h$ où $\mathcal{B}$ désigne l’aire d’une base et $h$ la hauteur relative à cette base.
\end{enumerate}

