On considère le cube $ABCDEFGH$ de côté 1 représenté ci-dessous.

\begin{center}
	\begin{tikzpicture}[line join=bevel]
		\PaveTikz[Cube,Largeur=6,Aff]
	\end{tikzpicture}
\end{center}

On munit l'espace du repère orthonormé $\left( A;\vect{AB};\vect{AC};\vect{AE} \right)$.

\begin{enumerate}
	\item 
	\begin{enumerate}
		\item Justifier que les droites $(AH)$ et $(ED)$ sont perpendiculaires.
		\item Justifier que la droite $(GH)$ est orthogonale au plan $(EDH)$.
		\item En déduire que la droite $(ED)$ est orthogonale au plan $(AGH)$.
	\end{enumerate}
	\item Donner les coordonnées du vecteur $\vect{ED}$.
	
	Déduire de la question 1.(c) qu'une équation cartésienne du plan $(AGH)$ est :\[y-z=0.\]
	\item On désigne par $L$ le point de coordonnées $\left(-\frac23;1;0)\right)$.
	\begin{enumerate}
		\item Déterminer une représentation paramétrique de la droite $(EL)$.
		\item Déterminer l'intersection de la droite $(EL)$ et du plan $(AGH)$.
		\item Démontrer que le projeté orthogonal de $L$ sur le plan $(AGH)$ est le point $K$ de coordonnées \mbox{$\left(\frac23;\frac12;\frac12\right)$}.
		\item Montrer que la distance du point $L$ au plan $(AGH)$ est égale à $\dfrac{\sqrt{2}}{2}$.
		\item Déterminer le volume du tétraèdre $LAGH$.
		
		On rappelle que le volume $\mathcal{V}$ d'un tétraèdre est donné par la formule : \[ \mathcal{V}=\dfrac13 \times (\text{aire de la base}) \times \text{hauteur}. \]
	\end{enumerate}
\end{enumerate}

