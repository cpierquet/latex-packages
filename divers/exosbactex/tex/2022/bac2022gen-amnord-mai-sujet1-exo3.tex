Dans l'espace muni d'un repère orthonormé $\Rijk$ d'unité 1 cm, on considère les points suivants : %
\[\text{J}(2;0;1) \text{, } \text{K}(1;2;1) \text{ et } \text{L}(-2;-2;-2).\]
%
\begin{enumerate}
	\item 
	\begin{enumerate}
		\item Montrer que le triangle $JKL$ est rectangle en $J$.
		\item Calculer la valeur exacte de l'aire du triangle $JKL$ en cm$^2$.
		\item Déterminer une valeur approchée au dixième près de l'angle géométrique $\widehat{\text{JKL}}$.
	\end{enumerate}
	\item
	\begin{enumerate}
		\item Démontrer que le vecteur $\vect{n}$ de coordonnées $\begin{pmatrix}3\\3\\-10\end{pmatrix}$ est un vecteur normal au plan $(JKL)$.
		\item En déduire une équation cartésienne du plan $(JKL)$.
	\end{enumerate}
\end{enumerate}

Dans la suite, $T$ désigne le point de coordonnées $(10;9;-6)$.

\begin{enumerate}[resume]
	\item
	\begin{enumerate}
		\item Déterminer une représentation paramétrique de la droite $\Delta$ orthogonale au plan $(JKL)$ et passant par $T$.
		\item Déterminer les coordonnées du point $H$, projeté orthogonal du point $T$ sur le plan $(JKL)$.
		\item On rappelle que le volume $\mathcal{V}$ d'un tétraèdre est donné par la formule : \[\mathcal{V}  = \dfrac13 \mathcal{B} \times h \text{ où }  \mathcal{B} \text{ désigne l'aire d'une base et } h \text{ la hauteur correspondante}.\]
		%
		Calculer la valeur exacte du volume du tétraèdre $JKLT$ en cm$^3$.
	\end{enumerate}
\end{enumerate}

