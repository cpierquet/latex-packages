Dans l'espace rapporté à un repère orthonormé $\Rijk$, on considère les points \[A(-1;-1;3),\qquad B(1;1;2),\qquad  C(1;-1;7).\]
%
On considère également la droite $\Delta$ passant par les points $D(-1;6;8)$ et $E(11;- 9;2)$.

\begin{enumerate}
	\item 
	\begin{enumerate}
		\item Vérifier que la droite $\Delta$ admet pour représentation paramétrique : \[ \begin{dcases} x=-1+4t \\ y = \phantom{-}6 - 5t \text{~~avec } t \in \R. \\ z = \phantom{-}8 - 2t \end{dcases}\]
		\item Préciser une représentation paramétrique de la droite $\Delta'$ parallèle à $\Delta$ et passant par l'origine O du repère.
		\item Le point F$(1,36;-1,7;-0,7)$ appartient-il à la droite $\Delta'$ ?
	\end{enumerate}	
	\item
	\begin{enumerate}
		\item Montrer que les points A, B et C définissent un plan.
		\item Montrer que la droite $\Delta$ est perpendiculaire au plan (ABC).
		\item Montrer qu'une équation cartésienne du plan (ABC) est: $4x - Sy - 2z + 5 = 0$.
	\end{enumerate}
	\item
	\begin{enumerate}
		\item Montrer que le point $G(7;-4;4)$ appartient à la droite $\Delta$.
		\item Déterminer les coordonnées du point $H$, projeté orthogonal du point $G$ sur le plan $(ABC)$.
		\item En déduire que la distance du point $G$ au plan $(ABC)$ est égale à $3\sqrt 5$.
	\end{enumerate}
	\item
	\begin{enumerate}
		\item Montrer que le triangle $ABC$ est rectangle en $A$.
		\item Calculer le volume $V$ du tétraèdre $ABCG$.
		
		\emph{On rappelle que le volume $V$ d'un tétraèdre est donné par la formule $V = \dfrac13 \times  B \times h$ où $B$ est l'aire d'une base et $h$ la hauteur correspondant à cette base.}
	\end{enumerate}
\end{enumerate}