Parmi les angines, un quart nécessite la prise d'antibiotiques, les autres non.

Afin d'éviter de prescrire inutilement des antibiotiques, les médecins disposent d'un test de diagnostic ayant les caractéristiques suivantes :

\begin{itemize}
	\item lorsque l'angine nécessite la prise d'antibiotiques, le test est positif dans 90\,\% des cas;
	\item lorsque l'angine ne nécessite pas la prise d'antibiotiques, le test est négatif dans 95\,\% des cas.
\end{itemize}

Les probabilités demandées dans la suite de l'exercice seront arrondies à $10^{-4}$ près si nécessaire.

\medskip

\textbf{Partie 1}

\medskip

Un patient atteint d'angine et ayant subi le test est choisi au hasard. 

On considère les évènements suivants :

\begin{itemize}
	\item $A$ : « le patient est atteint d'une angine nécessitant la prise d'antibiotiques »{} ; 
	\item $T$ : « le test est positif »{} ;
	\item $\overline{A}$ et $\overline{T}$ sont respectivement les évènements contraires de $A$ et $T$.
\end{itemize}

\begin{enumerate}
	\item Calculer $P(A \cap T)$. On pourra s'appuyer sur un arbre pondéré.
	\item Démontrer que $P(T) = \num{0,2625}$.
	\item On choisit un patient ayant un test positif. Calculer la probabilité qu'il soit atteint d'une angine nécessitant la prise d'antibiotiques.
	\item
	\begin{enumerate}
		\item Parmi les évènements suivants, déterminer ceux qui correspondent à un résultat erroné du test :  $A \cap T$, $\overline{A} \cap T$, $A \cap \overline{T}$, $\overline{A} \cap \overline{T}$.
		\item On définit l'évènement $E$ : « le test fournit un résultat erroné ». 
		
		Démontrer que $p(E) = \num{0,0625}$.
	\end{enumerate}
\end{enumerate}

\textbf{Partie 2}

\medskip

On sélectionne au hasard un échantillon de $n$ patients qui ont été testés.

On admet que l'on peut assimiler ce choix d'échantillon à un tirage avec remise.

On note $X$ la variable aléatoire qui donne le nombre de patients de cet échantillon ayant un test erroné.

\begin{enumerate}
	\item On suppose que $n = 50$.
	\begin{enumerate}
		\item Justifier que la variable aléatoire $X$ suit une loi binomiale $\mathcal{B}(n;p)$ de paramètres
		$n = 50$ et $p = \num{0,0625}.$
		\item Calculer $P(X = 7)$.
		\item Calculer la probabilité qu'il y ait au moins un patient dans l'échantillon dont le test est erroné.
	\end{enumerate}	
	\item Quelle valeur minimale de la taille de l'échantillon faut-il choisir pour que $P(X \geqslant 10)$ soit supérieure à $0,95$ ?
\end{enumerate}

