\textbf{Les parties A et B peuvent être traitées de façon indépendante.}

\medskip

Au cours de la fabrication d’une paire de lunettes, la paire de verres doit subir deux traitements notés T1 et T2.

\medskip

\textbf{Partie A}

\medskip

On prélève au hasard une paire de verres dans la production.

\smallskip

On désigne par A l’évènement : « la paire de verres présente un défaut pour le traitement T1 ».

On désigne par B l’évènement : « la paire de verres présente un défaut pour le traitement T2 ».

On note respectivement $\overline{A}$ et $\overline{B}$ les évènements contraires de A et B.

\smallskip

Une étude a montré que :

\begin{itemize}
	\item la probabilité qu’une paire de verres présente un défaut pour le traitement T1 notée $P(A)$ est égale à 0,1 ;
	\item la probabilité qu’une paire de verres présente un défaut pour le traitement T2 notée $P(B)$ est égale à 0,2 ;
	\item la probabilité qu’une paire de verres ne présente aucun des deux défauts est 0,75.
\end{itemize}

\begin{enumerate}
	\item Recopier et compléter le tableau suivant avec les probabilités correspondantes.
	
	\begin{center}
		\begin{tblr}{width=6cm,colspec={X[c]X[c]X[c]X[c]},vline{1}={2-Z}{solid},vline{2-Z}={solid},hline{1}={2-Z}{solid},hline{2-Z}={solid}}
			& A & $\overline{A}$ & Total \\
			B & & & \\
			$\overline{A}$ & & & \\
			Total & & & \\
		\end{tblr}
	\end{center}
	\item 
	\begin{enumerate}
		\item Déterminer, en justifiant la réponse, la probabilité qu’une paire de verres, prélevée au hasard dans la production, présente un défaut pour au moins un des deux traitements T1 ou T2.
		\item Donner la probabilité qu’une paire de verres, prélevée au hasard dans la production, présente deux défauts, un pour chaque traitement T1 et T2.
		\item Les évènements A et B sont-ils indépendants ? Justifier la réponse.
	\end{enumerate}
	\item Calculer la probabilité qu’une paire de verres, prélevée au hasard dans la production, présente un défaut pour un seul des deux traitements.
	\item Calculer la probabilité qu’une paire de verres, prélevée au hasard dans la production, présente un défaut pour le traitement T2, sachant que cette paire de verres présente un défaut pour le traitement T1.
\end{enumerate}

\textbf{Partie B}

\medskip

On prélève, au hasard, un échantillon de 50 paires de verres dans la production. On suppose que la production est suffisamment importante pour assimiler ce prélèvement à un tirage avec remise. On note X la variable aléatoire qui, à chaque échantillon de ce type, associe le nombre de paires de verres qui présentent le défaut pour le traitement T1.

\begin{enumerate}
	\item Justifier que la variable aléatoire X suit une loi binomiale et préciser les paramètres de cette loi.
	\item Donner l’expression permettant de calculer la probabilité d’avoir, dans un tel échantillon, exactement 10 paires de verres qui présentent ce défaut.
	
	Effectuer ce calcul et arrondir le résultat à $10^{-3}$.
	\item En moyenne, combien de paires de verres ayant ce défaut peut-on trouver dans un échantillon de 50 paires ?
\end{enumerate}