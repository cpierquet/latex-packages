Les douanes s'intéressent aux importations de casques audio portant le logo d'une certaine marque. Les saisies des douanes permettent d'estimer que :
%
\begin{itemize}
	\item 20\,\% des casques audio portant le logo de cette marque sont des contrefaçons ;
	\item 2\,\% des casques non contrefaits présentent un défaut de conception; 
	\item 10\,\% des casques contrefaits présentent un défaut de conception.
\end{itemize}

L'agence des fraudes commande au hasard sur un site internet un casque affichant le logo de la marque. On considère les événements suivants : 

\begin{itemize}
	\item $C$ : « le casque est contrefait » ;
	\item $D$ : « le casque présente un défaut de conception » ;
	\item $\overline{C}$ et $\overline{D}$ désignent respectivement les événements contraires de $C$ et $D$.
\end{itemize}

Dans l'ensemble de l'exercice, les probabilités seront arrondies à $10^{-3}$ si nécessaire.

\medskip

\textbf{Partie 1}

\begin{enumerate}
	\item Calculer $P(C \cap D)$. On pourra s'appuyer sur un arbre pondéré. 
	\item Démontrer que $P(D)=0,036$.
	\item Le casque a un défaut. Quelle est la probabilité qu'il soit contrefait ?
\end{enumerate}

\textbf{Partie 2}

\medskip

On commande $n$ casques portant le logo de cette marque. On assimile cette expérience 
à un tirage aléatoire avec remise. On note $X$ la variable aléatoire qui donne le nombre de casques présentant un défaut de conception dans ce lot.

\begin{enumerate}
	\item Dans cette question, $n=35$.
	\begin{enumerate}
		\item Justifier que $X$ suit une loi binomiale $\mathcal{B} (n;p)$ où $n=35$ et $p=0,036$.
		\item Calculer la probabilité qu'il y ait parmi les casques commandés, exactement un casque présentant un défaut de conception.
		\item Calculer $P(X \leqslant 1)$.
	\end{enumerate}
	\item Dans cette question, $n$ n'est pas fixé.
	
	Quel doit être le nombre minimal de casques à commander pour que la probabilité qu'au moins un casque présente un défaut soit supérieure à $0,992$.
\end{enumerate}

