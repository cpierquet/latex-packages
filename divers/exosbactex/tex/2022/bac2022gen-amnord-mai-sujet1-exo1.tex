Chaque chaque jour où il travaille, Paul doit se rendre à la gare pour rejoindre son lieu de travail en train. Pour cela, il prend son vélo deux fois sur trois et, si il ne prend pas son vélo, il prend sa voiture.

\begin{enumerate}
	\item lorsqu'il prend son vélo pour rejoindre la gare, Paul ne rate le train qu'une fois sur 50 alors que, lorsqu'il prend sa voiture pour rejoindre la gare Paul rate son train une fois sur 10.
	
	On considère une journée au hasard lors de laquelle Paul sera à la gare pour prendre le train qui le conduira au travail.
	
	On note:
	
	\begin{itemize}
		\item $V$ l'évènement \og Paul prend son vélo pour rejoindre la gare \fg{} ; 
		\item $R$ l'évènement \og Paul rate son train \fg.
	\end{itemize}
	
	\begin{enumerate}
		\item Faire un arbre pondéré résumant la situation.
		\item Montrer que la probabilité que Paul rate son train est égale à $\frac{7}{150}$.
		\item Paul a raté son train. Déterminer la valeur exacte de la probabilité qu'il ait pris son vélo pour rejoindre la gare.
	\end{enumerate}
	\item On choisit au hasard un mois pendant lequel Paul s'est rendu $20$ jours à la gare pour rejoindre son lieu de travail selon les modalités décrites en préambule. 
	
	On suppose que, pour chacun de ces 20 jours, le choix entre le vélo et
	la voiture est indépendant des choix des autres jours.
	
	On note $X$ la variable aléatoire donnant le nombre de jours où Paul prend son vélo sur ces $20$ jours.
	\begin{enumerate}
		\item Déterminer la loi suivie par la variable aléatoire $X$. Préciser ses paramètres.
		\item Quelle est la probabilité que Paul prenne son vélo exactement $10$ jours sur ces $20$ jours pour se rendre à la gare ? On arrondira la probabilité cherchée à $10^{-3}$.
		\item Quelle est la probabilité que Paul prenne son vélo au moins $10$ jours sur ces 20 jours pour se rendre à la gare ?
		On arrondira la probabilité cherchée à $10^{-3}$.
		\item En moyenne, combien de jours sur une période choisie au hasard de 20 jours pour se rendre à la gare, Paul prend-il son vélo ? On arrondira la réponse à l'entier.
	\end{enumerate}
	\item Dans le cas où Paul se rend à la gare en voiture, on note $T$ la variable aléatoire donnant le temps de trajet nécessaire pour se rendre à la gare. La durée du trajet est donnée en minutes, arrondie à la minute. La loi de probabilité de $T$ est donnée par le tableau ci-dessous:
	
	\begin{center}
		\begin{tabularx}{\linewidth}{|m{3.3cm}|*{9}{>{\centering \arraybackslash}X|}}\hline
			$k$ (en minutes)&10 &11&12 &13 &14 &15 &16 &17 &18\\ \hline
			$P(T = k)$&0,14&0,13 &0,13&0,12 &0,12&0,11 &0,10 &0,08&0,07\\ \hline
		\end{tabularx}
	\end{center}
	
	Déterminer l'espérance de la variable aléatoire $T$ et interpréter cette valeur dans le contexte de l'exercice.
\end{enumerate}

