\emph{Cet exercice est un questionnaire à choix multiples (QCM) qui comprend six questions. Les six questions sont indépendantes. Pour chacune des questions, \textbf{une seule des quatre réponses est exacte}. Le candidat indiquera sur sa copie le numéro de la question suivi de la lettre correspondant à la réponse exacte. \\
	Aucune justification n'est demandée.\\
	Une réponse fausse, une réponse multiples ou une absence de réponse ne rapporte ni n'enlève aucun point}

\medskip

\textbf{Question 1}

\medskip

Le réel $a$ est définie par $a = \ln (9) + \ln \left(\dfrac{\sqrt{3}}{3} \right)  + \ln \left(\dfrac19 \right)$ est égal à :

\begin{tblr}{width=\linewidth,colspec={X[l]X[l]X[l]X[l]}}
	(a)~~$1 - \dfrac12 \ln (3)$ & (b)~~$\dfrac12 \ln (3)$ & (c)~~$3 \ln (3) + \dfrac12$ & (d)~~$- \dfrac12 \ln (3)$
\end{tblr}

\bigskip

\textbf{Question 2}

\medskip

On note $(E)$ l'équation suivant $\ln (x) + \ln (x - 10) = \ln (3) + \ln (7)$ d'inconnue le réel $x$.

\medskip

\begin{tblr}{width=\linewidth,colspec={X[l]}}
	(a)~~3 est solution de $(E)$. \\
	(b)~~$5 - \sqrt{46}$ est solution de $(E)$.\\
	(c)~~L'équation $(E)$ admet une unique solution réelle.\\
	(d)~~L'équation $(E)$ admet deux solutions réelles.
\end{tblr}

\bigskip

\textbf{Question 3}

\medskip

La fonction $f$ est définie sur l'intervalle $]0;+\infty[$ par l'expression $f(x) = x^2(- 1 + \ln (x))$.

On note $\mathcal{C}_f$ sa courbe représentative dans le plan muni d'un repère.

\medskip

\begin{tblr}{width=\linewidth,colspec={X[l]}}
	(a)~~Pour tout réel $x$ de l'intervalle $]0;+\infty[$, $f'(x) = 2x + \dfrac1x$. \\
	(b)~~La fonction $f$ est croissante sur l'intervalle $]0;+\infty[$.\\
	(c)~~$f'\left(\sqrt{\text{e}} \right)$ est différent de $0$.\\
	(d)~~La droite d'équation $y = \dfrac12 \text{e}$ est tangente à la courbe $\mathcal{C}_f$ au point d'abscisse $\sqrt{\text{e}}$.
\end{tblr}

\pagebreak

\textbf{Question 4}

\medskip

Un sac contient 20 jetons jaunes et 30 jetons bleus. On tire successivement et avec remise 5 jetons du sac.

La probabilité de tirer exactement 2 jetons jaunes, arrondie au millième, est :

\medskip

\begin{tblr}{width=\linewidth,colspec={X[l]X[l]X[l]X[l]}}
	(a)~~$0,683$ & (b)~~$0,346$ & (c)~~$0,230$ & (d)~~$0,165$
\end{tblr}

\bigskip

\textbf{Question 5}

\medskip

Un sac contient 20 jetons jaunes et 30 jetons bleus. On tire successivement et avec remise 5 jetons du sac.

La probabilité de tirer au moins un jeton jaune, arrondie au millième, est :

\medskip

\begin{tblr}{width=\linewidth,colspec={X[l]X[l]X[l]X[l]}}
	(a)~~$0,078$ & (b)~~$0,259$ & (c)~~$0,337$ & (d)~~$0,922$
\end{tblr}

\bigskip

\textbf{Question 6}

\medskip

Un sac contient $20$ jetons jaunes et $30$ jetons bleus.

On réalise l'expérience aléatoire suivante : on tire successivement et avec remise cinq jetons du sac.

On note le nombre de jetons jaunes obtenus après ces cinq tirages. 

Si on répète cette expérience aléatoire un très grand nombre de fois alors, en moyenne, le nombre de jetons jaunes est égal à :

\medskip

\begin{tblr}{width=\linewidth,colspec={X[l]X[l]X[l]X[l]}}
	(a)~~$0,4$ & (b)~~$1,2$ & (c)~~$2$ & (d)~~$2,5$
\end{tblr}