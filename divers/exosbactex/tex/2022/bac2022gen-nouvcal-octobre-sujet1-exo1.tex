On considère la fonction $f$ définie sur l'intervalle $\intervOO{0}{+\infty}$ par \[f(x) = x^2 - 6x + 4\ln (x).\]
On admet que la fonction $f$ est deux fois dérivable sur l'intervalle $\intervOO{0}{+\infty}$.

On note $f$' sa dérivée et $f''$ sa dérivée seconde.

On note $\mathcal{C}_f$ la courbe représentative de la fonction $f$ dans un repère orthogonal.

\begin{enumerate}
	\item 
	\begin{enumerate}
		\item Déterminer $\displaystyle\lim_{x \to 0} f(x)$. 
		
		Interpréter graphiquement ce résultat.
		\item Déterminer $\displaystyle\lim_{x \to + \infty} f(x)$.
	\end{enumerate}
	\item
	\begin{enumerate}
		\item Déterminer $f'(x)$ pour tout réel $x$ appartenant à $\intervOO{0}{+\infty}$.
		\item Étudier le signe de $f'(x)$ sur l'intervalle $\intervOO{0}{+\infty}$.
		
		En déduire le tableau de variations de $f$.
	\end{enumerate}
	\item Montrer que l'équation $f(x) = 0$ admet une unique solution dans l'intervalle $\intervFF{4}{5}$.
	\item On admet que, pour tout $x$ de $\intervOO{0}{+\infty}$, on a : \[f''(x) = \dfrac{2x^2 - 4}{x^2}.\]
	\begin{enumerate}
		\item Étudier la convexité de la fonction $f$ sur $\intervOO{0}{+\infty}$. 
		
		On précisera les valeurs exactes des coordonnées des éventuels points d'inflexion de $\mathcal{C}_f$. 
		\item On note A le point de coordonnées $\left(\sqrt 2;f\left(\sqrt 2~\right)\right)$.
		
		Soit $t$ un réel strictement positif tel que $t \ne \sqrt 2$. Soit $M$ le point de coordonnées $(t;f(t))$.
		
		En utilisant la question 4.a., indiquer, selon la valeur de $t$, les positions relatives du segment $[AM]$ et de la courbe $\mathcal{C}_f$.
	\end{enumerate}
\end{enumerate}

