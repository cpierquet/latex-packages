\textbf{\large Partie A}

\medskip

On considère la fonction $f$ définie pour tout réel $x$ par : \[ f(x)=1+x-\e^{0,5x-2}. \]
%
On admet que la fonction $f$ est dérivable sur $\R$. On note $f'$ sa dérivée.

\begin{enumerate}
	\item 
	\begin{enumerate}
		\item Déterminer la limite de la fonction $f$ en $-\infty$.
		\item Démontrer que, pour tout réel $x$ non nul, $f(x)=1+0,5x\left( 2 - \dfrac{\e^{0,5x}}{0,5x} \times \e^{-2} \right)$.
		
		En déduire la limite de la fonction $f$ en $+\infty$
	\end{enumerate}
	\item 
	\begin{enumerate}
		\item Déterminer $f'(x)$ pour tout réel $x$.
		\item Démontrer que l'ensemble des solutions de l'inéquation $f'(x)<0$ est l'intervalle $]4+2\,\ln(2);+\infty[$.
	\end{enumerate}
	\item Déduire des questions précédentes le tableau de variation de la fonction $f$ sur $\R$.
	
	On fera figurer la valeur exacte de l'image de $4+2\,\ln(2)$ par $f$.
	\item Montrer que l'équation $f(x)=0$ admet une unique solution sur l'intervalle $[-1;0]$.
\end{enumerate}

\textbf{\large Partie B}

\medskip

On considère la suite $\suiten$ définie par $u_0=0$ et, pour tout entier naturel $n$, 

\smallskip

\hfill~$u_{n+1}=f\big(u_n\big)$ où $f$ est la fonction définie à la \textbf{partie A}.\hfill~

\begin{enumerate}
	\item 
	\begin{enumerate}
		\item Démontrer par récurrence que, pour tout entier naturel $n$, on a : \[ u_n \leqslant u_{n+1} \leqslant 4.\]
		\item En déduire que la suite $\suiten$ converge. On notera $\ell$ la limite. 
	\end{enumerate}
	\item 
	\begin{enumerate}
		\item On rappelle que $\ell$ vérifie la relation $\ell=f\big(\ell\big)$.
		
		Démontrer que $\ell=4$.
		\item On considère la fonction \texttt{valeur} écrite ci-contre dans le langage \textsf{Python} :
		
\begin{CodePythonLstAlt}*[Largeur=8cm]{center}
def valeur(a) :
	u = 0
	n = 0
	while u <= a :
		u = 1+u-exp(0.5*u-2)
		n = n+1
	return n
\end{CodePythonLstAlt}
		
		L'instruction \texttt{valeur(3.99)} renvoie la valeur \texttt{12}.
		
		Interpréter ce résultat dans le contexte de l'exercice.
	\end{enumerate}
\end{enumerate}

