Dans une station de ski, il existe deux types de forfait selon l'âge du skieur :

\begin{itemize}
	\item un forfait JUNIOR pour les personnes de moins de vingt-cinq ans ;
	\item un forfait SÉNIOR pour les autres.
\end{itemize}

Par ailleurs, un usager peut choisir, en plus du forfait correspondant à son âge, l'option coupe-file qui permet d'écourter le temps d'attente aux remontées mécaniques.

On admet que :

\begin{itemize}
	\item 20\,\% des skieurs ont un forfait JUNIOR ;
	\item 80\,\% des skieurs ont un forfait SÉNIOR ;
	\item parmi les skieurs ayant un forfait JUNIOR, 6\,\% choisissent l'option coupe-file;
	\item parmi les skieurs ayant un forfait SÉNIOR, $12,5$\,\% choisissent l'option coupe-file. 
\end{itemize}

On interroge un skieur au hasard et on considère les événements : 

\begin{itemize}
	\item J : « le skieur a un forfait JUNIOR » ;
	\item C : « le skieur choisit l'option coupe-file ».
\end{itemize}

\hfill~\textit{Les deux parties peuvent être traitées de manière indépendante.}\hfill~

\medskip

\textbf{\large Partie A}

\begin{enumerate}
	\item Traduire la situation par un arbre pondéré.
	\item Calculer la probabilité $P(J \cap C)$.
	\item Démontrer que la probabilité que le skieur choisisse l'option coupe-file est égale à $0,112$.
	\item Le skieur a choisi l'option coupe-file. Quelle est la probabilité qu'il s'agisse d'un skieur ayant un forfait SÉNIOR ? Arrondir le résultat à $10^{-3}$.
	\item Est-il vrai que les personnes de moins de vingt-cinq ans représentent moins de 15\,\% des skieurs ayant choisi l'option coupe-file ? Expliquer.
\end{enumerate}

\textbf{\large Partie B}

\medskip

On rappelle que la probabilité qu'un skieur choisisse l'option coupe-file est égale à $0,112$.

\smallskip

On considère un échantillon de 30 skieurs choisis au hasard.

\smallskip

Soit $X$ la variable aléatoire qui compte le nombre des skieurs de l'échantillon ayant choisi l'option coupe-file.

\begin{enumerate}
	\item On admet que la variable aléatoire X suit une loi binomiale.
	
	Donner les paramètres de cette loi.
	\item Calculer la probabilité qu'au moins un des 30 skieurs ait choisi l'option coupe-file.
	
	Arrondir le résultat à $10^{-3}$.
	\item Calculer la probabilité qu'au plus un des 30 skieurs ait choisi l'option coupe-file.
	
	Arrondir le résultat à $10^{-3}$.
	\item Calculer l'espérance mathématique de la variable aléatoire X.
\end{enumerate}

