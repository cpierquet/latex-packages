\textbf{Les parties B et C sont indépendantes}

\medskip

On considère la fonction $f$  définie sur $]0;+\infty[$ par \[f(x) = x - x \ln (x),\] où $\ln$ désigne la fonction logarithme népérien. 

\medskip

\textbf{Partie A}

\begin{enumerate}
	\item Déterminer la limite de $f(x)$ quand $x$ tend vers $0$.
	\item Déterminer la limite de $f(x)$ quand $x$ tend vers $+\infty$.
	\item On admet que la fonction $f$ est dérivable sur $]0;+\infty[$ et on note $f'$ sa fonction dérivée.
	\begin{enumerate}
		\item Démontrer que, pour tout réel $x > 0$, on a : $f'(x) = - \ln (x)$.
		\item En déduire les variations de la fonction $f$ sur $]0;+\infty[$ et dresser son tableau de variations.
	\end{enumerate}
	\item Résoudre l'équation $f(x) = x$ sur $]0;+\infty[$.
\end{enumerate}

\medskip

\textbf{Partie B}

\medskip

Dans cette partie, on pourra utiliser avec profit certains résultats de la \textbf{partie A}. 

On considère la suite $\left(u_n\right)$ définie par : \[\begin{dcases} u_0 = 0,5 \\ u_{n+1} =  u_n - u_n \ln \big(u_n\big) \:\text{ pour tout entier naturel } n \end{dcases}.\]
%
Ainsi, pour tout entier naturel $n$, on a : $u_{n+1} = f\left(u_n\right)$.

\begin{enumerate}
	\item On rappelle que la fonction $f$ est croissante sur l'intervalle $[0,5;1]$.
	
	Démontrer par récurrence que, pour tout entier naturel $n$, on a : $0,5 \leqslant  u_n \leqslant u_{n+1} \leqslant 1$.
	\item 
	\begin{enumerate}
		\item Montrer que la suite $\left(u_n\right)$ est convergente.
		\item On note $\ell$ la limite de la suite $\left(u_n\right)$. Déterminer la valeur de $\ell$.
	\end{enumerate}
\end{enumerate}

\medskip

\textbf{Partie C}

\medskip

Pour un nombre réel $k$ quelconque, on considère la fonction $f_k$ définie sur $]0~;~+\infty[$ par : \[f_k(x) = kx - x \ln (x).\]
%
\begin{enumerate}
	\item Pour tout nombre réel $k$, montrer que $f_k$ admet un maximum $y_k$ atteint en $x_k = \text{e}^k- 1$.
	\item  Vérifier que, pour tout nombre réel $k$, on a : $x_k = y_k$.
\end{enumerate}

