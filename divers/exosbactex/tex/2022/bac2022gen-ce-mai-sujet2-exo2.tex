Soit $f$ la fonction définie sur l'intervalle $]0;+\infty[$ par \[f(x) = x\ln (x) + 1.\]
%
On note $\mathcal{C}_f$ sa courbe représentative dans un repère du plan.

\begin{enumerate}
	\item Déterminer la limite de la fonction $f$ en $0$ ainsi que sa limite en $+\infty$.
	\item 
	\begin{enumerate}
		\item On admet que $f$ est dérivable sur $]0;+\infty[$ et on notera $f'$ sa fonction dérivée.
		
		Montrer que pour tout réel $x$ strictement positif : \[f'(x) = 1 + \ln (x).\]
		\item En déduire le tableau de variation de la fonction $f$ sur $]0;+\infty[$. On y fera figurer la valeur exacte de l'extremum de $f$ sur $]0;+\infty[$ et les limites.
		\item Justifier que pour tout $x \in  ]0;1[$, $f(x) \in  ]0;1[$.
	\end{enumerate}	
	\item 
	\begin{enumerate}
		\item Déterminer une équation de la tangente $(T)$ à la courbe $\mathcal{C}_f$ au point d'abscisse 1.
		\item Étudier la convexité de la fonction $f$ sur $]0;,+\infty[$.
		\item En déduire que pour tout réel $x$ strictement positif : \[f(x) \geqslant x.\]
	\end{enumerate}
	\item On définit la suite $\left(u_n\right)$ par son premier terme $u_0$ élément de l'intervalle $]0;1[$ et pour tout entier naturel $n$ : \[u_{n+1} = f\left(u_n\right).\]
	\begin{enumerate}
		\item Démontrer par récurrence que pour tout entier naturel $n$, on a : $0 < u_n < 1$.
		\item Déduire de la question 3.(c) la croissance de la suite $\left(u_n\right)$.
		\item En déduire que la suite $\left(u_n\right)$ est convergente.
	\end{enumerate}
\end{enumerate}

