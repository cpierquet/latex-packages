L'espace est rapporté un repère orthonormal où l'on considère :
%
\begin{itemize}
	\item les points $A(2;-1;0)$, $B(1;0;-3)$, $C(6;6;1)$ et $E(1;2;4)$ ;
	\item le plan $\mathcal{P}$ d'équation cartésienne $2x - y - z + 4 = 0$.
\end{itemize}

\begin{enumerate}
	\item
	\begin{enumerate}
		\item Démontrer que le triangle $ABC$ est rectangle en $A$.
		\item Calculer le produit scalaire $\vect{BA} \cdot \vect{BC}$ puis les longueurs $BA$ et $BC$.
		\item En déduire la mesure en degrés de l'angle $\widehat{ABC}$ arrondie au degré.
	\end{enumerate}
	\item
	\begin{enumerate}
		\item Démontrer que le plan $\mathcal{P}$ est parallèle au plan $(ABC)$.
		\item En déduire une équation cartésienne du plan $(ABC)$.
		\item Déterminer une représentation paramétrique de la droite $\mathcal{D}$ orthogonale au plan $(ABC)$ et passant par le point $E$.
		\item Démontrer que le projeté orthogonal $H$ du point $E$ sur le plan $(ABC)$ à à pour coordonnées $\left(4~;~\dfrac{1}{2}~;~\dfrac{5}{2}\right)$.
	\end{enumerate}
	\item On rappelle que le volume d'une pyramide est donné par $\mathcal{V} = \dfrac13 \mathcal{B}h$ où $\mathcal{B}$ désigne l'aire d'une base et $h$ la hauteur de la pyramide associée à cette base.
	
	Calculer l'aire du triangle $ABC$ puis démontrer que le volume de la pyramide à $ABCE$ est égal à $16,5$ unités de volume.
\end{enumerate}