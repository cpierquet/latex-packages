\textbf{Partie A}

\medskip 

Soit $p$ la fonction définie sur l'intervalle $[-3;4]$ par : \[p(x) = x^3 - 3x^2 + 5x + 1\]%
%
\begin{enumerate}
	\item Déterminer les variations de la fonction $p$ sur l'intervalle $[-3;4]$.
	\item Justifier que l'équation $p(x) = 0$ admet dans l'intervalle $[-3;4]$ une unique solution qui sera notée $\alpha$.
	\item Déterminer une valeur approchée du réel $\alpha$ au dixième près.
	\item Donner le tableau de signes de la fonction $p$ sur l'intervalle $[-3;4]$.
\end{enumerate}

\textbf{Partie B}

\medskip 

Soit $f$ la fonction définie sur l'intervalle $[-3;4]$ par : \[f(x) = \dfrac{\text{e}^x}{1 + x^2}.\]%
%
On note $\mathcal{C}_f$ sa courbe représentative dans un repère orthogonal.

\begin{enumerate}
	\item 
	\begin{enumerate}
		\item Déterminer la dérivée de la fonction $f$ sur l'intervalle $[-3;4]$.
		\item Justifier que la courbe $\mathcal{C}_f$ admet une tangente horizontale au point d'abscisse 1.
	\end{enumerate}
	\item  Les concepteurs d'un toboggan utilisent la courbe $\mathcal{C}_f$ comme profil d'un toboggan. Ils estiment que le toboggan assure de bonnes sensations si le profil possède au moins deux points d'inflexion.
	
	\medskip
	
	\hfill~\begin{tikzpicture}[x=0.75cm,y=0.75cm,xmin=-4,xmax=4,xgrille=1,xgrilles=0.5,ymin=-0,ymax=4,ygrille=1,ygrilles=0.5,line join=bevel]
		\GrilleTikz \AxesTikz \AxexTikz[Police=\small]{-4,-3,...,3} \AxeyTikz[Police=\small]{1,2,3}
		\draw[very thick,blue,samples=250,domain=\xmin:\xmax] plot (\x,{exp(\x)/(\x*\x+1)}) ;
		\draw[blue] (2.5,2.5) node[font=\large] {$\mathcal{C}_f$} ;
		\draw (0,-1.25) node {Représentation de la courbe $\mathcal{C}_f$} ;
		%\clip (\xmin,\ymin) rectangle (\xmax,\ymax) ;
	\end{tikzpicture}
	\hfill~
	\begin{tikzpicture}[x=0.75cm,y=0.75cm,xmin=-4,xmax=4,xgrille=1,xgrilles=0.5,ymin=-0,ymax=4,ygrille=1,ygrilles=0.5,line join=bevel]
		\fill [draw=black,very thick,fill=lightgray] (-4,0) -- plot[samples=250,domain=-4:4] (\x,{exp(\x)/(\x*\x+1)}) -- (4,0) -- cycle;
		\draw[very thick] (-4,0) --++ (-0.66,0.788) ;
		\draw[very thick] (4,3.212) --++ (-0.66,0.788) ;
		\begin{scope}[shift={(-0.66,0.788)}]
			\draw[very thick,samples=250,domain=\xmin:\xmax] plot (\x,{exp(\x)/(\x*\x+1)}) ;
		\end{scope}
		\draw (0,-1.25) node {Vue de profil du toboggan} ;
	\end{tikzpicture}
	\hfill~
	\begin{enumerate}
		\item D'après le graphique ci-dessus, le toboggan semble-t-il assurer de bonnes sensations ? Argumenter.
		\item On admet que la fonction $f''$, dérivée seconde de la fonction $f$, a pour expression pour tout réel $x$ de l'intervalle $[-3;4]$ : \[f''(x) = \dfrac{p(x)(x - 1)\text{e}^x}{\left(1 + x^2\right)^3}\]%
		où $p$ est la fonction définie dans la \textbf{partie A}.
		
		En utilisant l'expression précédente de $f''$, répondre à la question : \og le toboggan assure-t-il de bonnes sensations ? \fg. Justifier.
	\end{enumerate}
\end{enumerate}

