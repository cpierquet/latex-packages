\textbf{Partie I : lectures graphiques}

\medskip

$f$ désigne une fonction définie et dérivable sur $\R$.

On donne ci-dessous la courbe représentative de la fonction dérivée $f'$.

\begin{center}
	\begin{tikzpicture}[x=0.9cm,y=3cm,xmin=-7,xmax=7,ymin=-0.8,ymax=1.2,xgrille=1,ygrille=0.2]
		\GrilleTikz[Affs=false]
		\AxesTikz[ElargirOx=0/0,ElargirOy=0/0]
		\AxexTikz[Police=\small]{-7,-6,...,6} \AxeyTikz[Police=\small]{1}
		\draw[line width=1.25pt,red,domain=\xmin:\xmax,samples=2000] plot (\x,{(2*\x+1)/(\x*\x+\x+2.5)});
		\draw (0,1) node[above,red] {Courbe de la fonction dérivée $f'$} ;
	\end{tikzpicture}
\end{center}

\emph{Avec la précision permise par le graphique, répondre aux questions suivantes}

\begin{enumerate}
	\item Déterminer le coefficient directeur de la tangente à la courbe de la fonction $f$ en $O$.
	\item 
	\begin{enumerate}
		\item Donner les variations de la fonction dérivée $f'$.
		\item En déduire un intervalle sur lequel $f$ est convexe. 
	\end{enumerate}
\end{enumerate}

\textbf{Partie II : étude de fonction}

\medskip

La fonction $f$ est définie sur $\R$ par \[f(x) = \ln \left(x^2 + x + \dfrac{5}{2}\right).\]

\begin{enumerate}
	\item Calculer les limites de la fonction $f$ en $+\infty$ et en $-\infty$.
	\item Déterminer une expression $f'(x)$ de la fonction dérivée de $f$ pour tout $x \in \R$.
	\item En déduire le tableau des variations de $f$. On veillera à placer les limites dans ce tableau.
	\item 
	\begin{enumerate}
		\item Justifier que l'équation $f(x) = 2$ a une unique solution $\alpha$ dans l'intervalle $\left[-\dfrac{1}{2};+ \infty\right[$.
		\item Donner une valeur approchée de $\alpha$ à $10^{-1}$ près.
	\end{enumerate}
	\item La fonction $f'$ est dérivable sur $\R$. On admet que, pour tout $x \in  \R$, $f''(x) = \dfrac{-2x^2 - 2x + 4}{\left(x^2 + x + \dfrac{5}{2}\right)^2}$.
	
	Déterminer le nombre de points d'inflexion de la courbe représentative de $f$.
\end{enumerate}

