Une société de jeu en ligne propose une nouvelle application pour smartphone nommée \og Tickets cœurs ! \fg.

Chaque participant génère sur son smartphone un ticket comportant une grille de taille $3 \times 3$ sur laquelle sont placés trois cœurs répartis au hasard, comme par exemple ci-dessous.

\begin{center}
	\begin{tikzpicture}[xmin=0,xmax=3,ymin=0,ymax=3]
		\GrilleTikz[Affs=false][very thick][]
%		\tgrillep[line width=1pt]
		\foreach \Pos in {(1.5,2.5),(0.5,1.5),(2.5,0.5)} \draw \Pos node[font=\LARGE] {$\heartsuit$} ;
	\end{tikzpicture}
\end{center}

Le ticket est gagnant si les trois cœurs sont positionnés côte à côte sur une même ligne, sur une même colonne ou sur une même diagonale.

\begin{enumerate}
	\item Justifier qu'il y a exactement $84$ façons différentes de positionner les trois cœurs sur une grille.
	\item Montrer que la probabilité qu'un ticket soit gagnant est égale à $\dfrac{2}{21}$.
	\item Lorsqu'un joueur génère un ticket, la société prélève 1\,€ sur son compte en banque. Si le ticket est gagnant, la société verse alors au joueur $5$\,€. Le jeu est-il favorable au joueur?
	\item Un joueur décide de générer $20$ tickets sur cette application. On suppose que les générations des tickets sont indépendantes entre elles.
	\begin{enumerate}
		\item Donner la loi de probabilité de la variable aléatoire $X$ qui compte le nombre de tickets gagnants parmi les $20$ tickets générés.
		\item Calculer la probabilité, arrondie à $10^{-3}$, de l'évènement ($X$  =  5).
		\item Calculer la probabilité, arrondie à $10^{-3}$, de l'évènement $(X \geqslant 1)$ et interpréter le résultat dans le contexte de l'exercice.
	\end{enumerate}
\end{enumerate}

