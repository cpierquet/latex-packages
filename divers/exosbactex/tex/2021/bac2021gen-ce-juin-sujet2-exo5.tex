\textbf{Partie A : Détermination d’une fonction \boldmath$f$\unboldmath{} et résolution d’une équation différentielle}

\medskip

On considère la fonction $f$ définie sur $\R$ par : \[ f(x)=\e^x+ax+b\e^{-x} \]%
où $a$ et $b$ sont des nombres réels que l’on propose de déterminer dans cette partie.

\smallskip

Dans le plan muni d’un repère d’origine $O$, on a représenté ci-dessous la courbe $\mathcal{C}$, représentant la fonction $f$, et la tangente $(T)$ à la courbe $\mathcal{C}$ au point d’abscisse $0$.

\begin{center}
	\begin{tikzpicture}[x=0.8cm,y=0.8cm,xmin=-2.5,xmax=3.5,xgrilles=0.2,ymin=-0.5,ymax=6.5,ygrilles=0.2]
		\GrilleTikz
		\AxesTikz[ElargirOx=0/0,ElargirOy=0/0]
		\AxexTikz[Police=\small]{-2,-1,1,2,3} \AxeyTikz[Police=\small]{1,2,...,6} ;
		\draw (0,0) node[below left,font=\small] {0} ;
		\clip (\xmin,\ymin)rectangle(\xmax,\ymax);
		\draw[line width=1.25pt,red,samples=250,domain=\xmin:\xmax] plot (\x,{exp(\x)-\x+2*exp(-\x)}) ;
		\draw[line width=1.25pt,blue,samples=250,domain=\xmin:\xmax] plot (\x,{-2*\x+3}) ;
	\end{tikzpicture}
\end{center}

\begin{enumerate}
	\item Par lecture graphique, donner les valeurs de $f(0)$ et de $f'(0)$.
	\item En utilisant l’expression de la fonction $f$, exprimer $f(0)$ en fonction de $b$ et en déduire la valeur de $b$.
	\item On admet que la fonction $f$ est dérivable sur $\R$ et on note $f'$ sa fonction dérivée.
	\begin{enumerate}
		\item Donner, pour tout réel $x$, l’expression de $f'(x)$.
		\item Exprimer $f'(0)$ en fonction de $a$.
		\item En utilisant les questions précédentes, déterminer $a$, puis en déduire l’expression de $f(x)$.
	\end{enumerate}
	\item On considère l’équation différentielle : \[ (E) \quad : \quad y'+y=2\e^{x}-x-1. \]
	\begin{enumerate}
		\item Vérifier que la fonction $g$ définie sur $\R$ par : \[ g(x)=\e^x-x+2\e^{-x} \]%
		est solution de l’équation $(E)$.
		\item Résoudre l’équation différentielle $y'+y=0$.
		\item En déduire toutes les solutions de l’équation $(E)$.
	\end{enumerate}
\end{enumerate}

\pagebreak

\textbf{Partie B : Étude de la fonction} $\bm{g}$ \textbf{sur} $\bm{{\intervFO{1}{+\infty}}}$

\begin{enumerate}
	\item Vérifier que pour tout réel $x$, on a : \[ \e^{2x}-\e^x-2=\big(\e^x-2\big)\big(\e^x+1\big). \]
	\item En déduire une expression factorisée de $g'(x)$, pour tout réel $x$.
	\item On admettra que, pour tout $x \in \intervFO{1}{+\infty}$, $\e^x-2 > 0$.
	
	Étudier le sens de variation de la fonction $g$ sur $\intervFO{1}{+\infty}$.
\end{enumerate}