\textbf{Cet exercice est un questionnaire à choix multiples (QCM)}

\medskip

\emph{Pour chaque question, trois affirmations sont proposées, une seule de ces affirmations est exacte.\\
	Le candidat recopiera sur sa copie le numéro de chaque question et la lettre de la réponse choisie pour celle-ci.\\
	AUCUNE JUSTIFICATION n'est demandée. Une réponse fausse ou l'absence de réponse n'enlève aucun point.}

\medskip

\begin{enumerate}
	\item On considère la fonction $f$ définie sur $\R$ par \[f(x) = \left(x^2 - 2x - 1\right)\e^x.\]
	%
	\textbf{A.} La fonction dérivée de $f$ est la fonction définie par $f'(x) = (2x - 2)\e^x$.
	
	\textbf{B.} La fonction $f$ est décroissante sur l'intervalle $]-\infty;2]$.
	
	\textbf{C.} $\displaystyle\lim_{x \to -\infty} f(x) = 0$.
	\item On considère la fonction $f$ définie sur $\R$ par $f(x) = \dfrac{3}{5 + \e^x}$.
	
	Sa courbe représentative dans un repère admet :
	
	\textbf{A.} une seule asymptote horizontale;
	
	\textbf{B.} une asymptote horizontale et une asymptote verticale; 
	
	\textbf{C.} deux asymptotes horizontales.
	\item On donne ci-dessous la courbe $\mathcal{C}_{f''}$ représentant la fonction dérivée seconde $f''$  d'une fonction $f$ définie et deux fois dérivable sur l'intervalle $[-3,5;6]$.
	
	\begin{center}
		\begin{tikzpicture}[x=1cm,y=1cm,xmin=-4,xmax=7,ymin=-3,ymax=5,xgrille=1,ygrille=1]
			\GrilleTikz[Affs=false] \AxesTikz[ElargirOx=0/0,ElargirOy=0/0]
			\AxexTikz[Police=\small]{-4,-3,...,6} \AxeyTikz[Police=\small]{-3,-2,...,4}
			\draw[red,line width=1.25pt,domain=-3.5:6,samples=500] plot (\x,{0.1*(\x+3)*(\x-2)*(\x-5)}) ;
			\draw(1.5,4) node[above] {Courbe de la fonction dérivée seconde $f''$} ;
		\end{tikzpicture}
	\end{center}
	
	\textbf{A.} La fonction $f$ est convexe sur l'intervalle $[-3;3]$.
	
	\textbf{B.} La fonction $f$ admet trois points d'inflexion.
	
	\textbf{C.} La fonction dérivée $f'$ de $f$ est décroissante sur l'intervalle $[0;2]$.
	\item  On considère la suite $\left(u_n\right)$ définie pour tout entier naturel $n$ par $u_n = n^2 - 17n + 20$. 
	
	\textbf{A.} La suite $\left(u_n\right)$ est minorée.
	
	\textbf{B.} La suite $\left(u_n\right)$ est décroissante.
	
	\textbf{C.} L'un des termes de la suite $\left(u_n\right)$ est égal à \num{2021}.
	\item On considère la suite $\left(u_n\right)$ définie par $u_0 = 2$ et, pour tout entier naturel $n$, $u_{n+1} = 0,75u_n +5$.
	
	On considère la fonction \og \texttt{seuil} \fg{} suivante écrite en \textsf{Python} :
	
\begin{CodePythonLstAlt}*[Largeur=8cm]{center}
def seuil() :
	u = 2
	n = 0
	while u < 45 :
		u = 0,75*u + 5
		n = n+1
	return n
\end{CodePythonLstAlt}
	
	Cette fonction renvoie :
	
	\textbf{A.} la plus petite valeur de $n$ telle que $u_n \geqslant  45$ ;
	
	\textbf{B.} la plus petite valeur de $n$ telle que $u_n  < 45$ ;
	
	\textbf{C.} la plus grande valeur de $n$ telle que $u_n \geqslant 45$.
\end{enumerate}

