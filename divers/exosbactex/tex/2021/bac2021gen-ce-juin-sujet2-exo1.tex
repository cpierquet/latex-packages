\textit{Cet exercice est un questionnaire à choix multiples. Pour chacune des questions suivantes, une seule des quatre réponses proposées est exacte. Une réponse exacte rapporte un point. Une réponse fausse, une réponse multiple ou l'absence de réponse à une questionne rapporte ni n'enlève de point. Pour répondre, indiquer sur la copie le numéro de la question et la lettre de la réponse choisie. Aucune justification n'est demandée}

\bigskip

\textbf{\underline{Question 1 :}}

On considère la fonction $g$ définie sur $\intervOO{0}{+\infty}$ par $g(x)=x^2+2x-\dfrac{2}{x}$.

Une équation de la tangente à la courbe représentative de $g$ au point d’abscisse 1 est :

\medskip

\begin{tblr}{hlines,vlines,width=\linewidth,colspec={*{2}{X[l]}}}
	(a)~~$y=7(x-1)$ & (c)~~$y=7x+7$ \\
	(b)~~$y=x-1$ & (d)~~$y=x+1$
\end{tblr}

\bigskip

\textbf{\underline{Question 2 :}}

\medskip

On considère la suite $\suiten[v]$ définie sur $\N$ par $v_n = \dfrac{3n}{n+2}$. On cherche à déterminer la limite de $v_n$ lorsque $n$ tend vers $+\infty$.

\medskip

\newcommand\espv{\vphantom{\dfrac32}}
\begin{tblr}{hlines,vlines,width=\linewidth,colspec={*{2}{X[l,m]}}}
	(a)~~$\displaystyle\lim_{n \to +\infty} v_n = 1\espv$ & (c)~~$\displaystyle\lim_{n \to +\infty} v_n = \dfrac32$ \\
	(b)~~$\displaystyle\lim_{n \to +\infty} v_n = 3\espv$ & (d)~~On ne peut pas la déterminer$\espv$
\end{tblr}

\bigskip

\textbf{\underline{Question 3 :}}

\medskip

Dans une urne il y a 6 boules noires et 4 boules rouges. On effectue successivement 10 tirages aléatoires avec remise. Quelle est la probabilité (à $10^{-4}$ près) d’avoir 4 boules noires et 6 boules rouges ?

\medskip

\begin{tblr}{hlines,vlines,width=\linewidth,colspec={*{2}{X[l,m]}}}
	(a)~~$0,1662$ & (c)~~$0,115$ \\
	(b)~~$0,4$ & (d)~~$0,8886$
\end{tblr}

\bigskip

\textbf{\underline{Question  4 :}}

\medskip

On considère la fonction $f$ définie sur $\R$ par $f(x)=3\e^x-x$.

\medskip

\begin{tblr}{hlines,vlines,width=\linewidth,colspec={*{2}{X[l,m]}}}
	(a)~~$\displaystyle\lim_{x \to +\infty} f(x)=3$ & (c)~~$\displaystyle\lim_{x \to +\infty} f(x)=-\infty$ \\
	(b)~~$\displaystyle\lim_{x \to +\infty} f(x)=+\infty$ & (d)~~On ne peut pas déterminer la limite de $f$ en $+\infty$
\end{tblr}

\bigskip

\textbf{\underline{Question  5 :}}

\medskip

Un code inconnu est constitué de 8 signes. Chaque signe peut être une lettre ou un chiffre. Il y a donc 36 signes utilisables pour chacune des positions.

Un logiciel de cassage de code teste environ cent millions de codes par seconde.

En combien de temps au maximum le logiciel peut-il découvrir le code ?

\medskip

\begin{tblr}{hlines,vlines,width=\linewidth,colspec={*{2}{X[l,m]}}}
	(a)~~environ $0,3$ seconde & (c)~~environ 3 heures \\
	(b)~~environ 8 heures & (d)~~environ 470 heures
\end{tblr}

