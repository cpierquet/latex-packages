\begin{center}
	\textbf{Partie I}
\end{center}

On donne ci-dessous, dans le plan rapporté à un repère orthonormé, la courbe représentant la \textbf{fonction dérivée} $f'$ d’une fonction $f$ dérivable sur $\R$.

À l’aide de cette courbe, conjecturer, en justifiant les réponses :

\begin{enumerate}
	\item Le sens de variation de la fonction $f$ sur $\R$.
	\item La convexité de la fonction $f$ sur $\R$.
\end{enumerate}

\begin{center}
	\begin{tikzpicture}[x=1.25cm,y=1.25cm,xmin=-2,xmax=4,ymin=-2,ymax=4]
		\AxesTikz[ElargirOx=0/0,ElargirOy=0/0]
		\foreach \x in {-1,0,1,2,3} \draw[line width=1.25pt] (\x,4pt) -- (\x,-4pt) node[below left] {$\num{\x}$} ;
		\AxeyTikz{-1,1,2,3}
		\draw[line width=1.25pt,->,>=latex] (0,0)--(0,1);
		\draw[line width=1.25pt,->,>=latex] (0,0)--(1,0);
		\clip (\xmin,\ymin) rectangle (\xmax,\ymax) ;
		\draw[red,very thick,samples=250,domain=\xmin:\xmax] plot (\x,{(-\x-1)*exp(-\x)}) ;
		\draw[line width=1.25pt,fill=white] (0.5,1.75) rectangle (3.5,2.75) node[midway,font=\scriptsize] {\parbox{3.5cm}{Courbe représentant la \textbf{dérivée} $f'$ de la fonction $f$.}} ;
	\end{tikzpicture}
\end{center}

\begin{center}
	\textbf{Partie II}
\end{center}

On admet que la fonction $f$ mentionnée dans la \textbf{Partie I} est définie sur $\R$ par : \[ f(x)=(x+2)\e^{-x}. \]
On note $\mathcal{C}$ la courbe représentative de $f$ dans un repère orthonormé $\Rij$.

On admet que la fonction $f$ est deux fois dérivable sur $\R$, et on note $f'$ et $f''$ les fonctions dérivées première et seconde de $f$ respectivement.

\begin{enumerate}
	\item Montrer que, pour tout nombre réel $x$, \[ f(x)=\dfrac{x}{\e^x}+2\e^{-x}. \]
	En déduire la limite de $f$ en $+\infty$.
	
	Justifier que la courbe $\mathcal{C}$ admet une asymptote que l’on précisera.
	
	On admet que $\displaystyle\lim_{x \to -\infty} f(x)=-\infty$.
	\item 
	\begin{enumerate}
		\item Montrer que, pour tout nombre réel $x$, $f'(x)=(-x-1)\e^{-x}$.
		\item Étudier les variations sur $\R$ de la fonction $f$ et dresser son tableau de variations.
		\item Montrer que l’équation$f(x)=2$ admet une unique solution $\alpha$ sur l’intervalle $\intervFF{-2}{1}$ dont on donnera une valeur approchée à $10^{-1}$ près.
	\end{enumerate}
	\item Déterminer, pour tout nombre réel $x$, l’expression de $f''(x)$ et étudier la convexité de la fonction $f$. Que représente pour la courbe $\mathcal{C}$ son point $A$ d’abscisse $0$ ?
\end{enumerate}