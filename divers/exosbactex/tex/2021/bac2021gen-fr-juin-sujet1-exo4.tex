Dans un repère orthonormé $\Rijk$, on considère :

\begin{itemize}
	\item le point $A$ de coordonnées $(1;3;2)$ ;
	\item le vecteur $\vect{u}$ de coordonnées $\begin{pmatrix}1\\1\\0\end{pmatrix}$ ;
	\item la droite $d$ passant par l'origine $O$ du repère et admettant pour vecteur directeur $\vect{u}$.
\end{itemize}

Le but de cet exercice est de déterminer le point de $d$ le plus proche du point $A$ et d'étudier quelques propriétés de ce point.

\smallskip

On pourra s'appuyer sur la figure ci-dessous pour raisonner au fur et à mesure des questions.

\begin{center}
	\begin{tikzpicture}[x=1.5cm,y=1.2cm,>=stealth',font=\large]
		\clip (-2,-3.15) rectangle (4,2.5) ;
		\draw[thick,->] (-2,0)--(4,0) ; \draw[thick,->] (0,-3.5)--(0,3) ;
		\draw[very thick,->] (0,0)--(0,1) node[left] {$\vect{k}$}; ;
		\draw[thick] (0,0)--(-125:4) ;
		\draw[very thick,->] (0,0)--(-125:1.1) node[left] {$\vect{\imath}$};
		\draw[thick,fill=white] (0,0)--(2.25,1.5)--(2.4,-0.8)--(0.75,-1.5)--cycle ;
		\draw[thick,red,samples=2,domain=-2:4] plot (\x,{-2*\x}) ;
		\draw[red] (1.65,-3) node[above] {$d$} ;
		\draw[very thick,->,densely dashed] (0,0)--(1,0) node[above] {$\vect{\jmath}$};
		\draw[thick,densely dashed] (1,0)--(2.3,0) ;
		\draw[thick] (2.25,1.5)--(0.75,-1.5);
		\draw[thick,densely dashed] (0,0)--(2.4,-0.8);
		\draw[very thick,->] (0,0)--(0.35,-0.7) node[right] {$\vect{u}$};
		\foreach \Point/\Nom/\Pos in {(0,0)/O/below left,(2.25,1.5)/A/above,(2.4,-0.8)/{A'}/below,(0.75,-1.5)/{M_0}/left}%
		\filldraw \Point circle[radius=2pt] node[\Pos] {$\Nom$} ;
		\foreach \x in {-1,2,3} \draw[thick] (\x,3pt)--(\x,-3pt) ;
		\foreach \y in {-3,-2,-1,2} \draw[thick] (3pt,\y)--(-3pt,\y) ;
	\end{tikzpicture}
\end{center}

\begin{enumerate}
	\item Déterminer une représentation paramétrique de la droite $d$.
	\item Soit $t$ un nombre réel quelconque, et $M$ un point de la droite $d$, le point $M$ ayant pour coordonnées $(t;t;0)$.
	\begin{enumerate}
		\item On note $AM$ la distance entre les points $A$ et $M$. Démontrer que : \[ AM^2 = 2t^2-8t+14. \]
		\item Démontrer que le point $M_0$ de coordonnées $(2;2;0)$ est le point de la droite $d$ pour lequel la distance $AM$ est minimale. On admettra que la distance $AM$ est minimale lorsque son carré $AM^2$ est minimal.
	\end{enumerate}
	\item Démontrer que les droites $\big(AM_0\big)$ et $d$ sont orthogonales.
	\item On appelle $A'$ le projeté orthogonal du point $A$ sur le plan d'équation cartésienne $z = 0$. Le point $A'$ admet donc pour coordonnées $(1;3;0)$.
	
	Démontrer que le point $M_0$ est le point du plan $\big(AA'M_0\big)$ le plus proche du point $O$, origine du repère.
	\item Calculer le volume de la pyramide $OM_0A'A$.
	
	On rappelle que le volume d'une pyramide est donné par : $\mathcal{V} = \dfrac13 \mathcal{B} h$, où $\mathcal{B}$ est l'aire d'une base et $h$ est la hauteur de la pyramide correspondant à cette base.
\end{enumerate}

