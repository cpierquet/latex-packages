Un biologiste s’intéresse à l’évolution de la population d’une espèce animale sur une île du Pacifique.

Au début de l’année 2020, cette population comptait 600 individus. On considère que l’espèce sera menacée 
d’extinction sur cette île si sa population devient inférieure ou égale à 20 individus.

\smallskip

Le biologiste modélise le nombre d’individus par la suite $\suiten$ définie par :\[ \begin{dcases} u_0 = 600 \\ u_{n+1}=0,75u_n \big( 1-0,15u_n \big) \end{dcases} \] %
où pour tout entier naturel $n$, $u_n$ désigne le nombre d’individus, en milliers, au début de l’année $2020+n$.

\begin{enumerate}
	\item Estimer, selon ce modèle, le nombre d’individus présents sur l’île au début de l’année 2021 puis au début de l’année 2022.
\end{enumerate} 

Soit $f$ la fonction définie sur l’intervalle $\intervFF{0}{1}$ par $f(x)=0,75x(1-0,15x)$.

\begin{enumerate}[resume]
	\item Montrer que la fonction $f$ est croissante sur l’intervalle $\intervFF{0}{1}$ et dresser son tableau de variations.
	\item Résoudre dans l’intervalle $\intervFF{0}{1}$ l’équation $f(x)=x$.
\end{enumerate}

On remarquera pour la suite de l’exercice que, pour tout entier naturel $n$, $u_{n+1}=f \big(u_n\big)$.

\begin{enumerate}[resume]
	\item
	\begin{enumerate}
		\item Démontrer par récurrence que pour tout entier naturel $n$, $0 \leqslant u_{n+1} \leqslant u_n \leqslant 1$.
		\item En déduire que la suite $\suiten$ est convergente.
		\item Déterminer la limite $\ell$ de la suite $\suiten$.
	\end{enumerate}
	\item Le biologiste a l’intuition que l’espèce sera tôt ou tard menacée d’extinction.
	\begin{enumerate}
		\item Justifier que, selon ce modèle, le biologiste a raison.
		\item Le biologiste a programmé en langage \textsf{Python} la fonction \texttt{menace()} ci-dessous :
		
\begin{CodePythonLstAlt}*[Largeur=10cm]{center}
def menace():
	u = 0.6
	n = 0
	while u > 0.02 :
		u = 0.75*u*(1-0.15*u)
		n = n+1
	return n
\end{CodePythonLstAlt}
		
		Donner la valeur numérique renvoyée lorsqu'on appelle la fonction \texttt{menace()}.
		
		Interpréter ce résultat dans le contexte de l’exercice.
	\end{enumerate}
\end{enumerate}

