On considère l'équation différentielle $(E)$ : $y' = y + 2x\e^{x}$.

\smallskip

On cherche l'ensemble des fonctions définies et dérivables sur l'ensemble $\R$ des nombres réels qui sont solutions de cette équation.

\begin{enumerate}
	\item Soit u la fonction définie sur $\R$ par $u(x) = x^2\e^x$. On admet que $u$ est dérivable et on note $u'$ sa fonction dérivée. Démontrer que $u$ est une solution particulière de $(E)$.
	\item Soit $f$ une fonction définie et dérivable sur $\R$. On note $g$ la fonction définie sur $\R$ par : \[ g(x)=f(x)-u(x). \]
	\begin{enumerate}
		\item Démontrer que si la fonction $f$ est solution de l'équation différentielle $(E)$ alors la fonction $g$ est solution de l'équation différentielle : $y' = y$.
		
		On admet que la réciproque de cette propriété est également vraie.
		\item À l'aide de la résolution de l'équation différentielle $y' = y$, résoudre l'équation différentielle $(E)$.
	\end{enumerate}
	\item \textbf{Étude de la fonction} \boldmath$u$\unboldmath
	\begin{enumerate}
		\item Étudier le signe de $u'(x)$ pour $x$ variant dans $\R$.
		\item Dresser le tableau de variations de la fonction $u$ sur $\R$ (les limites ne sont pas demandées).
		\item Déterminer le plus grand intervalle sur lequel la fonction $u$ est concave.
	\end{enumerate}
\end{enumerate}