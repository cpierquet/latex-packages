\textit{Les probabilités demandées dans cet exercice seront arrondies à $10^{-3}$.}

\medskip

Un laboratoire pharmaceutique vient d’élaborer un nouveau test anti-dopage.

\medskip

\textbf{Partie A}

\medskip

Une étude sur ce nouveau test donne les résultats suivants : 

\begin{itemize}
	\item si un athlète est dopé, la probabilité que le résultat du test soit positif est $0,98$ (sensibilité du test) ; 
	\item si un athlète n’est pas dopé, la probabilité que le résultat du test soit négatif est $0,995$ (spécificité du test). 
\end{itemize}

On fait subir le test à un athlète sélectionné au hasard au sein des participants à une compétition d’athlétisme. On note $D$ l’événement « l’athlète est dopé » et $P$ l’événement « le test est positif ». On admet que la probabilité de l’événement $D$ est égale à $0,08$.

\begin{enumerate}
	\item Traduire la situation sous la forme d’un arbre pondéré. 
	\item Démontrer que $P(T)=0,083$. 
	\item 
	\begin{enumerate}
		\item Sachant qu’un athlète présente un test positif, quelle est la probabilité qu’il soit dopé ? 
		\item Le laboratoire décide de commercialiser le test si la probabilité de l’événement « un athlète présentant un test positif est dopé » est supérieure ou égale à $0,95$.
		
		Le test proposé par le laboratoire sera-t-il commercialisé ? Justifier.
	\end{enumerate}
\end{enumerate} 

\smallskip

\textbf{Partie B} 

\medskip

Dans une compétition sportive, on admet que la probabilité qu’un athlète contrôlé présente un test positif est $0,103$.

\begin{enumerate}
	\item Dans cette question 1., on suppose que les organisateurs décident de contrôler 5 athlètes au hasard parmi les athlètes de cette compétition. On note $X$ la variable aléatoire égale au nombre d’athlètes présentant un test positif parmi les 5 athlètes contrôlés. 
	\begin{enumerate}
		\item Donner la loi suivie par la variable aléatoire $X$. Préciser ses paramètres.
		\item Calculer l’espérance $\mathbb{E}(X)$ et interpréter le résultat dans le contexte de l’exercice. 
		\item Quelle est la probabilité qu’au moins un des 5 athlètes contrôlés présente un test positif ? 
	\end{enumerate}
	\item Combien d’athlètes faut-il contrôler au minimum pour que la probabilité de l’événement « au moins un athlète contrôlé présente un test positif » soit supérieure ou égale à $0,75$ ? Justifier.
\end{enumerate} 

