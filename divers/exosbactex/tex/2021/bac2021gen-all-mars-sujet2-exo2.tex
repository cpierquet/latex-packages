On considère les suites $\suiten$ et $\suiten[v]$ définies pour tout entier naturel $n$ par : \[\left\lbrace\begin{array}{r@{\:\:=\:\:}l} u_0&v_0=1\\u_{n+1}&u_n+v_n\\v_{n+1}&2u_n+v_n \end{array}\right.\]
Dans toute la suite de l’exercice, on admet que les suites $\suiten$ et $\suiten[v]$ sont strictement positives.

\begin{enumerate}
	\item 
	\begin{enumerate}
		\item Calculez $u_1$ et $v_1$.
		\item Démontrer que la suite $\suiten[v]$ est strictement croissante, puis en déduire que, pour tout entier naturel $n$, $v_n \geqslant 1$.
		\item Démontrer par récurrence que, pour tout entier naturel $n$,on a : $u_n \geqslant n+1$.
		\item En déduire la limite de la suite $\suiten$.
	\end{enumerate}
	\item On pose, pour tout entier naturel $n$ : \[r_n=\dfrac{v_n}{u_n}.\]
	On admet que : \[r_n^2=2+\dfrac{(-1)^{n+1}}{u_n^2}.\]
	\begin{enumerate}
		\item Démontrer que pour tout entier naturel $n$ : \[-\dfrac{1}{u_n^2} \leqslant \dfrac{(-1)^{n+1}}{u_n^2} \leqslant \dfrac{1}{u_n^2}.\]
		\item En déduire : \[\lim_{n \to +\infty} \dfrac{(-1)^{n+1}}{u_n^2}.\]
		\item Déterminer la limite de la suite $\left(r_n^2\right)$ et en déduire que $\suiten[r]$ converge vers $\sqrt{2}$.
		\item Démontrer que pour tout entier naturel $n$ : \[r_{n+1}=\dfrac{2+r_n}{1+r_n}.\]
		\item On considère le programme suivant écrit en langage \textsf{Python} :
\begin{CodePythonLstAlt}*[Largeur=10cm]{center}
def seuil() :
	n = 0
	r = 1
	while abs(r-sqrt(2)) > 10**(-4)
		r = (2+r)/(1+r)
		n = n+1
	return n
\end{CodePythonLstAlt}
		(\texttt{abs} désigne la valeur absolue, \texttt{sqrt} la racine carrée et \texttt{10**(-4)} représente $10^{-4}$).
		
		La valeur de \texttt{n} renvoyée par ce programme est \texttt{5}.
		
		À quoi correspond-elle ?
	\end{enumerate}
\end{enumerate}

