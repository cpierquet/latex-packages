Dans une boulangerie, les baguettes sortent du four à une température de 225°C.

On s’intéresse à l’évolution de la température d’une baguette après sa sortie du four.

On admet qu’on peut modéliser cette évolution à l’aide d’une fonction $f$ définie et dérivable sur l’intervalle $\intervFO{0}{+\infty}$. Dans cette modélisation, $f(t)$ représente la température en degré Celsius de la baguette au bout de la durée $t$, exprimée en heure, après la sortie du four.

Ainsi, $f(0,5)$ représente la température d’une baguette une demi-heure après la sortie du four.

Dans tout l’exercice, la température ambiante de la boulangerie est maintenue à 25°C.

On admet alors que la fonction $f$ est solution de l'équation différentielle $y'+6y = 150$.

\begin{enumerate}
	\item 
	\begin{enumerate}
		\item Préciser la valeur de $f(0)$.
		\item Résoudre l’équation différentielle $y'+ 6y = 150$.
		\item En déduire que pour tout réel $t \geqslant 0$, on a $f(t)= 200\e^{-6t}+25$.
	\end{enumerate}
	\item Par expérience, on observe que la température d’une baguette sortant du four :
	\begin{itemize}
		\item décroît ;
		\item tend à se stabiliser à la température ambiante.
	\end{itemize}
	La fonction $f$ fournit-elle un modèle en accord avec ces observations ?
	\item Montrer que l’équation $f(t)=40$ admet une unique solution dans $\intervFO{0}{+\infty}$.
	
	\medskip
	
	Pour mettre les baguettes en rayon, le boulanger attend que leur température soit inférieure ou égale à 40°C. On note $T_0$ le temps d’attente minimal entre la sortie du four d’une baguette et sa mise en rayon.
	
	On donne ci-dessous la représentation graphique de la fonction $f$ dans un repère orthogonal.
	
	\begin{center}
		\begin{tikzpicture}[x=5cm,y=0.025cm,xmin=0,xmax=2.2,xgrille=0.1,xgrilles=0.1,ymin=0,ymax=240,ygrille=20,ygrilles=20]
			\GrilleTikz[Affp=false]
			\AxesTikz[ElargirOx=0/0,ElargirOy=0/0,Labelx=$t$,PosLabelx=above,Labely=$y$,PosLabely=right] ;
			\AxexTikz[Police=\small]{0,0.5,1,1.5,2} ;
			\AxeyTikz[Police=\small]{0,20,...,220} ;
			\draw (0,230) node[right,font=\small\sffamily] {Température en degré Celsius} ;
			\draw (2.2,-15) node[below left,font=\small\sffamily] {Durée en heure} ;
			\draw[line width=1.25pt,red,samples=250,domain=0:2.2] plot (\x,{200*exp(-6*\x)+25}) ;
			\draw (0.3,60) node[above right,font=\large,red] {$\mathcal{C}_f$} ;
		\end{tikzpicture}
	\end{center}
	\item Avec la précision permise par le graphique, lire $T_0$. On donnera une valeur approchée de $T_0$ sous forme d’un nombre entier de minutes.
	\item On s’intéresse ici à la diminution, minute après minute, de la température d’une baguette à sa sortie du four.
	
	Ainsi, pour un entier naturel $n$, $D_n$ désigne la diminution de température en degré Celsius d’une baguette entre la $n$-ième et la $(n+1)$-ième minute après sa sortie du four.
	
	On admet que, pour tout entier naturel $n$ : \[ D_n=f\left(\dfrac{n}{60}\right) - f\left(\dfrac{n+1}{60}\right).\]
	\begin{enumerate}
		\item Vérifier que 19 est une valeur approchée de $D_0$ à 0,1 près, et interpréter ce résultat dans le contexte de l’exercice.
		\item  Vérifier que l’on a, pour tout entier naturel $n$ : $D_n = 200\e^{-0,1n}(1 - \e^{-0,1})$.
		
		En déduire le sens de variation de la suite $\suiten[D]$, puis la limite de la suite $\suiten[D]$.
		
		Ce résultat était-il prévisible dans le contexte de l’exercice ?
	\end{enumerate}
\end{enumerate}