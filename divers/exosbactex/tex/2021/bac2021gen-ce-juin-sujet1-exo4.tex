Dans un repère orthonormé de l’espace, on considère les points suivants :

\hfill~$A(2;-1;0)$ ; $B(3;-1;2)$ ; $C(0;4;1)$ et $S(0;1;4)$\hfill~

\begin{enumerate}
	\item Montrer que le triangle $ABC$ est rectangle en $A$
	\item 
	\begin{enumerate}
		\item Montrer que le vecteur $\vect{n} \begin{pmatrix}2\\1\\-1\end{pmatrix}$ est orthogonal au plan $(ABC)$.
		\item En déduire une équation cartésienne du plan $(ABC)$.
		\item Montrer que les points $A$, $B$, $C$ et $S$ ne sont pas coplanaires.
	\end{enumerate}
	\item  Soit $(d)$ la droite orthogonale au plan $(ABC)$ passant par $S$. Elle coupe le plan
	$(ABC)$ en $H$.
	\begin{enumerate}
		\item Déterminer une représentation paramétrique de la droite $(d)$.
		\item Montrer que les coordonnées du point $H$ sont $H(2;2;3)$.
	\end{enumerate}
	\item On rappelle que le volume $\mathcal{V}$ d’un tétraèdre est $\mathcal{V} = \dfrac{\text{Aire de la base} \times \text{hauteur}}{3}$.
	
	Calculer le volume du tétraèdre $SABC$
	\item 
	\begin{enumerate}
		\item Calculer la longueur $SA$.
		\item On indique que $SB = \sqrt{17}$.
		
		En déduire une mesure de l’angle $\widehat{ASB}$ approchée au dixième de degré.
	\end{enumerate}
\end{enumerate}

