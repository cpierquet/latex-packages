Dans cet exercice, les résultats des probabilités demandées seront, si nécessaire, arrondis au
millième.

La leucose féline est une maladie touchant les chats ; elle est provoquée par un virus.

Dans un grand centre vétérinaire, on estime à 40\,\% la proportion de chats porteurs de la maladie.

On réalise un test de dépistage de la maladie parmi les chats présents dans ce centre vétérinaire.

Ce test possède les caractéristiques suivantes :

\begin{itemize}
	\item lorsque le chat est porteur de la maladie, son test est positif dans 90\,\% des cas ;
	\item lorsque le chat n’est pas porteur de la maladie, son test est négatif dans 85\,\% des cas.
\end{itemize}

On choisit un chat au hasard dans le centre vétérinaire et on considère les événements suivants :

\begin{itemize}
	\item M : « Le chat est porteur de la maladie » ;
	\item T : « Le test du chat est positif » ;
	\item $\overline{M}$ et $\overline{T}$ désignent les événements contraires des événements M et T respectivement.
\end{itemize}

\begin{enumerate}
	\item 
	\begin{enumerate}
		\item Traduire la situation par un arbre pondéré.
		\item Calculer la probabilité que le chat soit porteur de la maladie et que son test soit positif.
		\item Montrer que la probabilité que le test du chat soit positif est égale à $0,45$.
		\item On choisit un chat parmi ceux dont le test est positif. Calculer la probabilité qu’il soit porteur de la maladie.
	\end{enumerate}
	\item On choisit dans le centre vétérinaire un échantillon de 20 chats au hasard. On admet que l’on peut assimiler ce choix à un tirage avec remise.
	
	On note $X$ la variable aléatoire donnant le nombre de chats présentant un test positif dans
	l’échantillon choisi.
	\begin{enumerate}
		\item Déterminer, en justifiant, la loi suivie par la variable aléatoire $X$.
		\item Calculer la probabilité qu’il y ait dans l’échantillon exactement 5 chats présentant un test positif.
		\item Calculer la probabilité qu’il y ait dans l’échantillon au plus 8 chats présentant un test positif.
		\item Déterminer l’espérance de la variable aléatoire $X$ et interpréter le résultat dans le contexte de l’exercice.
	\end{enumerate}
	\item Dans cette question, on choisit un échantillon de $n$ chats dans le centre, qu’on assimile encore à un tirage avec remise. On note $p_n$ la probabilité qu’il y ait au moins un chat présentant un test positif dans cet échantillon.
	\begin{enumerate}
		\item Montrer que $p_n = 1 - 0,55^n$.
		\item Décrire le rôle du programme ci-contre écrit en langage \textsf{Python}, dans lequel la variable \texttt{n} est un entier naturel et la variable \texttt{P} un nombre réel.
		%
\begin{CodePythonLstAlt}*[Largeur=8cm]{center}
def seuil() :
	n = 0
	P = 0
	while P < 0.99 :
		n = n+1
		P = 1-0.55**n
	return n
\end{CodePythonLstAlt}
		\item Déterminer, en précisant la méthode employée, la valeur renvoyée par ce programme.
	\end{enumerate}
\end{enumerate}

