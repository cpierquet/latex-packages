\textbf{Partie A :}

\medskip

Soit $f$ la fonction définie sur $\R$ par : \[ g(x)=2\e^{-\frac13x}+\dfrac23x-2. \]
\begin{enumerate}
	\item On admet que la fonction $g$ est dérivable sur $\R$ et on note $g'$ sa fonction dérivée.
	
	Montrer que, pour tout réel $x$ : \[ g'(x)=-\dfrac23 \e^{-\frac13x}+\dfrac23. \]
	\item En déduire le sens de variations de la fonction $g$ sur $\R$.
	\item Déterminer le signe de $g(x)$, pour tout $x$ réel.
\end{enumerate}

\smallskip

\textbf{Partie B :}

\begin{enumerate}
	\item On considère l’équation différentielle : \[ (E) \quad : \quad 3y'+y=0. \]
	Résoudre l’équation différentielle $(E)$.
	\item Déterminer la solution particulière dont la courbe représentative, dans un repère du plan, passe par le point $M(0;2)$.
	\item Soit $f$ la fonction définie sur $\R$ par : \[ f(x)=2\e^{-\frac13x} \] et soit $\mathcal{C}_f$ sa courbe représentative.
	\begin{enumerate}
		\item Montrer que la tangente $(\Delta_0)$ à la courbe $\mathcal{C}_f$ au point $M(0;2)$ admet une équation de la forme : \[ y=-\dfrac23x+2. \]
		\item Étudier, sur $\R$, la position de cette courbe $\mathcal{C}_f$ par rapport à la tangente $(\Delta_0)$.
	\end{enumerate}
\end{enumerate}

\smallskip

\textbf{Partie C :}

\begin{enumerate}
	\item Soit $A$ le point de la courbe $\mathcal{C}_f$ d’abscisse $a$, $a$ réel quelconque.
	
	Montrer que la tangente $(\Delta_a)$ à la courbe $\mathcal{C}_f$ au point $A$ coupe l’axe des
	abscisses en un point $P$ d’abscisse $-3$.
	\item Expliquer la construction de la tangente $(\Delta_{-2})$ à la courbe $\mathcal{C}_f$ au point $B$ d’abscisse $-2$.
\end{enumerate}