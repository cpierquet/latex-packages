\emph{Cet exercice est un questionnaire à choix multiples. Pour chacune des questions suivantes, une seule des quatre réponses proposées est exacte. Une réponse exacte rapporte un point. Une réponse fausse, une réponse multiple ou l'absence de réponse à une question ne rapporte ni n'enlève de point.\\[5pt]
	Pour répondre, indiquer sur la copie le numéro de la question et la lettre de la réponse choisie. Aucune justification n'est demandée.}

\medskip

Dans l'espace rapporté à un repère orthonormé $\Rijk$, on considère les points $A(1;0;2)$, $B(2;1;0)$, $C(0;1;2)$ et la droite $\Delta$ dont une représentation paramétrique est : $\begin{dcases} x = 1 + 2t\\ y = -2 + t\\ z = 4 - t \end{dcases},\: t\, \in \R$.

\begin{enumerate}
	\item Parmi les points suivants, lequel appartient à la droite $\Delta$ ?
	\begin{center}
		\begin{tblr}{width=\linewidth,colspec={X[m,l]X[m,l]}}
			\textbf{Réponse A :} $M(2;1;-1)$; & \textbf{Réponse B :} $N(-3;-4;6)$ ;\\
			\textbf{Réponse C :} $P(-3;-4;2)$ ; & \textbf{Réponse D :} $Q(-5;-5;1)$.
		\end{tblr}
	\end{center}
	
	\item Le vecteur $\vect{AB}$ admet pour coordonnées :
	
	\begin{center}
		\begin{tblr}{width=\linewidth,colspec={X[m,l]X[m,l]}}
			\textbf{Réponse A :} $\begin{pmatrix}1,5\\0,5\\1\end{pmatrix}$;& \textbf{Réponse B :}$\begin{pmatrix}-1\\-1\\2\end{pmatrix}$ ;\\
			\textbf{Réponse C :} $\begin{pmatrix}1\\1\\-2\end{pmatrix}$& \textbf{Réponse D :} $\begin{pmatrix}3\\1\\2\end{pmatrix}$.
		\end{tblr}
	\end{center}
	
	\item Une représentation paramétrique de la droite $(AB)$ est :
	
	\begin{center}
		\begin{tblr}{width=\linewidth,colspec={X[m,l]X[m,l]}}
			\textbf{Réponse A :}$\begin{dcases}x=1+2t\\y = t\\z = 2\end{dcases},\:t \in \R$&
			\textbf{Réponse B :} $\begin{dcases}x =2 - t\\y = 1 - t\\z = 2t\end{dcases},\:t \in \R$\\
			\textbf{Réponse C :} $\begin{dcases}x = 2 + t\\y = 1 + t\\z = 2t\end{dcases},\:t \in \R$&
			\textbf{Réponse D :} $\begin{dcases}x = 1 + t\\y = 1 + t\\z = 2 - 2t\end{dcases},\:t \in \R$
		\end{tblr}
	\end{center}
	
	\item Une équation cartésienne du plan passant par le point $C$ et orthogonal à la droite $\Delta$ est :
	
	\begin{center}
		\begin{tblr}{width=\linewidth,colspec={X[m,l]X[m,l]}}
			\textbf{Réponse A :} $x - 2y + 4z - 6 = 0$ ;& \textbf{Réponse B :} $2x + y - z + 1 = 0$ ;\\
			\textbf{Réponse C :} $2x + y - z- 1 = 0$ ;& \textbf{Réponse D :} $y + 2z - 5 = 0$.
		\end{tblr}
	\end{center}
	
	\item On considère le point $D$ défini par la relation vectorielle $\vect{{OD}} = 3\vect{{OA}} - \vect{{OB}} - \vect{{OC}}$.
	
	\begin{center}
		\begin{tblr}{width=\linewidth,colspec={X[m,l]X[m,l]}}
			\textbf{Réponse A :} $\vect{{AD}}$, $\vect{{AB}}$, $\vect{{AC}}$ sont coplanaires ; &\textbf{Réponse B :} $\vect{{AD}} = \vect{{BC}}$ ;\\
			\textbf{Réponse C :} $D$ a pour coordonnées $(3;-1;-1)$ ;&\textbf{Réponse D :} les points $A$, $B$, $C$ et $D$ sont alignés.
		\end{tblr}
	\end{center}
\end{enumerate}

