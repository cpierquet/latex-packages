Au 1\up{er} janvier 2020, la centrale solaire de Big Sun possédait \num{10560} panneaux solaires. On observe, chaque année, que 2\,\% des panneaux se sont détériorés et nécessitent d’être retirés tandis que 250 nouveaux panneaux solaires sont installés.

\medskip

\textbf{Partie A - Modélisation à l’aide d’une suite}

\medskip

On modélise l’évolution du nombre de panneaux solaires par la suite $\suiten$ définie par $u_0=\num{10560}$ et, pour tout entier naturel $n$, $u_{n+1}=0,98u_n+250$, où $u_n$ est le nombre de panneaux solaires au 1\up{er} janvier de l’année $2020+n$.

\begin{enumerate}
	\item 
	\begin{enumerate}
		\item Expliquer en quoi cette modélisation correspond à la situation étudiée.
		\item On souhaite savoir au bout de combien d’années le nombre de panneaux solaires sera strictement supérieur à \num{12000}. À l’aide de la calculatrice, donner la réponse à ce problème.
		\item Recopier et compléter le programme en \textsf{Python} ci-dessous de sorte que la valeur cherchée à la question précédente soit stockée dans la variable \texttt{n} à l’issue de l’exécution de ce dernier.
		%
\begin{CodePythonLstAlt}*[Largeur=8cm]{center}
u = 10560
n = 0
while ......... :
	u = .......
	n = .......
\end{CodePythonLstAlt}
	\end{enumerate}
	\item Démontrer par récurrence que, pour tout entier naturel $n$, on a $u_n \leqslant \num{12500}$.
	\item Démontrer que la suite $\suiten$ est croissante.
	\item En déduire que la suite $\suiten$ converge. Il n’est pas demandé, ici, de calculer sa limite.
	\item On définit la suite $\suiten[v]$ par $v_n=u_n-\num{12500}$, pour tout entier naturel $n$.
	\begin{enumerate}
		\item Démontrer que la suite $\suiten[v]$ est une suite géométrique de raison 0,98 dont on précisera le premier terme.
		\item Exprimer, pour tout entier naturel $n$, $v_n$ en fonction de $n$.
		\item En déduire, pour tout entier naturel $n$, $u_n$ en fonction de $n$.
		\item Déterminer la limite de la suite $\suiten$. Interpréter ce résultat dans le contexte du modèle.
	\end{enumerate}
\end{enumerate}

\smallskip

\textbf{Partie B - Modélisation à l’aide d’une fonction}

\medskip

Une modélisation plus précise a permis d’estimer le nombre de panneaux solaires de la centrale à l’aide de la fonction $f$ définie pour tout $x \in \intervFO{0}{+\infty}$ par $f(x)=\num{12500}-500\e^{-0,02x+1,4}$, où $x$ représente le nombre d’années écoulées depuis le 1\up{er} janvier 2020.

\begin{enumerate}
	\item Étudier le sens de variation de la fonction $f$.
	\item Déterminer la limite de la fonction $f$ en $+\infty$.
	\item En utilisant ce modèle, déterminer au bout de combien d’années le nombre de panneaux solaires dépassera \num{12000}.
\end{enumerate}

