On considère la suite $\suiten$ définie par : $u_0 = 1$ et, pour tout entier naturel $n$, \[ u_{n+1}=\dfrac{4u_n}{u_n+4}. \]

\begin{enumerate}
	\item La copie d’écran ci-dessous présente les valeurs, calculées à l’aide d’un tableur, des
	termes de la suite $\suiten$ pour $n$ variant de 0 à 12, ainsi que celles du quotient $\dfrac{4}{u_n}$ (avec, pour les valeurs de $u_n$, affichage de deux chiffres pour les parties décimales).
	
	\begin{center}
		%TABLEUR
		\begin{tikzpicture}
			\tableur*[14]{A/2cm,B/2cm,C/2cm}
			\celtxt*[c]{A}{1}{$n$}
			\celtxt*[c]{B}{1}{$u_n$}
			\celtxtmulti{C}{1}{1.75cm}{$\nicefrac{4}{u_n}$}
			\foreach \A in {2,3,...,14}{%
				\FPeval{nba}{clip(\A-2)}
				\celtxt*[c]{A}{\A}{$\nba$}
			}
			\foreach \C in {2,3,...,14}{%
				\FPeval{nbc}{clip(\C+2)} 
				\celtxt*[c]{C}{\C}{$\nbc$}
			}
			\foreach \B/\val in {2/{1,00},3/{0,80},4/{0,67},5/{0,57},6/{0,50},7/{0,44},8/{0,40},9/{0,36},10/{0,33},11/{0,29},12/{0,27},13/{0,27},14/{0,25}}{%
				\celtxt*[c]{B}{\B}{$\val$}} 
		\end{tikzpicture}
	\end{center}
	
	\smallskip
	
	À l’aide de ces valeurs, conjecturer l’expression de $\dfrac{4}{u_n}$ en fonction de $n$.
	
	\smallskip
	
	Le but de cet exercice est de démontrer cette conjecture (question 5.), et d’en déduire la limite de la suite $\suiten$ (question 6.).
	\item Démontrer par récurrence que, pour tout entier naturel $n$, on a : $u_n > 0$.
	\item Démontrer que la suite $\suiten$ est décroissante.
	\item Que peut-on conclure des questions 2. et 3. concernant la suite $\suiten$ ?
	\item On considère la suite $\suiten[v]$ définie pour tout entier naturel $n$ par : $v_n = \dfrac{4}{u_n}$.
	
	Démontrer que $\suiten[v]$ est une suite arithmétique. Préciser sa raison et son premier terme.
	
	En déduire, pour tout entier naturel $n$, l’expression de $v_n$ en fonction de $n$.
	\item Déterminer, pour tout entier naturel $n$, l’expression de $u_n$ en fonction de $n$.
	
	En déduire la limite de la suite $\suiten$.
\end{enumerate}

