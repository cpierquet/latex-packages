\textbf{Partie I : Étude d’une fonction auxiliaire}

\medskip

Soit $g$ la fonction définie sur $\intervOO{0}{+\infty}$ par : $g(x)=\ln(x)+2x-2$.

\begin{enumerate}
	\item Déterminer les limites de $g$ en $+\infty$ et $0$. 
	\item Déterminer le sens de variation de la fonction $g$ sur $\intervOO{0}{+\infty}$
	\item Démontrer que l'équation $g(x)=0$ admet une unique solution $\alpha$ sur $\intervOO{0}{+\infty}$. 
	\item Calculer $g(1)$ puis déterminer le signe de $g$ sur $\intervOO{0}{+\infty}$.
\end{enumerate}

\textbf{Partie II : Étude d'une fonction \boldmath$f$\unboldmath}

\medskip

On considère la fonction $f$, définie sur $\intervOO{0}{+\infty}$ par : $f(x)=\left(2-\dfrac{1}{x}\right)\big(\ln(x)-1\big)$.

\begin{enumerate}
	\item 
	\begin{enumerate}
		\item On admet que la fonction $f$ est dérivable sur $\intervOO{0}{+\infty}$ et on note $f'$ sa dérivée.
		
		Démontrer que, pour tout $x$ de $\intervOO{0}{+\infty}$, on a : \[f'(x)=\dfrac{g(x)}{x^2}.\]
		\item Dresser le tableau de variation de la fonction $f$ sur $\intervOO{0}{+\infty}$. Le calcul des limites n’est pas demandé.
	\end{enumerate}
	\item Résoudre l'équation $f(x)=0$ sur $\intervOO{0}{+\infty}$ puis dresser le tableau de signes de $f$ sur l’intervalle $\intervOO{0}{+\infty}$. 
\end{enumerate}

\textbf{Partie III : Étude d’une fonction $\bm{F}$ admettant pour dérivée la fonction $\bm{f}$}

\medskip

On admet qu’il existe une fonction $F$ dérivable sur $\intervOO{0}{+\infty}$ dont la dérivée $F'$ est la fonction $f$.

Ainsi, on a : $F'=f$.

\smallskip

On note $\mathcal{C}_F$ la courbe représentative de la fonction $F$ dans un repère orthonormé $\Rij$.

\smallskip

On ne cherchera pas à déterminer une expression de $F(x)$.

\begin{enumerate}
	\item Étudier les variations de $F$ sur $\intervOO{0}{+\infty}$.
	\item La courbe représentative $\mathcal{C}_F$ de $F$ admet-elle des tangentes parallèles à l’axe des abscisses ? Justifier la réponse.
\end{enumerate}