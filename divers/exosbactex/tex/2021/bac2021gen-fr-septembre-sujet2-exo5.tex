\textbf{Partie 1}

\medskip

Le graphique ci-dessous donne la représentation graphique dans un repère orthonormé de la fonction $f$
définie sur l'intervalle $]0;+\infty[$ par: \[f(x) = \dfrac{2\ln (x) - 1}{x}.\]
%
\begin{center}
	\begin{tikzpicture}[x=1.65cm,y=1.65cm,xmin=0,xmax=5,ymin=-2.25,ymax=1]
		\AxesTikz[ElargirOx=0/0,ElargirOx=0/0]
		\AxexTikz{0,1,2,3,4} \AxeyTikz{-2,-1,0}
		\clip (\xmin,\ymin) rectangle (\xmax,\ymax) ;
		\draw[line width=1.25pt,red,domain=0.1:5,samples=500] plot (\x,{(2*ln(\x)-1)/(\x)}) ;
	\end{tikzpicture}
\end{center}

\begin{enumerate}
	\item Déterminer par le calcul l'unique solution $\alpha$ de l'équation $f(x) = 0$.
	
	On donnera la valeur exacte de $\alpha$ ainsi que la valeur arrondie au centième.
	\item Préciser, par lecture graphique, le signe de $f(x)$ lorsque $x$ varie dans l'intervalle $]0;+\infty[$.
\end{enumerate}

\textbf{Partie II}

\medskip

On considère la fonction $g$ définie sur l'intervalle $]0;+\infty[$ par : \[g(x) = [\ln (x)]^2 - \ln (x).\]
%
\begin{enumerate}
	\item 
	\begin{enumerate}
		\item Déterminer la limite de la fonction $g$ en $0$.
		\item Déterminer la limite de la fonction $g$ en $+ \infty$.
	\end{enumerate}
	\item On note $g'$ la fonction dérivée de la fonction $g$ sur l'intervalle $]0;+\infty[$.
	
	Démontrer que, pour tout nombre réel $x$ de $]0;+\infty[$, on a : $g'(x) = f(x)$, où $f$ désigne la fonction définie dans la \textbf{partie I}.
	\item Dresser le tableau de variations de la fonction $g$ sur l'intervalle $]0;+\infty[$.
	
	On fera figurer dans ce tableau les limites de la fonction $g$ en $0$ et en $+\infty$, ainsi que la valeur du minimum de $g$ sur $]0;+\infty[$.
	\item Démontrer que, pour tout nombre réel $m > - 0,25$, l'équation $g(x) = m$ admet exactement deux solutions.
	\item Déterminer par le calcul les deux solutions de l'équation $g(x) = 0$.
\end{enumerate}

