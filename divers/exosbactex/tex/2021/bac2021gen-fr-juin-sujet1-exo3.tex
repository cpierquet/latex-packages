Cécile a invité des amis à déjeuner sur sa terrasse. Elle a prévu en dessert un assortiment de gâteaux individuels qu'elle a achetés surgelés.

Elle sort les gâteaux du congélateur à $-19$\textcelsius{} et les apporte sur la terrasse où la température ambiante est de $15$\textcelsius.

Au bout de 10 minutes, la température des gâteaux est de $1,3$\textcelsius.

\begin{center}
	\textbf{I -- Premier modèle}
\end{center}

On suppose pue la vitesse de décongélation est constante, c'est-à-dire que l'augmentation de la température des gâteaux est la même minute après minute.

\smallskip

Selon ce modèle, déterminer quelle serait la température des gâteaux 25 minutes après leur sortie du congélateur.

\smallskip

Ce modèle semble-t-il pertinent ?

\begin{center}
	\textbf{II -- Second modèle}
\end{center}

On note $T_n$ la température des gâteaux, en degré Celsius, au bout de $n$ minutes après leur sortie du congélateur ; ainsi $T_0 = -19$.

\smallskip

On admet que pour modéliser l'évolution de la température, on doit avoir la relation suivante :

\smallskip

\hfill~pour tout entier naturel $n$, $T_{n+1}-T_n = -0,06 \times (T_n-25)$\hfill~

\begin{enumerate}
	\item Justifier pue,pour tout entier naturel $n$, on a : $T_{n+1} = 0,94T_n + 1,5$.
	\item Calculer $T_1$ et $T_2$. On donnera des valeurs arrondies au dixième.
	\item Démontrer par récurrence que, pour tout entier naturel $n$, on a : $T_n < 25$.
	
	En revenant à fa situation étudiée, ce résultat était-il prévisible ?
	\item Étudier le sens de variation de la suite $\suiten[T]$.
	\item Démontrer que la suite $\suiten[T]$ est convergente.
	\item On pose, pour tout entier naturel $n$, $U_n = T_n - 25$.
	\begin{enumerate}
		\item Montrer que la suite $\suiten[U]$ est une suite géométrique dont on précisera la raison et le premier terme $U_0$.
		\item En déduire que pour tout entier naturel $n$, $T_n = 14 \times 0,94^n + 25$.
		\item En déduire la limite de la suite $\suiten[T]$. Interpréter ce résultat dans le contexte de la situation étudiée.
	\end{enumerate}
	\item 
	\begin{enumerate}
		\item Le fabricant conseille de consommer les gâteaux au bout d'une demi-heure à température ambiante après leur sortie du congélateur. Quelle est alors la température atteinte par les gâteaux ? On donnera une valeur arrondie à l'entier le plus proche.
		\item Cécile est une habituée de ces gâteaux, qu'elle aime déguster lorsqu'ils sont encore frais, à la température de $10$\textcelsius{}. Donner un encadrement entre deux entiers consécutifs du temps en minutes après lequel Cécile doit déguster son gâteau.
		\item Le programme suivant, écrit en langage \textsf{Python}, doit renvoyer après son exécution le plus petite valeur de l'entier $n$ pour laquelle $T_n \geqslant 10$.
		
		\smallskip
		
		\begin{minipage}{0.5\linewidth}
\begin{CodePythonLstAlt}*[Largeur=8cm]{center}
def seuil() :
	n = 0
	T = ......
	while T....... :
		T = ......
		n = n+1
	return n
\end{CodePythonLstAlt}
		\end{minipage}\hfill
		\begin{minipage}{0.33\linewidth}
			Recopier ce programme sur la copie et compléter les lignes incomplètes afin que le programme renvoie la valeur attendue.
		\end{minipage}\hfill~
	\end{enumerate}
\end{enumerate}

