En mai 2020, une entreprise fait le choix de développer le télétravail afin de s’inscrire
dans une démarche écoresponsable.

Elle propose alors à ses \num{5000} collaborateurs en France de choisir entre le télétravail et le travail au sein des locaux de l’entreprise.

En mai 2020, seuls 200 d’entre eux ont choisi le télétravail.

Chaque mois, depuis la mise en place de cette mesure, les dirigeants de l’entreprise constatent que 85\,\% de ceux qui avaient choisi le télétravail le mois précédent choisissent de continuer, et que, chaque mois, 450 collaborateurs supplémentaires choisissent le télétravail.

\smallskip

On modélise le nombre de collaborateurs de cette entreprise en télétravail par la $\suiten[a]$.

\smallskip

Le terme $a_n$ désigne ainsi une estimation du nombre de collaborateurs en télétravail
le $n$-ième mois après le mois de mai 2020. Ainsi $a_0 = 200$.

\medskip

\textbf{Partie A :}

\begin{enumerate}
	\item Calculer $a_1$.
	\item Justifier que pour tout entier naturel $n$, \[ a_{n+1}=0,85a_n+450. \]
	\item On considère la suite $\suiten[v]$ définie pour tout entier naturel $n$ par : \[ v_n = a_n - \num{3000}. \]
	\begin{enumerate}
		\item Démontrer que la suite $\suiten[v]$ est une suite géométrique de raison $0,85$.
		\item Exprimer $v_n$ en fonction de $n$ pour tout entier naturel $n$.
		\item En déduire que, pour tout entier naturel $n$, \[ a_n = -\num{2800} \times 0,85^n + \num{3000}. \]
	\end{enumerate}
	\item  Déterminer le nombre de mois au bout duquel le nombre de télétravailleurs sera strictement supérieur à \num{2500}, après la mise en place de cette mesure dans l’entreprise.
\end{enumerate}

\smallskip

\textbf{Partie B :}

\medskip

Afin d’évaluer l’impact de cette mesure sur son personnel, les dirigeants de l’entreprise sont parvenus à modéliser le nombre de collaborateurs satisfaits par ce dispositif à l’aide de la suite $\suiten$ définie par $u_0=1$ et, pour tout entier naturel $n$, \[ u_{n+1} = \dfrac{5u_n+4}{u_n+2} \]%
où $u_n$ désigne le nombre de milliers de collaborateurs satisfaits par cette nouvelle mesure au bout de $n$ mois après le mois de mai 2020

\begin{enumerate}
	\item Démontrer que la fonction $f$ définie pour tout $x \in \intervFO{0}{+\infty}$ par $f(x)=\dfrac{5x+4}{x+2}$ est strictement croissante sur $\intervFO{0}{+\infty}$.
	\item 
	\begin{enumerate}
		\item Démontrer par récurrence que pour tout entier naturel $n$, \[ 0 \leqslant u_n \leqslant u_{n+1} \leqslant 4. \]
		\item Justifier que la suite $\suiten$ est convergente.
	\end{enumerate}
	\item On admet que pour tout entier naturel $n$, \[ 0 \leqslant 4-u_n \leqslant 3 \times \left( \dfrac12\right)^n. \]%
	En déduire la limite de la suite $\suiten$ et l’interpréter dans le contexte de la modélisation.
\end{enumerate}

