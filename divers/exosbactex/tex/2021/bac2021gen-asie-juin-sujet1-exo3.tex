Un sac contient les huit lettres suivantes: A B C D E F G H (2 voyelles et 6 consonnes).

Un jeu consiste à tirer simultanément au hasard deux lettres dans ce sac.

On gagne si le tirage est constitué d'une voyelle \textbf{et} d'une consonne.

\begin{enumerate}
	\item Un joueur extrait simultanément deux lettres du sac.
	\begin{enumerate}
		\item Déterminer le nombre de tirages possibles.
		\item Déterminer la probabilité que le joueur gagne à ce jeu.
	\end{enumerate}
\end{enumerate}	
Les questions 2. et 3. de cet exercice sont indépendantes.

Pour la suite de l'exercice, on admet que la probabilité que le joueur gagne est égale à $\dfrac{3}{7}$.

\begin{enumerate}[resume]
	\item Pour jouer, le joueur doit payer $k$ euros, $k$ désignant un entier naturel non nul. 
	
	Si le joueur gagne, il remporte la somme de $10$ euros, sinon il ne remporte rien.
	
	On note $G$ la variable aléatoire égale au gain algébrique d'un joueur (c'est-à-dire la somme remportée à laquelle on soustrait la somme payée).
	\begin{enumerate}
		\item Déterminer la loi de probabilité de $G$.
		\item Quelle doit être la valeur maximale de la somme payée au départ pour que le jeu reste
		favorable au joueur ?
	\end{enumerate}
	\item Dix joueurs font chacun une partie. Les lettres tirées sont remises dans le sac après chaque partie.
	
	On note $X$ la variable aléatoire égale au nombre de joueurs gagnants.
	\begin{enumerate}
		\item Justifier que $X$ suit une loi binomiale et donner ses paramètres.
		\item Calculer la probabilité, arrondie à $10^{-3}$, qu'il y ait exactement quatre joueurs gagnants. 
		\item Calculer $P(X \geqslant 5)$ en arrondissant à $10^{-3}$. Donner une interprétation du résultat obtenu. 
		\item Déterminer le plus petit entier naturel $n$ tel que $P(X \leqslant n) \geqslant 0,9$.
	\end{enumerate}
\end{enumerate}

