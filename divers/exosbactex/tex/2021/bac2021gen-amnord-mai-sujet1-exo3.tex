\textbf{Les questions 1. à 5. de cet exercice peuvent être traitées de façon indépendante.}

\medskip

On considère un cube $ABCDEFGH$. Le point $I$ est le milieu du segment $[EF]$, le point $J$ est le milieu du segment $[BC]$ et le point $K$ est le milieu du segment $[AE]$.

\begin{center}
	\begin{tikzpicture}[scale=1,font=\small]
		\tkzDefPoint(0,0){A}\tkzDefPoint(4,-0.5){B}\tkzDefPoint(0,4.0311){E}\tkzDefPoint(4,3.5311){F}
		\begin{scope}[shift=(A)]\tkzDefPoint(1.5,1){D}\end{scope}
		\begin{scope}[shift=(B)]\tkzDefPoint(1.5,1){C}\end{scope}
		\begin{scope}[shift=(E)]\tkzDefPoint(1.5,1){H}\end{scope}
		\begin{scope}[shift=(F)]\tkzDefPoint(1.5,1){G}\end{scope}
		\tkzDefMidPoint(A,E) \tkzGetPoint{K}
		\tkzDefMidPoint(E,F) \tkzGetPoint{I}
		\tkzDefMidPoint(B,C) \tkzGetPoint{J}
		\tkzDrawPolygon[gray,fill=lightgray!15,line width=1pt](A,B,F,E)
		\tkzDrawPolygon[gray,fill=lightgray!15,line width=1pt](B,C,G,F)
		\tkzDrawPolygon[gray,fill=lightgray!15,line width=1pt](E,F,G,H)
		\tkzDrawSegment[gray,line width=1pt,dashed](A,D)
		\tkzDrawSegment[gray,line width=1pt,dashed](D,C)
		\tkzDrawSegment[gray,line width=1pt,dashed](H,D)
		\tkzDrawSegment[line width=1.5pt,densely dashed](K,H)
		\tkzDrawSegment[line width=1.5pt](A,I)
		\tkzDrawSegment[line width=1.5pt,densely dashed](I,J)
		\tkzDrawPoints[size=4](A,B,E,F,C,D,G,H,K,I,J)
		\tkzLabelPoints[left](A,E,K)
		\tkzLabelPoints[right](G,C)
		\tkzLabelPoints[above](F,H,I)
		\tkzLabelPoints[above right](D)
		\tkzLabelPoints[below right](B,J)
	\end{tikzpicture}
	
\end{center}

\begin{enumerate}
	\item Les droites $(AI)$ et $(KH)$ sont-elles parallèles ? Justifier votre réponse.
\end{enumerate}

Dans la suite, on se place dans le repère orthonormé $\left( A;\vect{AB},\vect{AD},\vect{AE} \right)$.

\begin{enumerate}[resume]
	\item
	\begin{enumerate}
		\item Donner les coordonnées des points $I$ et $J$.
		\item Montrer que les vecteurs $\vect{IJ}$, $\vect{AE}$ et $\vect{AC}$ sont coplanaires. 
	\end{enumerate}
\end{enumerate}

On considère le plan $\mathcal{P}$ d’équation $x+3y-2z+2=0$ ainsi que les droites $d_1$ et $d_2$ définies par les représentations paramétriques ci-dessous : \[ d_1 \: : \: \begin{dcases} x = 8+t \\ y=8-2t \\ z=-2+3t \end{dcases}, \: t \in \R \quad \text{ et } \quad d_2 \: : \: \begin{dcases} x = 4+t \\ y=1+t \\ z=8+2t \end{dcases}, \: t \in \R. \]

\begin{enumerate}[resume]
	\item Les droites $d_1$ et $d_2$ sont-elles parallèles ? Justifier votre réponse.
	\item Montrer que la droite $d_2$ est parallèle au plan $\mathcal{P}$.
	\item Montrer que le point $L(4;0;3)$ est le projeté orthogonal du point $M(5;3;1)$ sur le plan $\mathcal{P}$.
\end{enumerate}

