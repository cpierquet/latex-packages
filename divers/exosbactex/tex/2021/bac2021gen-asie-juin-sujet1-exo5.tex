\textbf{Partie I}

\medskip

Considérons l'équation différentielle \[y'= -0,4y + 0,4\] %
où $y$ désigne une fonction de la variable $t$, définie et dérivable sur $[0; + \infty[$.

\begin{enumerate}
	\item 
	\begin{enumerate}
		\item Déterminer une solution particulière constante de cette équation différentielle. 
		\item En déduire l'ensemble des solutions de cette équation différentielle.
		\item Déterminer la fonction $g$, solution de cette équation différentielle, qui vérifie $g(0) = 10$.
	\end{enumerate}
\end{enumerate}

\textbf{Partie II}

\medskip

Soit $p$ la fonction définie et dérivable sur l'intervalle $[0;+ \infty[$ par \[p(t) = \dfrac{1}{g(t)} = \dfrac{1}{1 + 9\e^{-0,4t}}.\]

\begin{enumerate}
	\item Déterminer la limite de $p$ en $+ \infty$. 
	\item Montrer que $p'(t) = \dfrac{3,6\e^{-0,4t}}{ \left(1 + 9\e^{-0,4t}\right)^2}$ pour tout $t \in  [0;+ \infty[$.
	
	\item
	\begin{enumerate}
		\item Montrer que l'équation $p(t) = \dfrac{1}{2}$ admet une unique solution $\alpha$ sur $[0;+ \infty[$. 
		\item Déterminer une valeur approchée de $\alpha$ à $10^{-1}$ près à l'aide d'une calculatrice.
	\end{enumerate}
\end{enumerate}

\textbf{Partie III}

\begin{enumerate}
	\item $p$ désigne la fonction de la partie II.
	
	Vérifier que $p$ est solution de l'équation différentielle $y' = 0,4y(1 - y)$ avec la condition initiale \mbox{$y(0) = \dfrac{1}{10}$} où $y$ désigne une fonction définie et dérivable sur $[0; + \infty[$.
	\item Dans un pays en voie de développement, en l'année 2020, 10\,\% des écoles ont accès à internet. 
	
	Une politique volontariste d'équipement est mise en œuvre et on s'intéresse à l'évolution de la proportion des écoles ayant accès à internet. 
	
	On note $t$ le temps écoulé, exprimé en année, depuis l'année 2020.
	
	La proportion des écoles ayant accès à internet à l'instant $t$ est modélisée par $p(t)$.
	
	Interpréter dans ce contexte la limite de la question \textbf{II.}1. puis la valeur approchée de $\alpha$ de la question \textbf{II.}3.(b) ainsi que la valeur $p(0)$.
\end{enumerate}