On considère la suite $\suiten$ définie par $u_0 = \num{10000}$ et pour tout entier naturel $n$ : \[ u_{n+1}=0,95u_n+200. \]

\begin{enumerate}
	\item Calculer $u_1$ et vérifier que $u_2 = \num{9415}$.
	\item 
	\begin{enumerate}
		\item Démontrer, à l’aide d’un raisonnement par récurrence, que pour tout entier $n$ : \[ u_n < \num{4000}. \]
		\item On admet que la suite $\suiten$ est décroissante. Justifier qu’elle converge.
	\end{enumerate}
	\item Pour tout entier naturel $n$, on considère la suite $\suiten[v]$ définie par : $v_n=u_n-\num{4000}$.
	\begin{enumerate}
		\item Calculer $v_0$.
		\item Démontrer que la suite $\suiten[v]$ est géométrique de raison égale à $0,95$.
		\item En déduire que pour tout entier naturel $n$ : \[ u_n = \num{4000} + \num{6000} \times 0,95^n. \]
		\item Quelle est la limite de la suite $\suiten$ ? Justifier la réponse.
	\end{enumerate}
	\item En 2020, une espèce animale comptait \num{10000} individus. L’évolution observée les années précédentes conduit à estimer qu’à partir de l’année 2021, cette population baissera de 5\,\% chaque début d’année.
	
	Pour ralentir cette baisse, il a été décidé de réintroduire 200 individus à la fin de chaque année, à partir de 2021.
	
	Une responsable d’une association soutenant cette stratégie affirme que : « l’espèce ne devrait pas s’éteindre, mais malheureusement, nous n’empêcherons pas une disparition de plus de la moitié de la population ».
	
	Que pensez-vous de cette affirmation ? Justifier la réponse.
\end{enumerate}

