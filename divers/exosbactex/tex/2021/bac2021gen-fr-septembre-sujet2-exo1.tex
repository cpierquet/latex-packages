Une entreprise reçoit quotidiennement de nombreux courriels (courriers électroniques).

Parmi ces courriels, 8\,\% sont du \og spam \fg, c'est-à-dire des courriers à intention publicitaire, voire malveillante, qu'il est souhaitable de ne pas ouvrir.

On choisit au hasard un courriel reçu par l'entreprise.

Les propriétés du logiciel de messagerie utilisé dans l'entreprise permettent d'affirmer que :

\begin{itemize}[leftmargin=*]
	\item La probabilité que le courriel choisi soit classé comme \og indésirable\fg{} sachant que c'est un spam est égale à $0,9$.
	\item La probabilité que le courriel choisi soit classé comme \og indésirable\fg{} sachant que ce n'est pas un spam est égale à $0,01$.
\end{itemize}

On note :

\begin{itemize}
	\item $S$ l'évènement \og le courriel choisi est un spam \fg ;
	\item $I$ l'évènement \og le courriel choisi est classé comme indésirable par le logiciel de messagerie \fg.
	\item $\overline{S}$ et $\overline{I}$ les évènements contraires de $S$ et $I$ respectivement.
\end{itemize}

\begin{enumerate}
	\item Modéliser la situation étudiée par un arbre pondéré, sur lequel on fera apparaître les probabilités associées à chaque branche.
	\item 
	\begin{enumerate}
		\item Démontrer que la probabilité que le courriel choisi soit un message de spam et qu'il soit classé indésirable est égale à $0,072$.
		\item Calculer la probabilité que le message choisi soit classé indésirable.
		\item Le message choisi est classé comme indésirable. Quelle est la probabilité que ce soit effectivement un message de spam ? On donnera un résultat arrondi au centième.
	\end{enumerate}
	\item On choisit au hasard $50$ courriels parmi ceux reçus par l'entreprise. On admet que ce choix se ramène à un tirage au hasard avec remise de $50$ courriels parmi l'ensemble des courriels reçus par l'entreprise.
	
	On appelle $Z$ la variable aléatoire dénombrant les courriels de spam parmi les $50$ choisis.
	\begin{enumerate}
		\item Quelle est la loi de probabilité suivie par la variable aléatoire $Z$, et quels sont ses paramètres ?
		\item Quelle est la probabilité que, parmi les 50 courriels choisis, deux au moins soient du spam ? On donnera un résultat arrondi au centième.
	\end{enumerate}
\end{enumerate}

