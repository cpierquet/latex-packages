Soit $f$ la fonction définie sur l’intervalle $\intervOO{0}{+\infty}$ par : \[f(x)=\dfrac{\e^x}{x}.\]
%
On note $\mathcal{C}_f$ la courbe représentative de la fonction $f$ dans un repère orthonormé.

\begin{enumerate}
	\item 
	\begin{enumerate}
		\item Préciser la limite de la fonction $f$ en $+\infty$.
		\item Justifier que l’axe des ordonnées est asymptote à la courbe $\mathcal{C}_f$.
	\end{enumerate}
	\item Montrer que, pour tout nombre réel $x$ de l’intervalle $\intervOO{0}{+\infty}$, on a : \[f'(x)= \dfrac{\e^x(x-1)}{x^2}\] où $f'$ désigne la fonction dérivée de la fonction $f$.
	\item Déterminer les variations de la fonction $f$ sur l’intervalle $\intervOO{0}{+\infty}$. On établira un tableau de variations de la fonction $f$ dans lequel apparaîtront les limites.
	\item Soit $m$ un nombre réel. Préciser, en fonction des valeurs du nombre réel $m$, le nombre de solutions de l’équation $f(x)=m$.
	\item On note $\Delta$ la droite d’équation $y=-x$.
	
	On note $A$ un éventuel point de $\mathcal{C}_f$ d’abscisse $a$ en lequel la tangente à la courbe $\mathcal{C}_f$ est parallèle à la droite $\Delta$.
	\begin{enumerate}
		\item Montrer que $a$ est solution de l’équation $\e^x(x-1)+x^2=0$.
		
		\smallskip
		
		On note $g$ la fonction définie sur $\intervFO{0}{+\infty}$ par $g(x)=\e^x(x-1)+x^2$.
		
		On admet que la fonction $g$ est dérivable et on note $g'$ sa fonction dérivée.
		\item Calculer $g'(x)$ pour tout nombre réel $x$ de l’intervalle $\intervFO{0}{+\infty}$, puis dresser le tableau de variations de $g$ sur $\intervFO{0}{+\infty}$.
		\item Montrer qu’il existe un unique point $A$ en lequel la tangente à $\mathcal{C}_f$ est parallèle à la droite $\Delta$.
	\end{enumerate}
\end{enumerate}

