Au cours d’une séance, un joueur de volley-ball s’entraîne à faire des services. La probabilité qu’il réussisse le premier service est égale à $0,85$.

\smallskip

On suppose de plus que les deux conditions suivantes sont réalisées :

\begin{itemize}
	\item si le joueur réussit un service, alors la probabilité qu’il réussisse le suivant est égale à $0,6$ ;
	\item si le joueur ne réussit pas un service, alors la probabilité qu’il ne réussisse pas le suivant est égale à $0,6$.
\end{itemize}

Pour tout entier naturel $n$ non nul, on note $R_n$ l’évènement « le joueur réussit le $n$-ième service » et $\overline{R_n}$ l’évènement contraire.

\bigskip

\textbf{\underline{Partie A :}}

\medskip

On s’intéresse aux deux premiers services de l’entraînement.

\begin{enumerate}
	\item Représenter la situation par un arbre pondéré.
	\item Démontrer que la probabilité de l’événement $R_2$ est égale à $0,57$.
	\item Sachant que le joueur a réussi le deuxième service, calculer la probabilité qu’il ait raté le premier.
	\item Soit $Z$ la variable aléatoire égale au nombre de services réussis au cours des deux premiers services.
	\begin{enumerate}
		\item  Déterminer la loi de probabilité de $Z$ (on pourra utiliser l’arbre pondéré de la question 1).
		\item Calculer l’espérance mathématique $\Esper{Z}$ de la variable aléatoire $Z$.
		
		Interpréter ce résultat dans le contexte de l’exercice.
	\end{enumerate}
\end{enumerate}

\smallskip

\textbf{\underline{Partie B :}}

\medskip

On s’intéresse maintenant au cas général.

Pour tout entier naturel $n$ non nul, on note $x_n$ la probabilité de l’évènement $R_n$.

\begin{enumerate}
	\item 
	\begin{enumerate}
		\item Donner les probabilités conditionnelles $P_{R_n} \big(R_{n+1}\big)$ et $P_{\overline{R_n}} \big(\overline{R_{n+1}}\big)$.
		\item Montrer que, pour tout entier naturel non nul $n$, on a : $x_{n+1}=0,2x_n+0,4$.
	\end{enumerate}
	\item Soit la suite $\Suite{u}$ définie pour tout entier naturel $n$ non nul par $u_n = x_n - 0,5$.
	\begin{enumerate}
		\item Montrer que la suite $\Suite{u}$ est une suite géométrique.
		\item Déterminer l’expression de $x_n$ en fonction de $n$. En déduire la limite de la suite $\Suite{x}$.
		\item Interpréter cette limite dans le contexte de l’exercice.
	\end{enumerate}
\end{enumerate}