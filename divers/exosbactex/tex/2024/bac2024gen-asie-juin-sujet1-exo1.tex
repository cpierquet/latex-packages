\textbf{Partie A}

\medskip

On considère une fonction $f$ définie sur $\IntervalleFO{0}{+\infty}$ représentée par la courbe $\mathcal{C}$ ci-dessous.

La droite $T$ est tangente à la courbe $\mathcal{C}$ au point $A$ d’abscisse $\frac52$.

\begin{Centrage}
	\begin{GraphiqueTikz}[x=2cm,y=1cm,Xmin=0,Xmax=5.5,Xgrille=0.5,Ymin=-5,Ymax=5.5,Xgrilles=0.1,Ygrille=1,Ygrilles=0.2]
		\TracerAxesGrilles{0.5,1,...,5}{-5,-4,...,5}
		\TracerCourbe[Couleur=red]{(4*x-2)*exp(-x+1)}
		\TracerCourbe[Couleur=blue]{(-4*2.5+6)*exp(-2.5+1)*(x-2.5)+(4*2.5-2)*exp(-2.5+1)}
		\DefinirPts{A/{2.5}/{(4*2.5-2)*exp(-2.5+1)}}
		\MarquerPts[Couleur=darkgray,Style=x,Taillex=3pt]{(A)/A/above right}
		\PlacerTexte[Couleur=red,Police=\large]{(0.5,-2)}{$\mathcal{C}$}
		\PlacerTexte[Couleur=blue,Police=\large]{(0.5,4)}{$T$}
		%\draw[blue] (0.5,4) node[font=\large] {$T$} ;
	\end{GraphiqueTikz}
\end{Centrage}

\begin{enumerate}
	\item Dresser, par lecture graphique, le tableau des variations de la fonction $f$ sur l’intervalle $\IntervalleFF{0}{5}$.
	\item Que semble présenter la courbe $\mathcal{C}$ au point $A$ ?
	\item La dérivée $f'$ et la dérivée seconde $f''$ de la fonction $f$ sont représentées par les courbes ci-dessous.
	
	Associer à chacune de ces deux fonctions la courbe qui la représente.
	
	Ce choix sera justifié.
\end{enumerate}

\begin{Centrage}
	\begin{GraphiqueTikz}[x=1cm,y=0.2cm,Xmin=0,Xmax=5.5,Xgrille=0.5,Ymin=-26,Ymax=1,Xgrilles=0.5,Ygrille=2,Ygrilles=2]
		\TracerAxesGrilles[Police=\small]{0.5,1,...,5}{-26,-24,...,0}
		\TracerCourbe[Couleur=purple]{(4*x-10)*exp(-x+1)}
		\draw[purple] (0.5,-20) node {$\mathcal{C}_1$} ;
	\end{GraphiqueTikz}
	\hspace*{5mm}
	\begin{GraphiqueTikz}[x=1cm,y=0.3cm,Xmin=0,Xmax=5.5,Xgrille=0.5,Ymin=-2,Ymax=16,Xgrilles=0.5,Ygrille=2,Ygrilles=2]
		\TracerAxesGrilles[Police=\small]{0.5,1,...,5}{-2,0,...,14}
		\TracerCourbe[Couleur=teal]{(-4*x+6)*exp(-x+1)}
		\draw[teal] (0.5,12) node {$\mathcal{C}_2$} ;
	\end{GraphiqueTikz}
\end{Centrage}

\begin{enumerate}[resume]
	\item La courbe $\mathcal{C}_3$ ci-dessous peut-elle être la représentation graphique sur $\IntervalleFO{0}{+\infty}$ d’une primitive de la fonction $f$ ? Justifier.
\end{enumerate}

\begin{Centrage}
	\begin{GraphiqueTikz}[x=1cm,y=0.5cm,Xmin=0,Xmax=5.5,Xgrille=0.5,Ymin=0,Ymax=7.5,Xgrilles=0.5,Ygrille=1,Ygrilles=1]
		\TracerAxesGrilles[Police=\small]{0,0.5,...,5}{0,2,...,6}
		\TracerCourbe[Couleur=violet]{-2*(-2*x-1)*exp(-x+1)}
		\draw[violet] (2,5) node {$\mathcal{C}_3$} ;
	\end{GraphiqueTikz}
\end{Centrage}

\smallskip

\textbf{Partie B}

\medskip

Dans cette partie, on considère que la fonction $f$, définie et deux fois dérivable sur $\IntervalleFO{0}{+\infty}$, est définie par \[ f(x)=(4x-2)\e^{-x+1}. \]
%
On notera respectivement $f'$ et $f''$ la dérivée et la dérivée seconde de la fonction $f$.

\begin{enumerate}
	\item Étude de la fonction $f$
	\begin{enumerate}
		\item Montrer que $f'(x) = (-4x +6)\e^{-x+1}$.
		\item Utiliser ce résultat pour déterminer le tableau complet des variations de la fonction $f$ sur $\IntervalleFO{0}{+\infty}$. On admet que $\lim\limits_{x\to+\infty} f(x)=0$.
		\item Étudier la convexité de la fonction $f$ et préciser l’abscisse d’un éventuel point d’inflexion de la courbe représentative de $f$.
	\end{enumerate}
	\item On considère une fonction $F$ définie sur $\IntervalleFO{0}{+\infty}$ par $F(x) = (ax +b)\e^{-x+1}$, où $a$ et $b$ sont deux nombres réels.
	\begin{enumerate}
		\item Déterminer les valeurs des réels $a$ et $b$ telles que la fonction $F$ soit une primitive de la fonction $f$ sur $\IntervalleFO{0}{+\infty}$.
		\item On admet que $F(x) = (-4x - 2)\e^{-x+1}$ est une primitive de la fonction $f$ sur $\IntervalleFO{0}{+\infty}$.
		
		En déduire la valeur exacte, puis une valeur approchée à $10^{-2}$ près, de l’intégrale \[ I = \int_{\frac32}^8 f(x)\dx.\]
	\end{enumerate}
	\item Une municipalité a décidé de construire une piste de trottinette freestyle.
	
	Le profil de cette piste est donné par la courbe représentative de la fonction $f$ sur l’intervalle $\IntervalleFF{\frac32}{8}$.
	
	L’unité de longueur est le mètre.
	
	\begin{Centrage}
		\begin{GraphiqueTikz}[x=1cm,y=1.5cm,Origx=1.5,Xmin=1.5,Xmax=8,Xgrille=0.5,Ymin=0,Ymax=2.5,Xgrilles=0.5,Ygrille=2,Ygrilles=2]
			\tikzset{pflaxes/.style={line width=0.8pt}}
			\TracerAxesGrilles[Grille=false,Police=\small]{1.5,2,...,8}{}
			\TracerCourbe[Couleur=gray]{(4*x-2)*exp(-x+1)}
			\draw[gray] (1.5,{(4*1.5-2)*exp(-1.5+1)}) node[left] {$D$} ;
		\end{GraphiqueTikz}
	\end{Centrage}
	\begin{enumerate}
		\item Donner une valeur approchée au cm près de la hauteur du point de départ D.
		\item La municipalité a organisé un concours de graffiti pour orner le mur de profil de la piste. L’artiste retenue prévoit de couvrir environ 75\,\% de la surface du mur.
		
		Sachant qu’une bombe aérosol de 150~mL permet de couvrir une surface de
		$0,8$~m\up{2}, déterminer le nombre de bombes qu’elle devra utiliser pour réaliser cette œuvre.
	\end{enumerate}
\end{enumerate}