\emph{L’exercice est constitué de deux parties indépendantes.}

\medskip

\textbf{\large Partie I}

\medskip

Pour tout entier $n$ supérieur ou égal à 1, on désigne par $f_n$ la fonction définie sur $\IntervalleFF{0}{1}$ par : \[ f_n(x)=x^n\,\e^{x}. \]
%
On note $\Courbe[n]$ la courbe représentative de la fonction $f_n$ dans un repère \RepereOij{} du plan.

On désigne par $\Suite{I}$ la suite définie pour tout entier $n$ supérieur ou égal à 1 par : \[ I_n = \int_0^1 f_n(x)\dx. \]

\begin{enumerate}
	\item 
	\begin{enumerate}
		\item On désigne par $F_1$  la fonction définie sur $\IntervalleFF{0}{1}$ par : \[ F_1(x)=(x-1)\,\e^{x}.\]
		%
		Vérifier que $F_1$ est une primitive de la fonction $f_1$.
		\item Calculer $I_1$.
	\end{enumerate}
	\item À l’aide d’une intégration par parties, établir la relation pour tout $n$ supérieur ou égal à 1, \[ I_{n+1} = \e - (n+1)I_n. \]
	\item Calculer $I_2$
	\item On considère la fonction \piton{mystere} écrite dans le langage \textsf{Python} :
	
	\begin{CodePiton}[Alignement=center,Largeur=13cm,Gobble=tabs]{}
	from math import e # la constante d'Euler e
	
	def mystere(n):
		a = 1
		L = [a]
		for i in range(1, n) :
			a = e - (i+1)*a
			L.append(a)
		return L
	\end{CodePiton}
	
	À l’aide des questions précédentes, expliquer ce que renvoie l’appel \piton{mystere(5)}.
\end{enumerate}

\pagebreak

\textbf{\large Partie II}

\begin{enumerate}
	\item Sur le graphique ci-dessous, on a représenté les courbes $\Courbe[1]$, $\Courbe[2]$, $\Courbe[3]$, $\Courbe[10]$, $\Courbe[20]$ et $\Courbe[30]$.
	
	\begin{center}
		\begin{tikzpicture}[x=12cm,y=2.4cm,xmin=-0.1,xmax=1.1,xgrille=0.1,xgrilles=0.025,ymin=-0.5,ymax=3,ygrille=0.5,ygrilles=0.1]
			\GrilleTikz\AxesTikz[ElargirOx=0,ElargirOy=0]\OrigineTikz[Police=\small]
			\AxexTikz[Police=\small]{-0.1,0.1,0.2,0.3,0.4,0.5,0.6,0.7,0.8,0.9,1}
			\AxeyTikz[Police=\small]{-0.5,0.5,1,1.5,2,2.5}
			\FenetreTikz
			\foreach \k in {1,2,3,10,20,30}{%
				\CourbeTikz[thick,darkgray,samples=250]{\x^\k*exp(\x)}{0:1}
			}
			\foreach \k/\absCk in {1/0.16,2/0.4,3/0.56,10/0.8,20/0.88,30/0.98}{%
				\draw[darkgray] (\absCk,0.45) node {$\Courbe[\k]$} ;
			}
			%\CourbeTikz[thick,darkgray,samples=250]{\x^1*exp(\x)}{0:1}
			%\CourbeTikz[thick,darkgray,samples=250]{\x^1*exp(\x)}{0:1}
		\end{tikzpicture}
	\end{center}
	\begin{enumerate}
		\item Donner une interprétation graphique de $I_n$.
		\item Quelle conjecture peut-on émettre sur la limite de la suite $\Suite{I}$ ?
	\end{enumerate}
	\item Montrer que pour tout $n$ supérieur ou égal à 1, \[ 0 \leqslant I_n \leqslant \e\int_0^1 x^n \dx. \]
	\item En déduire $\displaystyle\lim\limits_{n \to +\infty} I_n$.
\end{enumerate}