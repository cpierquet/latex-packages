L’objectif de cet exercice est de déterminer la distance entre deux droites non coplanaires.

\smallskip

Par définition, la distance entre deux droites non coplanaires de l’espace, $(d_1)$ et $(d_2)$ est la longueur du segment $[EF]$, où $E$ et $F$ sont des points appartenant respectivement à $(d_1)$ et à $(d_2)$ tels que la droite $(EF)$ est orthogonale à $(d_1)$ et $(d_2)$.

\smallskip

L’espace est muni d’un repère orthonormé $\Rijk$.

\smallskip

Soit $(d_1)$ la droite passant par $A(1;2;-1)$ de vecteur directeur $\Vecteur*{u}[1] \CoordVecEsp{1}{2}{0}$ et $(d_2)$ la droite dont
une représentation paramétrique est : $\begin{dcases} x = 0 \\ y = 1+ t \\z = 2+ t\end{dcases}$, $t \in \R$.

\begin{enumerate}
	\item  Donner une représentation paramétrique de la droite $(d_1)$.
	\item Démontrer que les droites $(d_1)$ et $(d_2)$ sont non coplanaires.
	\item Soit $\mathcal{P}$ le plan passant par $A$ et dirigé par les vecteurs non colinéaires $\Vecteur*{u}[1]$ et $\Vecteur{w} \CoordVecEsp{2}{-1}{1}$.
	
	Justifier qu’une équation cartésienne du plan $\mathcal{P}$ est : $-2x + y +5z +5 = 0$.
	\item 
	\begin{enumerate}
		\item Sans chercher à calculer les coordonnées du point d’intersection, justifier que la droite $(d_2)$ et le plan $\mathcal{P}$ sont sécants.
		\item On note $F$ le point d’intersection de la droite $(d_2)$ et du plan $\mathcal{P}$.
		
		Vérifier que le point $F$ a pour coordonnées $\CoordPtEsp{0}{-\frac53}{-\frac23}$.
	\end{enumerate}
\end{enumerate}

Soit $(\delta)$ la droite passant par $F$ et de vecteur directeur $\Vecteur{w}$. On admet que les droites $(\delta)$ et $(d_1)$ sont sécantes en un point $E$ de coordonnées $\CoordPtEsp{-\frac23}{-\frac43}{-1}$.

\begin{enumerate}[resume]
	\item 
	\begin{enumerate}
		\item Justifier que la distance $EF$ est la distance entre les droites $(d_1)$ et $(d_2)$.
		\item Calculer la distance entre les droites $(d_1)$ et $(d_2)$.
	\end{enumerate}
\end{enumerate}