On considère le pavé droit $ABCDEFGH$ tel que $AB = 3$ et $AD = AE =1$ représenté ci-dessous.

\begin{Centrage}
	\begin{EnvTikzEspace}[UniteX={0:2.75cm},UniteY={65:1.75cm},UniteZ={90:4cm}]
		%placement des points avec labels
		\PlacePointsEspace{A/0,0,0/bg B/3,0,0/bd C/3,1,0/d D/0,1,0/g E/0,0,1/g F/3,0,1/d G/3,1,1/h H/0,1,1/h I/1,0,0/b M/1.5,1,0/h}
		%segments pointillés
		\TraceSegmentsEspace[very thick,dashed]{A/D D/C D/H}
		%segments pleins
		\TraceSegmentsEspace[very thick]{A/B B/C C/G G/H H/E E/A E/F B/F F/G}
		%Marques points
		\MarquePointsEspace[StyleMarque=x,TailleMarque=3pt]{I,M}
	\end{EnvTikzEspace}
\end{Centrage}

On considère le point $I$ du segment $[AB]$ tel que $\vect{AB}=3\vect{AI}$ et on appelle $M$ le milieu du segment $[CD]$.

\smallskip

On se place dans le repère orthonormé $\left(A;\vect{AI},\vect{AD},\vect{AE}\right)$.

\begin{enumerate}
	\item Sans justifier, donner les coordonnées des points $F$, $H$ et $M$.
	\item 
	\begin{enumerate}
		\item Montrer que le vecteur $\vect{n} \begin{pmatrix}2\\6\\3\end{pmatrix}$ est un vecteur normal au plan $(HMF)$.
		\item En déduire qu'une équation cartésienne du plan $(HMF)$ est : \[ 2x + 6y + 3z - 9 = 0. \]
		\item Le plan $(\mathcal{P})$ dont une équation cartésienne est $5x + 15y - 3z + 7 = 0$ est-il parallèle au plan $(HMF)$ ? Justifier la réponse.
	\end{enumerate}
	\item Déterminer une représentation paramétrique de la droite $(DG)$.
	\item On appelle $N$ le point d'intersection de la droite $(DG)$ avec le plan $(HMF)$.
	
	Déterminer les coordonnées du point $N$.
	\item Le point $R$ de coordonnées $\left(3;\frac14;\frac12\right)$ est-il le projeté orthogonal du point $G$ sur le plan $(HMF)$ ? Justifier la réponse.
\end{enumerate}