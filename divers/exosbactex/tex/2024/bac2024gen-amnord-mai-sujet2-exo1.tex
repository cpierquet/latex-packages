Les données publiées le 1\up{er} mars 2023 par le ministère de la transition écologique sur les immatriculations de véhicules particuliers en France en 2022 contiennent les informations suivantes :

\begin{itemize}
	\item 22,86\,\% des véhicules étaient des véhicules neufs ;
	\item 8,08\,\% des véhicules neufs étaient des hybrides rechargeables ;
	\item 1,27\,\% des véhicules d'occasion (c'est-à-dire qui ne sont pas neufs) étaient des hybrides rechargeables.
\end{itemize}

\textbf{\textit{Dans tout l'exercice, les probabilités seront arrondies au dix-millième.}}

\medskip

\textbf{\large Partie A}

\medskip

Dans cette partie, on considère un véhicule particulier immatriculé en France en 2022.

On note:

\begin{itemize}
	\item $N$ l'événement « le véhicule est neuf » ;
	\item $R$ l'événement « le véhicule est hybride rechargeable » ;
	\item $\overline{N}$ et $\overline{R}$ les événements contraires des événements contraires de $N$ et $R$.
\end{itemize}

\begin{enumerate}
	\item Représenter la situation par un arbre pondéré.
	\item Calculer la probabilité que ce véhicule soit neuf et hybride rechargeable.
	\item Démontrer que la valeur arrondie au dix-millième de la probabilité que ce véhicule soit hybride rechargeable est \num{0,0283}.
	\item Calculer la probabilité que ce véhicule soit neuf sachant qu'il est hybride rechargeable.
\end{enumerate}

\smallskip

\textbf{\large Partie B}

\medskip

Dans cette partie, on choisit 500 véhicules particuliers hybrides rechargeables immatriculés en France en 2022. Dans la suite, on admettra que la probabilité qu'un tel véhicule soit neuf est égale à 0,65.

On assimile le choix de ces 500 véhicules à un tirage aléatoire avec remise.

\smallskip

On appelle $X$ la variable aléatoire représentant le nombre de véhicules neufs parmi les 500 véhicules choisis.

\begin{enumerate}
	\item On admet que la variable aléatoire $X$ suit une loi binomiale. Préciser la valeur de ses paramètres.
	\item Déterminer la probabilité qu'exactement 325 de ces véhicules soient neufs.
	\item Déterminer la probabilité $p(X \geqslant 325)$ puis interpréter le résultat dans le contexte de l'exercice.
\end{enumerate}

\smallskip

\textbf{\large Partie C}

\medskip

On choisit désormais $n$ véhicules particuliers hybrides rechargeables immatriculés en France en 2022, où $n$ désigne un entier naturel strictement positif.

\smallskip

On rappelle que la probabilité qu'un tel véhicule soit neuf est égale à 0,65.

\smallskip

On assimile le choix de ces n véhicules à un tirage aléatoire avec remise.

\begin{enumerate}
	\item Donner l'expression en fonction de $n$ de la probabilité p, que tous ces véhicules soient d'occasion.
	\item On note $q_n$ la probabilité qu'au moins un de ces véhicules soit neuf. En résolvant une inéquation, déterminer la plus petite valeur de $n$ telle que $q_n \geqslant \num{0,9999}$.
\end{enumerate}