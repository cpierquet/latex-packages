Pour tout entier naturel $n$, on considère les intégrales suivantes : \[ I_n = \int_0^{\pi} \e^{-nx}\,\sin(x) \dx \] \[J_n = \int_0^{\pi} \e^{-nx}\,\cos(x) \dx \]

\begin{enumerate}
	\item Calculer $I_0$.
	\item 
	\begin{enumerate}
		\item Justifier que, pour tout entier naturel $n$, on $I_n \geqslant 0$.
		\item Montrer que, pour tout entier naturel $n$, on a $I_{n+1}-I_n \leqslant 0$.
		\item Déduire des deux questions précédentes que la suite $\suiten[I]$.
	\end{enumerate}
	\item 
	\begin{enumerate}
		\item Montrer que, pour tout entier naturel $n$, on a : \[ I_n \leqslant \int_0^{\pi} \e^{-nx} \dx. \]
		\item Montrer que, pour tout entier naturel $n \geqslant 1$, on a : \[ \int_0^{\pi} \e^{-nx} \dx = \frac{1-\e^{-n\pi}}{n}. \]
		\item Déduire des deux questions précédentes la limite de la suite $\suiten[I]$.
	\end{enumerate}
	\item 
	\begin{enumerate}
		\item En intégrant par parties l'intégrale $I_n$ de deux façons différentes, établir les deux relations suivantes, pour tout entier naturel $n \geqslant 1$ : \[ I_n = 1+\e^{-n\pi} - n\,J_n \qquad \text{et} \qquad I_n = \frac{1}{n} J_n. \]
		\item En déduire que, pour tout entier naturel $n \geqslant 1$, on a $I_n = \dfrac{1+\e^{-n\pi}}{n^2+1}$.
	\end{enumerate}
	\item On souhaite obtenir le rang \texttt{n} à partir duquel la suite $\suiten[I]$ devient inférieure à $0,1$.
	
	Recopier et compléter la cinquième ligne du script \texttt{Python} ci-dessous avec la commande appropriée.
	
\begin{CodePythonLstAlt}[Largeur=0.75\linewidth]{center}
from math import *
def seuil() :
	n = 0
	I = 2
	.........
		n = n+1
		I = (1+exp(-n*pi))/(n*n+1)
	return n
\end{CodePythonLstAlt}
\end{enumerate}