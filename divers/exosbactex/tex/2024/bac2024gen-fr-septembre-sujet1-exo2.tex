\textit{Les deux parties sont indépendantes.}

\begin{Centrage}
	\textbf{Partie A}
\end{Centrage}

Un artisan crée des bonbons au chocolat dont la forme rappelle le profil de la montage locale représentée en \textsf{\textbf{Figure 1}}. La base d'un tel bonbon est modélisée par la surface grisée, définie ci-dessous dans un repère orthonormé d'unité 2~cm (\textsf{\textbf{Figure 2}}).

\begin{Centrage}
	\begin{tikzpicture}[line join=bevel,x=2cm,y=2cm,transform shape]
		\def\xoffset{0.35}
		\path[thick,->,>=latex] (0,-0.15)--(0,1.5) ;
		\filldraw[very thin,fill=Chocolat!70] (-1,0) -- plot[samples=500,domain=-1:1] (\x,{\xintfloateval{(1-(\x)^2)*exp(\x)}}) -- (1,0) -- cycle ;
		\filldraw[very thin,fill=Chocolat!85] (0.35,{(1-(0.35)*(0.35))*exp(0.35)}) --++ (0.6,0.2) -- plot[samples=500,domain=0.95:1.6] (\x,{(1-(\x-0.6)*(\x-0.6))*exp(\x-0.6)+0.2}) -- plot[draw=none,samples=500,domain=1:0.35] (\x,{(1-(\x)*(\x))*exp(\x)}) -- cycle ;
		\draw[thick,<->,>=latex] ([shift={(-73.3:0.1)}]1,0)--++(0.6,0.2) node[midway,sloped,below,font=\small] {3~cm} ;
		%label
		\draw (0.3,-0.45) node[font=\large\sffamily\bfseries] {Figure 1} ;
	\end{tikzpicture}
	%
	\hspace{2cm}
	%
	\begin{tikzpicture}[line join=bevel,x=2cm,y=2cm]
		\draw[thin,lightgray,xstep=0.2,ystep=0.2] (-1.1,0) grid (1.2,1.5) ;
		\filldraw[fill=Chocolat!70,fill opacity=0.85] (-1,0) -- plot[samples=500,domain=-1:1] (\x,{\xintfloateval{(1-(\x)^2)*exp(\x)}}) -- (1,0) -- cycle ;
		\draw[thick,->,>=latex] (-1.1,0)--(1.2,0) ;
		\draw[thick,->,>=latex] (0,-0.15)--(0,1.5) ;
		\draw (0,0) node[below left] {$O$} (-1,0) node[below] {$-1$} (1,0) node[below] {$1$} (0,1) node[left] {$1$} ;
		\draw (0.75,0.926) node[above right] {$\mathcal{C}_f$} ;
		%label
		\draw (0.15,-0.45) node[font=\large\sffamily\bfseries] {Figure 2} ;
	\end{tikzpicture}
\end{Centrage}

Cette surface est délimitée par l'axe des abscisses et la représentation graphique notée $\mathcal{C}_f$ de la fonction $f$ définie sur $\IntervalleFF{-1}{1}$ par : \[ f(x)=\big(1-x^2\big)\e^x.\]
L'objectif de cette partie est de calculer le volume de chocolat nécessaire à la fabrication d'un bonbon au chocolat.

\begin{enumerate}
	\item 
	\begin{enumerate}
		\item Justifier que pour tout $x$ appartenant à l'intervalle $\IntervalleFF{-1}{1}$ on a $f(x) \geqslant 0$.
		\item Montrer à l'aide d'une intégration par parties que : \[ \int_{-1}^1 f(x) \dx = 2 \int_{-1}^1 x\,\e^{x} \dx. \]
	\end{enumerate}
	\item Le volume $\mathcal{V}$ de chocolat, en cm\up{3}, nécessaire à la fabrication d'un bonbon est donné par : \[ \mathcal{V} = 3 \times S \]%
	où $S$ est l'aire, en cm\up{2}, de la surface colorée (\textsf{\textbf{Figure 2}}).
	
	En déduire que ce volume $\mathcal{V}$, arrondi à $0,1$~cm\up{3} près, est égal à $4,4$~cm\up{3}.
\end{enumerate}

\begin{Centrage}
	\textbf{Partie B}
\end{Centrage}

On s'intéresse maintenant au bénéfice réalisé par l'artisan sur la vente de ces bonbons au chocolat en fonction du volume hebdomadaire des ventes.

\smallskip

Ce bénéfice peut être modélisé par la fonction $B$ définie sur l'intervalle $\IntervalleFO{0,01}{+\infty}$ par : \[ B(q)=8q^2(2-3\,\ln(q))-3. \]
%
Le bénéfice est exprimé en dizaines d'euros et la quantité $q$ en centaines de bonbons.

\medskip

On admet que la fonction $B$ est dérivable sur $\IntervalleFO{0,01}{+\infty}$. On note $B'$ sa fonction dérivée. 

\begin{enumerate}
	\item 
	\begin{enumerate}
		\item Déterminer $\Limite{B(q)}{q \to +\infty}$.
		\item Montrer que, pour tout $q \geqslant 0,01$, $B'(q)=8q(1-6\,\ln(q))$.
		\item Étudier le signe de $B'(q)$, et en déduire le sens de variation de $B$ sur $\IntervalleFO{0,01}{+\infty}$.
		
		Dresser le tableau de variation complet de la fonction $B$.
		\item Quel est le bénéfice maximal, à l'euro près, que peut espérer l'artisan ?
	\end{enumerate}
	\item 
	\begin{enumerate}
		\item Montrer que l'équation $B(q)= 10$ admet une unique solution $\beta$ sur l'intervalle $\IntervalleFO{1,2}{+\infty}$.
		
		Donner une valeur approchée de $\beta$ à $10^{-3}$ près.
		\item On admet que l'équation $B(q)= 10$ admet une unique solution $\alpha$ sur $\IntervalleFF{0,01}{1,2}$.
		
		On donne $\alpha \approx 0,757$.
		
		En déduire le nombre minimal et le nombre maximal de bonbons au chocolat à vendre pour réaliser un bénéfice supérieur à 100 euros. 
	\end{enumerate}
\end{enumerate}