%script python
\begin{scontents}[overwrite,write-out=frs2024j2exo3.py]
def seuil(S) :
	n = 0
	u = 7
	while u < S :
		n = n + 1
		u = 1.05 * u + 3
	return n
\end{scontents}

\textit{Pour chacune des affirmations suivantes, indiquer si elle est vraie ou fausse. Chaque réponse doit être justifiée. Une réponse non justifiée ne rapporte aucun point.}

\textit{Les cinq questions de cet exercice sont indépendantes.}

\bigskip

\begin{enumerate}
	\item On considère le script écrit en langage \textsf{Python} ci-dessous.
	
	\CodePythonLstFichierAlt*[8cm]{center}{frs2024j2exo3.py}
	
	\medskip

	\textbf{Affirmation 1} : l'instruction \texttt{seuil(100)} renvoie la valeur \texttt{18}.
	
	\bigskip
	\item Soit $\Suite{S}$ la suite définie pour tout entier naturel $n$ par $S_n = 1 + \frac{1}{5} + \frac{1}{5^2} + \cdots + \frac{1}{5^n}$.
	
	\smallskip
	
	\textbf{Affirmation 2} : la suite $\Suite{S}$ converge vers $\frac{5}{4}$.
	
	\bigskip
	\item \textbf{Affirmation 3} : dans une classe composée de 30 élèves, on peut former 870 binômes de délégués différents.
	
	\bigskip
	\item On considère la fonction $f$ définie sur $\IntervalleFO{1}{+\infty}$ par $f(x) = x \big(\ln(x)\big)^2$.
	
	\smallskip
	
	\textbf{Affirmation 4} : l'équation $f(x) = 1$ admet une solution unique dans l'intervalle $\IntervalleFO{1}{+\infty}$.
	
	\bigskip
	\item \textbf{Affirmation 5} : \[ \int_{0}^{1} x \e^{-x} \dx = \frac{\e - 2}{\e}.\]
\end{enumerate}