On considère la fonction $f$ définie sur l'intervalle $\IntervalleFF{0}{1}$ par $f(x)=2x\,\e^{-x}$.

On admet que la fonction $f$ est dérivable sur l'intervalle $\IntervalleFF{0}{1}$.

\begin{enumerate}
	\item 
	\begin{enumerate}
		\item Résoudre sur l'intervalle $\IntervalleFF{0}{1}$ l'équation $f(x)=x$.
		\item Démontrer que, pour tout $x$ appartenant à l'intervalle $\IntervalleFF{0}{1}$, $f'(x)=2(1-x)\e^{-x}$.
		\item Donner le tableau de variations de la fonction $f$ sur l'intervalle $\IntervalleFF{0}{1}$.
	\end{enumerate}
\end{enumerate}


On considère la suite $\Suite{u}$ définie par $u_0=0,1$ et pour tout entier naturel $n$, \[ u_{n+1}=f\big(u_n\big).\]

\begin{enumerate}[resume]
	\item 
	\begin{enumerate}
		\item Démontrer par récurrence que, pour tout $n$ entier naturel, $0 \leqslant u_n \leqslant u_{n+1} \leqslant 1$.
		\item En déduire que la suite $\Suite{u}$ est convergente.
	\end{enumerate}
	\item Démontrer que la limite de la suite $\Suite{u}$ est $\ln(2)$.
	\item 
	\begin{enumerate}
		\item Justifier que pour tout entier naturel $n$, $\ln(2)-u_n$ est positif.
		\item On souhaite écrire un script \textsf{Python} qui renvoie une valeur approchée de $\ln(2)$ par défaut à $10^{-4}$ près, ainsi que le nombre d'étapes pour y parvenir.
		
		Recopier et compléter le script ci-dessous afin qu'il réponde au problème posé.

\begin{CodePythonLstAlt}*[Largeur=0.5\linewidth]{center}
def seuil() :
	n = 0
	u = 0.1
	while ln(2) - u ... 0.0001 :
		n = n+1
		u = ...
	return (u, n)
\end{CodePythonLstAlt}
	\item Donner la valeur de la variable \texttt{n} renvoyée par la fonction \texttt{seuil()}.
	\end{enumerate}
\end{enumerate}