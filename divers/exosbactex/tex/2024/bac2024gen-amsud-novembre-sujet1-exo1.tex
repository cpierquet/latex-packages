%script python
\begin{scontents}[overwrite,write-out=as2024j1exo1.py]
from math import exp

def f(x) :
	return (20*x+8)*exp(-1/4*x)

def solution_equation() :
	a, b = 14, 15
	while b-a > 0.1 :
		m = (a+b) / 2
		if f(m) > 8 :
			a = m
		else :
			b = m
	return a, b
\end{scontents}
\textbf{PARTIE A}

\medskip

On considère l’équation différentielle \[ \big(E\big)\::\:y'+\frac14y=20\e^{-\frac14x}, \]
%
d’inconnue $y$, fonction définie et dérivable sur l’intervalle $\IntervalleFO{0}{+\infty}$.

\begin{enumerate}
	\item Déterminer la valeur du réel $a$ tel que la fonction $g$ définie sur l’intervalle $\IntervalleFO{0}{+\infty}$ par $g(x) = ax\e^{-\frac14x}$ soit une solution particulière de l’équation différentielle $\big(E\big)$.
	\item On considère l’équation différentielle \[ \big(E'\big)\::\:y'+\frac14y=0, \]
	%
	d’inconnue y, fonction définie et dérivable sur l’intervalle $\IntervalleFO{0}{+\infty}$.
	
	Déterminer les solutions de l’équation différentielle $\big(E'\big)$.
	\item En déduire les solutions de l’équation différentielle $\big(E\big)$.
	\item Déterminer la solution $f$ de l’équation différentielle $\big(E\big)$ telle que $f(0) = 8$.
\end{enumerate}

\smallskip

\textbf{PARTIE B}

\medskip

On considère la fonction $f$ définie sur l’intervalle $\IntervalleFO{0}{+\infty}$ par \[ f(x)=(20x+8)\e^{-\frac14x}. \]
%
On admet que la fonction $f$ est dérivable sur l’intervalle $\IntervalleFO{0}{+\infty}$ et on note $f'$ sa fonction dérivée sur l’intervalle $\IntervalleFO{0}{+\infty}$. De plus, on admet que $\lim\lim\limits_{x\to+\infty} f (x) = 0$.

\begin{enumerate}
	\item 
	\begin{enumerate}
		\item Justifier que, pour tout réel $x$ positif, \[ f'(x)= (18x-5)\e^{-\frac14x}. \]
		\item En déduire le tableau de variations de la fonction $f$. On précisera la valeur exacte du maximum de la fonction $f$ sur l’intervalle $\IntervalleFO{0}{+\infty}$.
	\end{enumerate}
	\item Dans cette question on s’intéresse à l’équation $f (x) = 8$.
	\begin{enumerate}
		\item Justifier que l’équation $f (x) = 8$ admet une unique solution, notée $\alpha$, dans l’intervalle $\IntervalleFF{14}{15}$.
		\item Recopier et compléter le tableau ci-dessous en faisant tourner étape par étape la fonction, écrite en langage \textsf{Python}, \texttt{solution\_equation} ci-contre.
		
		\begin{minipage}{0.45\linewidth}
			\begin{tblr}{width=0.975\linewidth,hlines,vlines,colspec={Q[c,m]c*{4}{X[m,c]}}, cell{Y-Z}{Z}={gray}}
				$a$		&14		&&&&\\
				$b$		&15		&&&&\\
				$b-a$	&1 		&&&&\\
				$m$		&$14,5$	&&&&\\
				{Condition\\$f(m)<8$}&FAUX&&&&\\
			\end{tblr}
		\end{minipage}
		\begin{minipage}{0.55\linewidth}
		\CodePythonLstFichierAlt*[0.75\linewidth]{center}{as2024j1exo1.py}
		\end{minipage}
		
		\medskip
		\item Quel est l’objectif de la fonction \texttt{solution\_equation} dans le contexte de la question ?
	\end{enumerate}
\end{enumerate}