\textit{Pour chacune dos affirmations suivantes, indiquer si elle est vraie ou fausse. Chaque réponse doit étre justifiée. Une réponse non justifiée ne rapporte aucun point.\\
Les quatre questions de cet exercice sont indépendantes.}

\medskip

Dans l'espace rapporté à un repère orthonormé $\Rijk$, on considère les points $A(0;4;-1)$, $B(6;1;5)$ et $C(6;-2;-1)$. On admet que les points $A$, $B$ et $C$ ne sont pas alignés.

\begin{description}
	\item[\textbf{Affirmation 1} :] Le vecteur $\Vecteur{n}\begin{pmatrix}2\\2\\-1\end{pmatrix}$ est un vecteur normal au plan $(ABC)$.
	\item[\textbf{Affirmation 2} :] Une représentation paramétrique de la droite $(AB)$ est $\begin{dcases}x=2+2t\\y=3-t\\z=1+2t\end{dcases}$ où $t \in \R$.
	\item[\textbf{Affirmation 3} :] Une équation cartésienne du plan $\mathcal{P}$ passant par le point $C$ et orthogonal à la droite $(AB)$ est $2x+2y-z-9=0$.
\end{description}

On considère les droites $\mathcal{D}$ et $\mathcal{D}'$ dont on donne une représentation paramétrique : %
\[ \mathcal{D}~:~\begin{cases}x=3+t\\y=1+t\\z=2+t\end{cases}\text{ où } t \in \R \quad ; \quad \mathcal{D}'~:~\begin{cases}x=2t'\\y=4-t'\\z=-1+2t'\end{cases}\text{ où } t' \in \R. \]

\begin{description}
	\item[\textbf{Affirmation 4} :] $\mathcal{D}$ et $\mathcal{D}'$ ne sont pas coplanaires.
\end{description}