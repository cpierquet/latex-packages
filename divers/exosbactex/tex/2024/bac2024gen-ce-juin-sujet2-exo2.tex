On considère la fonction $f$ définie sur l’intervalle $\IntervalleOO{-\infty}{1}$ par $f(x)=\dfrac{\e^x}{x-1}$.

On admet que la fonction $f$ est dérivable sur l’intervalle $\IntervalleOO{-\infty}{1}$.

On appelle $\mathcal{C}$ sa courbe représentative dans un repère.

\begin{enumerate}%$\IntervalleOO{-\infty}{1}$
	\item 
	\begin{enumerate}
		\item Déterminer la limite de la fonction $f$ en 1.
		\item En déduire une interprétation graphique.
	\end{enumerate}
	\item Déterminer la limite de la fonction $f$ en $-\infty$.
	\item 
	\begin{enumerate}
		\item Montrer que pour tout réel $x$ de l’intervalle $\IntervalleOO{-\infty}{1}$, on a $f'(x)=\dfrac{(x-2)\e^x}{(x-1)^2}$.
		\item Dresser, en justifiant, le tableau de variations de la fonction $f$ sur
		l’intervalle $\IntervalleOO{-\infty}{1}$.
	\end{enumerate}
	\item On admet que pour tout réel $x$ de l’intervalle $\IntervalleOO{-\infty}{1}$, on a $f''(x)=\dfrac{\big(x^2-4x+5\big)\e^x}{(x-1)^3}$.
	\begin{enumerate}
		\item Étudier la convexité de la fonction $f$ sur l’intervalle $\IntervalleOO{-\infty}{1}$.
		\item Déterminer l’équation réduite de la tangente $T$ à la courbe $\mathcal{C}$ au point d’abscisse 0.
		\item En déduire que, pour tout réel $x$ de l’intervalle $\IntervalleOO{-\infty}{1}$ on a :
		\begin{Centrage}
			$\e^x \geqslant (-2x-1)(x-1).$
		\end{Centrage}
	\end{enumerate}
	\item 
	\begin{enumerate}
		\item Justifier que l’équation $f(x)=-2$ admet une unique solution $\alpha$ sur l’intervalle $\IntervalleOO{-\infty}{1}$.
		\item À l’aide de la calculatrice, déterminer un encadrement de $\alpha$ d’amplitude $10^{-2}$.
	\end{enumerate}
\end{enumerate}