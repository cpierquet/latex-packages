Une concession automobile vend deux sortes de véhicules :

\begin{itemize}
	\item 60\,\% sont des véhicules tout-électrique ;
	\item 40\,\% sont des véhicules hybrides rechargeables.
\end{itemize}

75\,\% des acheteurs de véhicules tout-électrique et 52\,\% des acheteurs de véhicules hybrides ont la possibilité matérielle d’installer une borne de recharge à domicile.

\smallskip

On choisit un acheteur au hasard et on considère les événements suivants :

\begin{itemize}
	\item E : « l’acheteur choisit un véhicule tout-électrique » ;
	\item B : « l’acheteur a la possibilité d’installer une borne de recharge à son domicile ».
\end{itemize}

\textit{Dans l’ensemble de l’exercice, les probabilités seront arrondies au millième si nécessaire.}

\begin{enumerate}
	\item Calculer la probabilité que l’acheteur choisisse un véhicule tout-électrique et qu’il ait la possibilité d’installer une borne de recharge à son domicile.
	
	\textit{On pourra s’appuyer sur un arbre pondéré.}
	\item Démontrer que $P(B) = 0,658$.
	\item Un acheteur a la possibilité d’installer une borne de recharge à son domicile. Quelle est la probabilité qu'il choisisse un véhicule tout-électrique ?
	\item On choisit un échantillon de 20 acheteurs. On assimile ce prélèvement à un tirage avec remise.
	
	On note $X$ la variable aléatoire qui donne le nombre total d’acheteurs pouvant installer une borne de recharge à leur domicile parmi l’échantillon de 20 acheteurs.
	\begin{enumerate}
		\item Déterminer la nature et les paramètres de la loi de probabilité suivie par $X$.
		\item Calculer $P(X = 8)$.
		\item Calculer la probabilité qu’au moins 10 acheteurs puissent installer une borne de recharge.
		\item Calculer l’espérance de $X$.
		\item La directrice de la concession décide d’offrir l’installation de la borne de recharge aux acheteurs ayant la possibilité d’en installer une à leur domicile. Cette installation coûte \num{1200}\,€.
		
		En moyenne, quelle somme doit-elle prévoir d’engager pour cette offre lors de la vente de 20 véhicules ?
	\end{enumerate}
\end{enumerate}