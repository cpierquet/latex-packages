L'espace est muni d'un repère orthonormé $\Rijk$.

On considère :

\begin{itemize}
	\item les points $A(-2;0;2)$, $B(-1;3;0)$, $C(1;-1;2)$ et $D(0;0;3)$ ;
	\item la droite $\mathcal{D}_1$ dont une représentation paramétrique est $\begin{dcases}x=t\\y=3t\\z=3+5t\end{dcases}$ avec $t \in \R$ ;
	\item la droite $\mathcal{D}_2$ dont une représentation paramétrique est $\begin{dcases}x=1+3s\\y=-1-5s\\z=2-6s\end{dcases}$ avec $s \in \R$.
\end{itemize}

\begin{enumerate}
	\item Démontrer que les points $A$, $B$ et $C$ ne sont pas alignés.
	\item 
	\begin{enumerate}
		\item Démontrer que le vecteur $\Vecteur{n}\begin{pmatrix}1\\3\\5\end{pmatrix}$ est orthogonal au plan $(ABC)$.
		\item Justifier qu'une équation cartésienne du plan $(ABC)$ est : \[ x+3y+5z-8=0. \]
		\item En déduire que les points $A$, $B$, $C$ et $D$ ne sont pas coplanaires.
	\end{enumerate}
	\item 
	\begin{enumerate}
		\item Justifier que la droite $\mathcal{D}_1$ est la hauteur du tétraèdre $ABCD$ issue de $D$.
	\end{enumerate}
	On admet que la droite $\mathcal{D}_2$ est la hauteur du tétraèdre $ABCD$ issue de $C$.
	\begin{enumerate}[resume]
		\item Démontrer que les droites $\mathcal{D}_1$ et $\mathcal{D}_2$ sont sécantes et déterminer les coordonnées de leur point d'intersection.
	\end{enumerate}
	\item 
	\begin{enumerate}
		\item Déterminer les coordonnées du projeté orthogonal $H$ du point $D$ sur le plan $(ABC)$.
		\item Calculer la distance du point $D$ au plan $(ABC)$. \textit{Arrondir le résultat au centième.}
	\end{enumerate}
\end{enumerate}