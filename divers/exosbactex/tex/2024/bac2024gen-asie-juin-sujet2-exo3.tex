Pour chacune des affirmations suivantes, indiquer si elle est vraie ou fausse.

Chaque réponse doit être justifiée. Une réponse non justifiée ne rapporte aucun point.

\medskip

\begin{enumerate}
	\item Soit $\Suite{u}$ une suite définie pour tout entier naturel $n$ et vérifiant la relation suivante : \[\text{pour tout entier naturel }n,\quad \frac12 \leqslant u_n \leqslant \frac{3n^2+4n+7}{6n^2+1}. \]
	\underline{Affirmation 1 :} $\lim\limits_{n \to +\infty} = \frac12$.
	\item Soit une fonction définie et dérivable sur l'intervalle $\IntervalleFF{-4}{4}$.
	
	La représentation graphique $\mathcal{C}_{h'}$ de sa fonction dérivée $h'$est donnée ci-dessous.
	
	\begin{Centrage}
		\begin{GraphiqueTikz}[x=1.8cm,y=0.495cm,Xmin=-4.5,Xmax=4.5,Xgrille=0.5,Xgrilles=0.1,Ymin=-6,Ymax=10.25,Ygrille=2,Ygrilles=0.5]
			\tikzset{pflgrilles/.style={very thin,lightgray!25}}
			\TracerAxesGrilles*[Police=\small]{auto}{auto}
			\TracerCourbe[Couleur=red,Debut=-4,Fin=4]{-0.00000890077723837785*x^9 - 0.00298469080256438*x^8 + 0.000100023513778871*x^7 + 0.0848399565601911*x^6 + 0.0297442435448854*x^5 - 0.652507018715297*x^4 - 0.822087746651839*x^3 + 0.562572122827126*x^2 + 5.30832018398037*x + 4.52414743374050}
			\draw[red] (-3,2) node[font=\Large] {$\mathcal{C}_{h'}$} ;
		\end{GraphiqueTikz}
	\end{Centrage}
	
	\underline{Affirmation 2 :} La fonction $h$ est convexe sur $\IntervalleFF{-1}{3}$.
	\item Le code d'un immeuble est composé de 4 chiffres (qui peuvent être identiques) suivis de deux lettres distinctes parmi A, B et C (exemple : 1232BA).
	
	\smallskip
	
	\underline{Affirmation 3 :} Il existe \num{20634} codes qui contiennent au moins un 0.
	\item On considère la fonction $f$ définie sur $\IntervalleOO{0}{+\infty}$ par $f(x)=x\,\ln(x)$.
	
	\smallskip
	
	\underline{Affirmation 4 :} La fonction $f$ est une solution sur $\IntervalleOO{0}{+\infty}$ de l'équation différentielle \[ x\,y' -y = x. \]
\end{enumerate}