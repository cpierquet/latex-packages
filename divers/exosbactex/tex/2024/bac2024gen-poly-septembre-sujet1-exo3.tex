On considère une pyramide à base carrée formée de boules identiques empilées les unes sur les autres :

\begin{wrapstuff}[r,abovesep=-0.25\baselineskip]
%https://tex.stackexchange.com/questions/648153/how-to-create-triangular-pyramid-of-oranges-using-tikz/648243#648243
\tdplotsetmaincoords{67.5}{12.5}
\begin{tikzpicture}[scale=0.4,tdplot_main_coords]
	\tikzstyle{ballstack} = [ball color=teal!50]
	\shade[ballstack] (-2.598, 3/2, 0) circle (0.866 cm);
	\shade[ballstack] (-0.866, 3/2, 0) circle (0.866 cm);
	\shade[ballstack] (0.866, 3/2, 0) circle (0.866 cm);
	\shade[ballstack] (2.598, 3/2, 0) circle (0.866 cm); 
	\shade[ballstack] (-1.732, 0, 0) circle (0.866 cm);
	\shade[ballstack] (0, 0, 0) circle (0.866 cm);
	\shade[ballstack] (1.732, 0, 0) circle (0.866 cm);
	\shade[ballstack] (-0.866, -3/2, 0) circle (0.866 cm);
	\shade[ballstack] (0.866, -3/2, 0) circle (0.866 cm);
	\shade[ballstack] (0, -3, 0) circle (0.866 cm);
	\shade[ballstack] (0, 1, 1.414) circle (0.866 cm);
	\shade[ballstack] (-1.732, 1, 1.414) circle (0.866 cm);
	\shade[ballstack] (1.732, 1, 1.414) circle (0.866 cm);
	\shade[ballstack] (-0.866, -1/2, 1.414) circle (0.866 cm);
	\shade[ballstack] (0.866, -1/2, 1.414) circle (0.866 cm);
	\shade[ballstack] (0, -2, 1.414) circle (0.866 cm);
	\shade[ballstack] (-0.866, 1/2, 2.828) circle (0.866 cm);
	\shade[ballstack] (0.866, 1/2, 2.828) circle (0.866 cm);
	\shade[ballstack] (0, -1, 2.828) circle (0.866 cm);
	\shade[ballstack] (0, 0, 4.242) circle (0.866 cm);
\end{tikzpicture}
\end{wrapstuff}

\begin{itemize}
	\item le 1\up{er} étage, situé au niveau le plus haut, est composé de 1 boule ;
	\item le 2\up{e} étage, niveau juste en dessous, est composé de 4 boules ;
	\item le 3\up{e} étage possède 9 boules ;
	\item \ldots
	\item le $n$-ième étage possède $n^2$ boules.
\end{itemize}

Pour tout entier $n \geqslant 1$, on note $u_n$ le nombre de boules qui composent le $n$-ième étage en partant du haut de la pyramide. Ainsi, $u_n = n^2$.

\begin{enumerate}
	\item Calculer le nombre total de boules d’une pyramide de 4 étages.
	\item On considère la suite $\Suite{S}$ définie pour tout entier $n \geqslant 1$ par $S_n = u_1+u_2+\ldots+u_n$.
	\begin{enumerate}
		\item Calculer $S_5$ et interpréter ce résultat.
		\item On considère la fonction \texttt{pyramide} ci-dessous écrite de manière incomplète en langage \textsf{Python}. Recopier et compléter sur la copie le cadre ci-dessous de sorte que, pour tout entier naturel non nul \texttt{n}, l’instruction \texttt{pyramide(n)} renvoie le nombre de boules composant une pyramide de \texttt{n} étages.
		
\begin{CodePythonLstAlt}*[Largeur=9cm]{center}
def pyramide(n) :
	S = 0
	for i in range(1, n+1) :
		S = ...
	return ...
\end{CodePythonLstAlt}
		\item Vérifier que pour tout entier naturel $n$ : \[ \frac{n(n+1)(2n+1)}{6}+(n+1)^2 = \frac{(n+1)(n+2)[2(n+1)+1]}{6}. \]
		\item Démontrer par récurrence que pour tout entier $n \geqslant 1$ : \[ S_n = \frac{n(n+1)(2n+1)}{6}. \]
	\end{enumerate}
	\item Un marchand souhaite disposer des oranges en pyramide à base carrée. Il possède 200 oranges. Combien d’oranges utilise-t-il pour construire la plus grande pyramide possible ?
\end{enumerate}