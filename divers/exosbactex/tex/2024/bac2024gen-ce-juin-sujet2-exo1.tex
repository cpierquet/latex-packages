Un sac opaque contient huit jetons numérotés de 1 à 8, indiscernables au toucher.

À trois reprises, un joueur pioche un jeton dans ce sac, note son numéro, puis le remet dans le sac.

Dans ce contexte, on appelle « tirage » la liste ordonnée des trois numéros obtenus.

Par exemple, si le joueur pioche le jeton numéro 4, puis le jeton numéro 5, puis le jeton numéro 1, alors le tirage correspondant est $(4;5;1)$.

\begin{enumerate}
	\item Déterminer le nombre de tirages possibles.
	\item 
	\begin{enumerate}
		\item Déterminer le nombre de tirages sans répétition de numéro.
		\item En déduire le nombre de tirages contenant au moins une répétition de numéro.
	\end{enumerate}
\end{enumerate}

On note $X_1$ la variable aléatoire égale au numéro du premier jeton pioché, $X_2$ celle égale au numéro du deuxième jeton pioché et $X_3$ celle égale au numéro du troisième jeton pioché.

Puisqu’il s’agit d’un tirage avec remise, les variables aléatoires $X_1$, $X_2$ et $X_3$ sont indépendantes et suivent la même loi de probabilité.

\begin{enumerate}[resume]
	\item Établir la loi de probabilité de la variable aléatoire $X_1$.
	\item Déterminer l’espérance de la variable aléatoire $X_1$.
\end{enumerate}

On note $S = X_1 + X_2 + X_3$ la variable aléatoire égale à la somme des numéros des trois
jetons piochés.

\begin{enumerate}[resume]
	\item Déterminer l’espérance de la variable aléatoire $S$.
	\item Déterminer $P(S=24)$.
	\item Si un joueur obtient une somme supérieure ou égale à 22, alors il gagne un lot.
	\begin{enumerate}
		\item Justifier qu’il existe exactement 10 tirages permettant de gagner un lot.
		\item En déduire la probabilité de gagner un lot.
	\end{enumerate}
\end{enumerate}