Pour chacune des affirmations suivantes, préciser si elle est vraie ou fausse puis \textbf{justifier} la réponse donnée.

Toute réponse non argumentée ne sera pas prise en compte.

\begin{enumerate}
	\item \textbf{Affirmation 1} : Toute suite décroissante et minorée par 0 converge vers 0.
	\item On considère une suite $\Suite{u}$ définie sur $\N$ telle que, pour tout entier $n$, on a $u_n \leqslant \dfrac{-9^n+3^n}{7^n}$.
	
	\textbf{Affirmation 2} : $\lim\limits_{n \to +\infty} u_n = -\infty$.
	\item On considère la fonction suivante écrite en langage \textsf{Python} :
	
\begin{CodePythonLstAlt}[Largeur=10cm]{center}
def terme(N) :
	U = 1
	for i in range(N) :
		U = U+i
	return U
\end{CodePythonLstAlt}

	\textbf{Affirmation 3} : \texttt{terme(4)} renvoie la valeur \texttt{7}.
	\item Lors d’un concours, le gagnant a le choix entre deux prix :
	
	\begin{itemize}
		\item Prix A : il reçoit \num{1000} euros par jour pendant 15 jours ;
		\item Prix B : il reçoit 1 euro le 1\up{er} jour, 2 euros le 2\up{e} jour, 4 euros le 3\up{e} jour et pendant 15 jours la somme reçue double chaque jour.
	\end{itemize}
	
	\textbf{Affirmation 4} : La valeur du prix A est plus élevée que la valeur du prix B.
	\item On considère la suite $\Suite{v}$ définie pour tout entier $n \geqslant 1$ par \[ v_n = \int_{1}^{n} \ln(x)\dx. \]
	\textbf{Affirmation 5} : La suite $\Suite{v}$ est croissante.
\end{enumerate}