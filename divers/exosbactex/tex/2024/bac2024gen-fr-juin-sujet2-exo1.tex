\textit{Les parties A et B sont indépendantes.}

\smallskip

\begin{Centrage}
	\textbf{Partie A}
\end{Centrage}

Une société de vente en ligne procède à une étude du niveau de fidélité de ses clients. Elle définit pour cela comme \og régulier \fg\ un client qui a fait des achats chaque année depuis trois ans. Elle constate que 60\,\% de ses clients sont des clients réguliers, et que parmi eux, 47\,\% ont acheté la carte de fidélité.

Par ailleurs, parmi l'ensemble de tous les clients de la société, 38\,\% ont acheté la carte de fidélité.

On interroge au hasard un client et on considère les évènements suivants :

\begin{itemize}
	\item R : \og le client est un client régulier \fg\ ;
	\item F : \og le client a acheté la carte de fidélité \fg.
\end{itemize}

Pour un évènement $E$ quelconque, on note $\overline{E}$ sont évènement contraire et $P(E)$ sa probabilité.

\begin{wrapstuff}[r]
\ArbreProbasTikz[PositionProbas=auto,EspaceNiveau=2]{$R$/\numdots/,$F$/\numdots/,$\overline{F}$/\numdots/,$\overline{R}$/\numdots/,$F$/\numdots/,$\overline{F}$/\numdots/}
\end{wrapstuff}

\begin{enumerate}
	\item 
	
	\begin{enumerate}
		\item Reproduire l'arbre ci-contre et compléter les pointillés.
		\item Calculer la probabilité que le client interrogé soit un client régulier et qu'il ait acheté la carte de fidélité.
		\item Déterminer la probabilité que le client ait acheté la carte de fidélité sachant que ce n'est pas un client régulier.
		\item Le directeur du service des ventes affirme que parmi les clients qui ont acheté la carte de fidélité, plus de 80\,\% sont des clients réguliers. Cette affirmation est-elle exacte ?
	\end{enumerate}
\end{enumerate}

\begin{Centrage}
	\textbf{Partie B}
\end{Centrage}

La société demande à un institut de sondage de faire une enquête sur le profil de ses clients réguliers. L'institut a élaboré un questionnaire en ligne constitué d'un nombre variable de questions.

On choisit au hasard un échantillon de \num{1000} clients réguliers, à qui le questionnaire est proposé. On considère que ces \num{1000} clients répondent.

\begin{itemize}
	\item Pour les remercier, la société offre un bon d'achat à chacun des clients de l'échantillon.
	
	Le montant de ce bon d'achat dépend du nombre de questions posées au client.
	\item La société souhaite récompenser particulièrement les clients de l'échantillon qui ont acheté une carte de fidélité et, en plus du bon d'achat, offre à chacun d'eux une prime d'un montant de 50 euros versée sur la carte de fidélité.
\end{itemize}

On note $Y_1$ la variable aléatoire qui, à chaque échantillon de \num{1000} clients réguliers, associe le total, en euros, des montants du bon d'achat des \num{1000} clients. On admet que son espérance $E\big(Y_1\big)$ est égale à \num{30000} et que sa variance $V\big(Y_1\big)$ est égale à \num{10000}.

\smallskip

On note $X_2$ la variable aléatoire qui, à chaque échantillon de \num{1000} clients réguliers, associe le nombre de clients ayant acheté la carte de fidélité parmi eux, et on note $Y_2$ la variable aléatoire qui, à chaque échantillon de \num{1000} clients, associe le total, en euros, des montants de la prime de fidélité versée.

On admet que $X_2$ suit la loi binomiale de paramètres \num{1000} et $0,47$ et que $Y_2 = 50X_2$.

\begin{enumerate}
	\item Calculer l'espérance $E\big(X_2\big)$ de la variable $X_2$ et interpréter le résultat dans le contexte de l'exercice.
\end{enumerate}

On note $Y=Y_1+Y_2$ la variable aléatoire égale au total général, en euros, des montants offerts (bon d'achat et prime de fidélité) aux \num{1000} clients. On admet que les variables aléatoires $Y_1$ et $Y_2$ sont indépendantes.

On note $Z$ la variable aléatoire définie par $Z=\frac{Y}{\num{1000}}$.

\begin{enumerate}[resume]
	\item Préciser ce que modélise la variable aléatoire $Z$ dans le contexte de l'exercice. Vérifier que son espérance $E(Z)$ est égale à $53,5$ et que sa variance $V(Z)$ est égale à \num{0.72275}.
	\item À l'aide de l'inégalité de Bienaymé-Tchebychev, vérifier que la probabilité que $Z$ soit strictement compris entre $51,7$ euros et $55,3$ euros est supérieure à $0,75$.
\end{enumerate}