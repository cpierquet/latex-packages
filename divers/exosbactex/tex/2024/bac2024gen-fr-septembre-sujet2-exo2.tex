\textit{La partie \textbf{C} est indépendante des parties \textbf{A} et \textbf{B}.} 

\medskip

Un robot est positionné sur un axe horizontal et se déplace plusieurs fois d'un mètre sur cet axe, aléatoirement vers la droite ou vers la gauche.

Lors du premier déplacement, la probabilité que le robot se déplace à droite est égale à $\frac{1}{3}$.

S'il se déplace à droite, la probabilité que le robot se déplace de nouveau à droite lors du déplacement suivant est égale à $\frac{3}{4}$.

S'il se déplace à gauche, la probabilité que le robot se déplace de nouveau à gauche lors du déplacement suivant est égale à $\frac{1}{2}$.

\smallskip

Pour tout entier naturel $n \geqslant 1$, on note :

\begin{itemize}
	\item $D_n$ l'événement : « le robot se déplace à droite lors du $n$-ième déplacement » ;
	\item $\overline{D_n}$ l'évènement contraire de $D_n$ ;
	\item $p_n$ la probabilité de l'événement $D_n$.
\end{itemize}

On a donc $p_1 = \frac{1}{3}$.

\medskip

\begin{Centrage}
	\textbf{Partie A : étude du cas particulier où} $\bm{n=2}$\textbf{.}
\end{Centrage}

Dans cette partie, le robot réalise deux déplacements successifs.

\begin{enumerate}
	\item Reproduire et compléter l'arbre pondéré suivant :
	
	\begin{Centrage}
		\ArbreProbasTikz[PositionProbas=auto]{$D_1$/\numdots/,$D_2$/\numdots/,$\overline{D_2}$/\numdots/,$\overline{D_1}$/\numdots/,$D_2$/\numdots/,$\overline{D_2}$/\numdots/}
	\end{Centrage}
	\item Déterminer la probabilité que le robot se déplace deux fois à droite.
	\item Montrer que $p_2 = \frac{7}{12}$.
	\item Le robot s'est déplacé à gauche lors du deuxième déplacement. Quelle est la probabilité qu'il se soit déplacé à droite lors du premier déplacement ?
\end{enumerate}

\begin{Centrage}
	\textbf{Partie B : étude de la suite} $\bm{\left(p_n\right)}$\textbf{.}
\end{Centrage}

On souhaite estimer le déplacement du robot au bout d'un nombre important d'étapes.

\begin{enumerate}
	\item Démontrer que pour tout entier naturel $n \geqslant 1$, on a : \[ p_{n+1} = \frac{1}{4} p_n + \frac{1}{2}. \]
	
	On pourra s'aider d'un arbre.
	\item 
	\begin{enumerate}
		\item Montrer par récurrence que pour tout entier naturel $n \geqslant 1$, on a : $p_n \leqslant p_{n+1} < \frac{2}{3}$.
		\item La suite $\Suite{p}$ est-elle convergente ? Justifier.
	\end{enumerate}
	\item On considère la suite $\Suite{u}$ définie pour tout entier naturel $n \geq 1$, par $u_n = p_n - \frac{2}{3}$.
	\begin{enumerate}
		\item Montrer que la suite $\Suite{u}$ est géométrique et préciser son premier terme et sa raison.
		\item Déterminer la limite de la suite $\Suite{p}$ et interpréter le résultat dans le contexte de l'exercice.
	\end{enumerate}
\end{enumerate}

\begin{Centrage}
	\textbf{Partie C}
\end{Centrage}

Dans cette partie, on considère un autre robot qui réalise dix déplacements d'un mètre indépendants les uns des autres, chaque déplacement vers la droite ayant une probabilité fixe égale à $\frac{3}{4}$.

Quelle est la probabilité qu'il revienne à son point de départ au bout des dix déplacements ? On arrondira le résultat à $10^{-3}$ près.