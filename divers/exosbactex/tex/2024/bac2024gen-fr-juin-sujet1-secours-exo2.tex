\begin{Centrage}
	\textbf{Partie A}
\end{Centrage}

Suite à une étude statistique réalisée dans la station-service Carbuplus, on évalue à $0,25$ la probabilité qu'un client venant alimenter son véhicule en carburant passe moins de 12 minutes dans la station avant de la quitter.

On choisit au hasard et de façon indépendante 10 clients de la station et on assimile ce choix à un tirage avec remise. On appelle $X$ la variable aléatoire qui à chaque échantillon de 10 clients associe le nombre de ces clients ayant passé moins de 12 minutes à la station.

\begin{enumerate}
	\item Quelle est la loi de probabilité suivie par la variable aléatoire $X$ ? Préciser ses paramètres.
	\item Quelle est la probabilité qu'au moins 4 clients dans un échantillon de 10 passent moins de 12 minutes à la station ? On arrondira si besoin le résultat à $10^{-3}$ près.
	\item Calculer l'espérance $E(X)$ et interpréter le résultat dans le contexte de l'exercice.
\end{enumerate}

\begin{Centrage}
	\textbf{Partie B}
\end{Centrage}

Un client arrive à la station et se dirige vers une pompe. Il constate que deux voitures sont devant lui, la première accédant à la pompe au moment de son arrivée.

On désigne par $T_1$, $T_2$, $T_3$ les variables aléatoires qui modélisent les temps passés en minute par chacun des trois clients, dans leur ordre d'arrivée, pour alimenter son véhicule entre l'instant où la pompe est disponible pour lui et celui où il la libère.

\smallskip

On suppose que $T_1$, $T_2$, $T_3$ sont des variables aléatoires indépendantes de même espérance égale à 6 et de même variance égale à 1.

\smallskip

On note $S$ la variable aléatoire correspondant au temps d'attente total passé à la station du troisième client entre son arrivée à la station et son départ de la pompe après avoir alimenté son véhicule.

\begin{enumerate}
	\item Exprimer $S$ en fonction de $T_1$, $T_2$ et $T_3$.
	\item 
	\begin{enumerate}
		\item Déterminer l'espérance de $S$ et interpréter ce résultat dans le contexte de l'exercice.
		\item Quelle est la variance du temps d'attente total $S$ de ce troisième client ?
	\end{enumerate}
	\item Montrer que la probabilité que le troisième client passe un temps strictement compris entre 14 et 22 minutes à la station est supérieure ou égale à $0,81$.
\end{enumerate}