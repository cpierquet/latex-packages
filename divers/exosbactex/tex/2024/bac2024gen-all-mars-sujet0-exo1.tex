\emph{L’exercice est constitué de deux parties indépendantes.}

\medskip

\textbf{\large Partie I}

\medskip

On considère l’équation différentielle \[ (E) \: : \: y'+y=\e^{-x}. \]

\begin{enumerate}
	\item Soit $u$ la fonction définie sur $\R$ par $u(x) = x\,\e^{-x}$.
	
	Vérifier que la fonction $u$ est une solution de l’équation différentielle $(E)$.
	\item On considère l’équation différentielle $(E')$ : $y'+y=0$.
	
	Résoudre l’équation différentielle $(E')$ sur $\R$.
	\item En déduire toutes les solution de l’équation différentielle $(E)$ sur $\R$.
	\item Déterminer l’unique solution $g$ de l’équation différentielle $(E)$ telle que $g(0)=2$.
\end{enumerate}

\medskip

\textbf{\large Partie II}

\medskip

Dans cette partie, $k$ est un nombre réel fixé que l’on cherche à déterminer.

On considère la fonction $f_k$ définie sur $\R$ par \[ f_k(x)=(x+k)\,\e^{-x}. \]
%
Soit $h$ la fonction définie sur $\R$ par \[ h(x)=\e^{-x}. \]
%
On note $\Courbe[k]$ la courbe représentative de la fonction $f_k$ dans un repère orthogonal et $\Courbe$ la courbe représentative de la fonction $h$.

On a représenté sur le graphique en annexe les courbes $\Courbe[k]$ et $\Courbe$ sans indiquer les unités sur les axes ni le nom des courbes.

\begin{enumerate}
	\item Sur le graphique en annexe à rendre avec la copie, l’une des courbes est en traits pointillés, l’autre est en trait plein. Laquelle est la courbe $\Courbe$ ?
	\item En expliquant la démarche utilisée, déterminer la valeur du nombre réel $k$ et placer sur l’annexe à rendre avec la copie l’unité sur chacun des axes du graphique.
\end{enumerate}