\textit{Cet exercice est un questionnaire à choix multiples (QCM) qui comprend cinq questions. Les cinq questions sont indépendantes. Pour chacune des questions, \textbf{une seule des quatre réponses est exacte}. Le candidat indiquera sur sa copie le numéro de la question suivi de la lettre correspondant à la réponse exacte.\\
Aucune justification n'est demandée.\\
Une réponse fausse, une réponse multiple ou une absence de réponse ne rapporte, ni n'enlève aucun point.}

\begin{enumerate}
	\item La solution $f$ de l'équation différentielle $y'=-3y+7$ telle que $f(0)=1$ est la fonction définie sur $\R$ par :
	
	\smallskip
	
	\ReponsesQCM[Labels={A.},PoliceLabels={\bfseries},NbCols=2,Swap]{%
		$f(x)=\e^{-3x}$ §
		$f(x)=-\frac43\e^{-3x}+\frac73$ §
		$f(x)=\e^{-3x}+\frac73$ §
		$f(x)=\frac{10}{3}\e^{-3x}-\frac73$}
	\item La courbe d'une fonction $f$ définie sur $\IntervalleFO{0}{+\infty}$ est donnée ci-dessous.
	
	\begin{Centrage}
		\begin{GraphiqueTikz}[x=1.5cm,y=1.5cm,Xmin=0,Xmax=6,Xgrille=1,Xgrilles=1,Ymin=0,Ymax=4,Ygrille=1,Ygrilles=1]
			\TracerAxesGrilles[Elargir=2.5mm]{auto}{auto}
			\TracerCourbe[Couleur=red]{0.00196*x^(5)-0.05764*x^(4)+0.62643*x^(3)-3.00524*x^(2)+5.43449*x}
		\end{GraphiqueTikz}
	\end{Centrage}
	
	Un encadrement de l'intégrale $I = \displaystyle\int_1^5 f(x) \dx$ est :
	
	\smallskip
	
	\ReponsesQCM[Labels={A.},PoliceLabels={\bfseries},NbCols=2,Swap]{%
		$0 \leqslant I \leqslant 4$ §
		$1 \leqslant I \leqslant 5$ §
		$5 \leqslant I \leqslant 10$ §
		$10 \leqslant I \leqslant 15$}
	\item On considère la fonction $g$ définie sur $\R$ par $g(x)=x^2\,\ln\big(x^2+4\big)$.
	
	Alors$\displaystyle\int_0^2 g(x) \dx$ vaut, à $10^{-1}$ près :
	
	\smallskip
	
	\ReponsesQCM[Labels={A.},PoliceLabels={\bfseries},NbCols=2,Swap]{%
		$4,9$ §
		$8,3$ §
		$1,7$ §
		$7,5$}
	\item Une professeure enseigne la spécialité mathématiques dans une classe de 31 élèves de terminale.
	
	Elle veut former un groupe de S élèves. De combien de façons différentes peut-elle former 
	un tel groupe de 5 élèves ?
	
	\smallskip
	
	\ReponsesQCM[Labels={A.},PoliceLabels={\bfseries},NbCols=2,Swap]{%
		$31^5$ §
		$31\times30\times29\times28\times27$ §
		$31+30+29+28+27$ §
		$\dbinom{31}{5}$}
	
	\pagebreak
	\item La professeure s'intéresse maintenant à l'autre spécialité des 31 élèves de son groupe :
	
	\begin{itemize}
		\item 10 élèves ont choisi la spécialité physique-chimie ;
		\item 20 élèves ont choisi la spécialité SES ;
		\item 1 élève a choisi la spécialité LLCE espagnol.
	\end{itemize}
	
	Elle veut former un groupe de 5 élèves comportant exactement 3 élèves ayant choisi la
	spécialité SES. De combien de façons différentes peut-elle former un tel groupe ?
	
	\smallskip
	
	\ReponsesQCM[Labels={A.},PoliceLabels={\bfseries},NbCols=2,Swap]{%
		$\dbinom{20}{3}\times\dbinom{11}{2}$ §
		$\dbinom{20}{3}+\dbinom{11}{2}$ §
		$\dbinom{20}{3}$ §
		$20^3\times11^2$}
\end{enumerate}