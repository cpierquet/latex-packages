Le but de cet exercice est d'étudier la fonction $f$ définie sur l'intervalle $\intervOO{0}{+\infty}$ par : \[ f(x)=x\,\ln{\big(x^2\big)}-\frac1x. \]

\textbf{\large\underline{Partie A : lectures graphiques}}

\medskip

On a tracé ci-dessous la courbe représentative $\left(C_f\right)$ de la fonction $f$, ainsi que la droite $(T)$, tangente à la courbe $\left(C_f\right)$ au point $A$ de coordonnées $(1;-1)$. Cette tangente passe également par le point $B(0;-4)$.

\begin{Centrage}
	\begin{tikzpicture}[x=2cm,y=0.5cm,xmin=0,xmax=4,xgrille=1,ymin=-7,ymax=7,ygrille=1]
		\FenetreSimpleTikz(ElargirOx=0,ElargirOy=0)<HautGrad=1.5pt>{1,2,3}<HautGrad=1.5pt>{-6,-5,...,6}
		\begin{scope}
			\FenetreTikz
			\draw[line width=1.1pt,red,samples=250,domain=0.05:4] plot (\x,{\x*ln((\x)^2)-1/(\x)}) ;
			\draw[line width=1.1pt,blue,samples=2,domain=0:4] plot (\x,{3*\x-4}) ;
		\end{scope}
		\draw (0,-4) pic[line width=0.8pt] {ptcroix=3pt} node[below right] {$B$} ;
		\draw (1,-1) pic[line width=0.8pt] {ptcroix=3pt} node[below right] {$A$} ;
		\draw (2.25,4.5) node[red] {$\left(C_f\right)$} ;
		\draw (2.75,3.5) node[blue] {$(T)$} ;
	\end{tikzpicture}
\end{Centrage}

\begin{enumerate}
	\item Lire graphiquement $f'(1)$ et donner l'équation réduite de la tangente $(T)$.
	\item Donner les intervalles sur lesquels la fonction $f$ semble convexe ou concave.
	
	Que semble représenter le point A pour la courbe $\left(C_f\right)$ ?
\end{enumerate}

\textbf{\large\underline{Partie A : étude analytique}}

\smallskip

\begin{enumerate}
	\item Déterminer, en justifiant, la limite de $f$ en $+\infty$, puis sa limite en 0.
	\item On admet que la fonction $f$ est deux fois dérivable sur l'intervalle $\intervOO{0}{+\infty}$.
	\begin{enumerate}
		\item Déterminer $f'(x)$ pour $x$ appartenant à l'intervalle $\intervOO{0}{+\infty}$.
		\item Montrer que pour tout $x$ appartenant à l'intervalle $\intervOO{0}{+\infty}$, \[ f''(x) = \frac{2(x+1)(x-1)}{x^3}. \]
	\end{enumerate}
	\item 
	\begin{enumerate}
		\item Étudier la convexité de la fonction $f$ sur l'intervalle $\intervOO{0}{+\infty}$.
		\item Étudier les variations de la fonction $f'$, puis le signe de $f'(x)$ pour $x$ appartenant à l'intervalle $\intervOO{0}{+\infty}$.
		
		En déduire le sens de variation de la fonction $f$ sur l'intervalle $\intervOO{0}{+\infty}$.
	\end{enumerate}
	\item 
	\begin{enumerate}
		\item Montrer que l'équation $f(x)=0$ admet une unique solution $\alpha$ sur l'intervalle $\intervOO{0}{+\infty}$.
		\item Donner la valeur arrondie au centième de $\alpha$ et montrer que $\alpha$ vérifie : \[ \alpha^2 = \text{exp}{\left(\frac{1}{\alpha^2}\right)} \]
	\end{enumerate}
\end{enumerate}