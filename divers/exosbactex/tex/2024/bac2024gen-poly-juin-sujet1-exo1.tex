\textit{Pour chacune des affirmations suivantes, indiquer si elle est vraie ou fausse.\\
Chaque réponse doit être justifiée. Une réponse non justifiée ne rapporte aucun point.\\
Dans cet exercice, les questions sont indépendantes les unes des autres.}

\medskip

Les quatre affirmations se placent dans la situation suivante.

\smallskip

Dans l’espace muni d’un repère orthonormé $\Rijk$, on considère les points : \[ A(2;1;-1),~B(-1;2;1) \text{ et } (5;0;-3). \]
%
On note $\mathcal{P}$ le plan d’équation cartésienne : \[ x +5y-2z +3 = 0. \]
%
On note $\mathcal{D}$ la droite de représentation paramétrique : \[ \begin{dcases} x=-3+t \\ y=t+2 \\ z=2t+1 \end{dcases},~t \in \R. \]

\smallskip

\textbf{Affirmation 1 :}

Le vecteur $\Vecteur{n}\begin{pmatrix}1\\0\\2\end{pmatrix}$ est normal au plan $(OAC)$.

\smallskip

\textbf{Affirmation 2 :}

Les droites $\mathcal{D}$ et $(AB)$ sont sécantes au point $C$.

\smallskip

\textbf{Affirmation 3 :}

La droite $\mathcal{D}$ est parallèle au plan $\mathcal{P}$.

\smallskip

\textbf{Affirmation 4 :}

Le plan médiateur du segment $[BC]$, noté $\mathcal{Q}$, a pour équation cartésienne : \[ 3x - y -2z -7 = 0. \]

\textit{On rappelle que le plan médiateur d’un segment est le plan perpendiculaire à ce segment et passant par son milieu.}