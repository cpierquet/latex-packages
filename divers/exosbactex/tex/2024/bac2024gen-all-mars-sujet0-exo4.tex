\emph{Cet exercice est un questionnaire à choix multiples.\\
Pour chaque question, une seule des quatre propositions est exacte. Indiquer sur la copie le numéro de la question et la lettre de la proposition choisie.\\
Aucune justification n’est demandée.\\
Pour chaque question, une réponse exacte rapporte un point. Une réponse fausse, une réponse multiple ou l’absence de réponse ne rapporte ni n’enlève de point.\\
Les questions sont indépendantes.}

\bigskip

On considère le prisme droit $ABFEDCGH$ tel que $AB = AD$.

Sa base $ABFE$ est un trapèze rectangle en $A$, vérifiant $\Vecteur{BF} = \dfrac12 \Vecteur{AE}$.

On note $I$ le milieu du segment $[EF]$.

On note $J$ le milieu du segment $[AE]$.

On associe à ce prisme le repère orthonormé \RepereEspace{A}{i}{j}{k} tel que :

$\Vecteur{\imath} = \Vecteur{AB}$ ; $\Vecteur{\jmath} = \Vecteur{AD}$ ; $\Vecteur{k} = \Vecteur{AJ}$.

\begin{center}
	\begin{EnvTikzEspace}
		\def\lgpr{4}
		\PlacePointsEspace{A/0,0,0/hd B/\lgpr,0,0/hd C/\lgpr,\lgpr,0/hd D/0,\lgpr,0/hd}
		\PlacePointsEspace{E/0,0,{2*\lgpr}/hd F/\lgpr,0,\lgpr/hd G/\lgpr,\lgpr,\lgpr/hd H/0,\lgpr,{2*\lgpr}/hd}
		\PlacePointsEspace{I/2,0,{1.5*\lgpr}/hd J/0,0,\lgpr/hd}
		\TraceSegmentsEspace[thick,dashed]{A/D D/C D/H}
		\TraceSegmentsEspace[thick]{A/B B/C C/G G/H H/E E/A E/F B/F F/G}
		\MarquePointsEspace{A,B,C,D,E,F,G,H,I,J}
	\end{EnvTikzEspace}
\end{center}

\begin{enumerate}
	\item On donne les coordonnées de quatre vecteurs dans la base $\left(\Vecteur{\vphantom{t}\imath},\Vecteur{\vphantom{t}\jmath},\Vecteur{\vphantom{t}\smash{k}}\right)$. Lequel est un vecteur normal au plan $(ABG)$ ?
	
	\ReponsesQCM[Labels=(a),EspaceLabels={~~},PoliceLabels={},Largeur=0.9\linewidth]{%
		$\Vecteur{n}\CoordVecEsp{1}{1}{1}$ § 
		$\Vecteur{n}\CoordVecEsp{-1}{1}{1}$ §
		$\Vecteur{n}\CoordVecEsp{0}{-1}{1}$ §
		$\Vecteur{n}\CoordVecEsp{0}{0}{1}$}
	\item Parmi les droites suivantes, laquelle est parallèle à la droite $(IJ)$ ?
	
	\ReponsesQCM[Labels=(a),EspaceLabels={~~},PoliceLabels={},Largeur=0.9\linewidth]{%
		$(DG)$ § 
		$(BD)$ §
		$(AG)$ §
		$(FG)$}
	\item Quels vecteurs forment une base de l’espace ?
	
	\ReponsesQCM[Labels=(a),EspaceLabels={~~},PoliceLabels={},Largeur=0.9\linewidth]{%
		$\left(\Vecteur{AB};\Vecteur{CG}\right)$ § 
		$\left(\Vecteur{AB};\Vecteur{AC};\Vecteur{AD}\right)$ §
		$\left(\Vecteur{DA};\Vecteur{DC};\Vecteur{DG}\right)$ §
		$\left(\Vecteur{CA};\Vecteur{CG};\Vecteur{CE}\right)$}
	\item Une décomposition du vecteur $\Vecteur{AG}$ comme somme de plusieurs vecteurs \textbf{deux à deux orthogonaux} est :
	
	\ReponsesQCM[NbCols=2,Labels=(a),EspaceLabels={~~},PoliceLabels={},Largeur=0.9\linewidth]{%
		$\Vecteur{AG}=\Vecteur{AB}+\Vecteur{HG}$ § 
		$\Vecteur{AG}=\Vecteur{AB}+\Vecteur{AD}+\Vecteur{AJ}$ §
		$\Vecteur{AG}=\Vecteur{AB}+\Vecteur{BJ}+\Vecteur{JG}$ §
		$\Vecteur{AG}=\Vecteur{AD}+\Vecteur{DH}+\Vecteur{HG}$}
	\item Le volume du prisme droit $ABFEDCGH$, est égal à :
	
	\ReponsesQCM[Labels=(a),EspaceLabels={~~},PoliceLabels={},Largeur=0.9\linewidth]{%
		$\dfrac58$ § 
		$\dfrac85$ §
		$\dfrac32$ §
		$2$}
\end{enumerate}