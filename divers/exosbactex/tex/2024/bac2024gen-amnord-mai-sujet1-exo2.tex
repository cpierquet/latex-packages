\emph{Cet exercice est un questionnaire à choix multiple.\\ Pour chaque question, une seule des quatre réponses proposées est exacte. Le candidat indiquera sur sa copie le numéro de la question et la réponse choisie. Aucune justification n'est demandée. Une réponse fausse, une réponse multiple ou l'absence de réponse à une question ne rapporte ni n'enlève de point.\\Les cinq questions sont indépendantes.}

\medskip

L'espace est rapporté à un repère orthonormé $\Rijk$.

\begin{enumerate}
	\item On considère les points $A(1;0;3)$ et $B(4;1;0)$.
	
	Une représentation paramétrique de la droite $(AB)$ est :
	
	\begin{MultiCols}[Type=enum](2)
		\item $\begin{dcases}x=3+t\\y=1\\z=-3+3t\end{dcases} \text{ avec }t \in \R$
		\item $\begin{dcases}x=1+4t\\y=t\\z=3\end{dcases} \text{ avec }t \in \R$
		\item $\begin{dcases}x=1+3t\\y=t\\z=3-3t\end{dcases} \text{ avec }t \in \R$
		\item $\begin{dcases}x=4+t\\y=1\\z=3-3t\end{dcases} \text{ avec }t \in \R$
	\end{MultiCols}
\end{enumerate}


On considère la droite $(d)$ de représentation paramétrique $\begin{dcases}x=3+4t\\y=6t\\z=4-2t\end{dcases} \text{ avec }t \in \R$.

\begin{enumerate}[resume]
	\item Parmi les points suivants, lequel appartient à la droite $(d)$ ?
	
	\begin{MultiCols}[Type=enum](4)
		\item $N(3;6;4)$
		\item $M(7;6;6)$
		\item $P(4;6;-2)$
		\item $R(-3;-9;7)$
	\end{MultiCols}
	\item On considère la droite $(d')$ de représentation paramétrique $\begin{dcases}x=-2+3k\\y=-1-2k\\z=1+k\end{dcases} \text{ avec }k \in \R$.
	
	Les droites $(d)$ et $(d')$ sont :
	
	\begin{MultiCols}[Type=enum](4)
		\item sécantes
		\item non coplanaires
		\item parallèles
		\item confondues
	\end{MultiCols}
	\item On considère le plan $(P)$ passant par le point $I(2;1;0)$ et perpendiculaire à la droite $(d)$. Une équation du plan $(P)$ est :
	
	\begin{MultiCols}[Type=enum](2)
		\item $2x + 3y - z -7 = 0$
		\item $-x+ y - 4z + 1 = 0$
		\item $4x +6y-2z + 9 = 0$
		\item $2x+y +1= 0$
	\end{MultiCols}
\end{enumerate}