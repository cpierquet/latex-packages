Répondre par VRAI ou FAUX à chacune des affirmations suivantes et justifier votre réponse.

Toute réponse non justifiée ne sera pas prise en compte dans la notation. Toutes les questions de cet exercice sont indépendantes.

\smallskip

\begin{enumerate}
	\item On considère la suite \Suite{u} définie pour tout entier naturel non nul $n$ par $u_n = \frac{25+(-1)^n}{n}$.
	
	\medskip
	
	\textbf{Affirmation 1} : La suite \Suite{u} est divergente.
	\item On considère la suite $\Suite{w}$ définie pour tout entier naturel $n$ par $\begin{dcases} w_0 = 1 \\ w_{n+1}=\frac{w_n}{1+w_n} \end{dcases}$.
	
	On admet que pour tout entier naturel $n$, $w_n > 0$.
	
	On considère la suite $\Suite{t}$ définie pour tout entier naturel $n$ par $t_n = \frac{k}{w_n}$ où $k$ est un nombre réel strictement positif.
	
	\medskip
	
	\textbf{Affirmation 2} : La suite $\Suite{t}$ est une suite arithmétique strictement croissante.
	\item On considère la suite $\Suite{v}$ définie pour tout entier naturel $n$ par $\begin{dcases} v_0 = 1 \\ v_{n+1}=\ln\big(1+v_n\big) \end{dcases}$.
	
	On admet que pour tout entier naturel $n$, $v_n > 0$.
	
	\medskip
	
	\textbf{Affirmation 3} : La suite $\Suite{v}$ est décroissante.
	\item On considère la suite $\Suite{I}$ définie pour tout entier naturel $n$ par $I_n = \displaystyle\int_1^{\e} {\big(\ln(x)\big)}^n \dx$.
	
	\medskip
	
	\textbf{Affirmation 4} : Pour tout entier naturel $n$, $I_{n+1} = \e-(n+1)I_n$.
\end{enumerate}