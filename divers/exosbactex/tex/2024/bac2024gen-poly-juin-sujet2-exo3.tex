On considère la suite $\Suite{u}$ définie par : \[ u_0=8 \text{ et pour tout entier naturel }n,~u_{n+1}=u_n-\ln\left(\frac{u_n}{4}\right). \]

\begin{enumerate}
	\item 
	\begin{enumerate}
		\item Donner les valeurs arrondies au centième de $u_1$ et $u_2$.
		\item On considère la fonction \texttt{mystere} définie ci-dessous en \textsf{Python}. On admet que, pour tout réel strictement positif \texttt{a}, \texttt{log(a)} renvoie la valeur du logarithme népérien de \texttt{a}.

\begin{CodePythonLstAlt}*[Largeur=8cm]{center}
def mystere(k) :
	u = 8
	S = 0
	for i in range(k) :
		S = S + u
		u = u - log( u / 4 )
	return S
\end{CodePythonLstAlt}
	
	L'exécution de \texttt{mystere(10)} renvoie \texttt{58.44045206721732}. Que représente ce résultat ?
	\item Modifier la fonction précédente afin qu'elle renvoie la moyenne des $k$ premiers termes de la suite $\Suite{u}$.
	\end{enumerate}
	\item On considère la fonction $f$ définie et dérivable sur $\IntervalleOO{0}{+\infty}$ par : \[ f(x)=x-\ln\left(\frac{x}{4}\right). \]
	On donne ci-dessous une représentation graphique $\mathcal{C}_f$ de la fonction $f$ pour les valeurs de $x$ comprises entre 0 et 6.
	
	\begin{Centrage}
		\begin{GraphiqueTikz}[x=2.4cm,y=1.2cm,Xmin=0,Xmax=6,Xgrille=1,Xgrilles=1,Ymin=0,Ymax=6,Ygrille=1,Ygrilles=1]
			\TracerAxesGrilles[Elargir=2.5mm]{auto}{auto}
			\TracerCourbe[Couleur=red,Debut=0.01]{x-log(x/4)}
			\draw[red] (5,5) node[above right,font=\large] {$\mathcal{C}_f$} ;
		\end{GraphiqueTikz}
	\end{Centrage}
	
	Étudier les variations de $f$ sur $\IntervalleOO{0}{+\infty}$ et dresser son tableau de variations.
	
	\textit{On précisera la valeur exacte du minimum de $f$ sur $\IntervalleOO{0}{+\infty}$. Les limites ne sont pas demandées.}
\end{enumerate}

Dans la suite de l'exercice, on remarquera que pour tout entier naturel $n$, $u_{n+1} = f\big(u_n\big)$.

\begin{enumerate}[resume]
	\item 
	\begin{enumerate}
		\item Démontrer, par récurrence, que pour tout entier naturel $n$, on a : \[ 1 \leqslant u_{n+1} \leqslant u_n. \]
		\item En déduire que la suite $\Suite{u}$ converge vers une limite réelle.
		
		On note $\ell$ la valeur de cette limite 
		\item Résoudre l'équation $f(x) = x$.
		\item En déduire la valeur de $\ell$.
	\end{enumerate}
\end{enumerate}