L'objectif de cet exercice est de conjecturer en \textbf{partie A} puis de démontrer en \textbf{partie B} le comportement d'une suite.

Les deux parties peuvent cependant être traitées de manière indépendante.

On considère la suite $\left(u_{n}\right)$ définie par $u_{0}=3$ et pour tout $n \in \N$ : \[ u_{n+1}=\frac{4}{5-u_{n}}. \]

\textbf{Partie A}

\smallskip

\begin{enumerate}
	\item Recopier et compléter la fonction \textsf{Python} suivante \texttt{suite(n)} qui prend comme paramètre le rang $n$ et renvoie la valeur du terme $u_{n}$.

\begin{CodePythonLstAlt}*[Largeur=7cm]{center}
def suite(n) :
	u = ...
	for i in range(n) :
		...
	return u
\end{CodePythonLstAlt}
	\item L'exécution de \texttt{suite(2)} renvoie \texttt{1.3333333333333333}.
	
	Effectuer un calcul pour vérifier et expliquer cet affichage.
	\item À l'aide des affichages ci-dessous, émettre une conjecture sur le sens de variation et une conjecture sur la convergence de la suite $\left(u_{n}\right)$.
	
\begin{CodePythonLstAlt}*[Largeur=7cm]{center,title={\scriptsize\faCode} Console Python}
>>> suite(2)
1.333333333333333
>>> suite(5)
1.0058479532163742
>>> suite(10)
1.0000057220349845
>>> suite(20)
1.000000000005457
\end{CodePythonLstAlt}
\end{enumerate}

\textbf{Partie B}

\medskip

On considère la fonction $f$ définie et dérivable sur l'intervalle $\IntervalleOO{-\infty}{5}$ par : \[ f(x)=\frac{4}{5-x}. \]
%
Ainsi, la suite $\Suite{u}$ est définie par $u_{0}=3$ et pour tout $n \in \N$, $ u_{n+1}=f\big(u_{n}\big)$.

\begin{enumerate}
	\item Montrer que la fonction $f$ est croissante sur l'intervalle $\IntervalleOO{-\infty}{5}$.
	\item Démontrer par récurrence que pour tout entier naturel $n$ on a : \[ 1 \leqslant u_{n+1} \leqslant u_{n} \leqslant 4. \]
	\item 
	\begin{enumerate}
		\item Soit $x$ un réel de l'intervalle $\IntervalleOO{-\infty}{5}$.
		
		Prouver l'équivalence suivante : \[ f(x)=x \Leftrightarrow x^{2}-5 x+4=0. \]
		\item Résoudre $f (x) = x$ dans l’intervalle $\IntervalleOO{-\infty}{5}$.
	\end{enumerate}
	\item Démontrer que la suite $\Suite{u}$ est convergente.
	
	Déterminer sa limite.
	\item Le comportement de la suite serait-il identique en choisissant comme terme initial $u_0 = 4$ au lieu de $u_0 = 3$ ?
\end{enumerate}