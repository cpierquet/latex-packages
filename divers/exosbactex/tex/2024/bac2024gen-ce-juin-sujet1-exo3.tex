On considère l'équation différentielle $\big(E_0\big)$ : $y'=y$ où $y$ est une fonction dérivable de la variable réelle $x$.

\begin{enumerate}
	\item Démontrer que l'unique fonction constante solution de l'équation différentielle $\big(E_0\big)$ est la fonction nulle.
	\item Déterminer toutes les solutions de l'équation différentielle $\big(E_0\big)$.
\end{enumerate}

On considère l'équation différentielle $\big(E\big)$ : $y'=y-\cos(x)-3\sin(x)$ où $y$ est une fonction dérivable de la variable réelle $x$.

\begin{enumerate}[resume]
	\item La fonction $h$ est définie sur $\R$ par $h(x) = 2\cos(x) + \sin(x)$.
	
	On admet qu'elle est dérivable sur $\R$.
	
	Démontrer que la fonction $h$ est solution de l'équation différentielle $\big(E\big)$.
	\item On considère une fonction $f$ définie et dérivable sur $\R$.
	
	Démontrer que : « $f$ est solution de $\big(E\big)$ » est équivalent à « $f-h$ est solution de $\big(E_0\big)$ ».
	\item En déduire toutes les solutions de l'équation différentielle $\big(E\big)$.
	\item Déterminer l'unique solution $g$ de l'équation différentielle $\big(E\big)$ telle que $g(0) = 0$.
	\item Calculer : \[ \int_{0}^{\tfrac{\pi}{2}} \left(-2\e^{x}+\sin(x)+2\cos(x)\right)\dx. \]
\end{enumerate}