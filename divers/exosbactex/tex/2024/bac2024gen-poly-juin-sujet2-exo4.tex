Une commune décide de remplacer le traditionnel feu d'artifice du 14 juillet par un
spectacle de drones lumineux.

\smallskip

Pour le pilotage des drones, l'espace est muni d'un repère orthonormé $\Rijk$ dont l'unité est la centaine de mètres.

\smallskip

La positon de chaque drone est modélisée par un point et chaque drone est envoyé d'un point de départ $D$ de coordonnées $(2;5;1)$.

\smallskip

On souhaite former avec des drones des figures en les positionnant dans un même plan $\mathcal{P}$.

\smallskip

Trois drones sont positionnés aux points $A(-1;-1;17)$, $B(4;-2;4)$ et $C(1;-3;7)$.

\begin{enumerate}
	\item Justifier que les points $A$, $B$ et $C$ ne sont pas alignés.
\end{enumerate} 

Dans la suite, on note $\mathcal{P}$ le plan $(ABC)$ et on considère le vecteur $\Vecteur{n} \begin{pmatrix}2\\-3\\1\end{pmatrix}$.

\begin{enumerate}[resume]
	\item 
	\begin{enumerate}
		\item Justifier que $\Vecteur{n}$ est normal au plan $(ABC)$.
		\item Démontrer qu'une équation cartésienne du plan $\mathcal{P}$ est $2x-3y+z-18=0$.
	\end{enumerate}
	\item Le pilote des drones décide d'envoyer un quatrième drone en prenant comme trajectoire la droite $d$ dont une représentation paramétrique est donnée par \[ d~:~\begin{dcases}x=3t+2\\y=t+5\\z=4t+1\end{dcases}\text{, avec } \in \R. \]
	\begin{enumerate}
		\item Déterminer un vecteur directeur de droite $d$.
		\item Afin que ce nouveau drone soit également placé dans le plan $\mathcal{P}$, déterminer par le calcul les coordonnées du point $E$, intersection de la droite $d$ avec le plan $\mathcal{P}$.
	\end{enumerate}
	\item Le pilote des drones décide d'envoyer un cinquième drone le long de la droite $\Delta$ qui passe par le point $D$ et qui est perpendiculaire au plan $\mathcal{P}$. Ce cinquième drone est placé lui aussi dans le plan $\mathcal{P}$, soit à l'intersection entre la droite $\Delta$ et le plan $\mathcal{P}$. On admet que le point $F(6;-1;3)$ correspond à cet emplacement.
	
	Démontrer que la distance entre le point de départ $D$ et le plan $\mathcal{P}$ vaut $2\sqrt{14}$ centaines de mètres.
	\item L'organisatrice du spectacle demande au pilote d'envoyer un nouveau drone dans le plan (peu importe sa position dans le plan), toujours à partir du point $D$. Sachant qu'il reste 40 secondes avant le début du spectacle et que le drone vole en trajectoire rectiligne à 18,6~$\text{m.s}^{-1}$, le nouveau drone peut-il arriver à temps ?
\end{enumerate}