Un sondage réalisé en France fournit les informations suivantes :

\begin{itemize}
	\item 60\,\% des plus de 15 ans ont l'intention de regarder les jeux Olympiques et Paralympiques (JOP) de Paris 2024 à la télévision ;
	\item parmi ceux qui ont l'intention de regarder les JOP, 8 personnes sur 9 déclarent pratiquer une activité sportive régulière.
\end{itemize}

On choisit au hasard une personne de plus de 15 ans. On considère les événements suivants :

\begin{itemize}
	\item $J$ : \og la personne a l'intention de regarder les JOP Paris 2024 à la télévision \fg\ ;
	\item $S$ : \og la personne choisie déclare pratiquer une activité sportive régulière \fg.
\end{itemize}

On note $\overline{J}$ et $\overline{S}$ leurs événements contraires.

\smallskip

\textit{Dans les questions 1. et 2., les probabilités seront données sous la forme d'une fraction irréductible.}

\begin{enumerate}
	\item Démontrer que la probabilité que la personne choisie ait l'intention de regarder les JOP de Paris 2024 à la télévision et déclare pratiquer une activité sportive régulière est de $\frac{8}{15}$.
	
	\textit{On pourra s'appuyer sur un arbre pondéré.}
\end{enumerate}

Selon ce sondage, deux personnes sur trois parmi les plus de 15 ans déclarent pratiquer une activité sportive régulière.

\begin{enumerate}[resume]
	\item 
	\begin{enumerate}
		\item Calculer la probabilité que la personne choisie n'ait pas l'intention de regarder les JOP de Paris 2024 à la télévision et déclare pratiquer une activité sportive régulière.
		\item En déduire la probabilité de $S$ sachant $\overline{J}$ notée $P_{\overline{J}}(S)$.
	\end{enumerate}
\end{enumerate}

\textit{Dans la suite de l'exercice, les résultats seront arrondis au millième.}

\begin{enumerate}[resume]
	\item Dans le cadre d'une opération de promotion, 30 personnes de plus de 15 ans sont choisies au hasard.
	
	On assimile ce choix à un tirage avec remise.
	
	On note $X$ la variable aléatoire qui donne le nombre de personnes déclarant pratiquer une activité sportive régulière parmi les 30 personnes.
	\begin{enumerate}
		\item Déterminer la nature et les paramètres de la loi de probabilité suivie par $X$.
		\item Calculer la probabilité qu'exactement 16 personnes déclarent pratiquer une activité sportive régulière parmi les 30 personnes.
		\item La fédération française de judo souhaite offrir une place pour la finale de l'épreuve par équipe mixte de judo à l'Arena Champ-de-Mars pour chaque personne déclare pratiquer une activité sportive régulière parmi ces 30 personnes.
		
		Le prix d'une place s'élève à 380\,€ et on dispose d'un budget de \num{10000} euros pour cette opération.
		
		Quelle est la probabilité que ce budget soit insuffisant ? 
	\end{enumerate}
\end{enumerate}