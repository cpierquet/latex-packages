\emph{Cet exercice est un questionnaire à choix multiples.\\
Pour chaque question, une seule des quatre propositions est exacte. Indiquer sur la copie le numéro de la question et la lettre de la proposition choisie.\\
Aucune justification n’est demandée.\\
Pour chaque question, une réponse exacte rapporte un point. Une réponse fausse, une réponse multiple ou l’absence de réponse ne rapporte ni n’enlève de point.\\
Les questions sont indépendantes.}

\bigskip

\begin{enumerate}
	\item Sur l’intervalle $\IntervalleFF{0}{2\pi}$, l’équation $\sin(x) = 0,1$ admet :
	
	\medskip
	
	\ReponsesQCM[NbCols=2,Labels=(a),EspaceLabels={~~},PoliceLabels={},Largeur=0.9\linewidth]{%
		zéro solution § 
		une solution §
		deux solutions §
		quatre solutions}
	\item On considère la fonction $f$ définie sur l’intervalle $\IntervalleFF{0}{\pi}$ par $f(x) = x + \sin(x)$. On admet que $f$ est deux fois dérivable.
	
	\medskip
	
	\ReponsesQCM[NbCols=2,Labels=(a),EspaceLabels={~~},PoliceLabels={},Largeur=0.9\linewidth]{%
		La fonction $f$ est convexe sur l’intervalle $\IntervalleFF{0}{\pi}$ § 
		La fonction $f$ est concave sur l’intervalle $\IntervalleFF{0}{\pi}$ §
		La fonction $f$ admet sur l’intervalle $\IntervalleFF{0}{\pi}$ un unique point d’inflexion §
		La fonction $f$ admet sur l’intervalle $\IntervalleFF{0}{\pi}$ exactement deux points d’inflexion}
	\item Une urne contient cinquante boules numérotées de 1 à 50. On tire successivement trois boules dans cette urne, sans remise. On appelle « tirage » la liste non ordonnée des numéros des trois boules tirées. Quel est le nombre de tirages possibles, \textbf{sans tenir compte de l’ordre des numéros} ?
	
	\medskip
	
	\ReponsesQCM[Labels=(a),EspaceLabels={~~},PoliceLabels={},Largeur=0.9\linewidth]{%
		$50^3$ § 
		$1\times2\times3$ §
		$50 \times 49 \times 48$ §
		$\dfrac{50 \times 49 \times 48}{1\times2\times3}$}
	\item On effectue dix lancers d’une pièce de monnaie. Le résultat d’un lancer est « pile » ou « face ». On note la liste ordonnée des dix résultats. Quel est le nombre de listes ordonnées possibles ?
	
	\medskip
	
	\ReponsesQCM[Labels=(a),EspaceLabels={~~},PoliceLabels={},Largeur=0.9\linewidth]{%
		$2\times10$ § 
		$2^{10}$ §
		$1 \times 2 \times 3 \times \ldots \times 10$ §
		$\dfrac{1 \times 2 \times 3 \times \ldots \times 10}{1\times2}$}
	\item On effectue $n$ lancers d’une pièce de monnaie équilibrée.
	
	Le résultat d’un lancer est « pile » ou « face ». On considère la liste ordonnée des $n$ résultats.
	
	Quelle est la probabilité d’obtenir au plus deux fois « pile » dans cette liste ?
	
	\medskip
	
	\ReponsesQCM[NbCols=2,Labels=(a),EspaceLabels={~~},PoliceLabels={},Largeur=0.9\linewidth]{%
		$\dfrac{n(n-1)}{2}$ § 
		$\dfrac{n(n-1)}{2} \times \left(\dfrac12\right)^n$ §
		$1+n+\dfrac{n(n-1)}{2}$ §
		$\left(1+n+\dfrac{n(n-1)}{2}\right) \times \left(\dfrac12\right)^n$}
\end{enumerate}