\textbf{Partie A}

\medskip

On définit la fonction $f$ sur l'intervalle $\IntervalleFF{0}{1}$ par \[ f(x)=\frac{0,96x}{0,93x+0,03}. \]

\begin{enumerate}
	\item Démontrer que, pour tout $x$ appartenant à l'intervalle $\IntervalleFF{0}{1}$, \[ f'(x)=\frac{\num{0,0288}}{(0,93x+0,03)^2}. \]
	\item Déterminer le sens de variation de la fonction $f$ sur l'intervalle $\IntervalleFF{0}{1}$.
\end{enumerate}

\smallskip

\textbf{Partie B}

\medskip


La lutte contre le dopage passe notamment par la réalisation de contrôles antidopage qui visent à déterminer si un sportif a fait usage de substances interdites.

Lors d'une compétition rassemblant \num{1000} sportifs, une équipe médicale teste tous les concurrents. On propose d'étudier la fiabilité de ce test.

\medskip

On appelle $x$ le réel compris entre 0 et 1 qui désigne la proportion de sportifs dopés.

\medskip

Lors de l'élaboration de ce test, on a pu déterminer que :

\begin{itemize}
	\item la probabilité qu'un sportif soit déclaré positif sachant qu'il est dopé est égale à $0,96$ ;
	\item la probabilité qu'un sportif déclaré positif sachant qu' il n'est pas dopé est égale à $0,03$.
\end{itemize}

On note :

\begin{itemize}
	\item $D$ l'événement: « le sportif est dopé » ;
	\item $T$ l'événement : « le test est positif » 
\end{itemize}

\begin{enumerate}
	\item Recopier et compléter l'arbre de probabilité ci-dessous :
	
	\begin{Centrage}
		\ArbreProbasTikz[PositionProbas=auto]{$D$/$x$/,$T$/\numdots/,$\overline{T}$/\numdots/,$\overline{D}$/$1-x$/,$T$/\numdots/,$\overline{T}$/\numdots/}
	\end{Centrage}
	\item Déterminer, en fonction de $x$ la probabilité qu'un sportif soit dopé et ait un test positif.
	\item Démontrer que la probabilité de l'événement $T$ est égale à $0,93x + 0,03$.
	\item Pour cette question uniquement, on suppose qu'il y a 50 sportifs dopés parmi les \num{1000} testés.
	
	La fonction $f$ désigne la fonction définie à la \textbf{partie A}.
	
	Démontrer que la probabilité qu'un sportif soit dopé sachant que son test est positif est égale à $f(0,05)$. En donner une valeur arrondie au centième.
	\item On appelle valeur prédictive positive d'un test la probabilité que le sportif soit réellement dopé lorsque le résultat du test est positif.
	\begin{enumerate}
		\item Déterminer à partir de quelle valeur de $x$ la valeur prédictive positive du test étudié sera supérieure ou égale à $0,9$. \textit{Arrondir le résultat au centième.}
		\item Un responsable de la compétition décide de ne plus tester l'ensemble des sportifs, mais de cibler les sportifs les plus performants supposés être plus fréquemment dopés.
		
		Quelle est la conséquence de cette décision sur la valeur prédictive positive du test ? \textit{Argumenter en utilisant un résultat de la \textbf{Partie A}.}
	\end{enumerate}
\end{enumerate}