Dans la revue \textit{Lancet Public Health}, les chercheurs affirment qu’au 11 mai 2020, $5,7$\,\% des adultes français avaient déjà été infectés par la COVID 19.

{\footnotesize Source : \url{https://www.thelancet.com/journals/lanpub/article/PIIS2468-2667(21)00064-5/fulltext}}

On se servira de cette donnée pour les \textbf{parties A} et \textbf{B} de cet exercice.

\medskip

\textbf{Partie A}

\smallskip

\begin{enumerate}
	\item On prélève un individu dans la population française adulte au 11 mai 2020.
	
	\smallskip
	
	On note $I$ l’évènement : « l’adulte a déjà été infecté par la COVID 19 ».
	
	\smallskip
	
	Quelle est la probabilité que cet individu prélevé ait déjà été infecté par la COVID
	19 ?
	\item On prélève un échantillon de 100 personnes de la population supposées choisie de façon indépendante les unes des autres.
	
	On assimile ce prélèvement à un tirage avec remise.
	
	On appelle $X$ la variable aléatoire qui compte le nombre de personnes ayant déjà été infectées.
	\begin{enumerate}
		\item Justifiez que $X$ suit une loi binomiale dont on donnera les paramètres.
		\item Calculer son espérance mathématique. Interpréter ce résultat dans le cadre de l’exercice.
		\item Quelle est la probabilité qu’il n’y ait aucune personne infectée dans l’échantillon ?
		
		On donnera une valeur approchée à $10^{-4}$ près du résultat.
		\item Quelle est la probabilité qu’il y ait au moins 2 personnes infectées dans l’échantillon ?
		
		On donnera une valeur approchée à $10^{-4}$ près du résultat.
		\item Déterminer le plus petit entier $n$ tel que $P(X \leqslant n) > 0,9$.
		
		Interpréter ce résultat dans le contexte de l’exercice.
	\end{enumerate}
\end{enumerate}

\smallskip

\textbf{Partie B}

\medskip

Un test a été mis en place : celui-ci permet de déterminer (même longtemps après l’infection), si une personne a ou non déjà été infectée par la COVID 19.

Si le test est positif, cela signifie que la personne a déjà été infectée par la COVID 19.

Deux paramètres permettent de caractériser ce test : sa sensibilité et sa spécificité.

La \textbf{sensibilité} d’un test est la probabilité qu’il soit positif sachant que la personne a été infectée par la maladie. (Il s’agit donc d’un vrai positif).

La \textbf{spécificité} d’un test est la probabilité que le test soit négatif sachant que la personne n’a pas été infectée par la maladie. (Il s’agit donc d’un vrai négatif).

\smallskip

Le fabricant du test fournit les caractéristiques suivantes :
%
\begin{itemize}
	\item sa sensibilité est de $0,8$ ;
	\item sa spécificité est de $0,99$.
\end{itemize}

On prélève un individu soumis au test dans la population française adulte au 11 mai 2020.

On note $T$ l’évènement « le test réalisé est positif ».

\begin{enumerate}
	\item Compléter l’arbre des probabilités ci-dessous avec les données de l’énoncé :
	
	\begin{Centrage}
		\ArbreProbasTikz[PositionProbas=auto]{$I$//,$T$//,$\overline{T}$//,$\overline{I}$//,$T$//,$\overline{T}$//}
	\end{Centrage}
	\item Montrer que $P(T) = \num{0,05503}$.
	\item Quelle est la probabilité qu’un individu ait été infecté sachant que son test est positif ?
	
	On donnera une valeur approchée à $10^{-4}$ près du résultat.
\end{enumerate}
\smallskip

\textbf{Partie C}

\medskip

On considère un groupe d’une population d’un autre pays soumis au même test de sensibilité $0,8$ et de spécificité $0,99$.

Dans ce groupe la proportion d’individus ayant un test positif est de $29,44$\,\%.

On choisit au hasard un individu de ce groupe; quelle est la probabilité qu’il ait été infecté ?