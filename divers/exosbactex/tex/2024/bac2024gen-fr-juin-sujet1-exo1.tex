\textit{Pour chacune des affirmations suivantes, indiquer si elle est vraie ou fausse. Chaque réponse doit être justifiée. Une réponse non justifiée ne rapporte aucun point.}

\begin{enumerate}
	\item On considère la fonction $f$ définie sur $\R$ par : $f(x)=5x\,\e^{-x}$.
	
	On note $\mathcal{C}_{f}$ la courbe représentative de $f$ dans un repère orthonormé.
	
	\medskip
	
	\textbf{Affirmation 1 :}
	
	L'axe des abscisses est une asymptote horizontale à la courbe $\mathcal{C}_{f}$.
	
	\medskip
	
	\textbf{Affirmation 2 :}
	
	La fonction $f$ est solution sur $\R$ de l'équation différentielle $(E): y^{\prime}+y=5 \e^{-x}$.
	\item On considère les suites $\left(u_{n}\right)$, $\left(v_{n}\right)$ et $\left(w_{n}\right)$, telles que, pour tout entier naturel $n$ :
	%
	\begin{Centrage}
		$u_{n} \leqslant v_{n} \leqslant w_{n}$.
	\end{Centrage}
	
	De plus, la suite $\left(u_{n}\right)$ converge vers $-1$ et la suite $\left(w_{n}\right)$ converge vers $1$.
	
	\medskip
	
	\textbf{Affirmation 3 :}
	
	La suite $\left(v_{n}\right)$ converge vers un nombre réel $\ell$ appartenant à l'intervalle $\IntervalleFF{-1}{1}$.
	
	On suppose de plus que la suite $\left(u_{n}\right)$ est croissante et que la suite $\left(w_{n}\right)$ est décroissante.
	
	\smallskip
	
	\textbf{Affirmation 4 :}
	
	Pour tout entier naturel $n$, on a alors : $u_{0} \leqslant v_{n} \leqslant w_{0}$.
\end{enumerate}