On considère la fonction $g$ définie sur l'intervalle $\intervFF{0}{1}$ par $g(x)= 2x - x^2$.

\begin{enumerate}
	\item Montrer que la fonction $g$ est strictement croissante sur l'intervalle $\intervFF{0}{1}$ et préciser les valeurs de $g(0)$ et de $g(1)$.
\end{enumerate}

On considère la suite $\suiten$ définie par $\begin{dcases} u_0=\tfrac12 \\ u_{n+1}=g\big(u_n\big)\end{dcases}$ pour tout entier naturel $n$.

\begin{enumerate}[resume]
	\item Calculer $u_1$ et $u_2$.
	\item Démontrer par récurrence que, pour tout entier naturel $n$, on a : $0 < u_n < u_{n+1} < 1$.
	\item En déduire que la suite $\suiten$ est convergente.
	\item Déterminer la limite $\ell$ de la suite $\suiten$.
\end{enumerate}

On considère la suite $\suiten[v]$ définie pour tout entier naturel $n$ par $v_n = \ln(1-u_n)$.

\begin{enumerate}[resume]
	\item Démontrer que la suite $\suiten[v]$ est une suite géométrique de raison 2 et préciser son premier terme.
	\item En déduire une expression de $v_n$ en fonction de $n$.
	\item En déduire une expression de $u_n$ en fonction de $n$ et retrouver la limite déterminée à la question 5.
	\item Recopier et compléter le script \texttt{Python} ci-dessous afin que celui-ci renvoie le rang \texttt{n} à partir duquel la suite dépasse 0,95.

\begin{CodePythonLstAlt}[Largeur=0.5\linewidth]{center}
def seuil() :
	n = 0
	u = 0.5
	while u < 0.95 :
		n = ......
		u = ......
	return n
\end{CodePythonLstAlt}
\end{enumerate}