Dans un repère orthonormé $\Rijk$ de l'espace, on considère le plan $(P)$ d'équation : \[ (P) \text{ : } 2x + 2y - 3z + 1 = 0. \]
%
On considère les trois points $A$, $B$ et $C$ de coordonnées : \[ A(1;0;1)\text{, } B(2;-1;1) \text{ et } C(-4;-6;5). \]
%
Le but de cet exercice est d'étudier le rapport des aires entre un triangle et son projeté orthogonal dans un plan.

\medskip

\textbf{Partie A}

\smallskip

\begin{enumerate}
	\item Pour chacun des points $A$, $B$ et $C$, vérifier s'il appartient au plan $(P)$.
	\item Montrer que le point $C'(0;-2;-1)$ est le projeté orthogonal du point $C$ sur le plan $(P)$.
	\item Déterminer une représentation paramétrique de la droite $(AB)$.
	\item On admet l'existence d'un unique point $H$ vérifiant les deux conditions \[ \begin{dcases} H \in (AB) \\ (AB) \text{ et } (HC) \text{ sont orthogonales} \end{dcases}. \]
	Déterminer les coordonnées du point $H$.
\end{enumerate}

\begin{Centrage}
	\begin{tikzpicture}
		\draw[semithick] (0,0) --++ (6,0) --++ (55:4.5) --++(-6,0) -- cycle ;
		\coordinate (A) at (2.3,0.6) ;
		\coordinate (B) at ($(A)+(28:2.7)$) ;
		\coordinate (C) at ($(B)+(145:2)$) ;
		\coordinate (C') at (3,2) ;
		\draw[semithick] (A) -- (B) -- (C) -- cycle ;
		\draw[semithick,densely dashed] (A) -- (C') -- (B) (C') -- (C) ;
		\MarquerPoints[StyleMarque=+,TailleMarque=1.75pt]{A,B,C,C'}
		\draw (6,0) node[above left] {$(P)$}
		      (A) node[below] {$A$}
		      (B) node[right] {$B$}
		      (C) node[above] {$C$}
		      (C') node[above right] {$C'$} ;
	\end{tikzpicture}
\end{Centrage}

\textbf{Partie B}

\medskip

On admet que les coordonnées du vecteur $\Vecteur{HC}$ sont : $\Vecteur{HC}\begin{pNiceMatrix}[cell-space-limits=1pt]-\tfrac{11}{2}\\-\tfrac{11}{2}\\4\end{pNiceMatrix}$.

\begin{enumerate}
	\item Calculer la valeur exacte de $\Norme{\Vecteur{HC}}$.
	\item Soit $\mathcal{S}$ l'aire du triangle $ABC$. Déterminer la valeur exacte de $\mathcal{S}$.
\end{enumerate}

\smallskip

\textbf{Partie C}

\medskip

On admet que $HC' = \sqrt{\frac{17}{2}}$.

\begin{enumerate}
	\item Soit $\alpha = \widehat{CHC'}$. Déterminer la valeur de $\cos(\alpha)$.
	\item 
	\begin{enumerate}
		\item Montrer que les droites $(C'H)$ et $(AB)$ sont perpendiculaires.
		\item Calculer $\mathcal{S}'$ l'aire du triangle $ABC'$, on donnera la valeur exacte.
		\item Donner une relation entre $\mathcal{S}$, $\mathcal{S}'$ et $\cos(\alpha)$.
	\end{enumerate}
\end{enumerate}