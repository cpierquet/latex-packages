\emph{Pour chacune des affirmations suivantes, indiquer si elle est juste ou fausse. Chaque réponse doit être justifiée. Une réponse non justifiée ne rapporte aucun point.}

\vspace*{0.75cm}

\textbf{\underline{Affirmation 1 :}} Soit $(E)$ l’équation différentielle : $y' - 2y = -6x +1$.

\medskip

La fonction $f$ définie sur $\R$ par $f(x)=\e^{2x}-6x+1$ est une solution de l’équation différentielle $(E)$.

\bigskip

\textbf{\underline{Affirmation 2 :}} On considère la suite $\Suite{u}$ définie sur $\N$ par%
\[ u_n  = 1 + \frac34 + {\left(\frac34\right)}^2 + \ldots {\left(\frac34\right)}^n. \]
La suite $\Suite{u}$ a pour limite $+\infty$.

\bigskip

\textbf{\underline{Affirmation 3 :}} On considère la suite $\Suite{u}$ définie dans l’affirmation 2.

\medskip

L’instruction \texttt{suite(50)} ci-dessous, écrite en langage \textsf{Python}, renvoie $u_{50}$.

\begin{CodePythonLstAlt}[Largeur=0.4\linewidth]{center}
def suite(k) :
	S = 0
	for i in range(k) :
		S = S + (3/4)**k
	return S
\end{CodePythonLstAlt}

\bigskip

\textbf{\underline{Affirmation 4 :}} Soit $a$ un réel et $f$ la fonction définie sur $\IntervalleOO{0}{+\infty}$ par%
\[ f(x)=a\,\ln(x)-2x. \]
Soit $\mathcal{C}$ la courbe représentative de la fonction $f$ dans un repère $\Rij$.

\smallskip

Il existe une valeur de $a$ pour laquelle la tangente à $\mathcal{C}$ au point d’abscisse 1 est parallèle à l’axe des abscisses.