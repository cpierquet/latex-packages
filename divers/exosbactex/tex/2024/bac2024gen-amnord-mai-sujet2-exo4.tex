Soit $a$ un réel strictement positif.

On considère la fonction $f$ définie sur l'intervalle $\intervOO{0}{+\infty}$ par $f(x) = a\, \ln(x)$.

On note $\mathcal{C}$ sa courbe représentative dans un repère orthonormé.

Soit $x_0$ un réel strictement supérieur à 1.

\begin{enumerate}
	\item Déterminer l'abscisse du point d'intersection de la courbe $\mathcal{C}$ et de l'axe des abscisses.
	\item Vérifier que la fonction $F$ définie par $F(x) = a(x\,\ln(x) - x)$ est une primitive de la fonction $f$ sur l'intervalle $\intervOO{0}{+\infty}$.
	\item En déduire l'aire du domaine grisé en fonction de $a$ et de $x_0$.
\end{enumerate}

\begin{Centrage}
	\begin{tikzpicture}[x=1.75cm,y=1.75cm,xmin=-1,xmax=5.5,ymin=-1,ymax=2.5]
		\fill[semithick,draw=blue,fill=cyan!50] plot[samples=250,domain=1:1.85] (\x,{ln(\x)}) -- (1.85,0)--cycle ;
		\AxesTikz\AxexTikz{1}
		\draw (0,0) node[below right] {$O$} ;
		\draw (\xmax,0) node[below left] {$x$} ;
		\draw (0,\ymax) node[below left] {$y$} ;
		\draw (1.85,0) node[below] {$x_0$} ;
		\draw[red] (5,1.5) node[below] {$\mathcal{C}$} ;
		\FenetreTikz
		\CourbeTikz[thick,red,samples=250]{ln(\x)}{0.1:5.5}
	\end{tikzpicture}
\end{Centrage}

On note $T$ la tangente à la courbe $\mathcal{C}$ au point $M$ d'abscisse $x_0$..

On appelle $A$ le point d'intersection de la tangente $T$ avec l'axe des ordonnées et $B$ le projeté orthogonal de $M$ sur l'axe des ordonnées.

\begin{Centrage}
	\begin{tikzpicture}[x=1.75cm,y=1.75cm,xmin=-1,xmax=5.5,ymin=-1,ymax=2.5]
		\AxesTikz\AxexTikz{1}
		\draw (0,0) node[below right] {$O$} ;
		\draw (\xmax,0) node[below left] {$x$} ;
		\draw (0,\ymax) node[below left] {$y$} ;
		\draw (1.85,0) node[below] {$x_0$} ;
		\draw[red] (5,1.5) node[below] {$\mathcal{C}$} ;
		\draw[blue] (-0.75,-0.5) node[below] {$T$} ;
		\draw (0,{1*ln(1.85)}) node[above left] {$B$} ;
		\draw (0,{1*ln(1.85)}) rectangle++ (3pt,-3pt) ;
		\draw[dashed] (1.85,0) --++ (0,{1*ln(1.85)});
		\FenetreTikz
		\CourbeTikz[thick,red,samples=250]{ln(\x)}{0.1:5.5}
		\CourbeTikz[thick,blue,samples=2]{1/1.85*(\x-1.85)+1*ln(1.85)}{\xmin:\xmax}
		\CourbeTikz[thick,olive,samples=2]{1*ln(1.85)}{\xmin:\xmax}
		\draw[semithick] (1.85,{1*ln(1.85)}) pic{PLdotcross=3pt/0} node[above] {$M$} ;
		\draw[semithick] (0,{-1+ln(1.85)}) pic{PLdotcross=3pt/0} node[below right] {$A$} ;
	\end{tikzpicture}
\end{Centrage}

\begin{enumerate}[resume]
	\item Démontrer que la longueur $AB$ est égale à une constante (c'est-à-dire à un nombre qui ne dépend pas de $x_0$) que l'on déterminera. \textit{Le candidat prendra soin d'expliciter sa démarche.}
\end{enumerate}