La directrice d'une école souhaite réaliser une étude auprès des étudiants qui ont passé l'examen de fin d'étude, pour analyser la façon dont ils pensent avoir réussi cet examen.

\smallskip

Pour cette étude, on demande aux étudiants à l'issue de l'examen de répondre individuellement à la question : \og Pensez-vous avoir réussi l'examen ? \fg. Seules les réponses \og oui \fg\ ou \og non \fg\ sont possibles, et on observe que 91,7\,\% des étudiants interrogés ont répondu \og oui \fg.

\smallskip

Suite à la publication des résultats à l'examen, on découvre que :

\begin{itemize}
	\item 65\,\% des étudiants ayant échoué ont répondu \og non \fg\ ;
	\item 98\,\% des étudiants qui ont réussi on répondu \og oui \fg.
\end{itemize}

On interroge au hasard un étudiant qui a passé l'examen.

\smallskip

On note $R$ l'évènement \og l'étudiant a réussi l'examen \fg\ et $Q$ l'évènement \og l'étudiant a répondu \og oui \fg\ à la question \fg.

\smallskip

Pour un évènement $A$ quelconque, on note $P(A)$ sa probabilité et $\overline{A}$ sont évènement contraire.

\medskip

\textit{\textbf{Dans tout l'exercice, les probabilités seront, si besoin, arrondies à $\bm{10^{-3}}$ près.}}

\begin{wrapstuff}[r]
	\ArbreProbasTikz[PositionProbas=auto,EspaceNiveau=2,EspaceFeuille=0.75]{$R$/$x$/,$Q$/\numdots/,$\overline{Q}$/\numdots/,$\overline{R}$/\numdots/,$Q$/\numdots/,$\overline{Q}$/\numdots/}
\end{wrapstuff}

\begin{enumerate}
	\item Préciser les valeurs des probabilités $P(Q)$ et $P_{\overline{R}} \big(\overline{Q}\big)$.
	\item On note $x$ la probabilité que l'étudiant interrogé ait réussi l'examen.
	
	\begin{enumerate}
		\item Recopier et compléter l'arbre pondéré ci-contre.
		\item Montrer que $x=0,9$.
	\end{enumerate}
	\item L'étudiant interrogé a répondu « oui » à la question.
	
	Quelle est la probabilité qu'il ait réussi l'examen ?
	\item La note obtenue par un étudiant interrogé au hasard est un nombre entier entre 0 et 20 . On suppose qu'elle est modélisée par une variable aléatoire $N$ qui suit la loi binomiale de paramètres $(20;0,615)$. La directrice souhaite attribuer une récompense aux étudiants ayant obtenu les meilleurs résultats.
	
	À partir de quelle note doit-elle attribuer les récompenses pour que 65\,\% des étudiants soient récompensés ?
	\item On interroge au hasard dix étudiants.
	
	Les variables aléatoires $N_{1}$, $N_{2}$, \ldots, $N_{10}$ modélisent la note sur 20 obtenue à l'examen par chacun d'entre eux. On admet que ces variables sont indépendantes et suivent la même loi binomiale de paramètres $(20;0,615)$.
	
	Soit $S$ la variable définie par $S=N_{1}+N_{2}+\ldots+N_{10}$.
	
	Calculer l'espérance $E(S)$ et la variance $V(S)$ de la variable aléatoire $S$.
	\item On considère la variable aléatoire $M=\frac{S}{10}$.
	
	\begin{enumerate}
		\item Que modélise cette variable aléatoire $M$ dans le contexte de l'exercice ?
		\item Justifier que $E(M)=12,3$ et $V(M)=\num{0,47355}$.
		\item À l'aide de l'inégalité de Bienaymé-Tchebychev, justifier l'affirmation ci-dessous.
		
		\medskip
		
		« La probabilité que la moyenne des notes des dix étudiants pris au hasard soit strictement comprise entre 10,3 et 14,3 est d'au moins 80\,\% ».
	\end{enumerate}
\end{enumerate}