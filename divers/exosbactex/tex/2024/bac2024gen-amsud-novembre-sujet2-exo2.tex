Cet exercice contient 5 affirmations.

\smallskip

Pour chaque affirmation, répondre par VRAI ou FAUX en justifiant la réponse.

Toute absence de justification ou justification incorrecte ne sera pas prise en compte dans la notation.

\medskip

\textbf{Partie 1}

\medskip

On considère la suite $\Suite{u}$ définie par :

$u_{0}=10$ et pour tout entier naturel $n$, $u_{n+1}=\frac{1}{3} u_{n}+2$.

\begin{enumerate}
	\item \textbf{Affirmation 1} : La suite $\Suite{u}$ est décroissante minorée par $0$.
	\item \textbf{Affirmation 2} : $\lim\limits_{n \to +\infty} u_{n}=0$.
	\item \textbf{Affirmation 3} : La suite $\Suite{v}$ définie pour tout entier naturel $n$ par $v_{n}=u_{n}-3$ est géométrique.
\end{enumerate}

\medskip

\textbf{Partie 2}

\medskip

On considère l'équation différentielle $(E)$ : $y'=\frac{3}{2} y+2$ d'inconnue $y$, fonction définie et dérivable sur $\R$.

\begin{enumerate}
	\item \textbf{Affirmation 4} : Il existe une fonction constante solution de l'équation différentielle $(E)$.
	\item Dans un repère orthonormé $\Rij$ on note $\mathcal{C}_{f}$ la courbe représentative de la fonction $f$ solution de $(E)$ tel que $f(0)=0$.
	
	\textbf{Affirmation 5} : La tangente au point d'abscisse 1 de $\mathcal{C}_{f}$ a pour coefficient directeur $2\,\e^{\frac{3}{2}}$.
\end{enumerate}