\emph{%
Léa passe une bonne partie de ses journées à jouer à un jeu vidéo et s'intéresse aux chances de victoire de ses prochaines parties.\\
Elle estime que si elle vient de gagner une partie, elle gagne la suivante dans 70\,\% des cas.\\
Mais si elle vient de subir une défaite, d'après elle, la probabilité qu'elle gagne la suivante est de 0,2.\\
De plus, elle pense avoir autant de chance de gagner la première partie que de la perdre.%
}

\smallskip

On s'appuiera sur les affirmations de Léa pour répondre aux questions de cet exercice.

\smallskip

Pour tout entier naturel $n$ non nul, on définit les événements suivants :

\begin{itemize}
	\item $G_n$ : « Léa gagne la $n$-ième partie de la journée » ;
	\item $D_n$ : « Léa perd la $n$-ième partie de la journée ».
\end{itemize}

Pour tout entier naturel $n$ non nul, on note $g_n$ la probabilité de l'événement $G_n$.

On a donc $g_1 = 0,5$.

\begin{enumerate}
	\item Quelle est la valeur de la probabilité conditionnelle $P_{G_1} \big(D_2\big)$ ?
	\item Recopier et compléter l'arbre des probabilités ci-dessous qui modélise la situation pour les deux premières parties de la journée :
	
	\begin{Centrage}
		\ArbreProbasTikz[PositionProbas=auto]{$G_1$/\numdots/,$G_2$/\numdots/,$D_2$/\numdots/,$D_1$/\numdots/,$G_2$/\numdots/,$D_2$/\numdots/}
	\end{Centrage}
	\item Calculer $g_2$.
	\item Soit $n$ un entier naturel non nul.
	\begin{enumerate}
		\item Recopier et compléter l'arbre des probabilités ci-dessous qui modélise la situation pour les \mbox{$n$-ième} et $(n+1)$-ième parties de la journée.
		
		\begin{Centrage}
			\ArbreProbasTikz[PositionProbas=auto]{$G_n$/\numdots/$g_n$,$G_{n+1}$/\numdots/,$D_{n+1}$/\numdots/,$D_n$/\numdots/,$G_{n+1}$/\numdots/,$D_{n+1}$/\numdots/}
		\end{Centrage}
		\item Justifier que pour tout entier naturel $n$ non nul, $g_{n+1}= 0,5g_n+0,2$.
	\end{enumerate}
	\item Pour tout entier naturel $n$ non nul, on pose $v_n=g_n-0,4$.
	\begin{enumerate}
		\item Montrer que la suite $\Suite{v}$ est géométrique. On précisera son premier terme et sa raison.
		\item Montrer que, pour tout entier naturel $n$ non nul : $g_n = 0,1 \times 0,5^{n-1}+0,4$.
	\end{enumerate}
	\item Étudier les variations de la suite $\Suite{g}$.
	\item Donner, en justifiant, la limite de la suite $\Suite{g}$.
	
	Interpréter le résultat dans le contexte de l'énoncé. 
	\item Déterminer, par le calcul, le plus petit entier $n$ tel que $g_n - 0,4 \leqslant 0,001$.
	\item Recopier et compléter les lignes \texttt{4}, \texttt{5} et \texttt{6} de la fonction suivante, écrite en langage \textsf{Python}, afin qu'elle renvoie le plus petit rang à partir duquel les termes de la suite $\Suite{g}$ sont tous inférieurs ou égaux à
	$0,4 + e$, où $e$ est un nombre réel strictement positif.
\end{enumerate}

\begin{CodePythonLstAlt}[Largeur=7cm,Filigrane]{center}
def seuil(e) :
	g = 0.5
	n = 1
	while ... :
		g = 0.5*g + 0.2
		n = ...
	return (n)
\end{CodePythonLstAlt}