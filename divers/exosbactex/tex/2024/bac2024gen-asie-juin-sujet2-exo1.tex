On considère la fonction $f$ définie sur $\IntervalleOO{0}{+\infty}$ par $f(x) = x^2 - x\,\ln(x)$.

On admet que $f$ est deux fois dérivable sur $\IntervalleOO{0}{+\infty}$.

On note $f'$ la fonction dérivée de la fonction $f$ et $f''$ la fonction dérivée de la fonction $f'$.

\medskip

\textbf{Partie A : Étude de la fonction \bm{$f$}}

\smallskip

\begin{enumerate}
	\item Déterminer les limites de la fonction $f$ en $0$ et en $+\infty$.
	\item Pour tout réel $x$ strictement positif, calculer $f'(x)$.
	\item Montrer que pour tout réel $x$ strictement positif : \[ f'(x)=\frac{2x-1}{x}. \]
	\item Étudier les variations de la fonction $f'$ sur $\IntervalleOO{0}{+\infty}$, puis dresser le tableau des variations de la fonction $f$ sur $\IntervalleOO{0}{+\infty}$.
	
	On veillera à faire apparaître la valeur exacte de l'extremum de la fonction $f'$ sur $\IntervalleOO{0}{+\infty}$.
	
	Les limites de la fonction $f'$ aux bornes de l'intervalle de définition ne sont pas attendues.
	\item Montrer que la fonction $f$ est strictement croissante sur $\IntervalleOO{0}{+\infty}$.
\end{enumerate}

\smallskip

\textbf{Partie B : Étude d'une fonction auxiliaire pour la résolution de l'équation \bm{$f(x) = x$}}

\medskip

On considère dans cette partie la fonction $g$ définie sur $\IntervalleOO{0}{+\infty}$ par $g(x) = x-\ln(x)$.

On admet que la fonction $g$ est dérivable sur $\IntervalleOO{0}{+\infty}$, on note $g'$ sa dérivée.

\begin{enumerate}
	\item Pour tout réel strictement positif, calculer $g'(x)$, puis dresser le tableau des variations de la fonction $g$. Les limites de la fonction $g$ aux bornes de l'intervalle de définition ne sont pas attendues.
	\item On admet que 1 est l'unique solution de l'équation $g(x) = 1$.
	
	Résoudre, sur l'intervalle $\IntervalleOO{0}{+\infty}$, l'équation $f(x) = x$.
\end{enumerate}

\smallskip

\textbf{Partie C : Étude d'une suite récurrente}

\medskip

On considère la suite $\Suite{u}$ définie par $u_0=\frac12$ et pour tout entier naturel $n$, \[ u_{n+1} = f\big(u_n\big)=u_n^2-u_n\,\ln\big(u_n\big). \]

\begin{enumerate}
	\item Montrer par récurrence que pour tout entier naturel $n$ : $\frac12 \leqslant u_n \leqslant u_{n+1} \leqslant 1$.
	\item Justifier que la suite $\Suite{u}$ converge.
\end{enumerate}

On appelle $\ell$ la limite de la suite $\Suite{u}$ et on admet que $\ell$ vérifie l'égalité $f(\ell) =\ell$.

\begin{enumerate}[resume]
	\item Déterminer la valeur de $\ell$.
\end{enumerate}